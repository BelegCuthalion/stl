\documentclass[a4paper, 12pt]{paper}
\usepackage{amsmath}
\usepackage{amssymb}
\usepackage{titlesec}
\usepackage{fullpage}
\usepackage{tikz-cd}
\usepackage{rotating}
\usepackage{pdflscape}
\usepackage{multicol}
\usepackage{multirow}
\usepackage{diagbox}
\usepackage[left=.5in,right=.5in,top=.5in,bottom=.5in]{geometry}
\usepackage{enumitem}
\usepackage[colorlinks]{hyperref}
\usepackage{bussproofs}

\setitemize{topsep=3pt,parsep=5pt,partopsep=0pt,label=,leftmargin=1.3pc}
\titleformat{\section}[runin]{\normalfont\bfseries}{\thesection}{0.5em}{}
\titlespacing{\section}{0pc}{1.5ex plus .1ex minus .2ex}{1pc}
\titleformat{\subsection}[runin]{\normalfont\bfseries}{\thesubsection}{0.7em}{}
\titlespacing{\subsection}{0pc}{2ex plus .1ex minus .2ex}{1pc}
\titleformat{\subsubsection}[runin]{\normalfont\bfseries}{\thesubsubsection}{0.7em}{}
\titlespacing{\subsubsection}{0pc}{2ex plus .1ex minus .2ex}{1pc}
\newcommand\eqn{\refstepcounter{equation}\tag{\theequation}}
\binoppenalty=\maxdimen
\relpenalty=\maxdimen
\newcommand{\ul}{\ulcorner}
\newcommand{\ur}{\urcorner}
\newcommand{\val}[1]{\ulcorner len1 \urcorner}
\newcommand{\caseref}[1]{\hyperref[#1]{\ref{#1}}}
\newcommand{\rot}{\rotatebox{90}}
\EnableBpAbbreviations

\begin{document}
{\noindent
	v 0.3.2 \\
{\large\textbf{The Cut Rule Is Admissible in GSTL}}
}
\\
\setcounter{section}{-1}

\section{GSTL}‌\\

\subsection{$\p$-formula}
A $\p$-formula is a regular proposition with zero or more ``$\p$'' operators applied. More precisely, if $\Phi$ is the set of all formulas, then the set of $\p$-formulas $P$ is a super-set of $\Phi$, defined recursively as follows:
\[ P ~~::=~~ \Phi ~|~ \p P \]
So if $A$ and $B$ are formulas, then $A$, $A \land B$, $\p A$, $\p \p ~ A$, $\p \p \p ~ \nabla A$ or $\p A \land B$ are all examples of $\p$-formulas. $\p^0 A$ means $A$ and $\p^{n+1} A$ means $\p \p^n A$. $\p \mathcal{S}$ means $\{\p^{m+1} A : \p^m A \in \mathcal{S}\}$. We also write $\p^m A$ for $\{ \p^m A \}$ and $\p^m A^{n+1}$ for $\p^m A^n,\p^m A$. $\mathcal{S}^n$ means $\{ \p^m A^n : \p^m A \in \mathcal{S} \}$.

\subsection{GSTL$^-$} A GSTL sequent $\mathcal{S} \Rightarrow \Delta$ is a binary relation between $\mathcal{S}$, a multi-set of $\p$-formulas, and $\Delta$, a sub-singleton of some formula, defined inductively by the following rules.

\begin{multicols}{3}
	\begin{prooftree}
		\RightLabel{$Id$}
		\AXC{}
		\UIC{$A \Rightarrow A$}
	\end{prooftree}
	\columnbreak
	\begin{prooftree}
		\RightLabel{$Ta$}
		\AXC{}
		\UIC{$\Rightarrow \top$}
	\end{prooftree}
	\columnbreak
	\begin{prooftree}
		\RightLabel{$Ex$}
		\AXC{}
		\UIC{$\p^n \bot \Rightarrow$}
	\end{prooftree}
\end{multicols}
‌\\
\begin{multicols}{3}
	\begin{prooftree}
		\RightLabel{$Lw$}
		\AXC{$\mathcal{S} \Rightarrow \Delta$}
		\UIC{$\mathcal{S}, \p^n A \Rightarrow \Delta$}
	\end{prooftree}
	\columnbreak
	\begin{prooftree}
		\RightLabel{$Rw$}
		\AXC{$\mathcal{S} \Rightarrow$}
		\UIC{$\mathcal{S} \Rightarrow A$}
	\end{prooftree}
	\columnbreak
	\begin{prooftree}
		\RightLabel{$Lc$}
		\AXC{$\mathcal{S} , \p^n A , \p^n A \Rightarrow \Delta$}
		\UIC{$\mathcal{S} , \p^n A \Rightarrow \Delta$}
	\end{prooftree}
\end{multicols}
‌\\
\begin{multicols}{3}
	\begin{prooftree}
		\RightLabel{$L\land_1$}
		\AXC{$\mathcal{S} , \p^n A \Rightarrow \Delta$}
		\UIC{$\mathcal{S} , \p^n A \land B \Rightarrow \Delta$}
	\end{prooftree}
	\columnbreak
	\begin{prooftree}
		\RightLabel{$L\land_2$}
		\AXC{$\mathcal{S} , \p^n B \Rightarrow \Delta$}
		\UIC{$\mathcal{S} , \p^n A \land B \Rightarrow \Delta$}
	\end{prooftree}
	\columnbreak
	\begin{prooftree}
		\RightLabel{$R\land$}
		\AXC{$\mathcal{S} \Rightarrow A$}
		\AXC{$\mathcal{S} \Rightarrow B$}
		\BIC{$\mathcal{S} \Rightarrow A \land B$}
	\end{prooftree}
\end{multicols}
‌\\
\begin{multicols}{3}
	\begin{prooftree}
		\RightLabel{$L\lor$}
		\AXC{$\mathcal{S} , \p^n A \Rightarrow \Delta$}
		\AXC{$\mathcal{S} , \p^n B \Rightarrow \Delta$}
		\BIC{$\mathcal{S} , \p^n A \lor B \Rightarrow \Delta$}
	\end{prooftree}
	\columnbreak
	\begin{prooftree}
		\RightLabel{$R\lor_1$}
		\AXC{$\mathcal{S} \Rightarrow A$}
		\UIC{$\mathcal{S} \Rightarrow A \lor B$}
	\end{prooftree}
	\columnbreak
	\begin{prooftree}
		\RightLabel{$R\lor_2$}
		\AXC{$\mathcal{S} \Rightarrow B$}
		\UIC{$\mathcal{S} \Rightarrow A \lor B$}
	\end{prooftree}
\end{multicols}
‌\\
\begin{multicols}{2}
	\begin{prooftree}
		\RightLabel{$L\rightarrow$}
		\AXC{$\mathcal{S} \Rightarrow \nabla^n A$}
		\AXC{$\mathcal{S} , \p^n B \Rightarrow \Delta$}
		\BIC{$\mathcal{S} , \p^{n+1} A \rightarrow B \Rightarrow \Delta$}
	\end{prooftree}
	\columnbreak
	\begin{prooftree}
		\RightLabel{$R\rightarrow$}
		\AXC{$\p \mathcal{S} , A \Rightarrow B$}
		\UIC{$\mathcal{S} \Rightarrow A \rightarrow B$}
	\end{prooftree}
\end{multicols}
‌\\
\begin{multicols}{2}
	\begin{prooftree}
		\RightLabel{$L\nabla$}
		\AXC{$\mathcal{S} , \p^{n+1} A \Rightarrow \Delta$}
		\UIC{$\mathcal{S} , \p^n \nabla A \Rightarrow \Delta$}
	\end{prooftree}
	
	\begin{prooftree}
		\RightLabel{$R\nabla$}
		\AXC{$\mathcal{S} \Rightarrow \Delta$}
		\UIC{$\p \mathcal{S} \Rightarrow \nabla \Delta$}
	\end{prooftree}
\end{multicols}
$\p \mathcal{S}$ in the premise of $R \rightarrow$ means that this rule is applicable only if all $\p$-formulas in the antecedent, except $A$, have at least one $\p$, which is removed after applying $R \rightarrow$. Also notice that $A$ and $B$ are names for formulas, not $\p$-formulas.
\subsection{$Cut$}
\begin{prooftree}
	\RightLabel{$Cut$}
	\AXC{$\mathcal{S} \Rightarrow A$}
	\AXC{$\mathcal{T} , \p^{l} A \Rightarrow \Delta$}
	\BIC{$\mathcal{S} , \mathcal{T} \Rightarrow \Delta$}
\end{prooftree}

\subsection{$\nabla Cut$} For any $l,k \ge 0$ and $n \ge 1$
\begin{prooftree}
	\AXC{$\mathcal{S} \Rightarrow \nabla^k A$}
	\AXC{$\mathcal{T} , \p^{l+k} A^n \Rightarrow \Delta$}
	\RightLabel{$\nabla Cut$}
	\BIC{$\p^l \mathcal{S} , \mathcal{T} \Rightarrow \Delta$}
\end{prooftree}
$A$ (as a formula, not a $\p$-formula) is called the \textit{cut-formula}.
By $\text{GSTL}$ we mean $\text{GSTL}^- + Cut$
\subsubsection{Rank} Rank of a formula $\varphi$ is defined as
\[ \rho(\varphi) = \begin{cases}
1 & \quad ; \varphi \in P \cup \{ \bot, \top \} \\
\rho(\psi) + 1 & \quad ; \varphi = \nabla \psi \\
max(\rho(\psi), \rho(\theta)) + 1 & \quad ; \varphi = \psi \Box \theta, \Box \in \{ \land , \lor, \rightarrow \}
\end{cases} \]
We also define rank for rule instances and proof trees. For an instance of the $\nabla Cut$ rule $c$ with cut-formula $\varphi$, $\rho(c) = \rho(\varphi)$, $0$ if it's not an instance of the $\nabla Cut$ rule.
For a proof tree $\mathbf{D}$, $\rho(\mathbf{D})$ is the maximum rank of its rule instances.

\subsection{Theorem} For any $\mathcal{S}$ and $\Delta$, if $\small\text{GSTL} \vdash \mathcal{S} \Rightarrow \Delta$ then $\small\text{GSTL}^- + \nabla Cut \vdash \mathcal{S} \Rightarrow \Delta$.

\textit{Proof}: Just replace any occurrence of $Cut$ with $\nabla Cut$, with $k := 0$ and $n := 1$.

\section{Translation} For any $\p$-formula $A$, $\tau : P \mapsto \Phi$ gives its translation as a formula.
\[ \tau(A) = \begin{cases}
	A & \quad ; A \in \Phi \\
	\nabla \tau(B) & \quad ; A = \p B
\end{cases} \]
We also use a natural extension of this translation from multi-sets of $\p$-formulas to multi-sets of formulas: $\tau(\mathcal{S}) = \{ \tau(A) : A \in \mathcal{S} \}$. Notice that $\tau|_\Phi = id$.

\section{Theorem} For any $\mathcal{S}$ and $\Delta$, $\text{GSTL} \vdash \mathcal{S} \Rightarrow \Delta$ if and only if $\text{iSTL} \vdash \tau(\mathcal{S}) \Rightarrow \Delta$.

\textit{Proof}: I. Suppose $\text{GSTL}$ proves $\mathcal{S} \Rightarrow \Delta$ by a proof tree $\mathbf{D}$. We will construct a proof tree in $\text{iSTL}$ for $\tau(\mathcal{S}) \Rightarrow \Delta$. By induction on $h(\mathbf{D})$, the induction hypothesis states that for any $\text{GSTL}$ proof tree $\mathbf{D}'$ for $\mathcal{S}' \Rightarrow \Delta'$ such that $h(\mathbf{D}') < h(\mathbf{D})$, there exists an $\text{iSTL}$ proof tree IH$(\mathbf{D'})$ for $\tau(\mathcal{S}') \Rightarrow \Delta'$. Now we consider different cases for the last rule of $\mathbf{D}$. In all cases, we denote the immediate sub-trees of $\mathbf{D}$ by $\mathbf{D_i} ~(0 \leq i)$. Notice that the rule name in parens is the last rule of $\mathcal{D}$ and in GSTL, but our construction in each case is in iSTL.
\begin{enumerate}
	\item[1,2.] ($Id$),($Ta$) iSTL has these as axioms.
	\setcounter{enumi}{2}

	\item ($Ex$) \todo{changing $N$ will probably fix this, but aint sure if it'll mess with completeness stuff}

	\item[4-6] ($Lw$),($Rw$),($Lc$) Just apply the same rule in iSTL on IH$(\mathbf{D_0})$.
	\setcounter{enumi}{6}

	\item ($L\land_1$) IH($\mathbf{D_0}$) is of the form $\tau(\mathcal{S}) , \nabla^n A \Rightarrow \Delta$. By $L\land_1$ we have $\tau(\mathcal{S}) , \nabla^n A \land \nabla^n B \Rightarrow \Delta$. By $Cut$ with lemma \ref{lem:i-nabla-dist-and} we have $\tau(\mathcal{S}) , \nabla^n (A \land B) \Rightarrow \Delta$.
	
	\item ($L\land_2$) The same.
	
	\item ($R\land$) Just apply iSTL's $R\land$ on IH$(\mathbf{D_0})$ and IH$(\mathbf{D_1})$.
	
	\item ($L\lor$) IH($\mathbf{D_0}$) and IH($\mathbf{D_1}$) are of the form $\tau(\mathcal{S}) , \nabla^n A \Rightarrow \Delta$ and $\tau(\mathcal{S}) , \nabla^n B \Rightarrow \Delta$ respectively. By $L\lor$ we have $\tau(\mathcal{S}) , \nabla^n A \lor \nabla^n B \Rightarrow \Delta$. By $Cut$ with lemma \ref{lem:i-nabla-dist-or} we have $\tau(\mathcal{S}) , \nabla^n (A \lor B) \Rightarrow \Delta$.
	
	\item[11,12.] ($R\lor_{1/2}$) Just apply iSTL's $R\lor_{1/2}$ on IH$(\mathbf{D_0})$.
	\setcounter{enumi}{12}
	
	\item ($L\rightarrow$) IH($\mathbf{D_0}$) and IH($\mathbf{D_1}$) are of the form $\tau(\mathcal{S}) \Rightarrow \nabla^n A$ and $\tau(\mathcal{S}) , \nabla^n B \Rightarrow \Delta$ respectively. By $L\rightarrow$ we have $\tau(\mathcal{S}) , \nabla (\nabla^n A \rightarrow \nabla^n B) \Rightarrow \Delta$. \todo{}
	
	\todo{}
\end{enumerate}


\section{Lemma}\label{true-assum} If $~\small\text{GSTL}^- + \nabla Cut \vdash \mathcal{S} \cup [\top^n | \epsilon_r] \Rightarrow \Delta$ then $\small\text{GSTL}^- + \nabla Cut \vdash \mathcal{S} \Rightarrow \Delta$ with a proof of at most the same rank.

\textit{Proof:} Suppose $\mathbf{D}$ is a proof tree for $\mathcal{S} \cup [\top^n | \epsilon_r] \Rightarrow \Delta$. By induction on $h(\mathbf{D})$, induction hypothesis states that for any proof tree $\mathbf{D}'$ with conclusion $\mathcal{S'} \cup [\top^{n'} | \epsilon_{r'}] \Rightarrow \Delta'$ such that $h(\mathbf{D}') < h(\mathbf{D})$, there is a proof $\mathbf{D}''$ of $\mathcal{S}' \Rightarrow \Delta'$ such that $\rho(\mathbf{D}'') \leq \rho(\mathbf{D}')$.

Consider different cases for the last rule of $\mathbf{D}$, with possible sub-trees $\mathbf{D_0}$ and $\mathbf{D_1}$. $Ta$ and $Ex$ cases are trivially ruled out. In $Id$ case (which implies $n = 1$ and $r = 0$), we have $\epsilon \Rightarrow \top$ by $Ta$. In $Lw$ case, where an instance of $\top$ is principal and $n = 1$, $\mathbf{D_0}$ itself proves the desired sequent. If $n > 1$, then the induction hypothesis with $n' = n - 1$ gives the desired sequent. $Lc$ on an instance of $\top$ is similar, with $n' = n + 1$. In all other cases, just apply induction hypothesis on $\mathbf{D_0}$ (and possibly $\mathbf{D_1}$), then the same last rule. Notice in all cases we must apply induction hypothesis with $n' = n$ and $r' = r$, except for $R\rightarrow$ and $R\nabla$, in which we apply it with $r' = r + 1$ and $r' = r - 1$ respectively. Also notice that $\nabla Cut$ is not used except in $\nabla Cut$ case, where it is applied with the same cut-formula, so the resulting proof tree is not of a higher rank than $\mathbf{D}$.

\section{Theorem}\label{cut-admis} \emph{Cut Reduction for GSTL: } For any $\mathcal{S}$, $n>0$, $A$, $\mathcal{T}$, $k$, $l$ and $\Delta$, if $~\small\text{GSTL}^- + \nabla Cut \vdash \mathcal{S} \Rightarrow \nabla^k A$ and $\small\text{GSTL}^- + \nabla Cut \vdash$ $\mathcal{T} \cup [A^n | \epsilon_{k+l}] \Rightarrow \Delta$ with proof trees of ranks less than $\rho(A)$, then
 $\small\text{GSTL}^- + \nabla Cut \vdash [ \mathcal{S} | \epsilon_l ] \cup \mathcal{T} \Rightarrow \Delta$ also with a proof tree of a rank less than $\rho(A)$.
 
\emph{Proof:}
Let $\mathbf{D_0}$ and $\mathbf{D_1}$ be proof trees of ranks less than $\rho(A)$, with conclusions $\mathcal{S} \Rightarrow \nabla^k A$ and $\mathcal{T} \cup [A^n | \epsilon_{k+l}] \Rightarrow \Delta$ and heights $h(\mathbf{D_0})$ and $h(\mathbf{D_1})$ respectively.
\begin{prooftree}
	\noLine
	\AXC{$\mathbf{D_0}$}
	\UIC{$\mathcal{S} \Rightarrow \nabla^k A$}
	
	\noLine
	\AXC{$\mathbf{D_1}$}
	\UIC{$\mathcal{T} \cup [A^n | \epsilon_{l+k}] \Rightarrow \Delta$}
	
	\dashedLine \RightLabel{$\nabla Cut$}
	\BIC{$[ \mathcal{S} | \epsilon_l ] \cup \mathcal{T} \Rightarrow \Delta$}
\end{prooftree}
By induction on $h(\mathbf{D_0}) + h(\mathbf{D_1})$, the induction hypothesis states that for any proof trees $\mathbf{D_0}'$ with conclusion $\mathcal{S}' \Rightarrow \nabla^{k'} A'$ and $\mathbf{D_1}'$ with conclusion $\mathcal{T}' \cup [A'^m | \epsilon_{k'+l'}] \Rightarrow \Delta'$ such that $h(\mathbf{D_0}') + h(\mathbf{D_1}') < h(\mathbf{D_0}) + h(\mathbf{D_1})$, if $\rho(\mathbf{D_0}'),\rho(\mathbf{D_1}') < \rho(A')$ then there exists a proof tree $\mathbf{D}'$ with conclusion $[ \mathcal{S}' | \epsilon_{l'} ] \cup \mathcal{T}' \Rightarrow \Delta'$ such that $\rho(\mathbf{D}') < \rho(A')$.

\begin{table}
	\centering
	\begin{tabular}{|c|p{.2cm}|*{17}{p{.4cm}|}}
		\hline
		\multicolumn{2}{|c|}{\backslashbox[2.3cm]{$\mathbf{D_0}$}{$\mathbf{D_1}$}} & \rot{$Id$} & \rot{$Ta$} & \rot{$Ex$} & \rot{$Lw$} & \rot{$Lc$} & \rot{$\nabla Cut$} & \rot{$L \land_1$} & \rot{$L \land_2$} & \rot{$L \lor$} & \rot{$L \nabla$} & \rot{$L \rightarrow$} & \rot{$Rw$} & \rot{$R \land$} & \rot{$R \lor_1$} & \rot{$R \lor_2$} & \rot{$R \rightarrow$} & \rot{$R \nabla$} \\
		\hline
		\multicolumn{2}{|c|}{$Id$} & \multicolumn{17}{c|}{\caseref{c:id-*}} \\ \hline
		\multicolumn{2}{|c|}{$Ta$} & \multicolumn{17}{c|}{\caseref{c:ta-*}} \\ \hline
		\multicolumn{2}{|c|}{$Ex$} & \multicolumn{17}{c|}{$\times$} \\ \hline
		\multicolumn{2}{|c|}{$Lw$} & \multicolumn{17}{c|}{\caseref{c:lw-*}} \\ \hline
		\multicolumn{2}{|c|}{$Lc$} & \multicolumn{17}{c|}{\caseref{c:lc-*}} \\ \hline
		\multicolumn{2}{|c|}{$\nabla Cut$} & \multicolumn{17}{c|}{\caseref{c:cut-*}} \\ \hline
		\multicolumn{2}{|c|}{$L \land_1$} & \multicolumn{17}{c|}{\caseref{c:la1-*}} \\ \hline
		\multicolumn{2}{|c|}{$L \land_2$} & \multicolumn{17}{c|}{\caseref{c:la2-*}} \\ \hline
		\multicolumn{2}{|c|}{$L \lor$} & \multicolumn{17}{c|}{\caseref{c:lo-*}} \\ \hline
		\multicolumn{2}{|c|}{$L \nabla$} & \multicolumn{17}{c|}{\caseref{c:ln-*}} \\ \hline
		\multicolumn{2}{|c|}{$L \rightarrow$} & \multicolumn{17}{c|}{\caseref{c:li-*}} \\ \hline
		\multicolumn{2}{|c|}{$Rw$} & \multicolumn{17}{c|}{\caseref{c:rw-*}} \\ \hline
		$R \lor_1,$ & \multirow{2}{*}{\tiny Prin.} &
		\multirow{2}{*}{\caseref{c:*-id}} &
		\multirow{2}{*}{$\times$} &
		\multirow{2}{*}{\caseref{c:*-ex}} &
		\multirow{2}{*}{\caseref{c:*-lw-p}} &
		\multirow{2}{*}{\caseref{c:*-lc-p}} &
		\multirow{2}{*}{$\times$} &
		\multirow{2}{*}{\caseref{c:ra-la1}} &
		\multirow{2}{*}{\caseref{c:ra-la2}} &
		\caseref{c:ro1-lo} &
		\multirow{2}{*}{\caseref{c:rn-*}} &
		\multirow{2}{*}{\caseref{c:ri-li}} &
		\multirow{2}{*}{$\times$} &
		\multirow{2}{*}{$\times$} &
		\multirow{2}{*}{$\times$} &
		\multirow{2}{*}{$\times$} &
		\multirow{2}{*}{$\times$} &
		\multirow{2}{*}{$\times$} \\
		$R \lor_2,$ & & & & & & & & & & \caseref{c:ro2-lo} & & & & & & & &\\
		\cline{2-19}
		$R \land,$ & \multirow{3}{*}{\tiny !Prin.} &
		\multirow{3}{*}{$\times$} &
		\multirow{3}{*}{$\times$} &
		\multirow{3}{*}{$\times$} &
		\multirow{3}{*}{\caseref{c:*-lw}} &
		\multirow{3}{*}{\caseref{c:*-lc}} &
		\multirow{3}{*}{\caseref{c:*-cut}} &
		\multirow{3}{*}{\caseref{c:*-la1}} &
		\multirow{3}{*}{\caseref{c:*-la2}} &
		\multirow{3}{*}{\caseref{c:*-lo}} &
		\multirow{3}{*}{\caseref{c:*-ln}} &
		\multirow{3}{*}{\caseref{c:*-li}} &
		\multirow{3}{*}{\caseref{c:*-rw}} &
		\multirow{3}{*}{\caseref{c:*-ra}} &
		\multirow{3}{*}{\caseref{c:*-ro1}} &
		\multirow{3}{*}{\caseref{c:*-ro2}} &
		\multirow{3}{*}{\caseref{c:*-ri}} &
		\multirow{3}{*}{\caseref{c:*-rn}} \\
		$R \rightarrow,$ & & & & & & & & & & & & & & & & & & \\
		$R \nabla$ & & & & & & & & & & & & & & & & & & \\
		\hline
		\multicolumn{2}{|c|}{\slashbox[2.3cm]{$\mathbf{D_0}$}{$\mathbf{D_1}$}} & \rot{$Id$} & \rot{$Ta$} & \rot{$Ex$} & \rot{$Lw$} & \rot{$Lc$} & \rot{$\nabla Cut$} & \rot{$L \land_1$} & \rot{$L \land_2$} & \rot{$L \lor$} & \rot{$L \nabla$} & \rot{$L \rightarrow$} & \rot{$Rw$} & \rot{$R \land$} & \rot{$R \lor_1$} & \rot{$R \lor_2$} & \rot{$R \rightarrow$} & \rot{$R \nabla$} \\
		\hline
	\end{tabular}
\end{table}

We proceed by case analysis on the last rule in $\mathbf{D_0}$. We can either apply the induction hypothesis on a pair of trees shorter than $\mathbf{D_0}$ and $\mathbf{D_1}$, or just cut a formula of a lower rank than $A$, to reach the desired sequent without increasing the cut-rank. This is easy when the cut-formula is also present in the immediate sub-tree(s) of $\mathbf{D_0}$, as it is the case for the left-rules. In these cases, we somehow ``commute'' the last rule of $\mathbf{D_0}$ with an application of induction hypothesis for its premise(s) and $\mathbf{D_1}$. In right-rule cases, where $A$ is principal and not present in the sub-tree(s) of $\mathbf{D_0}$, we consider further cases, this time for the last rule of $\mathbf{D_1}$. We can do the same with $\mathbf{D_1}$ in the cases that no instance of $A$ is principal in its last rule. Almost all other cases force a particular construction for $A$ so we can apply $\nabla Cut$ on its sub-formulas. The only exception is the case where $\mathbf{D_0}$ ends with $R \nabla$. The induction hypothesis is strong enough to handle this one in particular.

\begin{enumerate}
	\item ($Id$) \label{c:id-*} We have $\Gamma = A$ and the desired sequent should be of the form $\nabla^l A , \Sigma \Rightarrow \Delta$. $n-1$ applications of $Lc$ on $\mathbf{D_1}$ proves such sequent.

	\item ($Ta$) \label{c:ta-*} $\mathbf{D_0}$ proves $\Rightarrow \top$. $\mathcal{T} \Rightarrow \Delta$ is proved by lemma \ref{true-assum} for $\mathbf{D_1}$.

	\item ($Lw$) \label{c:lw-*} $\mathbf{D_0}$ proves $\Gamma , B \Rightarrow A$ and has a subproof $\mathbf{D_0}'$ of $\Gamma \Rightarrow A$.
	\begin{prooftree}
		\noLine
		\AXC{$\mathbf{D_0}'$}
		\UIC{$\Gamma \Rightarrow A$}
		\RightLabel{$Lw$}
		\UIC{$\Gamma , B \Rightarrow A$}

		
		\noLine
		\AXC{$\mathbf{D_1}$}
		\UIC{$\Sigma , (\nabla^l A)^n \Rightarrow \Delta$}

		\dashedLine\RightLabel{$MC$}
		\BIC{$\nabla^l \Gamma , \nabla^l B , \Sigma \Rightarrow \Delta$}
	\end{prooftree}
	From the induction hypothesis for $\mathbf{D_0}'$ and $\mathbf{D_1}$, we have a proof of $\nabla^l \Gamma , \Sigma \Rightarrow \Delta$ with a lower rank than $\rho(A)$. By applying $Lw$ we have $\nabla^l \Gamma , \nabla^l B , \Sigma \Rightarrow \Delta$, without increasing the cut-rank.
	\begin{prooftree}
		\noLine
		\AXC{$\mathbf{D_0}'$}
		\UIC{$\Gamma \Rightarrow A$}
		
		\noLine
		\AXC{$\mathbf{D_1}$}
		\UIC{$\Sigma , (\nabla^l A)^n \Rightarrow \Delta$}
		
		\RightLabel{IH}
		\BIC{$\nabla^l \Gamma , \Sigma \Rightarrow \Delta$}
		
		\RightLabel{$Lw$}
		\UIC{ $\nabla^l \Gamma , \nabla^l B , \Sigma \Rightarrow \Delta$}
	\end{prooftree}

	\item ($Lc$) \label{c:lc-*} $\mathbf{D_0}$ proves $\mathcal{S} | \Gamma , B | \mathcal{R} \Rightarrow \nabla^k A$ and has a subproof $\mathbf{D_0}'$ of $\mathcal{S} | \Gamma , B , B | \mathcal{R} \Rightarrow \nabla^k A$.
\begin{prooftree}
	\noLine
	\AXC{$\mathbf{D_0}'$}
	\UIC{$\mathcal{S} | \Gamma , B , B | \mathcal{R} \Rightarrow \nabla^k A$}
	\RightLabel{$Lc$}
	\UIC{$\mathcal{S} | \Gamma , B | \mathcal{R} \Rightarrow \nabla^k A$}
	
	
	\noLine
	\AXC{$\mathbf{D_1}$}
	\UIC{$\mathcal{T} \cup [A^n | \epsilon_{l+k}] \Rightarrow \Delta$}
	
	\dashedLine\RightLabel{$\nabla Cut$}
	\BIC{$[\mathcal{S} | \Gamma , B | \mathcal{R} | \epsilon_l] \cup \mathcal{T} \Rightarrow \Delta$}
\end{prooftree}
From the induction hypothesis for $\mathbf{D_0}'$ and $\mathbf{D_1}$, we have a proof of $[\mathcal{S} | \Gamma , B , B | \mathcal{R} | \epsilon_l] \cup \mathcal{T} \Rightarrow \Delta$ with rank lower than that of $A$. By applying $Lc$ in the proper place, we have $[\mathcal{S} | \Gamma , B | \mathcal{R} | \epsilon_l] \cup \mathcal{T} \Rightarrow \Delta$, without increasing the cut-rank.
\begin{prooftree}
	\noLine
	\AXC{$\mathbf{D_0}'$}
	\UIC{$\mathcal{S} | \Gamma , B , B | \mathcal{R} \Rightarrow \nabla^k A$}
	
	\noLine
	\AXC{$\mathbf{D_1}$}
	\UIC{$\mathcal{T} \cup [A^n | \epsilon_{l+k}] \Rightarrow \Delta$}
	
	\RightLabel{IH}
	\BIC{$[\mathcal{S} | \Gamma , B , B | \mathcal{R} | \epsilon_l] \cup \mathcal{T} \Rightarrow \Delta$}
	
	\RightLabel{$Lc$}
	\UIC{$[\mathcal{S} | \Gamma , B | \mathcal{R} | \epsilon_l] \cup \mathcal{T} \Rightarrow \Delta$}
\end{prooftree}

	\item \label{c:cut-*} ($\nabla Cut$) Assume $\mathbf{D_0}$ ends with a $\nabla Cut$ on $A'$, which by assumption must be of a lower rank than $A$.
	\begin{prooftree}
		\noLine
		\AXC{$\mathbf{D_0}'$}
		\UIC{$\mathcal{S} \Rightarrow \nabla^{k'} A'$}
		
		\noLine
		\AXC{$\mathbf{D_0}''$}
		\UIC{$\mathcal{R} , \p^{l'+k'} A'^{n'} \Rightarrow \nabla^k A$}
		
		\RightLabel{$\nabla Cut$}
		\BIC{$\p^{l'} \mathcal{S} , \mathcal{R} \Rightarrow \nabla^k A$}
		
		
		\noLine
		\AXC{$\mathbf{D_1}$}
		\UIC{$\mathcal{T} , \p^{l+k} A^n \Rightarrow \Delta$}
		
		\dashedLine\RightLabel{$\nabla Cut$}
		\BIC{$\p^{l'+l} \mathcal{S} , \p^l \mathcal{R} , \mathcal{T} \Rightarrow \Delta$}
	\end{prooftree}
	We can use the induction hypothesis to remove $A$ first, then cut $A'$.
	\begin{prooftree}
		\noLine
		\AXC{$\mathbf{D_0}'$}
		\UIC{$\mathcal{S} \Rightarrow \nabla^{k'} A'$}
		
		\noLine
		\AXC{$\mathbf{D_0}''$}
		\UIC{$\mathcal{R} , \p^{l'+k'} A'^{n'} \Rightarrow \nabla^k A$}

		\noLine
		\AXC{$\mathbf{D_1}$}
		\UIC{$\mathcal{T} , \p^{l+k} A^n \Rightarrow \Delta$}

		\RightLabel{IH}
		\BIC{$\p^l \mathcal{R} , \p^{l'+k'+l} A'^{n'} , \mathcal{T} \Rightarrow \Delta$}
		

		\RightLabel{$\nabla Cut$}
		\BIC{$\p^{l'+l} \mathcal{S} , \p^l \mathcal{R} , \mathcal{T} \Rightarrow \Delta$}
	\end{prooftree}

	\item \label{c:la1-*} ($L\land_1$) $\mathbf{D_0}$ proves $\Gamma , \nabla^r (B \land C) \Rightarrow A$ and has a subproof $\mathbf{D_0}'$ of $\Gamma , \nabla^r B \Rightarrow A$.
	\begin{prooftree}
		\noLine
		\AXC{$\mathbf{D_0}'$}
		\UIC{$\Gamma , \nabla^r B \Rightarrow A$}
		\UIC{$\Gamma , \nabla^r (B \land C) \Rightarrow A$}
		
		
		\noLine
		\AXC{$\mathbf{D_1}$}
		\UIC{$\Sigma , (\nabla^l A)^n \Rightarrow \Delta$}
		
		\dashedLine\RightLabel{$MC$}
		\BIC{$\nabla^l \Gamma , \nabla^{r+l} (B \land C) , \Sigma \Rightarrow \Delta$}
	\end{prooftree}
	From the induction hypothesis for $\mathbf{D_0}'$ and $\mathbf{D_1}$, we have a proof of $\nabla^l \Gamma , \nabla^{r+l} B , \Sigma \Rightarrow \Delta$ with rank lower than that of $A$. By applying $L\land$ we have $\nabla^l \Gamma , \nabla^{r+l} (B \land C) , \Sigma \Rightarrow \Delta$, without increasing the cut-rank.
	\begin{prooftree}
		\noLine
		\AXC{$\mathbf{D_0}'$}
		\UIC{$\Gamma , \nabla^r B \Rightarrow A$}
		
		\noLine
		\AXC{$\mathbf{D_1}$}
		\UIC{$\Sigma , (\nabla^l A)^n \Rightarrow \Delta$}
		
		\RightLabel{IH}
		\BIC{$\nabla^l \Gamma , \nabla^{r+l} B , \Sigma \Rightarrow \Delta$}
		
		\RightLabel{$L\land$}
		\UIC{$\nabla^l \Gamma , \nabla^{r+l} (B \land C) , \Sigma \Rightarrow \Delta$}
	\end{prooftree}

	\item \label{c:la2-*} ($L\land_2$) The other $L\land$ is the same.
	
	\item \label{c:lo-*} ($L\lor$)  $\mathbf{D_0}$ proves $\mathcal{S} | \Gamma , B \lor C | \mathcal{R} \Rightarrow \nabla^k A$ and has subproofs $\mathbf{D_0}'$ of $\mathcal{S} | \Gamma , B | \mathcal{R} \Rightarrow \nabla^k A$ and $\mathbf{D_0}''$ of $\mathcal{S} | \Gamma , C | \mathcal{R} \Rightarrow \nabla^k A$.
	\begin{prooftree}
		\noLine
		\AXC{$\mathbf{D_0}'$}
		\UIC{$\mathcal{S} | \Gamma , B | \mathcal{R} \Rightarrow \nabla^k A$}
		\noLine
		\AXC{$\mathbf{D_0}''$}
		\UIC{$\mathcal{S} | \Gamma , C | \mathcal{R} \Rightarrow \nabla^k A$}
		\RightLabel{$L\lor$}
		\BIC{$\mathcal{S} | \Gamma , B \lor C | \mathcal{R} \Rightarrow \nabla^k A$}
		
		
		\noLine
		\AXC{$\mathbf{D_1}$}
		\UIC{$\mathcal{T} \cup [A^n | \epsilon_{l+k}] \Rightarrow \Delta$}
		
		\dashedLine\RightLabel{$\nabla Cut$}
		\BIC{$[\mathcal{S} | \Gamma , B \lor C | \mathcal{R} | \epsilon_l] \cup \mathcal{T} \Rightarrow \Delta$}
	\end{prooftree}
	From the induction hypothesis for $\mathbf{D_0}'$ and $\mathbf{D_1}$, we have a proof of $[\mathcal{S} | \Gamma , B | \mathcal{R} | \epsilon_l] \cup \mathcal{T} \Rightarrow \Delta$ with rank lower than that of $A$. Also from the induction hypothesis for $\mathbf{D_0}''$ and $\mathbf{D_1}$, we have a proof of $[\mathcal{S} | \Gamma , C | \mathcal{R} | \epsilon_l] \cup \mathcal{T} \Rightarrow \Delta$ with rank lower than that of $A$. By applying $L\lor$ in the proper place, we have $[\mathcal{S} | \Gamma , B \lor C | \mathcal{R} | \epsilon_l] \cup \mathcal{T} \Rightarrow \Delta$, without increasing the cut-rank.
	\begin{prooftree}
		\noLine
		\AXC{$\mathbf{D_0}'$}
		\UIC{$\mathcal{S} | \Gamma , B | \mathcal{R} \Rightarrow \nabla^k A$}
		
		\noLine
		\AXC{$\mathbf{D_1}$}
		\UIC{$\mathcal{T} \cup [A^n | \epsilon_{l+k}] \Rightarrow \Delta$}
		
		\RightLabel{IH}
		\BIC{$[\mathcal{S} | \Gamma , B | \mathcal{R} | \epsilon_l] \cup \mathcal{T} \Rightarrow \Delta$}
		
		\noLine
		\AXC{$\mathbf{D_0}''$}
		\UIC{$\mathcal{S} | \Gamma , C | \mathcal{R} \Rightarrow \nabla^k A$}
		
		\noLine
		\AXC{$\mathbf{D_1}$}
		\UIC{$\mathcal{T} \cup [A^n | \epsilon_{l+k}] \Rightarrow \Delta$}
		
		\RightLabel{IH}
		\BIC{$[\mathcal{S} | \Gamma , C | \mathcal{R} | \epsilon_l] \cup \mathcal{T} \Rightarrow \Delta$}
		
		\RightLabel{$L\lor$}
		\BIC{$[\mathcal{S} | \Gamma , B \lor C | \mathcal{R} | \epsilon_l] \cup \mathcal{T} \Rightarrow \Delta$}
	\end{prooftree}
	
	\item \label{c:ln-*} ($L\nabla$) $\mathbf{D_0}$ proves $\mathcal{S} , \p^r \nabla B \Rightarrow \nabla^k A$ and has a subproof $\mathbf{D_0}'$ of $\mathcal{S} , \p{r+1} B \Rightarrow \nabla^k A$.
	\begin{prooftree}
		\noLine
		\AXC{$\mathbf{D_0}'$}
		\UIC{$\mathcal{S}  , \p^{r+1} B \Rightarrow \nabla^k A$}
		\RightLabel{$L\nabla$}
		\UIC{$\mathcal{S} , \p^r \nabla B \Rightarrow \nabla^k A$}

		\noLine
		\AXC{$\mathbf{D_1}$}
		\UIC{$\mathcal{T} , \p^{l+k} A^n \Rightarrow \Delta$}

		\dashedLine\RightLabel{$\nabla Cut$}
		\BIC{$\p^l \mathcal{S} , \p^{r+l} \nabla B , \mathcal{T} \Rightarrow \Delta$}
	\end{prooftree}
	From the induction hypothesis for $\mathbf{D_0}'$ and $\mathbf{D_1}$, we have a proof of $\p^l \mathcal{S} , \p^{r+l+1} B , \mathcal{T} \Rightarrow \Delta$ with rank lower than that of $A$. By applying $L\nabla$ we have $\p^l \mathcal{S} , \p^{r+l} \nabla B , \mathcal{T} \Rightarrow \Delta$, without increasing the cut-rank.
	\begin{prooftree}
		\noLine
		\AXC{$\mathbf{D_0}'$}
		\UIC{$\mathcal{S}  , \p^{r+1} B \Rightarrow \nabla^k A$}
		
		\noLine
		\AXC{$\mathbf{D_1}$}
		\UIC{$\mathcal{T} , \p^{l+k} A^n \Rightarrow \Delta$}
		
		\RightLabel{IH}
		\BIC{$\p^l \mathcal{S} , \p^{r+l+1} B , \mathcal{T} \Rightarrow \Delta$}
		
		\RightLabel{$L\nabla$}
		\UIC{$\p^l \mathcal{S} , \p^{r+l} \nabla B , \mathcal{T} \Rightarrow \Delta$}
	\end{prooftree}
	
	\item \label{c:li-*} ($L\rightarrow$) $\mathbf{D_0}$ proves $\Gamma , \nabla^{r+1} (B \rightarrow C) \Rightarrow A$ and has subproofs $\mathbf{D_0}'$ of $\Gamma \Rightarrow \nabla^r B$ and $\mathbf{D_0}''$ of $\Gamma , \nabla^r C \Rightarrow A$.
	\begin{prooftree}
		\noLine
		\AXC{$\mathbf{D_0}'$}
		\UIC{$\Gamma \Rightarrow \nabla^r B$}
		\noLine
		\AXC{$\mathbf{D_0}''$}
		\UIC{$\Gamma , \nabla^r C \Rightarrow A$}
		\RightLabel{$L\rightarrow$}
		\BIC{$\Gamma , \nabla^{r+1} (B \rightarrow C) \Rightarrow A$}
		
		
		\noLine
		\AXC{$\mathbf{D_1}$}
		\UIC{$\Sigma , (\nabla^l A)^n \Rightarrow \Delta$}
		
		\dashedLine\RightLabel{$MC$}
		\BIC{$\nabla^l \Gamma , \nabla^{r+l+1} (B \rightarrow C) , \Sigma \Rightarrow \Delta$}
	\end{prooftree}
	From the induction hypothesis for $\mathbf{D_0}''$ and $\mathbf{D_1}$, we have a proof of
	$\nabla^l \Gamma , \nabla^{r+l} C , \Sigma \Rightarrow \Delta$ with rank lower than that of $A$. Now in order to $L\rightarrow$ to be applicable on the resulting sequent and $\mathbf{D_0}'$, we must first identify their contexts. By $l$ times applications of $N'$ and then $Lw$ with $\Sigma$ on $\mathbf{D_0}'$, we have $\nabla^l \Gamma , \Sigma \Rightarrow \nabla^{r+l} B$. Then we can derive $\nabla^l \Gamma , \nabla^{r+l+1} (B \rightarrow C) , \Sigma \Rightarrow \Delta$, from $L\rightarrow$.
	\begin{prooftree}
		\noLine
		\AXC{$\mathbf{D_0}'$}
		\UIC{$\Gamma \Rightarrow \nabla^r B$}
		\doubleLine \RightLabel{$N'$}
		\UIC{$\nabla^l \Gamma \Rightarrow \nabla^{r+l} B$}
		\doubleLine \RightLabel{$Lw$}
		\UIC{$\nabla^l \Gamma , \Sigma \Rightarrow \nabla^{r+l} B$}
		
		\noLine
		\AXC{$\mathbf{D_0}''$}
		\UIC{$\Gamma , \nabla^r C \Rightarrow A$}
		\noLine
		\AXC{$\mathbf{D_1}$}
		\UIC{$\Sigma , (\nabla^l A)^n \Rightarrow \Delta$}
		\RightLabel{IH}
		\BIC{$\nabla^l \Gamma , \nabla^{r+l} C , \Sigma \Rightarrow \Delta$}
		
		\RightLabel{$L\rightarrow$}
		\BIC{$\nabla^l \Gamma , \nabla^{r+l+1} (B \rightarrow C) , \Sigma \Rightarrow \Delta$}
	\end{prooftree}
	
	\item \label{c:rw-*} ($Rw$) $\mathbf{D_0}$ proves $\mathcal{S} \Rightarrow \nabla^k A$ and has subproofs $\mathbf{D_0}'$ of $\mathcal{S} \Rightarrow$.
	\begin{prooftree}
		\noLine
		\AXC{$\mathbf{D_0}'$}
		\UIC{$\mathcal{S} \Rightarrow$}
		\RightLabel{$Rw$}
		\UIC{$\mathcal{S} \Rightarrow \nabla^k A$}
		
		\noLine
		\AXC{$\mathbf{D_1}$}
		\UIC{$\mathcal{T} , \p^{l+k} A^n \Rightarrow \Delta$}
		
		\dashedLine \RightLabel{$\nabla Cut$}
		\BIC{$\p^l \mathcal{S} , \mathcal{T} \Rightarrow \Delta$}
	\end{prooftree}
	$l$ times applications of $R\nabla$ on $\mathbf{D_0}'$ followed by proper $Lw$ and $Rw$ yields the desired sequent.
	\begin{prooftree}
		\noLine
		\AXC{$\mathbf{D_0}'$}
		\UIC{$\mathcal{S} \Rightarrow$}
		\doubleLine \RightLabel{$R\nabla$}
		\UIC{$\p^l \mathcal{S} \Rightarrow$}
		\doubleLine \RightLabel{$Lw$}
		\UIC{$\p^l \mathcal{S} , \mathcal{T} \Rightarrow$}
		\RightLabel{$Rw$}
		\UIC{$\p^l \mathcal{S} , \mathcal{T} \Rightarrow \Delta$}
	\end{prooftree}


	\item[13-17.] ($R*$) In the cases that the last rule in $\mathbf{D_0}$ is any of the logical right-rules $R*$ ($* \in \{ \land, \lor_{i}, \rightarrow, \nabla \}$), its principal formula would be $\nabla^k A$. The only admissible $R*$ with $k > 0$ would be $R\nabla$. For the other rules, this forces $k = 0$, plus a particular construction for $A$ based on the last rule. In each case, we will proceed by case analysis on the last rule of $\mathbf{D_1}$, but to avoid repeating similar cases, we will separate cases by whether the last rule of $\mathbf{D_1}$ is logical, and if $A$ is also the principal formula there or not. So we have the following groups of cases:
	(I) Cases in which $A$ is not principal in $\mathbf{D_1}$,
	(II) cases in which the last rule of $\mathbf{D_1}$ is any of the logical left-rules and (an instance of) $A$ is its principal formula, and (III) $A$ is principal in $\mathbf{D_1}$, but the rule is either structural or an axiom. Notice that the solution for the different cases of $\mathbf{D_1}$ in (I) and (III) is shared by all cases of $\mathbf{D_0}$, since it does not depend on the last rule of $\mathbf{D_0}$. This may seem like we have commuted the case analysis on $\mathbf{D_1}$ with the one on $\mathbf{D_0}$, which is not true. So each item in (I) or (III) handles many different cases with similar solutions.
	On the other hand, each case of $\mathbf{D_0}$ in (II) determines a single case for $\mathbf{D_1}$. The parens before each case show the last rule of $\mathbf{D_0}$ and $\mathbf{D_1}$ respectively.

	I. If $A$ is not principal in the last rule of $\mathbf{D_1}$.
	
	\begin{enumerate}[label={\alph*.}]
		\item \label{c:*-lw} ($R*, Lw$)
		\begin{prooftree}
			\noLine
			\AXC{$\mathbf{D_0}$}
			\UIC{$\mathcal{S} \Rightarrow \nabla^k A$}
			
			\noLine
			\AXC{$\mathbf{D_1}'$}
			\UIC{$\mathcal{T} \cup [A^n | \epsilon_{l+k} ] \Rightarrow \Delta$}
			\RightLabel{$Lw$}
			\UIC{$\mathcal{T} \cup [A^n | \epsilon_{l+k}] \cup [B | \epsilon_r] \Rightarrow \Delta$}
			
			\dashedLine\RightLabel{$\nabla Cut$}
			\BIC{$[\mathcal{S} | \epsilon_l] \cup \mathcal{T} \cup [B | \epsilon_r] \Rightarrow \Delta$}
		\end{prooftree}
		By induction hypothesis for $\mathbf{D_0}$ and $\mathbf{D_1}'$ we have
		\begin{prooftree}
			\noLine
			\AXC{$\mathbf{D_0}$}
			\UIC{$\mathcal{S} \Rightarrow \nabla^k A$}
			
			\noLine
			\AXC{$\mathbf{D_1}'$}
			\UIC{$\mathcal{T} \cup [A^n | \epsilon_{l+k}] \Rightarrow \Delta$}
			\RightLabel{IH}
			\BIC{$[\mathcal{S} | \epsilon_l] \cup \mathcal{T} \Rightarrow \Delta$}
			
			\RightLabel{$Lw$}
			\UIC{$[\mathcal{S} | \epsilon_l] \cup \mathcal{T} \cup [B | \epsilon_r] \Rightarrow \Delta$}
		\end{prooftree}
		
		\item \label{c:*-rw} ($R*, Rw$)
		\begin{prooftree}
			\noLine
			\AXC{$\mathbf{D_0}$}
			\UIC{$\Gamma \Rightarrow A$}

			\noLine
			\AXC{$\mathbf{D_1}'$}
			\UIC{$\Sigma , (\nabla^l A)^n \Rightarrow$}
			\RightLabel{$Rw$}
			\UIC{$\Sigma , (\nabla^l A)^n \Rightarrow B$}

			\dashedLine\RightLabel{$MC$}
			\BIC{$\nabla^l \Gamma , \Sigma \Rightarrow B$}
		\end{prooftree}
		By induction hypothesis for $\mathbf{D_0}$ and $\mathbf{D_1}'$ we have
		\begin{prooftree}
			\noLine
			\AXC{$\mathbf{D_0}$}
			\UIC{$\Gamma \Rightarrow A$}

			\noLine
			\AXC{$\mathbf{D_1}'$}
			\UIC{$\Sigma , (\nabla^l A)^n \Rightarrow$}
			\RightLabel{IH}
			\BIC{$\nabla^l \Gamma , \Sigma \Rightarrow$}

			\RightLabel{$Rw$}
			\UIC{$\nabla^l \Gamma , \Sigma \Rightarrow B$}
		\end{prooftree}
		
		\item \label{c:*-lc} ($R*, Lc$)
		\begin{prooftree}
			\noLine
			\AXC{$\mathbf{D_0}$}
			\UIC{$\Gamma \Rightarrow A$}

			\noLine
			\AXC{$\mathbf{D_1}'$}
			\UIC{$\Sigma , (\nabla^l A)^n , B , B \Rightarrow \Delta$}
			\RightLabel{$Lc$}
			\UIC{$\Sigma , (\nabla^l A)^n , B \Rightarrow \Delta$}

			\dashedLine\RightLabel{$MC$}
			\BIC{$\nabla^l \Gamma , \Sigma , B \Rightarrow \Delta$}
		\end{prooftree}
		By induction hypothesis for $\mathbf{D_0}$ and $\mathbf{D_1}'$ we have
		\begin{prooftree}
			\noLine
			\AXC{$\mathbf{D_0}$}
			\UIC{$\Gamma \Rightarrow A$}

			\noLine
			\AXC{$\mathbf{D_1}'$}
			\UIC{$\Sigma , (\nabla^l A)^n , B , B \Rightarrow \Delta$}
			\RightLabel{IH}
			\BIC{$\nabla^l \Gamma , \Sigma , B , B \Rightarrow \Delta$}

			\RightLabel{$Lc$}
			\UIC{$\nabla^l \Gamma , \Sigma , B \Rightarrow \Delta$}
		\end{prooftree}

		\item \label{c:*-cut} ($R*, \nabla Cut$) Assume $\mathbf{D_1}$ ends with a $\nabla Cut$ on $A'$, which by assumption must be of a lower rank than $A$.
	\begin{prooftree}
		\noLine
		\AXC{$\mathbf{D_0}$}
		\UIC{$\mathcal{S} \Rightarrow \nabla^k A$}

		\noLine
		\AXC{$\mathbf{D_1}'$}
		\UIC{$\mathcal{T} , \p^{l+k} A^n \Rightarrow \nabla^{k'} A'$}

		\noLine
		\AXC{$\mathbf{D_1}''$}
		\UIC{$\mathcal{T}' , \p^{l'+k'} A'^{n'} \Rightarrow \Delta$}

		\RightLabel{$\nabla Cut$}
		\BIC{$\p^{l'} \mathcal{T} , \p^{l+k+l'} A^n , \mathcal{T}' \Rightarrow \Delta$}

		\dashedLine\RightLabel{$\nabla Cut$}
		\BIC{$\p^{l+l'} \mathcal{S} , \p^{l'} \mathcal{T} , \mathcal{T}' \Rightarrow \Delta$}
	\end{prooftree}
	We can use the induction hypothesis to remove $A$ first, then cut $A'$.
	\begin{prooftree}
		\noLine
		\AXC{$\mathbf{D_0}$}
		\UIC{$\mathcal{S} \Rightarrow \nabla^k A$}
		
		\noLine
		\AXC{$\mathbf{D_1}'$}
		\UIC{$\mathcal{T} , \p^{l+k} A^n \Rightarrow \nabla^{k'} A'$}

		\RightLabel{IH}
		\BIC{$\p^l \mathcal{S} , \mathcal{T} \Rightarrow \nabla^{k'} A'$}
		
		\noLine
		\AXC{$\mathbf{D_1}''$}
		\UIC{$\mathcal{T}' , \p^{l'+k'} A'^{n'} \Rightarrow \Delta$}
		
		\RightLabel{$\nabla Cut$}
		\BIC{$\p^{l+l'} \mathcal{S} , \p^{l'} \mathcal{T} , \mathcal{T}' \Rightarrow \Delta$}
	\end{prooftree}


		\item \label{c:*-la1} ($R*, L\land_1$)
		\begin{prooftree}
			\noLine
			\AXC{$\mathbf{D_0}$}
			\UIC{$\Gamma \Rightarrow A$}

			\noLine
			\AXC{$\mathbf{D_1}'$}
			\UIC{$\Sigma , (\nabla^l A)^n , \nabla^r B \Rightarrow \Delta$}
			\RightLabel{$L\land_1$}
			\UIC{$\Sigma , (\nabla^l A)^n , \nabla^r (B \land C) \Rightarrow \Delta$}

			\dashedLine\RightLabel{$MC$}
			\BIC{$\nabla^l \Gamma , \Sigma , \nabla^r (B \land C) \Rightarrow \Delta$}
		\end{prooftree}
		By induction hypothesis for $\mathbf{D_0}$ and $\mathbf{D_1}'$ we have
		\begin{prooftree}
			\noLine
			\AXC{$\mathbf{D_0}$}
			\UIC{$\Gamma \Rightarrow A$}

			\noLine
			\AXC{$\mathbf{D_1}'$}
			\UIC{$\Sigma , (\nabla^l A)^n , \nabla^r B \Rightarrow \Delta$}
			\RightLabel{IH}
			\BIC{$\nabla^l \Gamma , \Sigma , \nabla^r B \Rightarrow \Delta$}

			\RightLabel{$L\land_1$}
			\UIC{$\nabla^l \Gamma , \Sigma , \nabla^r (B \land C) \Rightarrow \Delta$}
		\end{prooftree}
		
		\item \label{c:*-la2} ($R*, L\land_2$)
\begin{prooftree}
	\noLine
	\AXC{$\mathbf{D_0}$}
	\UIC{$\mathcal{S} \Rightarrow \nabla^k A$}
	
	\noLine
	\AXC{$\mathbf{D_1}'$}
	\UIC{$\mathcal{T} , \p^{l+k} A^n , \p^r C \Rightarrow \Delta$}
	\RightLabel{$L\land_2$}
	\UIC{$\mathcal{T} , \p^{l+k} A^n , \p^r B \land C \Rightarrow \Delta$}
	
	\dashedLine\RightLabel{$\nabla Cut$}
	\BIC{$\p^l \mathcal{S} , \mathcal{T} , \p^r B \land C \Rightarrow \Delta$}
\end{prooftree}
By induction hypothesis for $\mathbf{D_0}$ and $\mathbf{D_1}'$ we have
\begin{prooftree}
	\noLine
	\AXC{$\mathbf{D_0}$}
	\UIC{$\mathcal{S} \Rightarrow \nabla^k A$}
	
	\noLine
	\AXC{$\mathbf{D_1}'$}
	\UIC{$\mathcal{T} , \p^{l+k} A^n , \p^r C \Rightarrow \Delta$}
	\RightLabel{IH}
	\BIC{$\p^l \mathcal{S} , \mathcal{T} , \p^r C \Rightarrow \Delta$}
	
	\RightLabel{$L\land_2$}
	\UIC{$\p^l \mathcal{S} , \mathcal{T} , \p^r B \land C \Rightarrow \Delta$}
\end{prooftree}
		
		\item \label{c:*-ra} ($R*, R\land$)
		\begin{prooftree}
			\noLine
			\AXC{$\mathbf{D_0}$}
			\UIC{$\Gamma \Rightarrow A$}

			\noLine
			\AXC{$\mathbf{D_1}'$}
			\UIC{$\Sigma , (\nabla^l A)^n \Rightarrow B$}
			\noLine
			\AXC{$\mathbf{D_1}''$}
			\UIC{$\Sigma , (\nabla^l A)^n \Rightarrow C$}
			\RightLabel{$R\land$}
			\BIC{$\Sigma , (\nabla^l A)^n \Rightarrow B \land C$}

			\dashedLine\RightLabel{$MC$}
			\BIC{$\nabla^l \Gamma , \Sigma \Rightarrow B \land C$}
		\end{prooftree}
		By induction hypothesis, once for $\mathbf{D_0}$ and $\mathbf{D_1}'$ and again for $\mathbf{D_0}$ and $\mathbf{D_1}''$
		\begin{prooftree}
			\noLine
			\AXC{$\mathbf{D_0}$}
			\UIC{$\Gamma \Rightarrow A$}

			\noLine
			\AXC{$\mathbf{D_1}'$}
			\UIC{$\Sigma , (\nabla^l A)^n \Rightarrow B$}

			\RightLabel{IH}
			\BIC{$\nabla^l \Gamma , \Sigma \Rightarrow B$}

			\noLine
			\AXC{$\mathbf{D_0}$}
			\UIC{$\Gamma \Rightarrow A$}

			\noLine
			\AXC{$\mathbf{D_1}''$}
			\UIC{$\Sigma , (\nabla^l A)^n \Rightarrow C$}

			\RightLabel{IH}
			\BIC{$\nabla^l \Gamma , \Sigma \Rightarrow C$}

			\RightLabel{$R\land$}
			\BIC{$\nabla^l \Gamma , \Sigma \Rightarrow B \land C$}
		\end{prooftree}
		
		\item \label{c:*-lo} ($R*, L\lor$)
		\begin{prooftree}
			\noLine
			\AXC{$\mathbf{D_0}$}
			\UIC{$\mathcal{S} \Rightarrow \nabla^k A$}
			
			\noLine
			\AXC{$\mathbf{D_1}'$}
			\UIC{$\mathcal{T} \cup [A^n | \epsilon_{l+k}] \cup [B | \epsilon_r] \Rightarrow \Delta$}
			\noLine
			\AXC{$\mathbf{D_1}''$}
			\UIC{$\mathcal{T} \cup [A^n | \epsilon_{l+k}] \cup [C | \epsilon_r] \Rightarrow \Delta$}
			\RightLabel{$L\lor$}
			\BIC{$\mathcal{T} \cup [A^n | \epsilon_{l+k}] \cup [B \lor C | \epsilon_r] \Rightarrow \Delta$}
			
			\dashedLine\RightLabel{$\nabla Cut$}
			\BIC{$[\mathcal{S} | \epsilon_l] \cup \mathcal{T} \cup [B \lor C | \epsilon_r] \Rightarrow \Delta$}
		\end{prooftree}
		By induction hypothesis, once for $\mathbf{D_0}$ and $\mathbf{D_1}'$ and again for $\mathbf{D_0}$ and $\mathbf{D_1}''$
		\begin{prooftree}
			\noLine
			\AXC{$\mathbf{D_0}$}
			\UIC{$\mathcal{S} \Rightarrow \nabla^k A$}
			
			\noLine
			\AXC{$\mathbf{D_1}'$}
			\UIC{$\mathcal{T} \cup [A^n | \epsilon_{l+k}] \cup [B | \epsilon_r] \Rightarrow \Delta$}
			
			\RightLabel{IH}
			\BIC{$[\mathcal{S} | \epsilon_l] \cup \mathcal{T} \cup [B | \epsilon_r] \Rightarrow \Delta$}
			
			\noLine
			\AXC{$\mathbf{D_0}$}
			\UIC{$\mathcal{S} \Rightarrow \nabla^k A$}
			
			\noLine
			\AXC{$\mathbf{D_1}''$}
			\UIC{$\mathcal{T} \cup [A^n | \epsilon_{l+k}] \cup [C | \epsilon_r] \Rightarrow \Delta$}
			
			\RightLabel{IH}
			\BIC{$[\mathcal{S} | \epsilon_l] \cup \mathcal{T} \cup [C | \epsilon_r] \Rightarrow \Delta$}
			
			\RightLabel{$L\lor$}
			\BIC{$[\mathcal{S} | \epsilon_l] \cup \mathcal{T} \cup [B \lor C | \epsilon_r] \Rightarrow \Delta$}
		\end{prooftree}
		
		\item \label{c:*-ro1} ($R*, R\lor_1$)
		\begin{prooftree}
			\noLine
			\AXC{$\mathbf{D_0}$}
			\UIC{$\Gamma \Rightarrow A$}

			\noLine
			\AXC{$\mathbf{D_1}'$}
			\UIC{$\Sigma , (\nabla^l A)^n \Rightarrow B$}
			\RightLabel{$R\lor_1$}
			\UIC{$\Sigma , (\nabla^l A)^n \Rightarrow B \lor C$}

			\dashedLine\RightLabel{$MC$}
			\BIC{$\nabla^l \Gamma , \Sigma \Rightarrow B \lor C$}
		\end{prooftree}
		By induction hypothesis for $\mathbf{D_0}$ and $\mathbf{D_1}'$
		\begin{prooftree}
			\noLine
			\AXC{$\mathbf{D_0}$}
			\UIC{$\Gamma \Rightarrow A$}

			\noLine
			\AXC{$\mathbf{D_1}'$}
			\UIC{$\Sigma , (\nabla^l A)^n \Rightarrow B$}
			\RightLabel{IH}
			\BIC{$\nabla^l \Gamma , \Sigma \Rightarrow B$}

			\RightLabel{$R\lor_1$}
			\UIC{$\nabla^l \Gamma , \Sigma \Rightarrow B \lor C$}
		\end{prooftree}
		
		\item \label{c:*-ro2} ($R*, R\lor_2$)
		\begin{prooftree}
			\noLine
			\AXC{$\mathbf{D_0}$}
			\UIC{$\mathcal{S} \Rightarrow \nabla^k A$}
			
			\noLine
			\AXC{$\mathbf{D_1}'$}
			\UIC{$\mathcal{T} \cup [A^n | \epsilon_{l+k}] \Rightarrow C$}
			\RightLabel{$R\lor_1$}
			\UIC{$\mathcal{T} \cup [A^n | \epsilon_{l+k}] \Rightarrow B \lor C$}
			
			\dashedLine\RightLabel{$\nabla Cut$}
			\BIC{$[\mathcal{S} | \epsilon_l] \cup \mathcal{T} \Rightarrow B \lor C$}
		\end{prooftree}
		By induction hypothesis for $\mathbf{D_0}$ and $\mathbf{D_1}'$
		\begin{prooftree}
			\noLine
			\AXC{$\mathbf{D_0}$}
			\UIC{$\mathcal{S} \Rightarrow \nabla^k A$}
			
			\noLine
			\AXC{$\mathbf{D_1}'$}
			\UIC{$\mathcal{T} \cup [A^n | \epsilon_{l+k}] \Rightarrow C$}
			\RightLabel{IH}
			\BIC{$[\mathcal{S} | \epsilon_l] \cup \mathcal{T} \Rightarrow C$}
			
			\RightLabel{$R\lor_1$}
			\UIC{$[\mathcal{S} | \epsilon_l] \cup \mathcal{T} \Rightarrow B \lor C$}
		\end{prooftree}
		
		\item \label{c:*-li} ($R*, L\rightarrow$)
		\begin{prooftree}
			\noLine
			\AXC{$\mathbf{D_0}$}
			\UIC{$\mathcal{S} \Rightarrow \nabla^k A$}
			
			
			\noLine
			\AXC{$\mathbf{D_1}'$}
			\UIC{$\mathcal{T} , \p^{l+k} A^n \Rightarrow \nabla^r B$}
			
			\noLine
			\AXC{$\mathbf{D_1}''$}
			\UIC{$\mathcal{T} , \p^{l+k} A^n , \p^r C \Rightarrow \Delta$}
			
			\RightLabel{$L\rightarrow$}
			\BIC{$\mathcal{T} , \p^{l+k} A^n , \p^{r+1} B \rightarrow C \Rightarrow \Delta$}
			
			
			\dashedLine\RightLabel{$\nabla Cut$}
			\BIC{$\p^l \mathcal{S} , \mathcal{T} , \p^{r+1} B \rightarrow C \Rightarrow \Delta$}
		\end{prooftree}
		By induction hypothesis, once for $\mathbf{D_0}$ and $\mathbf{D_1}'$ and again for $\mathbf{D_0}$ and $\mathbf{D_1}''$
		\begin{prooftree}
			\noLine
			\AXC{$\mathbf{D_0}$}
			\UIC{$\mathcal{S} \Rightarrow \nabla^k A$}
			
			\noLine
			\AXC{$\mathbf{D_1}'$}
			\UIC{$\mathcal{T} , \p^{l+k} A^n \Rightarrow \nabla^r B$}
			
			\RightLabel{IH}
			\BIC{$\p^l \mathcal{S} , \mathcal{T} \Rightarrow \nabla^r B$}
			
			\noLine
			\AXC{$\mathbf{D_0}$}
			\UIC{$\mathcal{S} \Rightarrow \nabla^k A$}
			
			\noLine
			\AXC{$\mathbf{D_1}''$}
			\UIC{$\mathcal{T} , \p^{l+k} A^n, \p^r C \Rightarrow \Delta$}
			
			\RightLabel{IH}
			\BIC{$\p^l \mathcal{S} , \mathcal{T} , \p^r C \Rightarrow \Delta$}
			
			\RightLabel{$L\rightarrow$}
			\BIC{$\p^l \mathcal{S} , \mathcal{T} \p^{r+1} B \rightarrow C \Rightarrow \Delta$}
		\end{prooftree}
		
		\item \label{c:*-ri} ($R*, R\rightarrow$)
		\begin{prooftree}
			\noLine
			\AXC{$\mathbf{D_0}$}
			\UIC{$\Gamma \Rightarrow A$}
			
			\noLine
			\AXC{$\mathbf{D_1}'$}
			\UIC{$\nabla \Sigma , (\nabla^{l+1} A)^n , B \Rightarrow C$}
			\RightLabel{$R\rightarrow$}
			\UIC{$\Sigma , (\nabla^l A)^n \Rightarrow B \rightarrow C$}
			
			\dashedLine\RightLabel{$MC$}
			\BIC{$\nabla^l \Gamma , \Sigma \Rightarrow B \rightarrow C$}
		\end{prooftree}
		By induction hypothesis for $\mathbf{D_0}$ and $\mathbf{D_1}'$
		\begin{prooftree}
			\noLine
			\AXC{$\mathbf{D_0}$}
			\UIC{$\Gamma \Rightarrow A$}
			
			\noLine
			\AXC{$\mathbf{D_1}'$}
			\UIC{$\nabla \Sigma , (\nabla^{l+1} A)^n , B \Rightarrow C$}
			\RightLabel{IH}
			\BIC{$\nabla^{l+1} \Gamma , \nabla \Sigma , B \Rightarrow C$}
			
			\RightLabel{$R\rightarrow$}
			\UIC{$\nabla^l \Gamma , \Sigma \Rightarrow B \rightarrow C$}
		\end{prooftree}
		
		\item \label{c:*-ln} ($R*, L\nabla$)
		\begin{prooftree}
			\noLine
			\AXC{$\mathbf{D_0}$}
			\UIC{$\mathcal{S} \Rightarrow \nabla^k A$}

			\noLine
			\AXC{$\mathbf{D_1}'$}
			\UIC{$\mathcal{T} , \p^{l+k} A^n , \p^{r+1} B \Rightarrow \Delta$}
			\RightLabel{$L\nabla$}
			\UIC{$\mathcal{T} , \p^{l+k} A^n , \p^r \nabla B \Rightarrow \Delta$}

			\dashedLine\RightLabel{$\nabla Cut$}
			\BIC{$\p^l \mathcal{S} , \mathcal{T} , \p^r \nabla B \Rightarrow \Delta$}
		\end{prooftree}
		By induction hypothesis for $\mathbf{D_0}$ and $\mathbf{D_1}'$
		\begin{prooftree}
			\noLine
			\AXC{$\mathbf{D_0}$}
			\UIC{$\mathcal{S} \Rightarrow \nabla^k A$}

			\noLine
			\AXC{$\mathbf{D_1}'$}
			\UIC{$\mathcal{T} , \p^{l+k} A^n , \p^{r+1} B \Rightarrow \Delta$}
			\RightLabel{IH}
			\BIC{$\p^l \mathcal{S} , \mathcal{T} , \p^{r+1} B \Rightarrow \Delta$}

			\RightLabel{$L\nabla$}
			\UIC{$\p^l \mathcal{S} , \mathcal{T} , \p^r \nabla B \Rightarrow \Delta$}
		\end{prooftree}
		
		\item \label{c:*-rn} ($R*, R\nabla$) By the structure of $R \nabla$ we must have $l + k > 0$. Consider different cases for $l$.
		
		($\star$) If $l = 0$, then $k = k' + 1$ for some $k'$. Hence, the only possible right-rule as the last rule of $\mathbf{D_0}$ would be $R \nabla$. ($Rw$ for $\mathbf{D_0}$ is already discussed, other right-rules force a different structure for $A$)
		\begin{prooftree}
			\noLine
			\AXC{$\mathbf{D_0}'$}
			\UIC{$\mathcal{S} \Rightarrow \nabla^{k'} A$}
			\RightLabel{$R \nabla$}
			\UIC{$\p \mathcal{S} \Rightarrow \nabla^{k'+1} A$}
			
			\noLine
			\AXC{$\mathbf{D_1}'$}
			\UIC{$\mathcal{T} , \p^{k'} A^n \Rightarrow \Delta$}
			\RightLabel{$R\nabla$}
			\UIC{$\p \mathcal{T} , \p^{k'+1} A^n \Rightarrow \nabla \Delta$}
			
			\dashedLine\RightLabel{$\nabla Cut$}
			\BIC{$\p \mathcal{S} , \p \mathcal{T} \Rightarrow \nabla \Delta$}
		\end{prooftree}
		By induction hypothesis for $\mathbf{D_0}'$ and $\mathbf{D_1}'$
		\begin{prooftree}
			\noLine
			\AXC{$\mathbf{D_0}'$}
			\UIC{$\mathcal{S} \Rightarrow \nabla^{k'} A$}

			\noLine
			\AXC{$\mathbf{D_1}'$}
			\UIC{$\mathcal{T} , \p^{k'} A^n \Rightarrow \Delta$}

			\RightLabel{IH}
			\BIC{$\mathcal{S} , \mathcal{T} \Rightarrow \Delta$}
			
			\RightLabel{$R \nabla$}
			\UIC{$\p \mathcal{S} , \p \mathcal{T} \Rightarrow \nabla \Delta$}
		\end{prooftree}
		($\star \star$) Otherwise, if $l = l' + 1$ for some $l'$
		\begin{prooftree}
			\noLine
			\AXC{$\mathbf{D_0}$}
			\UIC{$\mathcal{S} \Rightarrow \nabla^k A$}
			
			\noLine
			\AXC{$\mathbf{D_1}'$}
			\UIC{$\mathcal{T} , \p^{l'+k} A^n \Rightarrow \Delta$}
			\RightLabel{$R\nabla$}
			\UIC{$\p \mathcal{T} , \p^{l'+k+1} A^n \Rightarrow \nabla \Delta$}
			
			\dashedLine\RightLabel{$\nabla Cut$}
			\BIC{$\p^{l'+1} \mathcal{S} , \p \mathcal{T} \Rightarrow \nabla \Delta$}
		\end{prooftree}
		then by the induction hypothesis for $\mathbf{D_0}$ and $\mathbf{D_1}'$
		\begin{prooftree}
			\noLine
			\AXC{$\mathbf{D_0}$}
			\UIC{$\mathcal{S} \Rightarrow \nabla^k A$}
			
			\noLine
			\AXC{$\mathbf{D_1}'$}
			\UIC{$\mathcal{T} , \p^{l'+k} A^n \Rightarrow \Delta$}
			
			\RightLabel{IH}
			\BIC{$\p^{l'} \mathcal{S} , \mathcal{T} \Rightarrow \Delta$}
			
			
			\RightLabel{$R\nabla$}
			\UIC{$\p^{l'+1} \mathcal{S} , \p \mathcal{T} \Rightarrow \nabla \Delta$}
		\end{prooftree}
	\end{enumerate}
	
		II. If the last rule in $\mathbf{D_1}$ is a logical left-rule and $A$ is also principal there. In 

	\begin{enumerate}[label={\Alph*.}]
		\item \label{c:ra-la1} ($R\land, L\land_1$)
		\begin{prooftree}
			\noLine
			\AXC{$\mathbf{D_0}'$}
			\UIC{$\Gamma \Rightarrow B$}
			\AXC{$\Gamma \Rightarrow C$}
			\RightLabel{$R\land$}
			\BIC{$\Gamma \Rightarrow B \land C$}
			
			\noLine
			\AXC{$\mathbf{D_1}'$}
			\UIC{$\Sigma , (\nabla^l (B \land C))^{n-1}, \nabla^l B \Rightarrow \Delta$}
			\RightLabel{$L\land_1$}
			\UIC{$\Sigma , (\nabla^l (B \land C))^n \Rightarrow \Delta$}
			
			\RightLabel{$MC$} \dashedLine
			\BIC{$\nabla^l \Gamma , \Sigma \Rightarrow \Delta$}
		\end{prooftree}
		Induction hypothesis for $\mathbf{D_0}$ and $\mathbf{D_1}'$ gives us $\nabla^l \Gamma , \Sigma , \nabla^l B \Rightarrow \Delta$. Then we remove $B$ by a low rank $MC$ on this sequent and $\mathbf{D_0}'$ to get $(\nabla^l \Gamma)^2 , \Sigma \Rightarrow \Delta$, which yields the desired sequent after enough applications of $Lc$.
		\begin{prooftree}
			\noLine
			\AXC{$\mathbf{D_0}'$}
			\UIC{$\Gamma \Rightarrow B$}

			\noLine
			\AXC{$\mathbf{D_0}$}
			\UIC{$\Gamma \Rightarrow B \land C$}
			\noLine
			\AXC{$\mathbf{D_1}'$}
			\UIC{$\Sigma , (\nabla^l (B \land C))^{n-1}, \nabla^l B \Rightarrow \Delta$}
			
			\RightLabel{IH}
			\BIC{$\nabla^l \Gamma , \Sigma , \nabla^l B \Rightarrow \Delta$}

			\RightLabel{$MC$}
			\BIC{$(\nabla^l \Gamma)^2 , \Sigma \Rightarrow \Delta$}
			
			\doubleLine \RightLabel{$Lc$}
			\UIC{$\nabla^l \Gamma , \Sigma \Rightarrow \Delta$}
		\end{prooftree}

		\item \label{c:ra-la2} ($R\land, L\land_1$)
\begin{prooftree}
	\noLine
	\AXC{$\Gamma \Rightarrow B$}
	\AXC{$\mathbf{D_0}'$}
	\UIC{$\Gamma \Rightarrow C$}
	\RightLabel{$R\land$}
	\BIC{$\Gamma \Rightarrow B \land C$}
	
	\noLine
	\AXC{$\mathbf{D_1}'$}
	\UIC{$\Sigma , (\nabla^l (B \land C))^{n-1}, \nabla^l C \Rightarrow \Delta$}
	\RightLabel{$L\land_2$}
	\UIC{$\Sigma , (\nabla^l (B \land C))^n \Rightarrow \Delta$}
	
	\RightLabel{$MC$} \dashedLine
	\BIC{$\nabla^l \Gamma , \Sigma \Rightarrow \Delta$}
\end{prooftree}
Induction hypothesis for $\mathbf{D_0}$ and $\mathbf{D_1}'$ gives us $\nabla^l \Gamma , \Sigma , \nabla^l C \Rightarrow \Delta$. Then we remove $C$ by a low rank $MC$ on this sequent and $\mathbf{D_0}'$ to get $\nabla^l \Gamma^2 , \Sigma \Rightarrow \Delta$, which yields the desired sequent after enough applications of $Lc$.
\begin{prooftree}
	\noLine
	\AXC{$\mathbf{D_0}'$}
	\UIC{$\Gamma \Rightarrow C$}
	
	\noLine
	\AXC{$\mathbf{D_0}$}
	\UIC{$\Gamma \Rightarrow B \land C$}
	\noLine
	\AXC{$\mathbf{D_1}'$}
	\UIC{$\Sigma , (\nabla^l (B \land C))^{n-1}, \nabla^l C \Rightarrow \Delta$}
	
	\RightLabel{IH}
	\BIC{$\nabla^l \Gamma , \Sigma , \nabla^l C \Rightarrow \Delta$}
	
	\RightLabel{$MC$}
	\BIC{$\nabla^l \Gamma^2 , \Sigma \Rightarrow \Delta$}
	
	\doubleLine \RightLabel{$Lc$}
	\UIC{$\nabla^l \Gamma , \Sigma \Rightarrow \Delta$}
\end{prooftree}

		\item \label{c:ro1-lo} ($R\lor_1, L\lor$)
		\begin{prooftree}
			\noLine
			\AXC{$\mathbf{D_0}'$}
			\UIC{$\mathcal{S} \Rightarrow B$}
			\RightLabel{$R\lor_1$}
			\UIC{$\mathcal{S} \Rightarrow B \lor C$}

			\noLine
			\AXC{$\mathbf{D_1}'$}
			\UIC{$\mathcal{T} | \Gamma , (B \lor C)^{n-1} , B | \mathcal{R} \Rightarrow \Delta$}
			\AXC{$\mathcal{T} | \Gamma , (B \lor C)^{n-1} , C | \mathcal{R} \Rightarrow \Delta$}
			\RightLabel{$L\lor$}
			\BIC{$\mathcal{T} | \Gamma , (B \lor C)^n | \mathcal{R} \Rightarrow \Delta$}

			\RightLabel{$\nabla Cut$} \dashedLine
			\BIC{$[ \mathcal{S} | \epsilon_{len(\mathcal{R})} ] \cup [ \mathcal{T} | \Gamma | \mathcal{R} ] \Rightarrow \Delta$}
		\end{prooftree}
		Induction hypothesis for $\mathbf{D_0}$ and $\mathbf{D_1}'$ gives us $[ \mathcal{S} | \epsilon_{len(\mathcal{R})} ] \cup [ \mathcal{T} | \Gamma , B | \mathcal{R} ] \Rightarrow \Delta$. Now removing $B$ with a low rank $\nabla Cut$ on this sequent and $\mathbf{D_0}'$ gives $[ \mathcal{S}^2 | \epsilon_{len(\mathcal{R})} ] \cup [ \mathcal{T} | \Gamma | \mathcal{R} ] \Rightarrow \Delta$, which yields the desired sequent with enough applications of $Lc$.
		\begin{prooftree}
			\noLine
			\AXC{$\mathbf{D_0}'$}
			\UIC{$\mathcal{S} \Rightarrow B$}

			\noLine
			\AXC{$\mathbf{D_0}$}
			\UIC{$\mathcal{S} \Rightarrow B \lor C$}

			\noLine
			\AXC{$\mathbf{D_1}'$}
			\UIC{$\mathcal{T} | \Gamma , (B \lor C)^{n-1} , B | \mathcal{R} \Rightarrow \Delta$}

			\RightLabel{IH}
			\BIC{$[ \mathcal{S} | \epsilon_{len(\mathcal{R})} ] \cup [ \mathcal{T} | \Gamma , B | \mathcal{R} ] \Rightarrow \Delta$}
			
			\RightLabel{$\nabla Cut$}
			\BIC{$[ \mathcal{S}^2 | \epsilon_{len(\mathcal{R})} ] \cup [ \mathcal{T} | \Gamma | \mathcal{R} ] \Rightarrow \Delta$}
			
			\doubleLine \RightLabel{$Lc$}
			\UIC{$[ \mathcal{S} | \epsilon_{len(\mathcal{R})} ] \cup [ \mathcal{T} | \Gamma | \mathcal{R} ] \Rightarrow \Delta$}
		\end{prooftree}

		\item \label{c:ro2-lo} ($R\lor_2, L\lor$)
\begin{prooftree}
	\noLine
	\AXC{$\mathbf{D_0}'$}
	\UIC{$\Gamma \Rightarrow C$}
	\RightLabel{$R\lor_2$}
	\UIC{$\Gamma \Rightarrow B \lor C$}
	
	\noLine
	\AXC{$\Sigma , (\nabla^l (B \lor C))^{n-1} , \nabla^l B \Rightarrow \Delta$}
	\AXC{$\mathbf{D_1}'$}
	\UIC{$\Sigma , (\nabla^l (B \lor C))^{n-1} , \nabla^l C \Rightarrow \Delta$}
	\RightLabel{$L\lor$}
	\BIC{$\Sigma , (\nabla^l (B \lor C))^n \Rightarrow \Delta$}
	
	\RightLabel{$MC$} \dashedLine
	\BIC{$\nabla^l \Gamma , \Sigma \Rightarrow \Delta$}
\end{prooftree}
Induction hypothesis for $\mathbf{D_0}$ and $\mathbf{D_1}'$ gives us $\nabla^l \Gamma , \Sigma , \nabla^l C \Rightarrow \Delta$. Then we remove $C$ by a low rank $MC$ on this sequent and $\mathbf{D_0}'$ to get $(\nabla^l \Gamma)^2 , \Sigma \Rightarrow \Delta$, which yields the desired sequent after enough applications of $Lc$.
\begin{prooftree}
	\noLine
	\AXC{$\mathbf{D_0}'$}
	\UIC{$\Gamma \Rightarrow C$}
	
	\noLine
	\AXC{$\mathbf{D_0}$}
	\UIC{$\Gamma \Rightarrow B \lor C$}
	
	\noLine
	\AXC{$\mathbf{D_1}'$}
	\UIC{$\Sigma , (\nabla^l (B \lor C))^{n-1} , \nabla^l C \Rightarrow \Delta$}
	
	\RightLabel{IH}
	\BIC{$\nabla^l \Gamma , \Sigma , \nabla^l C \Rightarrow \Delta$}
	
	\RightLabel{$MC$}
	\BIC{$(\nabla^l \Gamma)^2 , \Sigma \Rightarrow \Delta$}
	
	\doubleLine \RightLabel{$Lc$}
	\UIC{$\nabla^l \Gamma , \Sigma \Rightarrow \Delta$}
\end{prooftree}

		\item \label{c:rn-*} ($R\nabla, *$) This is the only case where $k$ can be more than $0$. First, in the case that $k = 0$, like previous cases, the last rule of $\mathbf{D_1}$ (and the construction of $A$) is determined by the last rule of $\mathbf{D_0}$; We have $A = \nabla B$ for some $B$.
		\begin{prooftree}
			\noLine
			\AXC{$\mathbf{D_0}$}
			\UIC{$\mathcal{S} | \epsilon_1 \Rightarrow \nabla B$}
			
			\noLine
			\AXC{$\mathbf{D_1}'$}
			\UIC{$\mathcal{T} \cup [B | (\nabla B)^{n-1} | \epsilon_l] \Rightarrow \Delta$}
			\RightLabel{$L \nabla$}
			\UIC{$\mathcal{T} \cup [(\nabla B)^n | \epsilon_l] \Rightarrow \Delta$}
			
			\RightLabel{$\nabla Cut$} \dashedLine
			\BIC{$[ \mathcal{S} | \epsilon_{l+1} ] \cup \mathcal{T} \Rightarrow \Delta$}
		\end{prooftree}
		First apply induction hypothesis for $\mathbf{D_0}$ and $\mathbf{D_1}'$ with $A' = A = \nabla B$ and $k' = 0$. Then remove $B$ by a $\nabla Cut$ on $\mathbf{D_0}$ and the resulting sequent.

		\begin{prooftree}
			\noLine
			\AXC{$\mathbf{D_0}$}
			\UIC{$\mathcal{S} | \epsilon_1 \Rightarrow \nabla B$}
			
			
			\noLine
			\AXC{$\mathbf{D_0}$}
			\UIC{$\mathcal{S} | \epsilon_1 \Rightarrow \nabla B$}
			
			\noLine
			\AXC{$\mathbf{D_1}'$}
			\UIC{$\mathcal{T} \cup [B | (\nabla B)^{n-1} | \epsilon_l] \Rightarrow \Delta$}
			
			\RightLabel{IH}
			\BIC{$[\mathcal{S} | \epsilon_{l+1}] \cup \mathcal{T} \cup [B | \epsilon_{l+1}] \Rightarrow \Delta$}
			
			
			\RightLabel{$\nabla Cut$}
			\BIC{$[\mathcal{S}^2 | \epsilon_{l+1}] \cup \mathcal{T} \Rightarrow \Delta$}
			
			\RightLabel{$Lc$}
			\UIC{$[\mathcal{S} | \epsilon_{l+1}] \cup \mathcal{T} \Rightarrow \Delta$}
		\end{prooftree}

		Now let $k = k' + 1$ for some $k'$. This does not necessarily determine the structure of $A$. Our construction, however, does not depend on then last rule of $\mathbf{D_1}$ or whether $A$ is principal there.
		\begin{prooftree}
			\noLine
			\AXC{$\mathbf{D_0}'$}
			\UIC{$\mathcal{S} \Rightarrow \nabla^{k'} A$}
			\RightLabel{$R \nabla$}
			\UIC{$\mathcal{S} | \epsilon_1 \Rightarrow \nabla^{k'+1} A$}
			
			\noLine
			\AXC{$\mathbf{D_1}$}
			\UIC{$\mathcal{T} \cup [A^n | \epsilon_{l+k'+1}] \Rightarrow \Delta$}
			
			\RightLabel{$\nabla Cut$} \dashedLine
			\BIC{$[ \mathcal{S} | \epsilon_{l+1} ] \cup \mathcal{T} \Rightarrow \Delta$}
		\end{prooftree}
		Induction hypothesis on $\mathbf{D_0}'$ and $\mathbf{D_1}$ does exactly the same.
		\begin{prooftree}
			\noLine
			\AXC{$\mathbf{D_0}'$}
			\UIC{$\mathcal{S} \Rightarrow \nabla^{k'} A$}

			\noLine
			\AXC{$\mathbf{D_1}$}
			\UIC{$\mathcal{T} \cup [A^n | \epsilon_{l+k'+1}] \Rightarrow \Delta$}
			
			\RightLabel{IH}
			\BIC{$[ \mathcal{S} | \epsilon_{l+1} ] \cup \mathcal{T} \Rightarrow \Delta$}
		\end{prooftree}

		\item \label{c:ri-li} ($R\rightarrow, L\rightarrow$)
		\begin{prooftree}
			\noLine
			\AXC{$\mathbf{D_0}'$}
			\UIC{$\p \mathcal{S} , B \Rightarrow C$}
			\RightLabel{$R\rightarrow$}
			\UIC{$\mathcal{S} \Rightarrow B \rightarrow C$}

			\noLine
			\AXC{$\mathbf{D_1}'$}
			\UIC{$\mathcal{T} , \p^{l+1} (B \rightarrow C)^{n-1} \Rightarrow \nabla^l B$}
			\noLine
			\AXC{$\mathbf{D_1}''$}
			\UIC{$\mathcal{T} , \p^{l+1} (B \rightarrow C)^{n-1} , \p^l C \Rightarrow \Delta$}
			\RightLabel{$L\rightarrow$}
			\BIC{$\mathcal{T} , \p^{l+1} (B \rightarrow C)^n \Rightarrow \Delta$}
			
			\RightLabel{$\nabla Cut$} \dashedLine
			\BIC{$\p^{l+1} \mathcal{S} , \mathcal{T} \Rightarrow \Delta$}
		\end{prooftree}
		From induction hypothesis for $\mathbf{D_0}$ and $\mathbf{D_1}'$ we have $\p^{l+1} \mathcal{S} , \mathcal{T} \Rightarrow \nabla^l B$. Call it $\mathbf{D}'$.
		\begin{prooftree}
			\noLine
			\AXC{$\mathbf{D_0}$}
			\UIC{$\mathcal{S} \Rightarrow B \rightarrow C$}
			
			\noLine
			\AXC{$\mathbf{D_1}'$}
			\UIC{$\mathcal{T} , \p^{l+1} (B \rightarrow C)^{n-1} \Rightarrow \nabla^l B$}
			
			\LeftLabel{$\mathbf{D}':~$}
			\RightLabel{IH}
			\BIC{$\p^{l+1} \mathcal{S} , \mathcal{T} \Rightarrow \nabla^l B$}
		\end{prooftree}
		Again from IH, this time for $\mathbf{D_0}$ and $\mathbf{D_1}''$ we have $\p^{l+1} \mathcal{S} , \mathcal{T} , \p^l C \Rightarrow \Delta$. Call it $\mathbf{D}''$.
		\begin{prooftree}
			\noLine
			\AXC{$\mathbf{D_0}$}
			\UIC{$\mathcal{S} \Rightarrow B \rightarrow C$}
			
			\noLine
			\AXC{$\mathbf{D_1}''$}
			\UIC{$\mathcal{T} , \p^{l+1} (B \rightarrow C)^{n-1} , \p^l C \Rightarrow \Delta$}
			
			\LeftLabel{$\mathbf{D}'':~$}
			\RightLabel{IH}
			\BIC{$\p^{l+1} \mathcal{S} , \mathcal{T} , \p^l C \Rightarrow \Delta$}
		\end{prooftree}
		Now we use $\nabla Cut$ (with $k:=0$) on $\mathbf{D_0}'$ and $\mathbf{D}''$ to remove $\p^l C$, and another $\nabla Cut$ (with $k:=l$) with $\mathbf{D}'$ to remove $\p^l B$. Applying enough contractions, we can derive the desired sequent.
		\begin{prooftree}
			\noLine
			\AXC{$\mathbf{D}'$}
			\UIC{$\p^{l+1} \mathcal{S} , \mathcal{T} \Rightarrow \nabla^l B$}


			\noLine
			\AXC{$\mathbf{D_0}'$}
			\UIC{$\p \mathcal{S} , B \Rightarrow C$}

			\noLine
			\AXC{$\mathbf{D}''$}
			\UIC{$\p^{l+1} \mathcal{S} , \mathcal{T} , \p^l C \Rightarrow \Delta$}

			\RightLabel{$\nabla Cut$}
			\BIC{$\p^{l+1} \mathcal{S}^2 , \p^l B , \mathcal{T} \Rightarrow \Delta$}



			\RightLabel{$\nabla Cut$}
			\BIC{$\p^{l+1} \mathcal{S}^3 , \mathcal{T}^2 \Rightarrow \Delta$}



			\doubleLine \RightLabel{$Lc$}
			\UIC{$\p^{l+1} \mathcal{S} , \mathcal{T} \Rightarrow \Delta$}
		\end{prooftree}

	\end{enumerate}

	III. If $A$ is principal in the last rule of $\mathbf{D_1}$, which is not logical.
	\begin{enumerate}[label={\roman*.}]
		\item \label{c:*-id} ($R*, Id$) $l$ has to be $0$, so $\mathbf{D_0}$ proves the desired sequent.

		\item \label{c:*-ex} ($R \nabla, Ex$) This forces $A = \bot$. So $R*$ is $R \nabla$ and $k > 0$.
		\begin{prooftree}
			\noLine
			\AXC{$\mathbf{D_0}'$}
			\UIC{$\mathcal{S} \Rightarrow \nabla^{k'} \bot$}
			\RightLabel{$R\nabla$}
			\UIC{$\p \mathcal{S} \Rightarrow \nabla^{k' + 1} \bot$}
			

			\AXC{$\mathbf{D_1}$}
			\RightLabel{$Ex$}
			\UIC{$\p^{l+k'+1} \bot^n \Rightarrow$}
			
			\dashedLine\RightLabel{$\nabla Cut$}
			\BIC{$\p^{l+1} \mathcal{S} \Rightarrow$}
		\end{prooftree}
		The desired sequent is exactly what we get from induction hypothesis for $\mathbf{D_0}'$ and $\mathbf{D_1}$.
		\begin{prooftree}
			\noLine
			\AXC{$\mathbf{D_0}'$}
			\UIC{$\mathcal{S} \Rightarrow \nabla^{k'} \bot$}
			
			\AXC{$\mathbf{D_1}$}
			\RightLabel{$Ex$}
			\UIC{$\p^{l+k'+1} \bot^n \Rightarrow$}
			
			\RightLabel{IH}
			\BIC{$\p^{l+k'+1} \mathcal{S} \Rightarrow$}
		\end{prooftree}

		\item \label{c:*-lw-p} ($R*, Lw$)
		\begin{prooftree}
			\noLine
			\AXC{$\mathbf{D_0}$}
			\UIC{$\mathcal{S} \Rightarrow \nabla^k A$}
			
			\noLine
			\AXC{$\mathbf{D_1}'$}
			\UIC{$\mathcal{T} \cup [A^{n-1} | \epsilon_{l+k}] \Rightarrow \Delta$}
			\RightLabel{$Lw$}
			\UIC{$\mathcal{T} \cup [A^n | \epsilon_{l+k}] \Rightarrow \Delta$}
			
			\dashedLine\RightLabel{$\nabla Cut$}
			\BIC{$[\mathcal{S} | \epsilon_l] \cup \mathcal{T} \Rightarrow \Delta$}
		\end{prooftree}
		If $n = 1$, then by $Lw$ on $\mathbf{D_1}'$ we have
		\begin{prooftree}
			\noLine
			\AXC{$\mathbf{D_1}'$}
			\UIC{$\mathcal{T} \Rightarrow \Delta$}
			
			\RightLabel{$Lw$}
			\UIC{$[\mathcal{S} | \epsilon_l] \cup \mathcal{T}\Rightarrow \Delta$}
		\end{prooftree}
		If $n > 1$, then by induction hypothesis for $\mathbf{D_0}$ and $\mathbf{D_1}'$ we have
		\begin{prooftree}
			\noLine
			\AXC{$\mathbf{D_0}$}
			\UIC{$\mathcal{S} \Rightarrow \nabla^k A$}
			
			\noLine
			\AXC{$\mathbf{D_1}'$}
			\UIC{$\mathcal{T} \cup [A^{n-1} | \epsilon_{l+k}] \Rightarrow \Delta$}
			
			\RightLabel{IH}
			\BIC{$[\mathcal{S} | \epsilon_l] \cup \mathcal{T}\Rightarrow \Delta$}
		\end{prooftree}
	
		\item \label{c:*-lc-p} ($R*, Lc$)
		\begin{prooftree}
			\noLine
			\AXC{$\mathbf{D_0}$}
			\UIC{$\mathcal{S} \Rightarrow \nabla^k A$}
			
			\noLine
			\AXC{$\mathbf{D_1}'$}
			\UIC{$\mathcal{T} , \p^{l+k} A^{n+1} \Rightarrow \Delta$}
			\RightLabel{$Lc$}
			\UIC{$\mathcal{T} , \p^{l+k} A^n \Rightarrow \Delta$}
			
			\dashedLine\RightLabel{$\nabla Cut$}
			\BIC{$\p^l \mathcal{S} , \mathcal{T} \Rightarrow \Delta$}
		\end{prooftree}
		By induction hypothesis for $\mathbf{D_0}$ and $\mathbf{D_1}'$ we have
		\begin{prooftree}
			\noLine
			\AXC{$\mathbf{D_0}$}
			\UIC{$\mathcal{S} \Rightarrow \nabla^k A$}
			
			\noLine
			\AXC{$\mathbf{D_1}'$}
			\UIC{$\mathcal{T} , \p^{l+k} A^{n+1} \Rightarrow \Delta$}
			
			\RightLabel{IH}
			\BIC{$\p^l \mathcal{S} , \mathcal{T} \Rightarrow \Delta$}
		\end{prooftree}
	\end{enumerate}
\end{enumerate}

\section{Corollary} \emph{$\nabla$Cut Elimination for GSTL: }
If $\small\text{GSTL}^- + \nabla Cut \vdash \Gamma \Rightarrow \Delta$, then $\small\text{GSTL}^- \vdash \Gamma \Rightarrow \Delta$.

\emph{Proof:} It suffices to show that for any proof tree of $\Gamma \Rightarrow \Delta$ like $\mathbf{D}$, there is another proof of it with a lower rank. Using induction on $h(\mathbf{D})$, the induction hypothesis states that for any proof of $\Gamma' \Rightarrow \Delta'$ like $\mathbf{D'}$ such that $h(\mathbf{D'}) < h(\mathbf{D})$, there is another proof of $\Gamma' \Rightarrow \Delta'$ like $\mathbf{D''}$ such that $\rho(\mathbf{D''}) < \rho(\mathbf{D'})$. Particularly for the immediate sub-trees of $\mathbf{D}$, this gives us sub-trees with lower cut-rank, which we call $\mathbf{D}_i$. We now consider two cases for the last rule of $\mathbf{D}$

\begin{enumerate}[label=\Roman*]
	\item If the last rule of $\mathbf{D}$ is of a lower rank than $\rho(\mathbf{D})$, i.e. the $\nabla Cut$ rule with the maximum rank is not the last rule in $\mathbf{D}$, then we can apply the same last rule on $\mathbf{D}_i$s and get a proof of $\Gamma \Rightarrow \Delta$ with a lower rank.
	
	\item If the last rule of $\mathbf{D}$ is an instance of $\nabla Cut$ rule with a cut-formula of rank $\rho(\mathbf{D})$, then we can apply theorem \ref{cut-admis} to $\mathbf{D_0}$ and $\mathbf{D_1}$ to get the same $\nabla Cut$ with a lower rank.
\end{enumerate}

\end{document}
