\section{Interpolation}

\subsection{Lemma}\label{lem:vdash} $\Rightarrow \Delta' \vdash_{iSTL} \Gamma \Rightarrow \Delta$ if and only if $\vdash_{iSTL} \{ \nabla^a \Box^b \Delta'\}_{(a,b) \in I} , \Gamma \Rightarrow \Delta$ for some finite set of pairs of natural numbers $I$.

\textit{Proof}:
If $\Delta' = \{\}$ then this is trivial by \todo{It is not true, actually.} So we can assume $\Delta' = A$ for some formula $A$. \\

I. Suppose we have $\Rightarrow A$ and $\{ \nabla^a \Box^b A\}_{(a,b) \in I} , \Gamma \Rightarrow \Delta$ for some finite set of pairs of natural numbers $I$. First for any $(a,b) \in I$, construct $\Rightarrow \nabla^a \Box^b A$ from $\Rightarrow A$ as follows. Introduce $\top$ in the left side with $Lw$ and apply $R\rightarrow$ to get $\Rightarrow \top \rightarrow A$. Repeat $b$ times to get $\Rightarrow \Box^b A$. Then $a$ applications of $N$ yields $\Rightarrow \nabla^a \Box^b A$. Now $Cut$ all of them into $\{ \nabla^a \Box^b A\}_{(a,b) \in I} , \Gamma \Rightarrow \Delta$ to get $\Gamma \Rightarrow \Delta$. \\

II. Suppose $\Rightarrow A \vdash_{iSTL} \Gamma \Rightarrow \Delta$ by a proof tree $\mathbf{D}$. By induction on the length of $\mathbf{D}$, for any $\Gamma'$, $\Delta'$ and any iSTL$+A$ proof tree $\mathbf{D}'$ which proves $\Gamma' \Rightarrow \Delta'$ in lower length than $\mathbf{D}$, we can assume there is an iSTL proof tree IH($\mathbf{D}'$) that proves $\{\nabla^a \Box^b A\}_{(a,b) \in I_{IH(\mathbf{D}')}} , \Gamma' \Rightarrow \Delta'$ for some set of pairs of natural numbers $I_{IH(\mathbf{D}')}$.

For different possible rules in iSTL$+A$ as the last rule of $\mathbf{D}$, we will construct an iSTL proof of $\{ \nabla^a \Box^b A\}_{(a,b) \in I} , \Gamma \Rightarrow \Delta$ for some finite set of pairs natural numbers $I$. In all cases, the possible immediate sub-trees of $\mathbf{D}$ are called $\mathbf{D_0}$ and $\mathbf{D_1}$.
\begin{enumerate}
	\setcounter{enumi}{-1}
	\item ($\Rightarrow A$) This implies $\Gamma = \{\}$ and $\Delta = A$. Then we have $A \Rightarrow A$ by $Id$.

	\item ($Id$) This implies $\Gamma = \Delta = A$. Then $A \Rightarrow A$ by $Id$.

	\item ($Ta$) This implies $\Gamma = \{\}$ and $\Delta = \top$. We have $\Rightarrow \top$ by $Ta$.

	\item ($Ex$) This implies $\Gamma = \bot$ and $\Delta = \{\}$. We have $\bot \Rightarrow$ by $Ex$.

	\item[4-10.] ($Lw$),($Lc$),($Rw$),($L\land_1$),($L\land_2$),($R\lor_1$),($R\lor_2$) Just apply the same rule on IH($\mathbf{D_0}$). Notice that $I$ would be the same set we get from the induction hypothesis.
	\setcounter{enumi}{10}

	\item ($N$) Same as the previous cases, except that $I = \{ (a + 1 , b) : (a , b) \in I_{IH(\mathbf{D_0})} \}$, where $I_{IH(\mathbf{D_0})}$ is the set from the induction hypothesis.

	\item ($Cut$) Just apply $Cut$ on IH($\mathbf{D_0}$) and IH($\mathbf{D_1}$). Notice that $I = I_{IH(\mathbf{D_0})} \cup I_{IH(\mathbf{D_1})}$.

	\item[13,14,15.] ($R\land$), ($L\lor$), ($L\rightarrow$) Same as the $Cut$ case, except that before applying the corresponding rule to IH($\mathbf{D_0}$) and IH($\mathbf{D_1}$), we must first equalize their contexts using enough applications of $Lw$, to $\{ \nabla^a \Box^b A \}_{(a,b) \in I_{IH(\mathbf{D_0})} \cup I_{IH(\mathbf{D_1})}}$.
	\setcounter{enumi}{15}

	\item ($R\rightarrow$) If $a = 0$ for some $(a,b) \in I_{IH(\mathbf{D_0})}$, then $R\rightarrow$ is not applicable on IH($\mathbf{D_0}$). We have $\nabla \Box^{b+1} A \Rightarrow \Box^b A$ from lemma \ref{lem:i-nabla-box}. For all such $(0,b) \in I_{IH(\mathbf{D_0})}$ we first $Cut$ this into IH($\mathbf{D_0}$), so that the context has at least one $\nabla$ before applying $R\rightarrow$. Notice that $I = \{ (a, b) : (a+1, b) \in I \} \cup$ $\{ (0, b+1) : (0, b) \in I \}$.
\end{enumerate}

\subsection{Theorem} \textit{Deductive Interpolation for iSTL: } For any $\Delta'$ and $\Delta$, if $\Rightarrow \Delta' \vdash_{iSTL} \Rightarrow \Delta$ then there exists a formula $C$ such that
\begin{enumerate}
	\item $\Rightarrow \Delta' \vdash_{iSTL} \Rightarrow C$,
	\item $\Rightarrow C \vdash_{iSTL} \Rightarrow \Delta$ and
	\item $P(C) \subseteq P(\Delta') \cap P(\Delta)$.
\end{enumerate}

\textit{Proof}: In case $\Delta' = \{\}$, just take $C = \bot$. Now let's assume $\Delta' = A$ for some formula $A$. It is also necessary to strengthen the statement of the theorem so that $\Rightarrow \Delta$ has an antecedent like $\Gamma$, i.e. $\Rightarrow A \vdash_{iSTL} \Gamma \Rightarrow \Delta$. By lemma \ref{lem:vdash}, theorem \ref{translation} and corollary \ref{cut-elim}, there is a GSTL$^-$ proof tree $\mathbf{D}$ for $\{ \nabla^a \Box^b A \}_{(a,b) \in I} , \Gamma \Rightarrow \Delta$ for some set of pairs of natural numbers $I$.

So we will prove for any $\Gamma$, $\Delta$ and $A$ that if $\vdash_{GSTL^-} \{ \nabla^a \Box^b A \}_{(a,b) \in I} , \Gamma \Rightarrow \Delta$ for some $I$, then there is $C$ such that
\begin{enumerate}
	\item $\Rightarrow A \vdash_{GSTL^-} \Rightarrow C$, 
	\item $\Rightarrow C \vdash_{GSTL^-} \Gamma \Rightarrow \Delta$ and
	\item $P(C) \subseteq P(A) \cap P(\Gamma,\Delta)$.
\end{enumerate}
By induction on the length of $\mathbf{D}$, we can assume for any $\Gamma'$, $\Delta'$, $I'$ and $\mathbf{D}'$, where $\mathbf{D}'$ is a GSTL$^-$ proof for $\{ \nabla^a \Box^b A \}_{(a,b) \in I'} , \Gamma' \Rightarrow \Delta'$ and $h(\mathbf{D}')<h(\mathbf{D})$, there is a formula IH($\mathbf{D}'$) such that
\begin{enumerate}
	\item $\Rightarrow A \vdash_{GSTL^-} IH(\mathbf{D}')$,
	\item $\Rightarrow IH(\mathbf{D}') \vdash_{GSTL^-} \Gamma' \Rightarrow \Delta'$ and
	\item $P(IH(\mathbf{D}')) \subseteq P(A) \cap P(\Gamma',\Delta')$.
\end{enumerate}
We now construct proper $C$ for different cases for the last rule of $\mathbf{D}$. In each case, the immediate sub-trees of $\mathbf{D}$ are denoted by $\mathbf{D_i} (i \in \{0,1\})$. First notice that if none of $\nabla^a \Box^b A$'s are principal in the last rule, we can just use the same interpolant from the induction hypothesis, IH($\mathbf{D_i}$). \todo{More justifications.}

Now suppose one $\nabla^{a_0} \Box^{b_0} A$ is principal in the last rule, which can be
\begin{enumerate}
	\item ($L\land_1$) This implies $b_0 = 0$ and $A = \nabla^n (A_1 \land A_2)$ for some formulas $A_1$ and $A_2$ and some natural number $n$. Let $I' = I - \{(a_0,b_0)\}$
	\begin{prooftree}
		\AXC{$\mathbf{D_0}$} \noLine
		\UIC{$\{ \nabla^a \Box^b \nabla^n (A_1 \land A_2) \}_{(a,b) \in I'} , \nabla^{a_0 + n} A_1 , \Gamma \Rightarrow \Delta$}
		\RightLabel{$L\land_1$}
		\UIC{$\{ \nabla^a \Box^b \nabla^n (A_1 \land A_2) \}_{(a,b) \in I'} , \nabla^{a_0 + n} (A_1 \land A_2) , \Gamma \Rightarrow \Delta$}
	\end{prooftree}
\end{enumerate}
By induction hypothesis
\begin{enumerate}
	\item $\Rightarrow \nabla^n (A_1 \land A_2) \vdash_{GSTL^-} IH(\mathbf{D_0})$,
	\item $\Rightarrow IH(\mathbf{D_0}) \vdash_{GSTL^-} \nabla^{a_0 + n} A_1 , \Gamma \Rightarrow \Delta$ and
	\item $P(IH(\mathbf{D_0})) \subseteq P(\nabla^n (A_1 \land A_2)) \cap P(\nabla^{a_0 + n} A_1,\Gamma,\Delta)$.
\end{enumerate}
\todo{no way}

\subsection{Theorem} \textit{Craig's Interpolation for GSTL$^-$: } For any $\Gamma_1$, $\Gamma_2$ and $\Delta$, if $\vdash_{GSTL^-} \Gamma_1 , \Gamma_2 \Rightarrow \Delta$, then there is a formula $C$ such that $P(C) \subseteq P(\Gamma_1) \cap P(\Gamma_2 , \Delta)$, $\vdash_{GSTL^-} \Gamma_1 \Rightarrow C$ and $\vdash_{GSTL^-} \Gamma_2 , C \Rightarrow \Delta$.

\textit{Proof}: Let $\mathbf{D}$ be the GSTL$^-$ proof of $\Gamma_1 , \Gamma_2 \Rightarrow \Delta$. We will use induction on $h(\mathbf{D})$. So for any $\Gamma_1'$, $\Gamma_2'$ and $\Delta'$ such that $\vdash_{GSTL^-} \Gamma_1' , \Gamma_2' \Rightarrow \Delta'$ with a proof smaller than $\mathbf{D}$, the induction hypothesis (IH) gives an interpolant $C_{\langle\Gamma_1'; \Gamma_2'; \Delta'\rangle}$ for which the statement of the theorem is true. We now build the desired interpolant $C$, in different cases for the last rule of $\mathbf{D}$. In cases for left-rules, we also need to consider whether the principal formula is in $\Gamma_1$ or $\Gamma_2$ in separate cases.
\begin{enumerate}
	\item ($Id$) We have $\Gamma_1,\Gamma_2 = \Delta = A$.
	\begin{enumerate}
		\item If $\Gamma_1 = \{\}$ and $\Gamma_2 = A$, then define $C = \top$. So we have $\Rightarrow \top$ by $Ta$ and $A , \top \Rightarrow A$ by $Id$ and $Lw$.
		
		\item If $\Gamma_1 = A$ and $\Gamma_2 = \{\}$ then define $C = A$. So we have $A \Rightarrow A$ by $Id$.
	\end{enumerate}
	\item ($Ta$) Take $C = \top$.
	
	\item ($Ex$) Take $C = \nabla^n \bot$.
	
	\item ($Lw$) $\mathbf{D}$ proves $\Gamma_1' , \Gamma_2' , A \Rightarrow \Delta$ and has a sub-proof for $\Gamma_1' , \Gamma_2' \Rightarrow \Delta$, for which IH gives an interpolant $C_{\langle\Gamma_1';\Gamma_2';\Delta\rangle}$ and proofs for $\Gamma_1' \Rightarrow C_{\langle\Gamma_1';\Gamma_2';\Delta\rangle}$ and $\Gamma_2 , C_{\langle\Gamma_1';\Gamma_2';\Delta\rangle} \Rightarrow \Delta$, such that $P(C_{\langle\Gamma_1';\Gamma_2';\Delta\rangle}) \subseteq$ $ P(\Gamma_1') \cap P(\Gamma_2' , \Delta)$.
	\begin{enumerate}
		\item If $\Gamma_1 = \Gamma_1'$ and $\Gamma_2 = \Gamma_2' , A$, take $C = C_{\langle\Gamma_1';\Gamma_2';\Delta\rangle}$. Then we have  $\Gamma_1' \Rightarrow C$ by IH and $\Gamma_2 , A , C \Rightarrow \Delta$ by $Lw$ and IH. From IH, we also have $P(C) \subseteq P(\Gamma_1') \cap P(\Gamma_2' , A , \Delta)$, since $P$ takes ``$,$'' to ``$\cup$'', which distributes over ``$\cap$'' and is increasing with respect to ``$\subseteq$''.
		
		\item If $\Gamma_1 = \Gamma_1' , A$ and $\Gamma_2 = \Gamma_2'$, again take $C = C_{\langle\Gamma_1';\Gamma_2';\Delta\rangle}$. Then we have  $\Gamma_1' , A \Rightarrow C$ by $Lw$ and IH, and $\Gamma_2 , C \Rightarrow \Delta$ by IH. We also have $P(C) \subseteq P(\Gamma_1' , A) \cap P(\Gamma_2' , \Delta)$ by IH and argument similar to the previous case.
	\end{enumerate}

	\item ($Lc$) $\mathbf{D}$ proves $\Gamma_1' , \Gamma_2' , A \Rightarrow \Delta$ and has a sub-proof for $\Gamma_1' , \Gamma_2' , A , A \Rightarrow \Delta$.
	\begin{enumerate}
		\item If $\Gamma_1 = \Gamma_1'$ and $\Gamma_2 = \Gamma_2' , A$, take $C = C_{\langle\Gamma_1';\Gamma_2',A,A;\Delta\rangle}$. Then we have $\Gamma_1' \Rightarrow C$ by IH and $\Gamma_2' , A \Rightarrow \Delta$ by IH and $Lc$. From IH, we also have $P(C) \subseteq P(\Gamma_1') \cap P(\Gamma_2',A,\Delta)$, since $P(\Gamma,X) = P(\Gamma,X,X)$.
		
		\item If $\Gamma_1 = \Gamma_1' , A$ and $\Gamma_2 = \Gamma_2'$, take $C = C_{\langle\Gamma_1',A,A;\Gamma_2';\Delta\rangle}$. Then we have $\Gamma_1' , A \Rightarrow C$ by IH and $Lc$, and $\Gamma_2' \Rightarrow \Delta$ by IH. We also have $P(C) \subseteq P(\Gamma_1',A) \cap P(\Gamma_2',\Delta)$ as justified before.
	\end{enumerate}

	\item[6,7.] ($L\land_i$, {\small$i \in \{1,2\}$}) $\mathbf{D}$ proves $\Gamma_1' , \Gamma_2' , \nabla^n (A_1 \land A_2) \Rightarrow \Delta$ and has a sub-proof for $\Gamma_1' , \Gamma_2' , \nabla^n A_i \Rightarrow \Delta$.
	\begin{enumerate}
		\item If $\Gamma_1 = \Gamma_1'$ and $\Gamma_2 = \Gamma_2' , \nabla^n (A_1 \land A_2)$, take $C = C_{\langle\Gamma_1';\Gamma_2',\nabla^n A_i;\Delta\rangle}$. Then we have $\Gamma_1' \Rightarrow C$ by IH and $\Gamma_2' , \nabla^n (A_1 \land A_2) \Rightarrow \Delta$ by IH and $L\land_i$. From IH, we also have $P(C) \subseteq$ $P(\Gamma_1') \cap P(\Gamma_2',\nabla^n(A_1 \land A_2),\Delta)$, since $P(\nabla^n X) = P(X)$ and $P$ takes sub-formula ordering to ``$\subseteq$''.
		
		\item If $\Gamma_1 = \Gamma_1' , \nabla^n (A_1 \land A_2)$ and $\Gamma_2 = \Gamma_2'$, take $C = C_{\langle\Gamma_1',\nabla^n A_i;\Gamma_2';\Delta\rangle}$. Then we have $\Gamma_1' , \nabla^n (A_1 \land A_2)$ $\Rightarrow C$ by IH and $L\land_i$. Also from IH we have $\Gamma_2' \Rightarrow \Delta$. We also have $P(C) \subseteq P(\Gamma_1',\nabla^n (A_1 \land A_2))$ $\cap P(\Gamma_2',\Delta)$ as justified in the previous case.
	\end{enumerate}
	\setcounter{enumi}{7}

	\item ($R\land$) $\mathbf{D}$ proves $\Gamma_1 , \Gamma_2 \Rightarrow A \land B$ and has sub-proofs for $\Gamma_1 , \Gamma_2 \Rightarrow A$ and $\Gamma_1 , \Gamma_2 \Rightarrow B$.\\
	Let $C_1 = C_{\langle\Gamma_1;\Gamma_2;A\rangle}$ and $C_2 = C_{\langle\Gamma_1;\Gamma_2;B\rangle}$, and then take $C = C_1 \land C_2$.
	We have $\Gamma_1 \Rightarrow C_1$ and $\Gamma_1 \Rightarrow C_2$, both from IH. Then by $R\land$ we have $\Gamma_1 \Rightarrow C_1 \land C_2$.
	We also have $\Gamma_2 , C_1 \Rightarrow A$ and $\Gamma_2 , C_2 \Rightarrow B$, again from IH.
	We can then derive $\Gamma_2 , C_1 \land C_2 \Rightarrow A$ and $\Gamma_2 , C_1 \land C_2 \Rightarrow B$, respectively by $L\land_1$ and $L\land_2$, and finally  $\Gamma_2 , C_1 \land C_2 \Rightarrow A \land B$ by $R\land$.
	We also have $P(C_1) \subseteq P(\Gamma_1) \cap P(\Gamma_2 , A)$ and $P(C_2) \subseteq P(\Gamma_1) \cap P(\Gamma_2 , B)$. So $P(C_1 , C_2) \subseteq P(\Gamma_1) \cap P(\Gamma_2 , A , B)$ as it was justified before, and then $P(C_1 \land C_2) \subseteq P(\Gamma_1) \cap P(\Gamma_2 , A \land B)$.
	
	\item ($L\lor$) $\mathbf{D}$ proves $\Gamma_1' , \Gamma_2' , \nabla^n (A \lor B) \Rightarrow \Delta$ and has sub-proofs for $\Gamma_1' , \Gamma_2' , \nabla^n A \Rightarrow \Delta$ and $\Gamma_1' , \Gamma_2' , \nabla^n B \Rightarrow \Delta$.
	\begin{enumerate}
		\item If $\Gamma_1 = \Gamma_1'$ and $\Gamma_2 = \Gamma_2' , \nabla^n (A \lor B)$, let $C_1 = C_{\langle\Gamma_1';\Gamma_2',\nabla^n A;\Delta\rangle}$ and $C_2 = C_{\langle\Gamma_1';\Gamma_2',\nabla^n B;\Delta\rangle}$, and then take $C = C_1 \land C_2$.
		We have $\Gamma_1' \Rightarrow C_1 \land C_2$ from IH and $R\land$.
		From IH, by $L\land_1$ and $L\land_2$ we can derive $\Gamma_2' , \nabla^n A , C_1 \land C_2 \Rightarrow \Delta$ and $\Gamma_2' , \nabla^n B , C_1 \land C_2 \Rightarrow \Delta$ respectively, to which we apply $L\lor$ to get to $\Gamma_2' , \nabla^n (A \lor B) , C_1 \land C_2 \Rightarrow \Delta$.
		From IH, we also have $P(C_1) \subseteq P(\Gamma_1') \cap P(\Gamma_2' , \nabla^n A , \Delta)$ and $P(C_2) \subseteq P(\Gamma_1') \cap P(\Gamma_2' , \nabla^n B , \Delta)$. Just like the previous case, we can deduce that $P(C_1 \land C_2) \subseteq P(\Gamma_1') \cap P(\Gamma_2' , \nabla^n (A \land B) , \Delta)$.

		\item If $\Gamma_1 = \Gamma_1' , \nabla^n (A \lor B)$ and $\Gamma_2 = \Gamma_2'$, let $C_1 = C_{\langle\Gamma_1',\nabla^n A;\Gamma_2';\Delta\rangle}$ and $C_2 = C_{\langle\Gamma_1',\nabla^n B;\Gamma_2';\Delta\rangle}$, and then take $C = C_1 \lor C_2$.
		From IH, by $R\lor_1$ and $R\lor_2$ we can derive $\Gamma_1' , \nabla^n A \Rightarrow C_1 \lor C_2$ and $\Gamma_1' , \nabla^n B \Rightarrow C_1 \lor C_2$ respectively, to which we apply $L\lor$ to get to $\Gamma_1' , \nabla^n (A \lor B) \Rightarrow C_1 \lor C_2$.
		We have $\Gamma_2' , C_1 \lor C_2 \Rightarrow \Delta$ from IH and $L\lor$.
		From IH, we also have $P(C_1) \subseteq P(\Gamma_1' , \nabla^n A) \cap$ $P(\Gamma_2' , \Delta)$ and $P(C_2) \subseteq P(\Gamma_1' , \nabla^n B) \cap P(\Gamma_2' , \Delta)$. Just like the previous case, we can deduce that $P(C_1 \lor C_2) \subseteq P(\Gamma_1' , \nabla^n (A \land B)) \cap P(\Gamma_2' , \Delta)$.
	\end{enumerate}

	\item[10,11.] ($R\lor_i$, {\small$i \in \{1,2\}$}) $\mathbf{D}$ proves $\Gamma_1 , \Gamma_2 \Rightarrow A_1 \lor A_2$ and has a sub-proof for $\Gamma_1 , \Gamma_2 \Rightarrow A_i$. Take $C = C_{\langle\Gamma_1;\Gamma_2;A_i\rangle}$. Then we have $\Gamma_1 \Rightarrow C$ from IH and $\Gamma_2 , C \Rightarrow A_1 \lor A_2$ from IH and $R\lor_i$.
	From IH, we also have $P(C) \subseteq P(\Gamma_1) \cap P(\Gamma_2 , A_1 \lor A_2)$, as was justified before.
	\setcounter{enumi}{11}
	
	\item ($L\rightarrow$) $\mathbf{D}$ proves $\Gamma_1' , \Gamma_2' , \nabla^{n+1} (A \rightarrow B) \Rightarrow \Delta$ and has sub-proofs for $\Gamma_1' , \Gamma_2' \Rightarrow \nabla^n A$ and $\Gamma_1' , \Gamma_2' , \nabla^n B \Rightarrow \Delta$.
	\begin{enumerate}
		\item If $\Gamma_1 = \Gamma_1'$ and $\Gamma_2 = \Gamma_2' , \nabla^{n+1} (A \rightarrow B)$, let $C_1 = C_{\langle\Gamma_1';\Gamma_2';\nabla^n A\rangle}$ and $C_2 = C_{\langle\Gamma_1';\Gamma_2',\nabla^n B;\Delta\rangle}$, and take $C = C_1 \land C_2$.
		We have $\Gamma_1' \Rightarrow C_1 \land C_2$ from IH and $R\land$.
		From IH, by $L\land_1$ and $L\land_2$ we can derive $\Gamma_2' , C_1 \land C_2 \Rightarrow \nabla^n A$ and $\Gamma_2' , \nabla^n B , C_1 \land C_2 \Rightarrow \Delta$ respectively, to which we apply $L\rightarrow$ to get to $\Gamma_2' , \nabla^{n+1} (A \rightarrow B) , C_1 \land C_2 \Rightarrow \Delta$.
		From IH, we also have $P(C_1) \subseteq P(\Gamma_1') \cap$ $P(\Gamma_2' , \nabla^n A)$ and $P(C_2) \subseteq P(\Gamma_1') \cap P(\Gamma_2' , \nabla^n B , \Delta)$. This implies $P(C_1 \land C_2) \subseteq P(\Gamma_1') \cap P(\Gamma_2' , \nabla^{n+1} (A \rightarrow B) , \Delta)$.

		\item If $\Gamma_1 = \Gamma_1' , \nabla^{n+1} (A \rightarrow B)$ and $\Gamma_2 = \Gamma_2'$, let $C_1 = C_{\langle\Gamma_2';\Gamma_1';\nabla^n A\rangle}$ and $C_2 = C_{\langle\Gamma_1',\nabla^n B;\Gamma_2';\Delta\rangle}$, and take $C = \nabla (C_1 \rightarrow C_2)$.
		From IH we have $\Gamma_1' , C_1 \Rightarrow \nabla^n A$. Also from IH, with a $Lw$ to add $C_1$ to the left, we have $\Gamma_1' , \nabla^n B , C_1 \Rightarrow C_2$. By applying $L\rightarrow$ we get $\Gamma_1 , \nabla^{n+1} (A \rightarrow B) , C_1 \Rightarrow C_2$.
		\todo{}
		{\color{red} If $\nabla C_1 \rightarrow \nabla C_2 \Rightarrow \nabla (C_1 \rightarrow C_2)$}
		\begin{prooftree}
			\AXC{$\Gamma_1 , \nabla^{n+1} (A \rightarrow B) , C_1 \Rightarrow C_2$}
			\RightLabel{$N'$}
			\UIC{$\nabla \Gamma_1 , \nabla^{n+2} (A \rightarrow B) , \nabla C_1 \Rightarrow \nabla C_2$}
			\RightLabel{$R\rightarrow$}
			\UIC{$\Gamma_1 , \nabla^{n+1} (A \rightarrow B) \Rightarrow \nabla C_1 \rightarrow \nabla C_2$}
			
			\AXC{}
			\RightLabel{{\color{red}Only if this was true}}
			\UIC{$\nabla C_1 \rightarrow \nabla C_2 \Rightarrow \nabla (C_1 \rightarrow C_2)$}

			\RightLabel{$Cut$\tiny of the other logic}
			\BIC{$\Gamma_1 , \nabla^{n+1} (A \rightarrow B) \Rightarrow \nabla (C_1 \rightarrow C_2)$}
		\end{prooftree}
		{\color{red} Or if there was an intuitionistic implication $\supset$, we could take $C = C_1 \supset C_2$}
		\begin{prooftree}
			\AXC{$\Gamma_1 , \nabla^{n+1} (A \rightarrow B) , C_1 \Rightarrow C_2$}
			\RightLabel{$\supset$}
			\UIC{$\Gamma_1 , \nabla^{n+1} (A \rightarrow B) \Rightarrow C_1 \supset C_2$}
		\end{prooftree}


		We have from IH $\Gamma_2' \Rightarrow C_1$ and $\Gamma_2' , C_2 \Rightarrow \Delta$, from which we can derive $\Gamma_2' , \nabla (C_1 \rightarrow C_2)$ by an application of $L\rightarrow$. We also have from IH $P(C_1) \subseteq P(\Gamma_2') \cap P(\Gamma_1' , \nabla^n A)$ and $P(C_2) \subseteq P(\Gamma_1' , \nabla^n B) \cap P(\Gamma_2' , \Delta)$. Then $P(\nabla (C_1 \rightarrow C_2)) \subseteq P(\Gamma_1' , \nabla^{n+1} (A \rightarrow B)) \cap P(\Gamma_2' , \Delta)$.
	\end{enumerate}

	\item ($R\rightarrow$) $\mathbf{D}$ proves $\Gamma_1 , \Gamma_2 \Rightarrow A \rightarrow B$ and has a sub-proof for $\nabla \Gamma_1 , \nabla \Gamma_2 , A \Rightarrow B$. Let $C = C_{\langle\nabla\Gamma_1;\nabla\Gamma_2,A;B\rangle}$ and we have
	\begin{multicols}{2}
		\begin{prooftree}
			\AXC{IH} \noLine
			\UIC{$\nabla \Gamma_1 \Rightarrow C$}
			\RightLabel{$Lw$}
			\UIC{$\nabla \Gamma_1 , \top \Rightarrow C$}
			\RightLabel{$R\rightarrow$}
			\UIC{$\Gamma_1 \Rightarrow \top \rightarrow C$}
		\end{prooftree}
		\columnbreak
		\begin{prooftree}
			\AXC{Lemma \ref{lem:i-nabla-box}} \noLine
			\UIC{$\nabla (\top \rightarrow C) \Rightarrow C$}

			\AXC{IH} \noLine
			\UIC{$\nabla \Gamma_2 , A , C \Rightarrow B$}

			\RightLabel{$Cut$}
			\BIC{$\nabla \Gamma_2 , A , \nabla (\top \rightarrow C) \Rightarrow B$}

			\RightLabel{$R\rightarrow$}
			\UIC{$\Gamma_2 , \top \rightarrow C \Rightarrow A \rightarrow B$}
		\end{prooftree}
	\end{multicols}
	We also have $P(\top \rightarrow C) \subseteq P(\Gamma_1) \cap P(\Gamma_2 , A \rightarrow B)$ from IH and the fact that $P$ preserves sub-formula ordering in $\subseteq$ and $\top$ does not introduce new atomic formulas.

	\item ($N'$) $\mathbf{D}$ proves $\nabla \Gamma_1 , \nabla \Gamma_2 \Rightarrow \nabla \Delta$ and has a sub-proof for $\Gamma_1 , \Gamma_2 \Rightarrow \Delta$. Just take $C = C(\Gamma_1;\Gamma_2;\Delta)$ and apply $N'$ on the sequents from IH. The variable condition is also trivial from IH.
\end{enumerate}