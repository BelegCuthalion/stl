\section{Interpolation}

\subsection{Theorem} \textit{Craig's Interpolation for GSTL$^-$: } For any $\Gamma_1$, $\Gamma_2$ and $\Delta$, if GSTL$^-\vdash \Gamma_1 , \Gamma_2 \Rightarrow \Delta$, then there is a formula $C$ such that $P(C) \subseteq P(\Gamma_1) \cap P(\Gamma_2 , \Delta)$, GSTL$^-\vdash \Gamma_1 \Rightarrow C$ and GSTL$^-\vdash \Gamma_2 , C \Rightarrow \Delta$.

\textit{Proof}: Let $\mathbf{D}$ be the GSTL$^-$ proof of $\Gamma_1 , \Gamma_2 \Rightarrow \Delta$. We will use induction on $h(\mathbf{D})$. So for any $\Gamma_1'$, $\Gamma_2'$ and $\Delta'$ such that GSTL$^-\vdash \Gamma_1' , \Gamma_2' \Rightarrow \Delta'$ with a proof smaller than $\mathbf{D}$, the induction hypothesis (IH) gives an interpolant $C_{\langle\Gamma_1'; \Gamma_2'; \Delta'\rangle}$ for which the statement of the theorem is true. We now build the desired interpolant $C$, in different cases for the last rule of $\mathbf{D}$. In cases for left-rules, we also need to consider whether the principal formula is in $\Gamma_1$ or $\Gamma_2$ in separate cases.
\begin{enumerate}
	\item ($Id$) We have $\Gamma_1,\Gamma_2 = \Delta = A$.
	\begin{enumerate}
		\item If $\Gamma_1 = \{\}$ and $\Gamma_2 = A$, then define $C = \top$. So we have $\Rightarrow \top$ by $Ta$ and $A , \top \Rightarrow A$ by $Id$ and $Lw$.
		
		\item If $\Gamma_1 = A$ and $\Gamma_2 = \{\}$ then define $C = A$. So we have $A \Rightarrow A$ by $Id$.
	\end{enumerate}
	\item ($Ta$) Take $C = \top$.
	
	\item ($Ex$) Take $C = \nabla^n \bot$.
	
	\item ($Lw$) $\mathbf{D}$ proves $\Gamma_1' , \Gamma_2' , A \Rightarrow \Delta$ and has a sub-proof for $\Gamma_1' , \Gamma_2' \Rightarrow \Delta$, for which IH gives an interpolant $C_{\langle\Gamma_1';\Gamma_2';\Delta\rangle}$ and proofs for $\Gamma_1' \Rightarrow C_{\langle\Gamma_1';\Gamma_2';\Delta\rangle}$ and $\Gamma_2 , C_{\langle\Gamma_1';\Gamma_2';\Delta\rangle} \Rightarrow \Delta$, such that $P(C_{\langle\Gamma_1';\Gamma_2';\Delta\rangle}) \subseteq$ $ P(\Gamma_1') \cap P(\Gamma_2' , \Delta)$.
	\begin{enumerate}
		\item If $\Gamma_1 = \Gamma_1'$ and $\Gamma_2 = \Gamma_2' , A$, take $C = C_{\langle\Gamma_1';\Gamma_2';\Delta\rangle}$. Then we have  $\Gamma_1' \Rightarrow C$ by IH and $\Gamma_2 , A , C \Rightarrow \Delta$ by $Lw$ and IH. From IH, we also have $P(C) \subseteq P(\Gamma_1') \cap P(\Gamma_2' , A , \Delta)$, since $P$ takes ``$,$'' to ``$\cup$'', which distributes over ``$\cap$'' and is increasing with respect to ``$\subseteq$''.
		
		\item If $\Gamma_1 = \Gamma_1' , A$ and $\Gamma_2 = \Gamma_2'$, again take $C = C_{\langle\Gamma_1';\Gamma_2';\Delta\rangle}$. Then we have  $\Gamma_1' , A \Rightarrow C$ by $Lw$ and IH, and $\Gamma_2 , C \Rightarrow \Delta$ by IH. We also have $P(C) \subseteq P(\Gamma_1' , A) \cap P(\Gamma_2' , \Delta)$ by IH and argument similar to the previous case.
	\end{enumerate}

	\item ($Lc$) $\mathbf{D}$ proves $\Gamma_1' , \Gamma_2' , A \Rightarrow \Delta$ and has a sub-proof for $\Gamma_1' , \Gamma_2' , A , A \Rightarrow \Delta$.
	\begin{enumerate}
		\item If $\Gamma_1 = \Gamma_1'$ and $\Gamma_2 = \Gamma_2' , A$, take $C = C_{\langle\Gamma_1';\Gamma_2',A,A;\Delta\rangle}$. Then we have $\Gamma_1' \Rightarrow C$ by IH and $\Gamma_2' , A \Rightarrow \Delta$ by IH and $Lc$. From IH, we also have $P(C) \subseteq P(\Gamma_1') \cap P(\Gamma_2',A,\Delta)$, since $P(\Gamma,X) = P(\Gamma,X,X)$.
		
		\item If $\Gamma_1 = \Gamma_1' , A$ and $\Gamma_2 = \Gamma_2'$, take $C = C_{\langle\Gamma_1',A,A;\Gamma_2';\Delta\rangle}$. Then we have $\Gamma_1' , A \Rightarrow C$ by IH and $Lc$, and $\Gamma_2' \Rightarrow \Delta$ by IH. We also have $P(C) \subseteq P(\Gamma_1',A) \cap P(\Gamma_2',\Delta)$ as justified before.
	\end{enumerate}

	\item[6,7.] ($L\land_i$, {\small$i \in \{1,2\}$}) $\mathbf{D}$ proves $\Gamma_1' , \Gamma_2' , \nabla^n (A_1 \land A_2) \Rightarrow \Delta$ and has a sub-proof for $\Gamma_1' , \Gamma_2' , \nabla^n A_i \Rightarrow \Delta$.
	\begin{enumerate}
		\item If $\Gamma_1 = \Gamma_1'$ and $\Gamma_2 = \Gamma_2' , \nabla^n (A_1 \land A_2)$, take $C = C_{\langle\Gamma_1';\Gamma_2',\nabla^n A_i;\Delta\rangle}$. Then we have $\Gamma_1' \Rightarrow C$ by IH and $\Gamma_2' , \nabla^n (A_1 \land A_2) \Rightarrow \Delta$ by IH and $L\land_i$. From IH, we also have $P(C) \subseteq$ $P(\Gamma_1') \cap P(\Gamma_2',\nabla^n(A_1 \land A_2),\Delta)$, since $P(\nabla^n X) = P(X)$ and $P$ takes sub-formula ordering to ``$\subseteq$''.
		
		\item If $\Gamma_1 = \Gamma_1' , \nabla^n (A_1 \land A_2)$ and $\Gamma_2 = \Gamma_2'$, take $C = C_{\langle\Gamma_1',\nabla^n A_i;\Gamma_2';\Delta\rangle}$. Then we have $\Gamma_1' , \nabla^n (A_1 \land A_2)$ $\Rightarrow C$ by IH and $L\land_i$. Also from IH we have $\Gamma_2' \Rightarrow \Delta$. We also have $P(C) \subseteq P(\Gamma_1',\nabla^n (A_1 \land A_2))$ $\cap P(\Gamma_2',\Delta)$ as justified in the previous case.
	\end{enumerate}
	\setcounter{enumi}{7}

	\item ($R\land$) $\mathbf{D}$ proves $\Gamma_1 , \Gamma_2 \Rightarrow A \land B$ and has sub-proofs for $\Gamma_1 , \Gamma_2 \Rightarrow A$ and $\Gamma_1 , \Gamma_2 \Rightarrow B$.\\
	Let $C_1 = C_{\langle\Gamma_1;\Gamma_2;A\rangle}$ and $C_2 = C_{\langle\Gamma_1;\Gamma_2;B\rangle}$, and then take $C = C_1 \land C_2$.
	We have $\Gamma_1 \Rightarrow C_1$ and $\Gamma_1 \Rightarrow C_2$, both from IH. Then by $R\land$ we have $\Gamma_1 \Rightarrow C_1 \land C_2$.
	We also have $\Gamma_2 , C_1 \Rightarrow A$ and $\Gamma_2 , C_2 \Rightarrow B$, again from IH.
	We can then derive $\Gamma_2 , C_1 \land C_2 \Rightarrow A$ and $\Gamma_2 , C_1 \land C_2 \Rightarrow B$, respectively by $L\land_1$ and $L\land_2$, and finally  $\Gamma_2 , C_1 \land C_2 \Rightarrow A \land B$ by $R\land$.
	We also have $P(C_1) \subseteq P(\Gamma_1) \cap P(\Gamma_2 , A)$ and $P(C_2) \subseteq P(\Gamma_1) \cap P(\Gamma_2 , B)$. So $P(C_1 , C_2) \subseteq P(\Gamma_1) \cap P(\Gamma_2 , A , B)$ as it was justified before, and then $P(C_1 \land C_2) \subseteq P(\Gamma_1) \cap P(\Gamma_2 , A \land B)$.
	
	\item ($L\lor$) $\mathbf{D}$ proves $\Gamma_1' , \Gamma_2' , \nabla^n (A \lor B) \Rightarrow \Delta$ and has sub-proofs for $\Gamma_1' , \Gamma_2' , \nabla^n A \Rightarrow \Delta$ and $\Gamma_1' , \Gamma_2' , \nabla^n B \Rightarrow \Delta$.
	\begin{enumerate}
		\item If $\Gamma_1 = \Gamma_1'$ and $\Gamma_2 = \Gamma_2' , \nabla^n (A \lor B)$, let $C_1 = C_{\langle\Gamma_1';\Gamma_2',\nabla^n A;\Delta\rangle}$ and $C_2 = C_{\langle\Gamma_1';\Gamma_2',\nabla^n B;\Delta\rangle}$, and then take $C = C_1 \land C_2$.
		We have $\Gamma_1' \Rightarrow C_1 \land C_2$ from IH and $R\land$.
		From IH, by $L\land_1$ and $L\land_2$ we can derive $\Gamma_2' , \nabla^n A , C_1 \land C_2 \Rightarrow \Delta$ and $\Gamma_2' , \nabla^n B , C_1 \land C_2 \Rightarrow \Delta$ respectively, to which we apply $L\lor$ to get to $\Gamma_2' , \nabla^n (A \lor B) , C_1 \land C_2 \Rightarrow \Delta$.
		From IH, we also have $P(C_1) \subseteq P(\Gamma_1') \cap P(\Gamma_2' , \nabla^n A , \Delta)$ and $P(C_2) \subseteq P(\Gamma_1') \cap P(\Gamma_2' , \nabla^n B , \Delta)$. Just like the previous case, we can deduce that $P(C_1 \land C_2) \subseteq P(\Gamma_1') \cap P(\Gamma_2' , \nabla^n (A \land B) , \Delta)$.

		\item If $\Gamma_1 = \Gamma_1' , \nabla^n (A \lor B)$ and $\Gamma_2 = \Gamma_2'$, let $C_1 = C_{\langle\Gamma_1',\nabla^n A;\Gamma_2';\Delta\rangle}$ and $C_2 = C_{\langle\Gamma_1',\nabla^n B;\Gamma_2';\Delta\rangle}$, and then take $C = C_1 \lor C_2$.
		From IH, by $R\lor_1$ and $R\lor_2$ we can derive $\Gamma_1' , \nabla^n A \Rightarrow C_1 \lor C_2$ and $\Gamma_1' , \nabla^n B \Rightarrow C_1 \lor C_2$ respectively, to which we apply $L\lor$ to get to $\Gamma_1' , \nabla^n (A \lor B) \Rightarrow C_1 \lor C_2$.
		We have $\Gamma_2' , C_1 \lor C_2 \Rightarrow \Delta$ from IH and $L\lor$.
		From IH, we also have $P(C_1) \subseteq P(\Gamma_1' , \nabla^n A) \cap$ $P(\Gamma_2' , \Delta)$ and $P(C_2) \subseteq P(\Gamma_1' , \nabla^n B) \cap P(\Gamma_2' , \Delta)$. Just like the previous case, we can deduce that $P(C_1 \lor C_2) \subseteq P(\Gamma_1' , \nabla^n (A \land B)) \cap P(\Gamma_2' , \Delta)$.
	\end{enumerate}

	\item[10,11.] ($R\lor_i$, {\small$i \in \{1,2\}$}) $\mathbf{D}$ proves $\Gamma_1 , \Gamma_2 \Rightarrow A_1 \lor A_2$ and has a sub-proof for $\Gamma_1 , \Gamma_2 \Rightarrow A_i$. Take $C = C_{\langle\Gamma_1;\Gamma_2;A_i\rangle}$. Then we have $\Gamma_1 \Rightarrow C$ from IH and $\Gamma_2 , C \Rightarrow A_1 \lor A_2$ from IH and $R\lor_i$.
	From IH, we also have $P(C) \subseteq P(\Gamma_1) \cap P(\Gamma_2 , A_1 \lor A_2)$, as was justified before.
	\setcounter{enumi}{11}
	
	\item ($L\rightarrow$) $\mathbf{D}$ proves $\Gamma_1' , \Gamma_2' , \nabla^{n+1} (A \rightarrow B) \Rightarrow \Delta$ and has sub-proofs for $\Gamma_1' , \Gamma_2' \Rightarrow \nabla^n A$ and $\Gamma_1' , \Gamma_2' , \nabla^n B \Rightarrow \Delta$.
	\begin{enumerate}
		\item If $\Gamma_1 = \Gamma_1'$ and $\Gamma_2 = \Gamma_2' , \nabla^{n+1} (A \rightarrow B)$, let $C_1 = C_{\langle\Gamma_1';\Gamma_2';\nabla^n A\rangle}$ and $C_2 = C_{\langle\Gamma_1';\Gamma_2',\nabla^n B;\Delta\rangle}$, and take $C = C_1 \land C_2$.
		We have $\Gamma_1' \Rightarrow C_1 \land C_2$ from IH and $R\land$.
		From IH, by $L\land_1$ and $L\land_2$ we can derive $\Gamma_2' , C_1 \land C_2 \Rightarrow \nabla^n A$ and $\Gamma_2' , \nabla^n B , C_1 \land C_2 \Rightarrow \Delta$ respectively, to which we apply $L\rightarrow$ to get to $\Gamma_2' , \nabla^{n+1} (A \rightarrow B) , C_1 \land C_2 \Rightarrow \Delta$.
		From IH, we also have $P(C_1) \subseteq P(\Gamma_1') \cap$ $P(\Gamma_2' , \nabla^n A)$ and $P(C_2) \subseteq P(\Gamma_1') \cap P(\Gamma_2' , \nabla^n B , \Delta)$. This implies $P(C_1 \land C_2) \subseteq P(\Gamma_1') \cap P(\Gamma_2' , \nabla^{n+1} (A \rightarrow B) , \Delta)$.

		\item If $\Gamma_1 = \Gamma_1' , \nabla^{n+1} (A \rightarrow B)$ and $\Gamma_2 = \Gamma_2'$, let $C_1 = C_{\langle\Gamma_2';\Gamma_1';\nabla^n A\rangle}$ and $C_2 = C_{\langle\Gamma_1',\nabla^n B;\Gamma_2';\Delta\rangle}$, and take $C = \nabla (C_1 \rightarrow C_2)$.
		From IH we have $\Gamma_1' , C_1 \Rightarrow \nabla^n A$. Also from IH, with a $Lw$ to add $C_1$ to the left, we have $\Gamma_1' , \nabla^n B , C_1 \Rightarrow C_2$. By applying $L\rightarrow$ we get $\Gamma_1 , \nabla^{n+1} (A \rightarrow B) , C_1 \Rightarrow C_2$.
		\todo{}
		{\color{red} If $\nabla C_1 \rightarrow \nabla C_2 \Rightarrow \nabla (C_1 \rightarrow C_2)$}
		\begin{prooftree}
			\AXC{$\Gamma_1 , \nabla^{n+1} (A \rightarrow B) , C_1 \Rightarrow C_2$}
			\RightLabel{$N'$}
			\UIC{$\nabla \Gamma_1 , \nabla^{n+2} (A \rightarrow B) , \nabla C_1 \Rightarrow \nabla C_2$}
			\RightLabel{$R\rightarrow$}
			\UIC{$\Gamma_1 , \nabla^{n+1} (A \rightarrow B) \Rightarrow \nabla C_1 \rightarrow \nabla C_2$}
			
			\AXC{}
			\RightLabel{{\color{red}Only if this was true}}
			\UIC{$\nabla C_1 \rightarrow \nabla C_2 \Rightarrow \nabla (C_1 \rightarrow C_2)$}

			\RightLabel{$Cut$\tiny of the other logic}
			\BIC{$\Gamma_1 , \nabla^{n+1} (A \rightarrow B) \Rightarrow \nabla (C_1 \rightarrow C_2)$}
		\end{prooftree}
		{\color{red} Or if there was an intuitionistic implication $\supset$, we could take $C = C_1 \supset C_2$}
		\begin{prooftree}
			\AXC{$\Gamma_1 , \nabla^{n+1} (A \rightarrow B) , C_1 \Rightarrow C_2$}
			\RightLabel{$\supset$}
			\UIC{$\Gamma_1 , \nabla^{n+1} (A \rightarrow B) \Rightarrow C_1 \supset C_2$}
		\end{prooftree}


		We have from IH $\Gamma_2' \Rightarrow C_1$ and $\Gamma_2' , C_2 \Rightarrow \Delta$, from which we can derive $\Gamma_2' , \nabla (C_1 \rightarrow C_2)$ by an application of $L\rightarrow$. We also have from IH $P(C_1) \subseteq P(\Gamma_2') \cap P(\Gamma_1' , \nabla^n A)$ and $P(C_2) \subseteq P(\Gamma_1' , \nabla^n B) \cap P(\Gamma_2' , \Delta)$. Then $P(\nabla (C_1 \rightarrow C_2)) \subseteq P(\Gamma_1' , \nabla^{n+1} (A \rightarrow B)) \cap P(\Gamma_2' , \Delta)$.
	\end{enumerate}

	\item ($R\rightarrow$) $\mathbf{D}$ proves $\Gamma_1 , \Gamma_2 \Rightarrow A \rightarrow B$ and has a sub-proof for $\nabla \Gamma_1 , \nabla \Gamma_2 , A \Rightarrow B$. Let $C = C_{\langle\Gamma_1';\Gamma_2',A;B\rangle}$.
	\todo{}
	{\color{red} Intuitionistic implication does not help.\\  If we strengthen the theorem so that interpolants are $\nabla$-irrelevant, i.e. $C \Leftrightarrow \nabla C$}
	\begin{prooftree}
		\AXC{} \RightLabel{IH}
		\UIC{$\nabla C \Rightarrow C$}

		\AXC{} \RightLabel{IH}
		\UIC{$\nabla \Gamma_2 , A , C \Rightarrow B$}

		\RightLabel{$Cut$}
		\BIC{$\nabla \Gamma_2 , A , \nabla C \Rightarrow B$}
		\RightLabel{$R\rightarrow$}
		\UIC{$\Gamma_2 , C , \Rightarrow A \rightarrow B$}
	\end{prooftree}
	IH also gives $\nabla \Gamma_1 \Rightarrow C$ and $C \Rightarrow \nabla C$. We cut them to $\nabla \Gamma_1 \Rightarrow \nabla C$. Since $N'$ is invertible {\color{red} (it is so)}, we have $\Gamma_1 \Rightarrow C$. {\color{red} Constructed interpolant in other cases are also $\nabla$-irrelevant, except $L\rightarrow$, that needs $\nabla C_1 \rightarrow \nabla C_2 \Rightarrow \nabla (C_1 \rightarrow C_2)$ again.}
	
	\item ($N'$) $\mathbf{D}$ proves $\nabla \Gamma_1 , \nabla \Gamma_2 \Rightarrow \nabla \Delta$ and has a sub-proof for $\Gamma_1 , \Gamma_2 \Rightarrow \Delta$. Just take $C = C(\Gamma_1;\Gamma_2;\Delta)$ and apply $N'$ on the sequents from IH. The variable condition is also trivial from IH.
\end{enumerate}