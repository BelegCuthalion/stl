\section{GSTL} A GSTL sequent $\Gamma \Rightarrow \Delta$ is a binary relation between $\Gamma$, a multi-set of formulas, and $\Delta$, a sub-singleton of some formula, defined inductively by the following rules.

\begin{multicols}{3}
	\begin{prooftree}
		\RightLabel{$Id$}
		\AXC{}
		\UIC{$A \Rightarrow A$}
	\end{prooftree}
	\columnbreak
	\begin{prooftree}
		\RightLabel{$Ta$}
		\AXC{}
		\UIC{$\Rightarrow \top$}
	\end{prooftree}
	\columnbreak
	\begin{prooftree}
		\RightLabel{$Ex$}
		\AXC{}
		\UIC{$\nabla^n \bot \Rightarrow$}
	\end{prooftree}
\end{multicols}
‌\\
\begin{multicols}{3}
	\begin{prooftree}
		\RightLabel{$Lw$}
		\AXC{$\Gamma \Rightarrow \Delta$}
		\UIC{$\Gamma, A \Rightarrow \Delta$}
	\end{prooftree}
	\columnbreak
	\begin{prooftree}
		\RightLabel{$Rw$}
		\AXC{$\Gamma \Rightarrow$}
		\UIC{$\Gamma \Rightarrow A$}
	\end{prooftree}
	\columnbreak
	\begin{prooftree}
		\RightLabel{$Lc$}
		\AXC{$\Gamma , A , A \Rightarrow \Delta$}
		\UIC{$\Gamma , A \Rightarrow \Delta$}
	\end{prooftree}
\end{multicols}
‌\\
\begin{multicols}{3}
	\begin{prooftree}
		\RightLabel{$L\land_1$}
		\AXC{$\Gamma , \nabla^n A \Rightarrow \Delta$}
		\UIC{$\Gamma , \nabla^n (A \land B) \Rightarrow \Delta$}
	\end{prooftree}
	\columnbreak
	\begin{prooftree}
		\RightLabel{$L\land_2$}
		\AXC{$\Gamma , \nabla^n B \Rightarrow \Delta$}
		\UIC{$\Gamma , \nabla^n (A \land B) \Rightarrow \Delta$}
	\end{prooftree}
	\columnbreak
	\begin{prooftree}
		\RightLabel{$R\land$}
		\AXC{$\Gamma \Rightarrow A$}
		\AXC{$\Gamma \Rightarrow B$}
		\BIC{$\Gamma \Rightarrow A \land B$}
	\end{prooftree}
\end{multicols}
‌\\
\begin{multicols}{3}
	\begin{prooftree}
		\RightLabel{$L\lor$}
		\AXC{$\Gamma , \nabla^n A \Rightarrow \Delta$}
		\AXC{$\Gamma , \nabla^n B \Rightarrow \Delta$}
		\BIC{$\Gamma , \nabla^n (A \lor B) \Rightarrow \Delta$}
	\end{prooftree}
	\columnbreak
	\begin{prooftree}
		\RightLabel{$R\lor_1$}
		\AXC{$\Gamma \Rightarrow A$}
		\UIC{$\Gamma \Rightarrow A \lor B$}
	\end{prooftree}
	\columnbreak
	\begin{prooftree}
		\RightLabel{$R\lor_2$}
		\AXC{$\Gamma \Rightarrow B$}
		\UIC{$\Gamma \Rightarrow A \lor B$}
	\end{prooftree}
\end{multicols}
‌\\
\begin{multicols}{2}
	\begin{prooftree}
		\RightLabel{$L\rightarrow$}
		\AXC{$\Gamma \Rightarrow \nabla^n A$}
		\AXC{$\Gamma , \nabla^n B \Rightarrow \Delta$}
		\BIC{$\Gamma , \nabla^{n+1} (A \rightarrow B) \Rightarrow \Delta$}
	\end{prooftree}
	\columnbreak
	\begin{prooftree}
		\RightLabel{$R\rightarrow$}
		\AXC{$\nabla \Gamma , A \Rightarrow B$}
		\UIC{$\Gamma \Rightarrow A \rightarrow B$}
	\end{prooftree}
\end{multicols}
‌\\
\begin{prooftree}
	\RightLabel{$N'$}
	\AXC{$\Gamma \Rightarrow \Delta$}
	\UIC{$\nabla \Gamma \Rightarrow \nabla \Delta$}
\end{prooftree}

\subsection{$MC$} For any $l \ge 0$ and $n \ge 1$
\begin{prooftree}
	\AXC{$\Gamma \Rightarrow A$}
	\AXC{$\Sigma , (\nabla^l A)^n \Rightarrow \Delta$}
	\RightLabel{$MC$}
	\BIC{$\nabla^l \Gamma , \Sigma \Rightarrow \Delta$}
\end{prooftree}
$A$ is called the \textit{cut-formula}.
By $\text{GSTL}$ we mean $\text{GSTL}^- + MC$

\subsubsection{Rank} Rank of a formula $A$ is defined as
\[ \rho(A) = \begin{cases}
1 & \quad ; A \in P \cup \{ \bot, \top \} \\
\rho(B) & \quad ; A = \nabla B \\
max(\rho(B), \rho(C)) + 1 & \quad ; A = B \Box C, \Box \in \{ \land , \lor, \rightarrow \}
\end{cases} \]
We also define rank for rule instances and proof trees. For an instance of the $MC$ rule $c$ with cut-formula $A$, $\rho(c) = \rho(A)$, $0$ if it's not an instance of the $MC$ rule.
For a proof tree $\mathbf{D}$, $\rho(\mathbf{D})$ is the maximum rank of its rule instances.

\subsection{Theorem} \label{translation} For any $\Gamma$ and $\Delta$, $\text{GSTL} \vdash \Gamma \Rightarrow \Delta$ if and only if $\text{iSTL} \vdash \Gamma \Rightarrow \Delta$.

\textit{Proof}: I. Suppose $\text{GSTL}$ proves $\Gamma \Rightarrow \Delta$ by a proof tree $\mathbf{D}$. We will construct a proof tree in $\text{iSTL}$ for $\Gamma \Rightarrow \Delta$. By induction on $h(\mathbf{D})$, the induction hypothesis states that for any $\text{GSTL}$ proof tree $\mathbf{D}'$ for $\Gamma' \Rightarrow \Delta'$ such that $h(\mathbf{D}') < h(\mathbf{D})$, there exists an $\text{iSTL}$ proof tree IH$(\mathbf{D'})$ for $\Gamma' \Rightarrow \Delta'$. Now we consider different cases for the last rule of $\mathbf{D}$. In all cases, we denote the immediate sub-trees of $\mathbf{D}$ by $\mathbf{D_i} ~(0 \leq i)$. The rule name in parens is the last rule of $\mathbf{D}$ and in GSTL, but our construction in each case is in iSTL.
\begin{enumerate}
	\item[1,2.] ($Id$),($Ta$) iSTL has these as axioms.
	\setcounter{enumi}{2}

	\item ($Ex$) Using $Cut$ on lemma \ref{lem:i-nabla-n-bot} and (iSTL's) $Ex$, we get $\nabla^n \bot \Rightarrow$.

	\item[4-6] ($Lw$),($Rw$),($Lc$) Just apply the same rule in iSTL on IH$(\mathbf{D_0})$.
	\setcounter{enumi}{6}

	\item ($L\land_1$) IH($\mathbf{D_0}$) is of the form $\Gamma , \nabla^n A \Rightarrow \Delta$. By $L\land_1$ we have $\Gamma , \nabla^n A \land \nabla^n B \Rightarrow \Delta$. By $Cut$ with lemma \ref{lem:i-nabla-dist-and} we have $\Gamma , \nabla^n (A \land B) \Rightarrow \Delta$.
	
	\item ($L\land_2$) The same.
	
	\item ($R\land$) Just apply iSTL's $R\land$ on IH$(\mathbf{D_0})$ and IH$(\mathbf{D_1})$.
	
	\item ($L\lor$) IH($\mathbf{D_0}$) and IH($\mathbf{D_1}$) are of the form $\Gamma , \nabla^n A \Rightarrow \Delta$ and $\Gamma , \nabla^n B \Rightarrow \Delta$ respectively. By $L\lor$ we have $\Gamma , \nabla^n A \lor \nabla^n B \Rightarrow \Delta$. By $Cut$ with lemma \ref{lem:i-nabla-n-dist-or} we have $\Gamma , \nabla^n (A \lor B) \Rightarrow \Delta$.
	
	\item[11,12.] ($R\lor_{1/2}$) Just apply iSTL's $R\lor_{1/2}$ on IH$(\mathbf{D_0})$.
	\setcounter{enumi}{12}
	
	\item ($L\rightarrow$) IH($\mathbf{D_0}$) and IH($\mathbf{D_1}$) are of the form $\Gamma \Rightarrow \nabla^n A$ and $\Gamma , \nabla^n B \Rightarrow \Delta$ respectively. By $L\rightarrow$ we have $\Gamma , \nabla (\nabla^n A \rightarrow \nabla^n B) \Rightarrow \Delta$. We also have $\Gamma , \nabla^{n+1} (A \rightarrow B) \Rightarrow \nabla (\nabla^n A \rightarrow \nabla^n B)$ from $Lw$ and $N$ applied to lemma \ref{lem:i-nabla-dist-imp}.
	
	\item[14,15.] ($R\rightarrow$),($N'$) Use the same rule in iSTL on IH($\mathbf{D_0}$).
	\setcounter{enumi}{15}
	
	\item ($MC$) Apply $N$ on IH($\mathbf{D_0}$) $l$ times, before cutting it with IH($\mathbf{D_1}$).
\end{enumerate}
II. To prove the other direction it suffices to observe that iSTL rules are just specific cases of GSTL ones.