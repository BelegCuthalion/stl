\section{Interpolation}
\setcounter{subsection}{-1}
\subsection{Notation}
\begin{enumerate}
	\item Let $a$ be a finite sequence of natural numbers of length $k$ and $A$ be a formula. We will write $(\nabla\Box)^a A$ for the formula constructed from applying $a_k$ times $\Box$, $a_{k-1}$ times $\nabla$, ..., $a_2$ times $\Box$ and $a_1$ times $\nabla$ to $A$. More precisely
	\[ (\nabla\Box)^a A = \nabla^{a_1} \Box^{a_2} \dots \nabla^{a_{k-1}} \Box^{a_k} A \]

	\item Let $\Gamma$ be a finite multi-set of formulas and $f$ a function that assigns to each member of $\Gamma$ a finite set of finite sequences of natural numbers. By $N(\Gamma, f)$ we denote the finite multi-set of formulas consisting of members of $\Gamma$ like $A$, with some $\nabla$'s and $\Box$'s as defined by memebers of $f(A)$, i.e.
	\[ N(\Gamma, f) = \{ \nabla^{a_1} \Box^{a_2} \dots \nabla^{a_{k-1}} \Box^{a_k} A : A \in \Gamma ~\&~ a \in f(A) ~\&~ |a| = k \} \]
	\[ = \{ (\nabla\Box)^a A : A \in \Gamma ~\&~ a \in f(A) \} \]
	So for any $B \in N(\Gamma, f)$, there is $A$ in $\Gamma$ such that $B = \nabla^{a_1} \Box^{a_2} \dots \nabla^{a_{k-1}} \Box^{a_k} A$, for some finite sequence of natural numbers $a_1.\dots,a_k$. \\
	By $N(A,f)$ we mean $N(\{A\},f)$. We might also write $N(\Gamma)$ when $f$ is not important.
\end{enumerate}

\subsection{Lemma}\label{lem:box-rule} $A \Rightarrow B \vdash_{iSTL} \Box A \Rightarrow \Box B$.

\textit{Proof:}
\begin{prooftree}
	\AXC{}
	\RightLabel{$Ta$}
	\UIC{$\Rightarrow \top$}
	\AXC{$A \Rightarrow B$}
	\RightLabel{$L\rightarrow$}
	\BIC{$\nabla (\top \rightarrow A) \Rightarrow B$}
	\RightLabel{$Lw$}
	\UIC{$\nabla (\top \rightarrow A) , \top \Rightarrow B$}
	\RightLabel{$R\rightarrow$}
	\UIC{$\top \rightarrow A \Rightarrow \top \rightarrow B$}
\end{prooftree}

\subsection{Lemma}\label{lem:nabla-box-rule} For any finite sequence of natural numbers $a$, $A \Rightarrow B \vdash_{iSTL} (\nabla\Box)^a A \Rightarrow (\nabla\Box)^a B$.

\textit{Proof:}
\begin{prooftree}
	\AXC{} \noLine
	\UIC{$A \Rightarrow B$}
	\RightLabel{Lemma \ref{lem:box-rule}$^{a_k}$} \doubleLine
	\UIC{$\Box^{a_k} A \Rightarrow \Box^{a_k} B$}
	\RightLabel{$N^{a_{k-1}}$} \doubleLine
	\UIC{$\nabla^{a_{k-1}} \Box^{a_k} A \Rightarrow \nabla^{a_{k-1}} \Box^{a_k} B$} \noLine
	\UIC{\vdots} \noLine
	\UIC{$\nabla^{a_3} \dots \nabla^{a_{k-1}} \Box^{a_k} A \Rightarrow \nabla^{a_3} \dots \nabla^{a_{k-1}} \Box^{a_k} B$}
	\RightLabel{Lemma \ref{lem:box-rule}$^{a_2}$} \doubleLine
	\UIC{$\Box^{a_2} \nabla^{a_3} \dots \nabla^{a_{k-1}} \Box^{a_k} A \Rightarrow \Box^{a_2} \nabla^{a_3} \dots \nabla^{a_{k-1}} \Box^{a_k} B$}
	\RightLabel{$N^{a_1}$} \doubleLine
	\UIC{$\nabla^{a_1} \Box^{a_2} \dots \nabla^{a_{k-1}} \Box^{a_k} A \Rightarrow \nabla^{a_1} \Box^{a_2} \dots \nabla^{a_{k-1}} \Box^{a_k} B$}
\end{prooftree}

\subsection{Lemma}\label{lem:vdash} $\Rightarrow A \vdash_{iSTL} \Gamma \Rightarrow \Delta$ if and only if $\vdash_{iSTL} N(A, f) , \Gamma \Rightarrow \Delta$ for some $f$.

\textit{Proof}:

I. Suppose we have $\Rightarrow A$ and $N(A, f) , \Gamma \Rightarrow \Delta$ for some $f$. First for any $a \in f(A)$, construct $\Rightarrow \nabla^{a_1} \Box^{a_2} \dots \nabla^{a_{k-1}} \Box^{a_k} A$ from $\Rightarrow A$ as follows: Introduce $\top$ in the left side with $Lw$ and apply $R\rightarrow$ to get $\Rightarrow \top \rightarrow A$. Repeat $a_k$ times to get $\Rightarrow \Box^{a_k} A$. Then $a_{k-1}$ applications of $N$ yields $\Rightarrow \nabla^{a_{k-1}} \Box^{b_k} A$. Repeat this process for all elements of $a$, and do this for all $a \in f(A)$. Now $Cut$ all of them into $N(A, f) , \Gamma \Rightarrow \Delta$ to get $\Gamma \Rightarrow \Delta$. \\

II. For the other direction, we can prove a stronger version: Suppose $\Rightarrow A \vdash_{iSTL} \Gamma \Rightarrow \Delta$ by a proof tree $\mathbf{D}$. We want to prove $\vdash_{iSTL} \{ \nabla^a \Box^b A\}_{(a,b) \in I} , \Gamma \Rightarrow \Delta$ for some finite set of pairs of natural numbers $I$. By induction on the length of $\mathbf{D}$, for any $\Gamma'$, $\Delta'$ and any iSTL$+A$ proof tree $\mathbf{D}'$ which proves $\Gamma' \Rightarrow \Delta'$ in lower length than $\mathbf{D}$, we can assume there is an iSTL proof tree IH($\mathbf{D}'$) that proves $\{\nabla^a \Box^b A\}_{(a,b) \in I_{IH(\mathbf{D}')}} , \Gamma' \Rightarrow \Delta'$ for some set of pairs of natural numbers $I_{IH(\mathbf{D}')}$.

For different possible rules in iSTL$+A$ as the last rule of $\mathbf{D}$, we will construct an iSTL proof of $\{ \nabla^a \Box^b A\}_{(a,b) \in I} , \Gamma \Rightarrow \Delta$ for some finite set of pairs natural numbers $I$. In all cases, the possible immediate sub-trees of $\mathbf{D}$ are called $\mathbf{D_0}$ and $\mathbf{D_1}$.
\begin{enumerate}
	\setcounter{enumi}{-1}
	\item ($\Rightarrow A$) This implies $\Gamma = \{\}$ and $\Delta = A$. Then we have $A \Rightarrow A$ by $Id$.

	\item ($Id$) This implies $\Gamma = \Delta = A$. Then $A \Rightarrow A$ by $Id$.

	\item ($Ta$) This implies $\Gamma = \{\}$ and $\Delta = \top$. We have $\Rightarrow \top$ by $Ta$.

	\item ($Ex$) This implies $\Gamma = \bot$ and $\Delta = \{\}$. We have $\bot \Rightarrow$ by $Ex$.

	\item[4-10.] ($Lw$),($Lc$),($Rw$),($L\land_1$),($L\land_2$),($R\lor_1$),($R\lor_2$) Just apply the same rule on IH($\mathbf{D_0}$). Notice that $I$ would be the same set we get from the induction hypothesis.
	\setcounter{enumi}{10}

	\item ($N$) Same as the previous cases, except that $I = \{ (a + 1 , b) : (a , b) \in I_{IH(\mathbf{D_0})} \}$, where $I_{IH(\mathbf{D_0})}$ is the set from the induction hypothesis.

	\item ($Cut$) Just apply $Cut$ on IH($\mathbf{D_0}$) and IH($\mathbf{D_1}$). Notice that $I = I_{IH(\mathbf{D_0})} \cup I_{IH(\mathbf{D_1})}$.

	\item[13,14,15.] ($R\land$), ($L\lor$), ($L\rightarrow$) Same as the $Cut$ case, except that before applying the corresponding rule to IH($\mathbf{D_0}$) and IH($\mathbf{D_1}$), we must first equalize their contexts using enough applications of $Lw$, to $\{ \nabla^a \Box^b A \}_{(a,b) \in I_{IH(\mathbf{D_0})} \cup I_{IH(\mathbf{D_1})}}$.
	\setcounter{enumi}{15}

	\item ($R\rightarrow$) If $a = 0$ for some $(a,b) \in I_{IH(\mathbf{D_0})}$, then $R\rightarrow$ is not applicable on IH($\mathbf{D_0}$). We have $\nabla \Box^{b+1} A \Rightarrow \Box^b A$ from lemma \ref{lem:i-nabla-box}. For all such $(0,b) \in I_{IH(\mathbf{D_0})}$ we first $Cut$ this into IH($\mathbf{D_0}$), so that the context has at least one $\nabla$ before applying $R\rightarrow$. Notice that $I = \{ (a, b) : (a+1, b) \in I \} \cup$ $\{ (0, b+1) : (0, b) \in I \}$.
\end{enumerate}

\subsection{Theorem} \textit{Deductive Interpolation for iSTL: } For any $\Delta'$ and $\Delta$, if $\Rightarrow \Delta' \vdash_{iSTL} \Rightarrow \Delta$ then there exists a formula $C$ such that
\begin{enumerate}[label=(\arabic*)]
	\item $\Rightarrow \Delta' \vdash_{iSTL} \Rightarrow C$,
	\item $\Rightarrow C \vdash_{iSTL} \Rightarrow \Delta$ and
	\item $P(C) \subseteq P(\Delta') \cap P(\Delta)$.
\end{enumerate}

\textit{Proof}: In case $\Delta' = \{\}$, just take $C = \bot$. Now let's assume $\Delta' = A$ for some formula $A$.

We will first prove the following stronger version: For any $\Gamma_1$, $\Gamma_2$ and $\Delta$, if $\vdash_{GSTL^-} \Gamma_1 , \Gamma_2 \Rightarrow \Delta$ then there is a formula $C$ and a finite multi-set of formulas $\Pi$ (consisting of members of $\Gamma_1$ with some $\nabla$ and $\Box$) such that
\begin{enumerate}[label=(\arabic**)]
	\item $\vdash_{GSTL^-} \Pi \Rightarrow C$,
	\item $\vdash_{GSTL^-} \Gamma_2 , C \Rightarrow \Delta$,
	\item $P(C) \subseteq P(\Gamma_1) \cap P(\Gamma_2,\Delta)$, and
	\item for any $B$ in $\Pi$ there is $A$ in $\Gamma_1$ such that $B = \nabla^{n_0} \Box^{n_1} \dots \nabla^{n_{k-1}} \Box^{n_k} A$ for some finite sequence of natural numbers $n_1,...,n_k$.
\end{enumerate}
Notice that using the notation defined above we can ommit (4*) and replace $\Pi$ in (1*) by $N(\Gamma_1, f)$ for some $f$.

Then to prove the original statement, first convert $\Rightarrow A \vdash_{iSTL} \Rightarrow \Delta$ to $\vdash_{GSTL^-} \{\nabla^a\Box^b A : (a,b) \in I\} \Rightarrow \Delta$ using lemma \ref{lem:vdash} and theorems \ref{translation} and \ref{cut-elim}, and then let $\Gamma_1 = \{\nabla^a \Box^b A : (a,b) \in I \}$, $\Gamma_2 = \{\}$. Notice that we can again translate (2*) and (3*) to (2) and (3) respectively, using \ref{lem:vdash}, \ref{cut-elim} and \ref{translation}. To convert (1*) to (1), i.e. $\vdash_{iSTL} \{\nabla^a \Box^b \nabla^c \Box^d A : (a,b) \in I , (c,d) \in J \} \Rightarrow C$ to $\Rightarrow A \vdash_{iSTL} C$ we must construct $\Rightarrow \nabla^a \Box^b \nabla^c \Box^d A$ for each $(a,b)$ and $(c,d)$ in $I$ and $J$ from $\Rightarrow A$, in a manner similar to the proof of \ref{lem:vdash}, and then cut them to get $\Rightarrow \Delta$.

Now back to the stronger form, using induction on the length of $\mathbf{D}$, we can assume for any $\Gamma_1'$, $\Gamma_2'$, $\Delta'$ and $\mathbf{D}'$, where $\mathbf{D}'$ is a GSTL$^-$ proof for $\Gamma_1' , \Gamma_2' \Rightarrow \Delta'$ and $h(\mathbf{D}')<h(\mathbf{D})$, there is a formula $C_{\langle \Gamma_1';\Gamma_2';\Delta' \rangle}$, more concisely $C'$, and a finite multi-set of formulas $\Pi'$ ($= N(\Gamma_1', f)$ for some proper f) such that
\begin{enumerate}[label=(\arabic*')]
	\item $\vdash_{GSTL^-} \Pi' \Rightarrow C'$,
	\item $\vdash_{GSTL^-} \Gamma_2' , C' \Rightarrow \Delta'$,
	\item $P(C') \subseteq P(\Gamma_1') \cap P(\Gamma_2',\Delta')$, and
	\item for any $B$ in $\Pi'$ there is $A$ in $\Gamma_1'$ such that $B = \nabla^{n_0} \Box^{n_1} \dots \nabla^{n_{k-1}} \Box^{n_k} A$ for some finite sequence of natural numbers $n_1,...,n_k$.
\end{enumerate}
Now construct proper $C$ and $\Pi$ for different cases for the last rule of $\mathbf{D}$. In each case, the immediate sub-trees of $\mathbf{D}$ are denoted by $\mathbf{D_i} (i \in \{0,1\})$.
\begin{enumerate}
	\item ($Id$) We have $\Gamma_1,\Gamma_2 = \Delta = A$.
	\begin{enumerate}
		\item If $\Gamma_1 = \{\}$ and $\Gamma_2 = A$, then let $C = \top$. So we have $\Rightarrow \top$ by $Ta$ and $A , \top \Rightarrow A$ by $Id$ and $Lw$.

		\item If $\Gamma_1 = A$ and $\Gamma_2 = \{\}$ then let $C = A$. So we have $A \Rightarrow A$ by $Id$.
	\end{enumerate}
	\item ($Ta$) Take $C = \top$.

	\item ($Ex$) Take $C = \nabla^n \bot$.

	\item ($Lw$) $\mathbf{D}$ proves $\Gamma_1' , \Gamma_2' , A \Rightarrow \Delta$ and has a sub-proof for $\Gamma_1' , \Gamma_2' \Rightarrow \Delta$, for which IH gives an interpolant $C_{\langle\Gamma_1';\Gamma_2';\Delta\rangle}$, a multi-set $\Pi$ consisting of formulas made up of some $\nabla$'s and $\Box$'s applied to members of $\Gamma_1'$, and proofs for $\Pi \Rightarrow C_{\langle\Gamma_1';\Gamma_2';\Delta\rangle}$ and $\Gamma_2 , C_{\langle\Gamma_1';\Gamma_2';\Delta\rangle} \Rightarrow \Delta$, such that $P(C_{\langle\Gamma_1';\Gamma_2';\Delta\rangle}) \subseteq$ $ P(\Gamma_1') \cap P(\Gamma_2' , \Delta)$.
	\begin{enumerate}
		\item If $\Gamma_1 = \Gamma_1'$ and $\Gamma_2 = \Gamma_2' , A$, take $C = C_{\langle\Gamma_1';\Gamma_2';\Delta\rangle}$. Then we have  $\Pi \Rightarrow C$ by IH and $\Gamma_2 , A , C \Rightarrow \Delta$ by $Lw$ and IH. From IH, we also have $P(C) \subseteq P(\Gamma_1') \cap P(\Gamma_2' , A , \Delta)$, since $P$ takes ``$,$'' to ``$\cup$'', which distributes over ``$\cap$'' and is increasing with respect to ``$\subseteq$''.

		\item If $\Gamma_1 = \Gamma_1' , A$ and $\Gamma_2 = \Gamma_2'$, again take $C = C_{\langle\Gamma_1';\Gamma_2';\Delta\rangle}$. Then we have  $\Pi , A \Rightarrow C$ by $Lw$ and IH, and $\Gamma_2 , C \Rightarrow \Delta$ by IH. We also have $P(C) \subseteq P(\Gamma_1' , A) \cap P(\Gamma_2' , \Delta)$ by IH and argument similar to the previous case.
	\end{enumerate}

	\item ($Lc$) $\mathbf{D}$ proves $\Gamma_1' , \Gamma_2' , A \Rightarrow \Delta$ and has a sub-proof for $\Gamma_1' , \Gamma_2' , A , A \Rightarrow \Delta$.
	\begin{enumerate}
		\item If $\Gamma_1 = \Gamma_1'$ and $\Gamma_2 = \Gamma_2' , A$, take $C = C_{\langle\Gamma_1';\Gamma_2',A,A;\Delta\rangle}$. Then we have $\Pi \Rightarrow C$ by IH and $\Gamma_2' , A \Rightarrow \Delta$ by IH and $Lc$. From IH, we also have $P(C) \subseteq P(\Gamma_1') \cap P(\Gamma_2',A,\Delta)$, since $P(\Gamma,X) = P(\Gamma,X,X)$.
		
		\item If $\Gamma_1 = \Gamma_1' , A$ and $\Gamma_2 = \Gamma_2'$, take $C = C_{\langle\Gamma_1',A,A;\Gamma_2';\Delta\rangle}$. Then we have $N(\Gamma_1',A,A,f) \Rightarrow C$ by IH (which is just $N(\Gamma_1',A,f')$ for some $f'$), and $\Gamma_2' \Rightarrow \Delta$ by IH. We also have $P(C) \subseteq P(\Gamma_1',A) \cap$ $P(\Gamma_2',\Delta)$ as justified before.
	\end{enumerate}

	\item[6,7.] ($L\land_i$, {\small$i \in \{1,2\}$}) $\mathbf{D}$ proves $\Gamma_1' , \Gamma_2' , \nabla^n (A_1 \land A_2) \Rightarrow \Delta$ and has a sub-proof for $\Gamma_1' , \Gamma_2' , \nabla^n A_i \Rightarrow \Delta$.
	\begin{enumerate}
		\item If $\Gamma_1 = \Gamma_1'$ and $\Gamma_2 = \Gamma_2' , \nabla^n (A_1 \land A_2)$, take $C = C_{\langle\Gamma_1';\Gamma_2',\nabla^n A_i;\Delta\rangle}$. Then we have $\Pi \Rightarrow C$ by IH and $\Gamma_2' , \nabla^n (A_1 \land A_2) \Rightarrow \Delta$ by IH and $L\land_i$. From IH, we also have $P(C) \subseteq$ $P(\Gamma_1')$ $\cap P(\Gamma_2',\nabla^n(A_1 \land A_2),\Delta)$, since $P(\nabla^n X) = P(X)$ and $P$ takes sub-formula ordering to ``$\subseteq$''.
		
		\item If $\Gamma_1 = \Gamma_1' , \nabla^n (A_1 \land A_2)$ and $\Gamma_2 = \Gamma_2'$, take $C = C_{\langle\Gamma_1',\nabla^n A_i;\Gamma_2';\Delta\rangle}$. Then we have $N(\Gamma_1',\nabla^n A_i,f)$ $\Rightarrow C$ by IH. For any $a \in f(\nabla^n A_i)$, we have $(\nabla\Box)^a \nabla^n (A_1 \land A_2) \Rightarrow (\nabla\Box)^a \nabla^n A_i$ from lemmas \ref{lem:nabla-box-rule} and \ref{translation}, from which we can get $N(\Gamma_1',\nabla^n (A_1 \land A_2),f') \Rightarrow C$ using $Cut$. Also from IH we have $\Gamma_2' \Rightarrow \Delta$. We also have $P(C) \subseteq P(\Gamma_1',\nabla^n (A_1 \land A_2))$ $\cap P(\Gamma_2',\Delta)$ as justified in the previous case.
	\end{enumerate}
	\setcounter{enumi}{7}

	\item ($R\land$) $\mathbf{D}$ proves $\Gamma_1 , \Gamma_2 \Rightarrow A \land B$ and has sub-proofs for $\Gamma_1 , \Gamma_2 \Rightarrow A$ and $\Gamma_1 , \Gamma_2 \Rightarrow B$.\\
	Let $C_1 = C_{\langle\Gamma_1;\Gamma_2;A\rangle}$ and $C_2 = C_{\langle\Gamma_1;\Gamma_2;B\rangle}$, and then take $C = C_1 \land C_2$.
	We have $N(\Gamma_1,f_1) \Rightarrow C_1$ and $N(\Gamma_1,f_2) \Rightarrow C_2$, both from IH, to which we apply proper $Lw$ and equalize their contexts to get $N(\Gamma_1,f) \Rightarrow C_1$ and $N(\Gamma_1,f) \Rightarrow C_2$ for some $f$. Then by $R\land$ we have $N(\Gamma_1,f) \Rightarrow C_1 \land C_2$.
	We also have $\Gamma_2 , C_1 \Rightarrow A$ and $\Gamma_2 , C_2 \Rightarrow B$, again from IH.
	We can then derive $\Gamma_2 , C_1 \land C_2 \Rightarrow A$ and $\Gamma_2 , C_1 \land C_2 \Rightarrow B$, respectively by $L\land_1$ and $L\land_2$, and finally  $\Gamma_2 , C_1 \land C_2 \Rightarrow A \land B$ by $R\land$.
	We also have $P(C_1) \subseteq P(\Gamma_1) \cap P(\Gamma_2 , A)$ and $P(C_2) \subseteq P(\Gamma_1) \cap P(\Gamma_2 , B)$. So $P(C_1 , C_2) \subseteq P(\Gamma_1) \cap P(\Gamma_2 , A , B)$ as it was justified before, and then $P(C_1 \land C_2) \subseteq P(\Gamma_1) \cap P(\Gamma_2 , A \land B)$.

	\item ($L\lor$) $\mathbf{D}$ proves $\Gamma_1' , \Gamma_2' , \nabla^n (A \lor B) \Rightarrow \Delta$ and has sub-proofs for $\Gamma_1' , \Gamma_2' , \nabla^n A \Rightarrow \Delta$ and $\Gamma_1' , \Gamma_2' ,$ $\nabla^n B \Rightarrow \Delta$.
	\begin{enumerate}
		\item If $\Gamma_1 = \Gamma_1'$ and $\Gamma_2 = \Gamma_2' , \nabla^n (A \lor B)$, let $C_1 = C_{\langle\Gamma_1';\Gamma_2',\nabla^n A;\Delta\rangle}$ and $C_2 = C_{\langle\Gamma_1';\Gamma_2',\nabla^n B;\Delta\rangle}$, and then take $C = C_1 \land C_2$.
		We have $N(\Gamma_1',f_1) \Rightarrow C_1$ and $N(\Gamma_1',f_2) \Rightarrow C_2$ from IH, to which we apply $R\land$ after equalizing their contexts with $Lw$, to get $N(\Gamma_1',f) \Rightarrow C_1 \land C_2$ for some $f$.
		From IH, by $L\land_1$ and $L\land_2$ we can derive $\Gamma_2' , \nabla^n A , C_1 \land C_2 \Rightarrow \Delta$ and $\Gamma_2' , \nabla^n B , C_1 \land C_2 \Rightarrow \Delta$ respectively, to which we apply $L\lor$ to get to $\Gamma_2' , \nabla^n (A \lor B) , C_1 \land C_2 \Rightarrow \Delta$.
		From IH, we also have $P(C_1) \subseteq P(\Gamma_1') \cap P(\Gamma_2' , \nabla^n A , \Delta)$ and $P(C_2) \subseteq P(\Gamma_1') \cap P(\Gamma_2' , \nabla^n B , \Delta)$. Just like the previous case, we can deduce that $P(C_1 \land C_2) \subseteq P(\Gamma_1') \cap P(\Gamma_2' , \nabla^n (A \land B) , \Delta)$.

		\item If $\Gamma_1 = \Gamma_1' , \nabla^n (A \lor B)$ and $\Gamma_2 = \Gamma_2'$, let $C_1 = C_{\langle\Gamma_1',\nabla^n A;\Gamma_2';\Delta\rangle}$ and $C_2 = C_{\langle\Gamma_1',\nabla^n B;\Gamma_2';\Delta\rangle}$, and then take $C = C_1 \lor C_2$.
		From IH, by $R\lor_1$ and $R\lor_2$ we can derive $N(\Gamma_1',\nabla^n A,f_1) \Rightarrow C_1 \lor C_2$ and $N(\Gamma_1',\nabla^n B,f_2) \Rightarrow C_1 \lor C_2$ respectively, to which we apply \todo{what?} to get to $N(\Gamma_1',\nabla^n (A \lor B),f) \Rightarrow C_1 \lor C_2$ for some $f$.
		We have $\Gamma_2' , C_1 \lor C_2 \Rightarrow \Delta$ from IH and $L\lor$.
		From IH, we also have $P(C_1) \subseteq P(\Gamma_1' , \nabla^n A) \cap$ $P(\Gamma_2' , \Delta)$ and $P(C_2) \subseteq P(\Gamma_1' , \nabla^n B) \cap P(\Gamma_2' , \Delta)$. Just like the previous case, we can deduce that $P(C_1 \lor C_2) \subseteq P(\Gamma_1' , \nabla^n (A \land B)) \cap P(\Gamma_2' , \Delta)$.
	\end{enumerate}

	\item[10,11.] ($R\lor_i$, {\small$i \in \{1,2\}$}) $\mathbf{D}$ proves $\Gamma_1 , \Gamma_2 \Rightarrow A_1 \lor A_2$ and has a sub-proof for $\Gamma_1 , \Gamma_2 \Rightarrow A_i$. Take $C = C_{\langle\Gamma_1;\Gamma_2;A_i\rangle}$. Then we have $N(\Gamma_1,f) \Rightarrow C$ from IH and $\Gamma_2 , C \Rightarrow A_1 \lor A_2$ from IH and $R\lor_i$.
	From IH, we also have $P(C) \subseteq P(\Gamma_1) \cap P(\Gamma_2 , A_1 \lor A_2)$, as was justified before.
	\setcounter{enumi}{11}

	\item ($L\rightarrow$) $\mathbf{D}$ proves $\Gamma_1' , \Gamma_2' , \nabla^{n+1} (A \rightarrow B) \Rightarrow \Delta$ and has sub-proofs for $\Gamma_1' , \Gamma_2' \Rightarrow \nabla^n A$ and $\Gamma_1' , \Gamma_2' , \nabla^n B \Rightarrow \Delta$.
	\begin{enumerate}
		\item If $\Gamma_1 = \Gamma_1'$ and $\Gamma_2 = \Gamma_2' , \nabla^{n+1} (A \rightarrow B)$, let $C_1 = C_{\langle\Gamma_1';\Gamma_2';\nabla^n A\rangle}$ and $C_2 = C_{\langle\Gamma_1';\Gamma_2',\nabla^n B;\Delta\rangle}$, and take $C = C_1 \land C_2$.
		We have $N(\Gamma_1',f) \Rightarrow C_1 \land C_2$ from IH and $R\land$.
		From IH, by $L\land_1$ and $L\land_2$ we can derive $\Gamma_2' , C_1 \land C_2 \Rightarrow \nabla^n A$ and $\Gamma_2' , \nabla^n B , C_1 \land C_2 \Rightarrow \Delta$ respectively, to which we apply $L\rightarrow$ to get to $\Gamma_2' , \nabla^{n+1} (A \rightarrow B) , C_1 \land C_2 \Rightarrow \Delta$.
		From IH, we also have $P(C_1) \subseteq P(\Gamma_1') \cap$ $P(\Gamma_2' , \nabla^n A)$ and $P(C_2) \subseteq P(\Gamma_1') \cap P(\Gamma_2' , \nabla^n B , \Delta)$. This implies $P(C_1 \land C_2) \subseteq P(\Gamma_1') \cap P(\Gamma_2' , \nabla^{n+1} (A \rightarrow B) , \Delta)$.

		\item If $\Gamma_1 = \Gamma_1' , \nabla^{n+1} (A \rightarrow B)$ and $\Gamma_2 = \Gamma_2'$, let $C_1 = C_{\langle\Gamma_2';\Gamma_1';\nabla^n A\rangle}$ and $C_2 = C_{\langle\Gamma_1',\nabla^n B;\Gamma_2';\Delta\rangle}$, and take $C = \nabla (C_1 \rightarrow C_2)$.
		From IH we have $\Gamma_1' , C_1 \Rightarrow \nabla^n A$. Also from IH, with a $Lw$ to add $C_1$ to the left, we have $N(\Gamma_1',\nabla^n B,f_1) , C_1 \Rightarrow C_2$. By applying $L\rightarrow$ \todo{how?} we get $N(\Gamma_1,\nabla^{n+1}(A \rightarrow B),f) , C_1 \Rightarrow C_2$ for some $f$. By lemma \ref{lem:i-nabla-box} and $Cut$ we can add a $\nabla\Box$ to all formulas in $N(\Gamma_1',\nabla^{n+1}(A \rightarrow B),f)$, so we can move $C_1$ to the right by $R\rightarrow$. With $N'$ we will have $N(\Gamma_1',\nabla^{n+1}(A \rightarrow B),f') , \Rightarrow \nabla (C_1 \rightarrow C_2)$ for some $f'$.
		From IH we also have $N(\Gamma_2',f_2) \Rightarrow C_1$ and $\Gamma_2' , C_2 \Rightarrow \Delta$. With proper applications of $Lw$, we can equalize their contexts to $N(\Gamma_2',f_2) , \Gamma_2' \Rightarrow C_1$ and $N(\Gamma_2',f_2) , \Gamma_2' , C_2 \Rightarrow \Delta$, from which we then derive $N(\Gamma_2',f_2) , \Gamma_2' , \nabla (C_1 \rightarrow C_2) \Rightarrow \Delta$ by an application of $L\rightarrow$. \todo{which is not what we want.} We also have from IH $P(C_1) \subseteq P(\Gamma_2') \cap P(\Gamma_1' , \nabla^n A)$ and $P(C_2) \subseteq P(\Gamma_1' , \nabla^n B) \cap P(\Gamma_2' , \Delta)$. Then $P(\nabla (C_1 \rightarrow C_2)) \subseteq P(\Gamma_1' , \nabla^{n+1} (A \rightarrow B)) \cap P(\Gamma_2' , \Delta)$.
	\end{enumerate}

	\item ($N'$) $\mathbf{D}$ proves $\nabla \Gamma_1 , \nabla \Gamma_2 \Rightarrow \nabla \Delta$ and has a sub-proof for $\Gamma_1 , \Gamma_2 \Rightarrow \Delta$. Just take $C = C(\Gamma_1;\Gamma_2;\Delta)$ and apply $N'$ on the sequents from IH. \todo{!} The variable condition is also trivial from IH.
\end{enumerate}