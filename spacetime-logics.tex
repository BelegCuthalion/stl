\documentclass[a4paper, 12pt]{paper}
\usepackage{amsmath}
\usepackage{amssymb}
\usepackage{mathabx}
\usepackage{titlesec}
\usepackage{fullpage}
\usepackage{tikz-cd}
\usepackage{rotating}
\usepackage{pdflscape}
\usepackage{multicol}
\usepackage{multirow}
\usepackage{diagbox}
\usepackage[left=.5in,right=.5in,top=.5in,bottom=.5in]{geometry}
\usepackage{enumitem}
\usepackage[colorlinks]{hyperref}
\usepackage{bussproofs}

\setitemize{topsep=3pt,parsep=5pt,partopsep=0pt,label=,leftmargin=1.3pc}
\titleformat{\section}[runin]{\normalfont\bfseries}{\thesection}{0.5em}{}
\titlespacing{\section}{0pc}{5ex plus .1ex minus .2ex}{1pc}
\titleformat{\subsection}[runin]{\normalfont\bfseries}{\thesubsection}{0.7em}{}
\titlespacing{\subsection}{0pc}{2ex plus .1ex minus .2ex}{1pc}
\titleformat{\subsubsection}[runin]{\normalfont\bfseries}{\thesubsubsection}{0.7em}{}
\titlespacing{\subsubsection}{0pc}{2ex plus .1ex minus .2ex}{1pc}
\newcommand\eqn{\refstepcounter{equation}\tag{\theequation}}
\binoppenalty=\maxdimen
\relpenalty=\maxdimen
\newcommand{\ul}{\ulcorner}
\newcommand{\ur}{\urcorner}
\newcommand{\val}[1]{\ulcorner len1 \urcorner}
\newcommand{\caseref}[1]{\hyperref[#1]{\ref{#1}}}
\newcommand{\rot}{\rotatebox{90}}
\newcommand{\p}{\partial}
\newcommand{\todo}[1]{{\color{red}\textbf{TODO} #1}}
\EnableBpAbbreviations

\begin{document}
{\noindent
	v 0.7.1 \\
{\large\textbf{Some notes about iSTL}}
}
\\
\setcounter{section}{-1}
\section{Notation} In the following $\Gamma$, $\Sigma$ and $\Pi$ are names for finite multi-sets of formulas, $\Delta$ for sub-singleton of some formula, $A$, $B$ and $C$ for formulas, $p$ for propositional variables and $n$, $l$, $r$ and $k$ for natural numbers. We denote by $P$ the set of all propositional variables, and by $P(A)$ the set of those that occur in the formula $A$.

We will write $A$ for the singleton $\{A\}$ where ever it is inferable from the context.
``$,$'' is the multi-set union.
We will also write $A^1$ for $\{A\}$, write $A^{n+1}$ for $A^n, A$, write $\Gamma^n$ for $\{ A^n : A \in \Gamma \}$ and write $P(\Gamma)$ for $\bigcup_{A \in \Gamma} P(A)$

A sequent $\Gamma \Rightarrow \Delta$ is a binary relation between $\Gamma$, a multi-set of formulas, and $\Delta$, a sub-singleton of some formula

We will also write $\Box A$ for $\top \rightarrow A$.


\section{iSTL$^-$} An iSTL sequent $\Gamma \Rightarrow \Delta$ is a binary relation between $\Gamma$, a multi-set of formulas, and $\Delta$, a sub-singleton of some formula, defined inductively by the following rules.
\begin{multicols}{3}
	\begin{prooftree}
		\AXC{}
		\RightLabel{$Id$}
		\UIC{$A \Rightarrow A$}
	\end{prooftree}
\columnbreak
	\begin{prooftree}
		\AXC{}
		\RightLabel{$Ta$}
		\UIC{$\Rightarrow \top$}
	\end{prooftree}
\columnbreak
	\begin{prooftree}
		\AXC{}
		\RightLabel{$Ex$}
		\UIC{$\bot \Rightarrow$}
	\end{prooftree}
\end{multicols}
‌\\
\begin{multicols}{3}
	\begin{prooftree}
		\RightLabel{$Lw$}
		\AXC{$\Gamma \Rightarrow \Delta$}
		\UIC{$\Gamma , A \Rightarrow \Delta$}
	\end{prooftree}
	\columnbreak
	\begin{prooftree}
		\RightLabel{$Rw$}
		\AXC{$\Gamma \Rightarrow$}
		\UIC{$\Gamma \Rightarrow A$}
	\end{prooftree}
	\columnbreak
	\begin{prooftree}
		\RightLabel{$Lc$}
		\AXC{$\Gamma , A , A \Rightarrow \Delta$}
		\UIC{$\Gamma , A \Rightarrow \Delta$}
	\end{prooftree}
\end{multicols}
‌\\
\begin{multicols}{3}
	\begin{prooftree}
		\RightLabel{$L\land$}
		\AXC{$\Gamma , A \Rightarrow \Delta$}
		\UIC{$\Gamma , A \land B \Rightarrow \Delta$}
	\end{prooftree}
	\columnbreak
	\begin{prooftree}
		\RightLabel{$L\land$}
		\AXC{$\Gamma , B \Rightarrow \Delta$}
		\UIC{$\Gamma , A \land B \Rightarrow \Delta$}
	\end{prooftree}
	\columnbreak
	\begin{prooftree}
		\RightLabel{$R\land$}
		\AXC{$\Gamma \Rightarrow A$}
		\AXC{$\Gamma \Rightarrow B$}
		\BIC{$\Gamma \Rightarrow A \land B$}
	\end{prooftree}
\end{multicols}
‌\\
\begin{multicols}{3}
	\begin{prooftree}
		\RightLabel{$L\lor$}
		\AXC{$\Gamma , A \Rightarrow \Delta$}
		\AXC{$\Gamma , B \Rightarrow \Delta$}
		\BIC{$\Gamma , A \lor B \Rightarrow \Delta$}
	\end{prooftree}
	\columnbreak
	\begin{prooftree}
		\RightLabel{$R\lor$}
		\AXC{$\Gamma \Rightarrow A$}
		\UIC{$\Gamma \Rightarrow A \lor B$}
	\end{prooftree}
	\columnbreak
	\begin{prooftree}
		\RightLabel{$R\lor$}
		\AXC{$\Gamma \Rightarrow B$}
		\UIC{$\Gamma \Rightarrow A \lor B$}
	\end{prooftree}
\end{multicols}
‌\\
\begin{multicols}{2}
	\begin{prooftree}
		\RightLabel{$L\rightarrow$}
		\AXC{$\Gamma \Rightarrow A$}
		\AXC{$\Gamma , B \Rightarrow \Delta$}
		\BIC{$\Gamma, \nabla(A \rightarrow B) \Rightarrow \Delta$}
	\end{prooftree}
	\columnbreak
	\begin{prooftree}
		\RightLabel{$R\rightarrow$}
		\AXC{$\nabla\Gamma , A \Rightarrow B$}
		\UIC{$\Gamma \Rightarrow A \rightarrow B$}
	\end{prooftree}
\end{multicols}
‌\\
\begin{prooftree}
	\RightLabel{$N$}
	\AXC{$\Gamma \Rightarrow A$}
	\UIC{$\nabla\Gamma \Rightarrow \nabla A$}
\end{prooftree}
‌\\
\subsection{Cut}
\begin{prooftree}
	\RightLabel{$Cut$}
	\AXC{$\Gamma \Rightarrow A$}
	\AXC{$\Gamma' , A \Rightarrow \Delta$}
	\BIC{$\Gamma , \Gamma' \Rightarrow \Delta$}
\end{prooftree}
By $\text{iSTL}$ we mean $\text{iSTLL}^- + Cut$
\pagebreak


\section{GSTL} A GSTL sequent $\Gamma \Rightarrow \Delta$ is a binary relation between $\Gamma$, a multi-set of formulas, and $\Delta$, a sub-singleton of some formula, defined inductively by the following rules.

\begin{multicols}{3}
	\begin{prooftree}
		\RightLabel{$Id$}
		\AXC{}
		\UIC{$A \Rightarrow A$}
	\end{prooftree}
	\columnbreak
	\begin{prooftree}
		\RightLabel{$Ta$}
		\AXC{}
		\UIC{$\Rightarrow \top$}
	\end{prooftree}
	\columnbreak
	\begin{prooftree}
		\RightLabel{$Ex$}
		\AXC{}
		\UIC{$\nabla^n \bot \Rightarrow$}
	\end{prooftree}
\end{multicols}
‌\\
\begin{multicols}{3}
	\begin{prooftree}
		\RightLabel{$Lw$}
		\AXC{$\Gamma \Rightarrow \Delta$}
		\UIC{$\Gamma, A \Rightarrow \Delta$}
	\end{prooftree}
	\columnbreak
	\begin{prooftree}
		\RightLabel{$Rw$}
		\AXC{$\Gamma \Rightarrow$}
		\UIC{$\Gamma \Rightarrow A$}
	\end{prooftree}
	\columnbreak
	\begin{prooftree}
		\RightLabel{$Lc$}
		\AXC{$\Gamma , A , A \Rightarrow \Delta$}
		\UIC{$\Gamma , A \Rightarrow \Delta$}
	\end{prooftree}
\end{multicols}
‌\\
\begin{multicols}{3}
	\begin{prooftree}
		\RightLabel{$L\land_1$}
		\AXC{$\Gamma , \nabla^n A \Rightarrow \Delta$}
		\UIC{$\Gamma , \nabla^n (A \land B) \Rightarrow \Delta$}
	\end{prooftree}
	\columnbreak
	\begin{prooftree}
		\RightLabel{$L\land_2$}
		\AXC{$\Gamma , \nabla^n B \Rightarrow \Delta$}
		\UIC{$\Gamma , \nabla^n (A \land B) \Rightarrow \Delta$}
	\end{prooftree}
	\columnbreak
	\begin{prooftree}
		\RightLabel{$R\land$}
		\AXC{$\Gamma \Rightarrow A$}
		\AXC{$\Gamma \Rightarrow B$}
		\BIC{$\Gamma \Rightarrow A \land B$}
	\end{prooftree}
\end{multicols}
‌\\
\begin{multicols}{3}
	\begin{prooftree}
		\RightLabel{$L\lor$}
		\AXC{$\Gamma , \nabla^n A \Rightarrow \Delta$}
		\AXC{$\Gamma , \nabla^n B \Rightarrow \Delta$}
		\BIC{$\Gamma , \nabla^n (A \lor B) \Rightarrow \Delta$}
	\end{prooftree}
	\columnbreak
	\begin{prooftree}
		\RightLabel{$R\lor_1$}
		\AXC{$\Gamma \Rightarrow A$}
		\UIC{$\Gamma \Rightarrow A \lor B$}
	\end{prooftree}
	\columnbreak
	\begin{prooftree}
		\RightLabel{$R\lor_2$}
		\AXC{$\Gamma \Rightarrow B$}
		\UIC{$\Gamma \Rightarrow A \lor B$}
	\end{prooftree}
\end{multicols}
‌\\
\begin{multicols}{2}
	\begin{prooftree}
		\RightLabel{$L\rightarrow$}
		\AXC{$\Gamma \Rightarrow \nabla^n A$}
		\AXC{$\Gamma , \nabla^n B \Rightarrow \Delta$}
		\BIC{$\Gamma , \nabla^{n+1} (A \rightarrow B) \Rightarrow \Delta$}
	\end{prooftree}
	\columnbreak
	\begin{prooftree}
		\RightLabel{$R\rightarrow$}
		\AXC{$\nabla \Gamma , A \Rightarrow B$}
		\UIC{$\Gamma \Rightarrow A \rightarrow B$}
	\end{prooftree}
\end{multicols}
‌\\
\begin{prooftree}
	\RightLabel{$N'$}
	\AXC{$\Gamma \Rightarrow \Delta$}
	\UIC{$\nabla \Gamma \Rightarrow \nabla \Delta$}
\end{prooftree}

\subsection{$MC$} For any $l \ge 0$ and $n \ge 1$
\begin{prooftree}
	\AXC{$\Gamma \Rightarrow A$}
	\AXC{$\Sigma , (\nabla^l A)^n \Rightarrow \Delta$}
	\RightLabel{$MC$}
	\BIC{$\nabla^l \Gamma , \Sigma \Rightarrow \Delta$}
\end{prooftree}
$A$ is called the \textit{cut-formula}.
By $\text{GSTL}$ we mean $\text{GSTL}^- + MC$

\subsubsection{Rank} Rank of a formula $A$ is defined as
\[ \rho(A) = \begin{cases}
1 & \quad ; A \in P \cup \{ \bot, \top \} \\
\rho(B) & \quad ; A = \nabla B \\
max(\rho(B), \rho(C)) + 1 & \quad ; A = B \Box C, \Box \in \{ \land , \lor, \rightarrow \}
\end{cases} \]
We also define rank for rule instances and proof trees. For an instance of the $MC$ rule $c$ with cut-formula $A$, $\rho(c) = \rho(A)$, $0$ if it's not an instance of the $MC$ rule.
For a proof tree $\mathbf{D}$, $\rho(\mathbf{D})$ is the maximum rank of its rule instances.

\subsection{Theorem} \label{translation} For any $\Gamma$ and $\Delta$, $\text{GSTL} \vdash \Gamma \Rightarrow \Delta$ if and only if $\text{iSTL} \vdash \Gamma \Rightarrow \Delta$.

\textit{Proof}: I. Suppose $\text{GSTL}$ proves $\Gamma \Rightarrow \Delta$ by a proof tree $\mathbf{D}$. We will construct a proof tree in $\text{iSTL}$ for $\Gamma \Rightarrow \Delta$. By induction on $h(\mathbf{D})$, the induction hypothesis states that for any $\text{GSTL}$ proof tree $\mathbf{D}'$ for $\Gamma' \Rightarrow \Delta'$ such that $h(\mathbf{D}') < h(\mathbf{D})$, there exists an $\text{iSTL}$ proof tree IH$(\mathbf{D'})$ for $\Gamma' \Rightarrow \Delta'$. Now we consider different cases for the last rule of $\mathbf{D}$. In all cases, we denote the immediate sub-trees of $\mathbf{D}$ by $\mathbf{D_i} ~(0 \leq i)$. The rule name in parens is the last rule of $\mathbf{D}$ and in GSTL, but our construction in each case is in iSTL.
\begin{enumerate}
	\item[1,2.] ($Id$),($Ta$) iSTL has these as axioms.
	\setcounter{enumi}{2}

	\item ($Ex$) Using $Cut$ on lemma \ref{lem:i-nabla-n-bot} and (iSTL's) $Ex$, we get $\nabla^n \bot \Rightarrow$.

	\item[4-6] ($Lw$),($Rw$),($Lc$) Just apply the same rule in iSTL on IH$(\mathbf{D_0})$.
	\setcounter{enumi}{6}

	\item ($L\land_1$) IH($\mathbf{D_0}$) is of the form $\Gamma , \nabla^n A \Rightarrow \Delta$. By $L\land_1$ we have $\Gamma , \nabla^n A \land \nabla^n B \Rightarrow \Delta$. By $Cut$ with lemma \ref{lem:i-nabla-dist-and} we have $\Gamma , \nabla^n (A \land B) \Rightarrow \Delta$.
	
	\item ($L\land_2$) The same.
	
	\item ($R\land$) Just apply iSTL's $R\land$ on IH$(\mathbf{D_0})$ and IH$(\mathbf{D_1})$.
	
	\item ($L\lor$) IH($\mathbf{D_0}$) and IH($\mathbf{D_1}$) are of the form $\Gamma , \nabla^n A \Rightarrow \Delta$ and $\Gamma , \nabla^n B \Rightarrow \Delta$ respectively. By $L\lor$ we have $\Gamma , \nabla^n A \lor \nabla^n B \Rightarrow \Delta$. By $Cut$ with lemma \ref{lem:i-nabla-n-dist-or} we have $\Gamma , \nabla^n (A \lor B) \Rightarrow \Delta$.
	
	\item[11,12.] ($R\lor_{1/2}$) Just apply iSTL's $R\lor_{1/2}$ on IH$(\mathbf{D_0})$.
	\setcounter{enumi}{12}
	
	\item ($L\rightarrow$) IH($\mathbf{D_0}$) and IH($\mathbf{D_1}$) are of the form $\Gamma \Rightarrow \nabla^n A$ and $\Gamma , \nabla^n B \Rightarrow \Delta$ respectively. By $L\rightarrow$ we have $\Gamma , \nabla (\nabla^n A \rightarrow \nabla^n B) \Rightarrow \Delta$. We also have $\Gamma , \nabla^{n+1} (A \rightarrow B) \Rightarrow \nabla (\nabla^n A \rightarrow \nabla^n B)$ from $Lw$ and $N$ applied to lemma \ref{lem:i-nabla-dist-imp}.
	
	\item[14,15.] ($R\rightarrow$),($N'$) Use the same rule in iSTL on IH($\mathbf{D_0}$).
	\setcounter{enumi}{15}
	
	\item ($MC$) Apply $N$ on IH($\mathbf{D_0}$) $l$ times, before cutting it with IH($\mathbf{D_1}$).
\end{enumerate}
II. To prove the other direction it suffices to observe that iSTL rules are just specific cases of GSTL ones.

\section{Lemma}\label{true-assum} If $~\small\text{GSTL}^- + \nabla Cut \vdash \mathcal{S} , \p^r \top^n \Rightarrow \Delta$ then $\small\text{GSTL}^- + \nabla Cut \vdash \mathcal{S} \Rightarrow \Delta$ with a proof of at most the same rank.

\textit{Proof:} Suppose $\mathbf{D}$ is a proof tree for $\mathcal{S} , \p^r \top^n \Rightarrow \Delta$. By induction on $h(\mathbf{D})$, induction hypothesis states that for any proof tree $\mathbf{D}'$ with conclusion $\mathcal{S'} , \p^{r'} \top^{n'} \Rightarrow \Delta'$ such that $h(\mathbf{D}') < h(\mathbf{D})$, there is a proof $\mathbf{D}''$ of $\mathcal{S}' \Rightarrow \Delta'$ such that $\rho(\mathbf{D}'') \leq \rho(\mathbf{D}')$.

Consider different cases for the last rule of $\mathbf{D}$, with possible sub-trees $\mathbf{D_0}$ and $\mathbf{D_1}$. $Ta$ and $Ex$ cases are trivially ruled out. In $Id$ case (which implies $n = 1$ and $r = 0$), we have $\Rightarrow \top$ by $Ta$. In $Lw$ case, where an instance of $\top$ is principal and $n = 1$, $\mathbf{D_0}$ itself proves the desired sequent. If $n > 1$, then the induction hypothesis with $n' = n - 1$ gives the desired sequent. $Lc$ on an instance of $\top$ is similar, with $n' = n + 1$. In all other cases, just apply induction hypothesis on $\mathbf{D_0}$ (and possibly $\mathbf{D_1}$), then the same last rule. Notice in all cases we must apply induction hypothesis with $n' = n$ and $r' = r$, except for $R\rightarrow$ and $R\nabla$, in which we apply it with $r' = r + 1$ and $r' = r - 1$ respectively. Also notice that $\nabla Cut$ is not used except in $\nabla Cut$ case, where it is applied with the same cut-formula, so the resulting proof tree is not of a higher rank than $\mathbf{D}$.

\section{Theorem}\label{cut-admis} \emph{Cut Reduction for GSTL: } For any $\mathcal{S}$, $n>0$, $A$, $\mathcal{T}$, $k$, $l$ and $\Delta$, if $~\small\text{GSTL}^- + \nabla Cut \vdash \mathcal{S} \Rightarrow \nabla^k A$ and $\small\text{GSTL}^- + \nabla Cut \vdash$ $\mathcal{T} , \p^{l+k} A^n \Rightarrow \Delta$ with proof trees of ranks less than $\rho(A)$, then
 $\small\text{GSTL}^- + \nabla Cut \vdash \p^l \mathcal{S} , \mathcal{T}$ $\Rightarrow \Delta$ also with a proof tree of a rank less than $\rho(A)$.
 
\emph{Proof:}
Let $\mathbf{D_0}$ and $\mathbf{D_1}$ be proof trees of ranks less than $\rho(A)$, with conclusions $\mathcal{S} \Rightarrow \nabla^k A$ and $\mathcal{T} , \p^{l+k} A^n \Rightarrow \Delta$ and heights $h(\mathbf{D_0})$ and $h(\mathbf{D_1})$ respectively.
\begin{prooftree}
	\noLine
	\AXC{$\mathbf{D_0}$}
	\UIC{$\mathcal{S} \Rightarrow \nabla^k A$}
	
	\noLine
	\AXC{$\mathbf{D_1}$}
	\UIC{$\mathcal{T} , \p^{l+k} A^n \Rightarrow \Delta$}
	
	\dashedLine \RightLabel{$\nabla Cut$}
	\BIC{$\p^l \mathcal{S} , \mathcal{T} \Rightarrow \Delta$}
\end{prooftree}
By induction on $h(\mathbf{D_0}) + h(\mathbf{D_1})$, the induction hypothesis states that for any proof trees $\mathbf{D_0}'$ with conclusion $\mathcal{S}' \Rightarrow \nabla^{k'} A'$ and $\mathbf{D_1}'$ with conclusion $\mathcal{T}' , \p^{l'+k'} A'^{n'} \Rightarrow \Delta'$ such that $h(\mathbf{D_0}') + h(\mathbf{D_1}') < h(\mathbf{D_0}) + h(\mathbf{D_1})$, if $\rho(\mathbf{D_0}'),\rho(\mathbf{D_1}') < \rho(A')$ then there exists a proof tree $\mathbf{D}'$ with conclusion $\p^{l'} \mathcal{S}' , \mathcal{T}' \Rightarrow \Delta'$ such that $\rho(\mathbf{D}') < \rho(A')$.

\begin{table}
	\centering
	\begin{tabular}{|c|p{.2cm}|*{17}{p{.4cm}|}}
		\hline
		\multicolumn{2}{|c|}{\backslashbox[2.3cm]{$\mathbf{D_0}$}{$\mathbf{D_1}$}} & \rot{$Id$} & \rot{$Ta$} & \rot{$Ex$} & \rot{$Lw$} & \rot{$Lc$} & \rot{$\nabla Cut$} & \rot{$L \land_1$} & \rot{$L \land_2$} & \rot{$L \lor$} & \rot{$L \nabla$} & \rot{$L \rightarrow$} & \rot{$Rw$} & \rot{$R \land$} & \rot{$R \lor_1$} & \rot{$R \lor_2$} & \rot{$R \rightarrow$} & \rot{$R \nabla$} \\
		\hline
		\multicolumn{2}{|c|}{$Id$} & \multicolumn{17}{c|}{\caseref{c:id-*}} \\ \hline
		\multicolumn{2}{|c|}{$Ta$} & \multicolumn{17}{c|}{\caseref{c:ta-*}} \\ \hline
		\multicolumn{2}{|c|}{$Ex$} & \multicolumn{17}{c|}{$\times$} \\ \hline
		\multicolumn{2}{|c|}{$Lw$} & \multicolumn{17}{c|}{\caseref{c:lw-*}} \\ \hline
		\multicolumn{2}{|c|}{$Lc$} & \multicolumn{17}{c|}{\caseref{c:lc-*}} \\ \hline
		\multicolumn{2}{|c|}{$\nabla Cut$} & \multicolumn{17}{c|}{\caseref{c:cut-*}} \\ \hline
		\multicolumn{2}{|c|}{$L \land_1$} & \multicolumn{17}{c|}{\caseref{c:la1-*}} \\ \hline
		\multicolumn{2}{|c|}{$L \land_2$} & \multicolumn{17}{c|}{\caseref{c:la2-*}} \\ \hline
		\multicolumn{2}{|c|}{$L \lor$} & \multicolumn{17}{c|}{\caseref{c:lo-*}} \\ \hline
		\multicolumn{2}{|c|}{$L \nabla$} & \multicolumn{17}{c|}{\caseref{c:ln-*}} \\ \hline
		\multicolumn{2}{|c|}{$L \rightarrow$} & \multicolumn{17}{c|}{\caseref{c:li-*}} \\ \hline
		\multicolumn{2}{|c|}{$Rw$} & \multicolumn{17}{c|}{\caseref{c:rw-*}} \\ \hline
		$R \lor_1,$ & \multirow{2}{*}{\tiny Prin.} &
		\multirow{2}{*}{\caseref{c:*-id}} &
		\multirow{2}{*}{$\times$} &
		\multirow{2}{*}{\caseref{c:*-ex}} &
		\multirow{2}{*}{\caseref{c:*-lw-p}} &
		\multirow{2}{*}{\caseref{c:*-lc-p}} &
		\multirow{2}{*}{$\times$} &
		\multirow{2}{*}{\caseref{c:ra-la1}} &
		\multirow{2}{*}{\caseref{c:ra-la2}} &
		\caseref{c:ro1-lo} &
		\multirow{2}{*}{\caseref{c:rn-*}} &
		\multirow{2}{*}{\caseref{c:ri-li}} &
		\multirow{2}{*}{$\times$} &
		\multirow{2}{*}{$\times$} &
		\multirow{2}{*}{$\times$} &
		\multirow{2}{*}{$\times$} &
		\multirow{2}{*}{$\times$} &
		\multirow{2}{*}{$\times$} \\
		$R \lor_2,$ & & & & & & & & & & \caseref{c:ro2-lo} & & & & & & & &\\
		\cline{2-19}
		$R \land,$ & \multirow{3}{*}{\tiny !Prin.} &
		\multirow{3}{*}{$\times$} &
		\multirow{3}{*}{$\times$} &
		\multirow{3}{*}{$\times$} &
		\multirow{3}{*}{\caseref{c:*-lw}} &
		\multirow{3}{*}{\caseref{c:*-lc}} &
		\multirow{3}{*}{\caseref{c:*-cut}} &
		\multirow{3}{*}{\caseref{c:*-la1}} &
		\multirow{3}{*}{\caseref{c:*-la2}} &
		\multirow{3}{*}{\caseref{c:*-lo}} &
		\multirow{3}{*}{\caseref{c:*-ln}} &
		\multirow{3}{*}{\caseref{c:*-li}} &
		\multirow{3}{*}{\caseref{c:*-rw}} &
		\multirow{3}{*}{\caseref{c:*-ra}} &
		\multirow{3}{*}{\caseref{c:*-ro1}} &
		\multirow{3}{*}{\caseref{c:*-ro2}} &
		\multirow{3}{*}{\caseref{c:*-ri}} &
		\multirow{3}{*}{\caseref{c:*-rn}} \\
		$R \rightarrow,$ & & & & & & & & & & & & & & & & & & \\
		$R \nabla$ & & & & & & & & & & & & & & & & & & \\
		\hline
		\multicolumn{2}{|c|}{\slashbox[2.3cm]{$\mathbf{D_0}$}{$\mathbf{D_1}$}} & \rot{$Id$} & \rot{$Ta$} & \rot{$Ex$} & \rot{$Lw$} & \rot{$Lc$} & \rot{$\nabla Cut$} & \rot{$L \land_1$} & \rot{$L \land_2$} & \rot{$L \lor$} & \rot{$L \nabla$} & \rot{$L \rightarrow$} & \rot{$Rw$} & \rot{$R \land$} & \rot{$R \lor_1$} & \rot{$R \lor_2$} & \rot{$R \rightarrow$} & \rot{$R \nabla$} \\
		\hline
	\end{tabular}
\end{table}

We proceed by case analysis on the last rule in $\mathbf{D_0}$. We can either apply the induction hypothesis on a pair of trees shorter than $\mathbf{D_0}$ and $\mathbf{D_1}$, or just cut a formula of a lower rank than $A$, to reach the desired sequent without increasing the cut-rank. This is easy when the cut-formula is also present in the immediate sub-tree(s) of $\mathbf{D_0}$, as it is the case for the left-rules. In these cases, we somehow ``commute'' the last rule of $\mathbf{D_0}$ with an application of induction hypothesis for its premise(s) and $\mathbf{D_1}$. In right-rule cases, where $A$ is principal and not present in the sub-tree(s) of $\mathbf{D_0}$, we consider further cases, this time for the last rule of $\mathbf{D_1}$. We can do the same with $\mathbf{D_1}$ in the cases that no instance of $A$ is principal in its last rule. Almost all other cases force a particular construction for $A$ so we can apply $\nabla Cut$ on its sub-formulas. The only exception is the case where $\mathbf{D_0}$ ends with $R \nabla$. The induction hypothesis is strong enough to handle this one in particular.

\begin{enumerate}
	\item ($Id$) \label{c:id-*} We have $\Gamma = A$ and the desired sequent should be of the form $\nabla^l A , \Sigma \Rightarrow \Delta$. $n-1$ applications of $Lc$ on $\mathbf{D_1}$ proves such sequent.

	\item ($Ta$) \label{c:ta-*} $\mathbf{D_0}$ proves $\epsilon \Rightarrow \top$. $\mathcal{T} \Rightarrow \Delta$ is proved by lemma \ref{true-assum} for $\mathbf{D_1}$.

	\item ($Lw$) \label{c:lw-*} $\mathbf{D_0}$ proves $\mathcal{S} , \p^r B \Rightarrow \nabla^k A$ and has a subproof $\mathbf{D_0}'$ of $\mathcal{S} \Rightarrow \nabla^k A$.
	\begin{prooftree}
		\noLine
		\AXC{$\mathbf{D_0}'$}
		\UIC{$\mathcal{S} \Rightarrow \nabla^k A$}
		\RightLabel{$Lw$}
		\UIC{$\mathcal{S} , \p^r B \Rightarrow \nabla^k A$}

		
		\noLine
		\AXC{$\mathbf{D_1}$}
		\UIC{$\mathcal{T} , \p^{l+k} A^n \Rightarrow \Delta$}

		\dashedLine\RightLabel{$\nabla Cut$}
		\BIC{$\p^l \mathcal{S} , \p^{r+l} B , \mathcal{T} \Rightarrow \Delta$}
	\end{prooftree}
	From the induction hypothesis for $\mathbf{D_0}'$ and $\mathbf{D_1}$, we have a proof of $\p^l \mathcal{S} , \mathcal{T} \Rightarrow \Delta$ with rank lower than that of $A$. By applying $Lw$ we have $\p^l \mathcal{S} , \Gamma , \p^{r+l} B , \mathcal{T} \Rightarrow \Delta$, without increasing the cut-rank.
	\begin{prooftree}
		\noLine
		\AXC{$\mathbf{D_0}'$}
		\UIC{$\mathcal{S} \Rightarrow \nabla^k A$}
		
		\noLine
		\AXC{$\mathbf{D_1}$}
		\UIC{$\mathcal{T} , \p^{l+k} A^n \Rightarrow \Delta$}
		
		\RightLabel{IH}
		\BIC{$\p^l \mathcal{S} , \mathcal{T} \Rightarrow \Delta$}
		
		\RightLabel{$Lw$}
		\UIC{ $\p^l \mathcal{S} , \p^{r+l} B , \mathcal{T} \Rightarrow \Delta$}
	\end{prooftree}

	\item ($Lc$) \label{c:lc-*} $\mathbf{D_0}$ proves $\mathcal{S} , \p^r B \Rightarrow \nabla^k A$ and has a subproof $\mathbf{D_0}'$ of $\mathcal{S} , \p^r B , \p^r B \Rightarrow \nabla^k A$.
\begin{prooftree}
	\noLine
	\AXC{$\mathbf{D_0}'$}
	\UIC{$\mathcal{S} , \p^r B , \p^r B \Rightarrow \nabla^k A$}
	\RightLabel{$Lc$}
	\UIC{$\mathcal{S} , \p^r B \Rightarrow \nabla^k A$}
	
	
	\noLine
	\AXC{$\mathbf{D_1}$}
	\UIC{$\mathcal{T} , \p^{l + k} A^n \Rightarrow \Delta$}
	
	\dashedLine\RightLabel{$\nabla Cut$}
	\BIC{$\p^l \mathcal{S} , \p^{r+l} B , \mathcal{T} \Rightarrow \Delta$}
\end{prooftree}
From the induction hypothesis for $\mathbf{D_0}'$ and $\mathbf{D_1}$, we have a proof of $\p^l \mathcal{S} , \p^{r+l} B , \p^{r+l} B , \mathcal{T} \Rightarrow \Delta$ with rank lower than that of $A$. By applying $Lc$ we have $\p^l \mathcal{S} , \p^{r+l} B , \mathcal{T} \Rightarrow \Delta$, without increasing the cut-rank.
\begin{prooftree}
	\noLine
	\AXC{$\mathbf{D_0}'$}
	\UIC{$\mathcal{S} , \p^r B , \p^r B \Rightarrow \nabla^k A$}
	
	\noLine
	\AXC{$\mathbf{D_1}$}
	\UIC{$\mathcal{T} , \p^{l+k} A^n \Rightarrow \Delta$}
	
	\RightLabel{IH}
	\BIC{$\p^l \mathcal{S} , \p^{r+l} B , \p^{r+l} B , \mathcal{T} \Rightarrow \Delta$}
	
	\RightLabel{$Lc$}
	\UIC{$\p^l \mathcal{S} , \p^{r+l} B , \mathcal{T} \Rightarrow \Delta$}
\end{prooftree}

	\item \label{c:cut-*} ($MC$) Assume $\mathbf{D_0}$ ends with a $MC$ on $A'$, which by assumption must be of a lower rank than $A$.
	\begin{prooftree}
		\noLine
		\AXC{$\mathbf{D_0}'$}
		\UIC{$\Gamma \Rightarrow A'$}
		
		\noLine
		\AXC{$\mathbf{D_0}''$}
		\UIC{$\Pi , (\nabla^{l'} A')^{n'} \Rightarrow A$}
		
		\RightLabel{$MC$}
		\BIC{$\nabla^{l'} \Gamma , \Pi \Rightarrow A$}
		
		
		\noLine
		\AXC{$\mathbf{D_1}$}
		\UIC{$\Sigma , (\nabla^l A)^n \Rightarrow \Delta$}
		
		\dashedLine\RightLabel{$MC$}
		\BIC{$\nabla^{l'+l} \Gamma , \nabla^l \Pi , \Sigma \Rightarrow \Delta$}
	\end{prooftree}
	We can use the induction hypothesis to remove $A$ first, then cut $A'$.
	\begin{prooftree}
		\noLine
		\AXC{$\mathbf{D_0}'$}
		\UIC{$\Gamma \Rightarrow A'$}
		
		\noLine
		\AXC{$\mathbf{D_0}''$}
		\UIC{$\Pi , (\nabla^{l'} A')^{n'} \Rightarrow A$}

		\noLine
		\AXC{$\mathbf{D_1}$}
		\UIC{$\Sigma , (\nabla^l A)^n \Rightarrow \Delta$}

		\RightLabel{IH}
		\BIC{$\nabla^l \Pi , (\nabla^{l'+l} A')^{n'} , \Sigma \Rightarrow \Delta$}
		

		\RightLabel{$MC$}
		\BIC{$\nabla^{l'+l} \Gamma , \nabla^l \Pi , \Sigma \Rightarrow \Delta$}
	\end{prooftree}

	\item \label{c:la1-*} ($L\land_1$) $\mathbf{D_0}$ proves $\Gamma , \nabla^r (B \land C) \Rightarrow A$ and has a subproof $\mathbf{D_0}'$ of $\Gamma , \nabla^r B \Rightarrow A$.
	\begin{prooftree}
		\noLine
		\AXC{$\mathbf{D_0}'$}
		\UIC{$\Gamma , \nabla^r B \Rightarrow A$}
		\UIC{$\Gamma , \nabla^r (B \land C) \Rightarrow A$}
		
		
		\noLine
		\AXC{$\mathbf{D_1}$}
		\UIC{$\Sigma , (\nabla^l A)^n \Rightarrow \Delta$}
		
		\dashedLine\RightLabel{$MC$}
		\BIC{$\nabla^l \Gamma , \nabla^{r+l} B \land C , \Sigma \Rightarrow \Delta$}
	\end{prooftree}
	From the induction hypothesis for $\mathbf{D_0}'$ and $\mathbf{D_1}$, we have a proof of $\nabla^l \Gamma , \nabla^{r+l} B , \Sigma \Rightarrow \Delta$ with rank lower than that of $A$. By applying $L\land$ we have $\nabla^l \Gamma , \nabla^{r+l} (B \land C) , \Sigma \Rightarrow \Delta$, without increasing the cut-rank.
	\begin{prooftree}
		\noLine
		\AXC{$\mathbf{D_0}'$}
		\UIC{$\Gamma , \nabla^r B \Rightarrow A$}
		
		\noLine
		\AXC{$\mathbf{D_1}$}
		\UIC{$\Sigma , (\nabla^l A)^n \Rightarrow \Delta$}
		
		\RightLabel{IH}
		\BIC{$\nabla^l \Gamma , \nabla^{r+l} B , \Sigma \Rightarrow \Delta$}
		
		\RightLabel{$L\land$}
		\UIC{$\nabla^l \Gamma , \nabla^{r+l} (B \land C) , \Sigma \Rightarrow \Delta$}
	\end{prooftree}

	\item \label{c:la2-*} ($L\land_2$) The other $L\land$ is the same.
	
	\item \label{c:lo-*} ($L\lor$)  $\mathbf{D_0}$ proves $\Gamma , \nabla^r (B \lor C) \Rightarrow A$ and has subproofs $\mathbf{D_0}'$ of $\Gamma , \nabla^r B \Rightarrow A$ and $\mathbf{D_0}''$ of $\Gamma , \nabla^r C \Rightarrow A$.
	\begin{prooftree}
		\noLine
		\AXC{$\mathbf{D_0}'$}
		\UIC{$\Gamma , \nabla^r B \Rightarrow A$}
		\noLine
		\AXC{$\mathbf{D_0}''$}
		\UIC{$\Gamma , \nabla^r C \Rightarrow A$}
		\RightLabel{$L\lor$}
		\BIC{$\Gamma , \nabla^r (B \lor C) \Rightarrow A$}
		
		
		\noLine
		\AXC{$\mathbf{D_1}$}
		\UIC{$\Sigma , (\nabla^l A)^n \Rightarrow \Delta$}
		
		\dashedLine\RightLabel{$MC$}
		\BIC{$\nabla^l \Gamma , \nabla^{r+l} (B \lor C) , \Sigma \Rightarrow \Delta$}
	\end{prooftree}
	From the induction hypothesis for $\mathbf{D_0}'$ and $\mathbf{D_1}$, we have a proof of $\nabla^l \Gamma , \nabla^{r+l} B , \Sigma \Rightarrow \Delta$ with rank lower than that of $A$. Also from the induction hypothesis for $\mathbf{D_0}''$ and $\mathbf{D_1}$, we have a proof of $\nabla^l \Gamma , \nabla^{r+l} C , \Sigma \Rightarrow \Delta$ with rank lower than that of $A$. By applying $L\lor$ we have $\nabla^l \Gamma , \nabla^{r+l} (B \lor C) , \Sigma \Rightarrow \Delta$, without increasing the cut-rank.
	\begin{prooftree}
		\noLine
		\AXC{$\mathbf{D_0}'$}
		\UIC{$\Gamma , \nabla^r B \Rightarrow A$}
		
		\noLine
		\AXC{$\mathbf{D_1}$}
		\UIC{$\Sigma , (\nabla^l A)^n \Rightarrow \Delta$}
		
		\RightLabel{IH}
		\BIC{$\nabla^l \Gamma , \nabla^{r+l} B , \Sigma \Rightarrow \Delta$}
		
		\noLine
		\AXC{$\mathbf{D_0}''$}
		\UIC{$\Gamma , \nabla^r C \Rightarrow A$}
		
		\noLine
		\AXC{$\mathbf{D_1}$}
		\UIC{$\Sigma , (\nabla^l A)^n \Rightarrow \Delta$}
		
		\RightLabel{IH}
		\BIC{$\nabla^l \Gamma , \nabla^{r+l} C , \Sigma \Rightarrow \Delta$}
		
		\RightLabel{$L\lor$}
		\BIC{$\nabla^l \Gamma , \nabla^{r+l} (B \lor C) , \Sigma \Rightarrow \Delta$}
	\end{prooftree}
	
	\item \label{c:ln-*} ($L\nabla$) $\mathbf{D_0}$ proves $\mathcal{S} | \Gamma | \Sigma , \nabla B | \mathcal{R} \Rightarrow \nabla^k A$ and has a subproof $\mathbf{D_0}'$ of $\mathcal{S} | \Gamma , B | \Sigma | \mathcal{R} \Rightarrow \nabla^k A$.
	\begin{prooftree}
		\noLine
		\AXC{$\mathbf{D_0}'$}
		\UIC{$\mathcal{S} | \Gamma , B | \Sigma | \mathcal{R} \Rightarrow \nabla^k A$}
		\RightLabel{$L\nabla$}
		\UIC{$\mathcal{S} | \Gamma | \Sigma , \nabla B | \mathcal{R} \Rightarrow \nabla^k A$}

		\noLine
		\AXC{$\mathbf{D_1}$}
		\UIC{$\mathcal{T} \cup [A^n | \epsilon_{l+k}] \Rightarrow \Delta$}

		\dashedLine\RightLabel{$\nabla Cut$}
		\BIC{$[\mathcal{S} | \Gamma | \Sigma , \nabla B | \mathcal{R} | \epsilon_l] \cup \mathcal{T} \Rightarrow \Delta$}
	\end{prooftree}
	From the induction hypothesis for $\mathbf{D_0}'$ and $\mathbf{D_1}$, we have a proof of $[\mathcal{S} | \Gamma , B | \Sigma | \mathcal{R} | \epsilon_l] \cup \mathcal{T} \Rightarrow \Delta$ with rank lower than that of $A$. By applying $L\nabla$ in the proper place, we have $[\mathcal{S} | \Gamma | \Sigma , \nabla B | \mathcal{R} | \epsilon_l] \cup \mathcal{T} \Rightarrow \Delta$, without increasing the cut-rank.
	\begin{prooftree}
		\noLine
		\AXC{$\mathbf{D_0}'$}
		\UIC{$\mathcal{S} | \Gamma , B | \Sigma | \mathcal{R} \Rightarrow \nabla^k A$}
		
		\noLine
		\AXC{$\mathbf{D_1}$}
		\UIC{$\mathcal{T} \cup [A^n | \epsilon_{l+k}] \Rightarrow \Delta$}
		
		\RightLabel{IH}
		\BIC{$[\mathcal{S} | \Gamma , B | \Sigma | \mathcal{R} | \epsilon_l] \cup \mathcal{T} \Rightarrow \Delta$}
		
		\RightLabel{$L\nabla$}
		\UIC{$[\mathcal{S} | \Gamma | \Sigma , \nabla B | \mathcal{R} | \epsilon_l] \cup \mathcal{T} \Rightarrow \Delta$}
	\end{prooftree}
	
	\item \label{c:li-*} ($L\rightarrow$) $\mathbf{D_0}$ proves $\Gamma , \nabla^{r+1} (B \rightarrow C) \Rightarrow A$ and has subproofs $\mathbf{D_0}'$ of $\Gamma \Rightarrow \nabla^r B$ and $\mathbf{D_0}''$ of $\Gamma , \nabla^r C \Rightarrow A$.
	\begin{prooftree}
		\noLine
		\AXC{$\mathbf{D_0}'$}
		\UIC{$\Gamma \Rightarrow \nabla^r B$}
		\noLine
		\AXC{$\mathbf{D_0}''$}
		\UIC{$\Gamma , \nabla^r C \Rightarrow A$}
		\RightLabel{$L\nabla$}
		\BIC{$\Gamma , \nabla^{r+1} (B \rightarrow C) \Rightarrow A$}
		
		
		\noLine
		\AXC{$\mathbf{D_1}$}
		\UIC{$\Sigma , (\nabla^l A)^n \Rightarrow \Delta$}
		
		\dashedLine\RightLabel{$MC$}
		\BIC{$\nabla^l \Gamma , \nabla^{r+l+1} (B \rightarrow C) , \Sigma \Rightarrow \Delta$}
	\end{prooftree}
	From the induction hypothesis for $\mathbf{D_0}''$ and $\mathbf{D_1}$, we have a proof of
	$\nabla^l \Gamma , \nabla^{r+l} C , \Sigma \Rightarrow \Delta$ with rank lower than that of $A$. Now in order to $L\rightarrow$ to be applicable on the resulting sequent and $\mathbf{D_0}'$, we must first identify their contexts. By $l$ times applications of $N'$ and then $Lw$ with $\Sigma$ on $\mathbf{D_0}'$, we have $\nabla^l \Gamma , \Sigma \Rightarrow \nabla^{r+l} B$. Then we can derive $\nabla^l \Gamma , \nabla^{r+l+1} (B \rightarrow C) , \Sigma \Rightarrow \Delta$, from $L\rightarrow$.
	\begin{prooftree}
		\noLine
		\AXC{$\mathbf{D_0}'$}
		\UIC{$\Gamma \Rightarrow \nabla^r B$}
		\doubleLine \RightLabel{$N'$}
		\UIC{$\nabla^l \Gamma \Rightarrow \nabla^{r+l} B$}
		\doubleLine \RightLabel{$Lw$}
		\UIC{$\nabla^l \Gamma , \Sigma \Rightarrow \nabla^{r+l} B$}
		
		\noLine
		\AXC{$\mathbf{D_0}''$}
		\UIC{$\Gamma , \nabla^r C \Rightarrow A$}
		\noLine
		\AXC{$\mathbf{D_1}$}
		\UIC{$\Sigma , (\nabla^l A)^n \Rightarrow \Delta$}
		\RightLabel{IH}
		\BIC{$\nabla^l \Gamma , \nabla^{r+l} C , \Sigma \Rightarrow \Delta$}
		
		\RightLabel{$L\rightarrow$}
		\BIC{$\nabla^l \Gamma , \nabla^{r+l+1} (B \rightarrow C) , \Sigma \Rightarrow \Delta$}
	\end{prooftree}
	
	\item \label{c:rw-*} ($Rw$) $\mathbf{D_0}$ proves $\mathcal{S} \Rightarrow \nabla^k A$ and has subproofs $\mathbf{D_0}'$ of $\mathcal{S} \Rightarrow$.
	\begin{prooftree}
		\noLine
		\AXC{$\mathbf{D_0}'$}
		\UIC{$\mathcal{S} \Rightarrow$}
		\RightLabel{$Rw$}
		\UIC{$\mathcal{S} \Rightarrow \nabla^k A$}
		
		\noLine
		\AXC{$\mathbf{D_1}$}
		\UIC{$\mathcal{T} \cup [A^n | \epsilon_{l+k}] \Rightarrow \Delta$}
		
		\dashedLine \RightLabel{$\nabla Cut$}
		\BIC{$[\mathcal{S} | \epsilon_l] \cup \mathcal{T} \Rightarrow \Delta$}
	\end{prooftree}
	$l$ times applications of $R\nabla$ on $\mathbf{D_0}'$ followed by proper $Lw$ and $Rw$ yields the desired sequent.
	\begin{prooftree}
		\noLine
		\AXC{$\mathbf{D_0}'$}
		\UIC{$\mathcal{S} \Rightarrow$}
		\doubleLine \RightLabel{$R\nabla$}
		\UIC{$\mathcal{S} | \epsilon_l \Rightarrow$}
		\RightLabel{$Lw$}
		\UIC{$[\mathcal{S} | \epsilon_l] \cup \mathcal{T} \Rightarrow$}
		\RightLabel{$Rw$}
		\UIC{$[\mathcal{S} | \epsilon_l] \cup \mathcal{T} \Rightarrow \Delta$}
	\end{prooftree}


	\item[13-17.] ($R*$) In the cases that the last rule in $\mathbf{D_0}$ is any of the logical right-rules $R*$ ($* \in \{ \land, \lor_{i}, \rightarrow, \nabla \}$), its principal formula would be $\nabla^k A$. The only admissible $R*$ with $k > 0$ would be $R\nabla$. For the other rules, this forces $k = 0$, plus a particular construction for $A$ based on the last rule. In each case, we will proceed by case analysis on the last rule of $\mathbf{D_1}$, but to avoid repeating similar cases, we will separate cases by whether the last rule of $\mathbf{D_1}$ is logical, and if $A$ is also the principal formula there or not. So we have the following groups of cases:
	(I) Cases in which $A$ is not principal in $\mathbf{D_1}$,
	(II) cases in which the last rule of $\mathbf{D_1}$ is any of the logical left-rules and (an instance of) $A$ is its principal formula, and (III) $A$ is principal in $\mathbf{D_1}$, but the rule is either structural or an axiom. Notice that the solution for the different cases of $\mathbf{D_1}$ in (I) and (III) is shared by all cases of $\mathbf{D_0}$, since it does not depend on the last rule of $\mathbf{D_0}$. This may seem like we have commuted the case analysis on $\mathbf{D_1}$ with the one on $\mathbf{D_0}$, which is not true. So each item in (I) or (III) handles many different cases with similar solutions.
	On the other hand, each case of $\mathbf{D_0}$ in (II) determines a single case for $\mathbf{D_1}$. The parens before each case show the last rule of $\mathbf{D_0}$ and $\mathbf{D_1}$ respectively.

	\paragraph{} I. If $A$ is not principal in the last rule of $\mathbf{D_1}$.
	
	\begin{enumerate}[label={\alph*.}]
		\item \label{c:*-lw} ($R*, Lw$)
		\begin{prooftree}
			\noLine
			\AXC{$\mathbf{D_0}$}
			\UIC{$\mathcal{S} \Rightarrow \nabla^k A$}
			
			\noLine
			\AXC{$\mathbf{D_1}'$}
			\UIC{$\mathcal{T} , \p^{l+k} A^n \Rightarrow \Delta$}
			\RightLabel{$Lw$}
			\UIC{$\mathcal{T} , \p^{l+k} A^n , \p^r B \Rightarrow \Delta$}
			
			\dashedLine\RightLabel{$\nabla Cut$}
			\BIC{$\p^l \mathcal{S} , \mathcal{T} , \p^r B \Rightarrow \Delta$}
		\end{prooftree}
		By induction hypothesis for $\mathbf{D_0}$ and $\mathbf{D_1}'$ we have
		\begin{prooftree}
			\noLine
			\AXC{$\mathbf{D_0}$}
			\UIC{$\mathcal{S} \Rightarrow \nabla^k A$}
			
			\noLine
			\AXC{$\mathbf{D_1}'$}
			\UIC{$\mathcal{T} , \p^{l+k} A^n \Rightarrow \Delta$}
			\RightLabel{IH}
			\BIC{$\p^l \mathcal{S} , \mathcal{T} \Rightarrow \Delta$}
			
			\RightLabel{$Lw$}
			\UIC{$\p^l \mathcal{S} , \mathcal{T} , \p^r B \Rightarrow \Delta$}
		\end{prooftree}
		
		\item \label{c:*-rw} ($R*, Rw$)
		\begin{prooftree}
			\noLine
			\AXC{$\mathbf{D_0}$}
			\UIC{$\mathcal{S} \Rightarrow \nabla^k A$}
			
			\noLine
			\AXC{$\mathbf{D_1}'$}
			\UIC{$\mathcal{T} , \p^{l+k} A^n \Rightarrow$}
			\RightLabel{$Rw$}
			\UIC{$\mathcal{T} , \p^{l+k} A^n \Rightarrow B$}
			
			\dashedLine\RightLabel{$\nabla Cut$}
			\BIC{$\p^l \mathcal{S} , \mathcal{T} \Rightarrow B$}
		\end{prooftree}
		By induction hypothesis for $\mathbf{D_0}$ and $\mathbf{D_1}'$ we have
		\begin{prooftree}
			\noLine
			\AXC{$\mathbf{D_0}$}
			\UIC{$\mathcal{S} \Rightarrow \nabla^k A$}
			
			\noLine
			\AXC{$\mathbf{D_1}'$}
			\UIC{$\mathcal{T} , \p^{l+k} A^n \Rightarrow$}
			\RightLabel{IH}
			\BIC{$\p^l \mathcal{S} , \mathcal{T} \Rightarrow$}
			
			\RightLabel{$Rw$}
			\UIC{$\p^l \mathcal{S} , \mathcal{T} \Rightarrow B$}
		\end{prooftree}
		
		\item \label{c:*-lc} ($R*, Lc$)
		\begin{prooftree}
			\noLine
			\AXC{$\mathbf{D_0}$}
			\UIC{$\Gamma \Rightarrow A$}

			\noLine
			\AXC{$\mathbf{D_1}'$}
			\UIC{$\Sigma , (\nabla^l A)^n , B , B \Rightarrow \Delta$}
			\RightLabel{$Lc$}
			\UIC{$\Sigma , (\nabla^l A)^n , B \Rightarrow \Delta$}

			\dashedLine\RightLabel{$MC$}
			\BIC{$\nabla^l \Gamma , \Sigma , B \Rightarrow \Delta$}
		\end{prooftree}
		By induction hypothesis for $\mathbf{D_0}$ and $\mathbf{D_1}'$ we have
		\begin{prooftree}
			\noLine
			\AXC{$\mathbf{D_0}$}
			\UIC{$\Gamma \Rightarrow A$}

			\noLine
			\AXC{$\mathbf{D_1}'$}
			\UIC{$\Sigma , (\nabla^l A)^n , B , B \Rightarrow \Delta$}
			\RightLabel{IH}
			\BIC{$\nabla^l \Gamma , \Sigma , B , B \Rightarrow \Delta$}

			\RightLabel{$Lc$}
			\UIC{$\nabla^l \Gamma , \Sigma , B \Rightarrow \Delta$}
		\end{prooftree}

		\item \label{c:*-cut} ($R*, \nabla Cut$) Assume $\mathbf{D_1}$ ends with a $\nabla Cut$ on $A'$, which by assumption must be of a lower rank than $A$.
	\begin{prooftree}
		\noLine
		\AXC{$\mathbf{D_0}$}
		\UIC{$\mathcal{S} \Rightarrow \nabla^k A$}

		\noLine
		\AXC{$\mathbf{D_1}'$}
		\UIC{$\mathcal{T} , \p^{l+k} A^n \Rightarrow \nabla^{k'} A'$}

		\noLine
		\AXC{$\mathbf{D_1}''$}
		\UIC{$\mathcal{T}' , \p^{l'+k'} A'^{n'} \Rightarrow \Delta$}

		\RightLabel{$\nabla Cut$}
		\BIC{$\p^{l'} \mathcal{T} , \p^{l+k+l'} A^n , \mathcal{T}' \Rightarrow \Delta$}

		\dashedLine\RightLabel{$\nabla Cut$}
		\BIC{$\p^{l+l'} \mathcal{S} , \p^{l'} \mathcal{T} , \mathcal{T}' \Rightarrow \Delta$}
	\end{prooftree}
	We can use the induction hypothesis to remove $A$ first, then cut $A'$.
	\begin{prooftree}
		\noLine
		\AXC{$\mathbf{D_0}$}
		\UIC{$\mathcal{S} \Rightarrow \nabla^k A$}
		
		\noLine
		\AXC{$\mathbf{D_1}'$}
		\UIC{$\mathcal{T} , \p^{l+k} A^n \Rightarrow \nabla^{k'} A'$}

		\RightLabel{IH}
		\BIC{$\p^l \mathcal{S} , \mathcal{T} \Rightarrow \nabla^{k'} A'$}
		
		\noLine
		\AXC{$\mathbf{D_1}''$}
		\UIC{$\mathcal{T}' , \p^{l'+k'} A'^{n'} \Rightarrow \Delta$}
		
		\RightLabel{$\nabla Cut$}
		\BIC{$\p^{l+l'} \mathcal{S} , \p^{l'} \mathcal{T} , \mathcal{T}' \Rightarrow \Delta$}
	\end{prooftree}


		\item \label{c:*-la1} ($R*, L\land_1$)
		\begin{prooftree}
			\noLine
			\AXC{$\mathbf{D_0}$}
			\UIC{$\mathcal{S} \Rightarrow \nabla^k A$}
			
			\noLine
			\AXC{$\mathbf{D_1}'$}
			\UIC{$\mathcal{T} [A^n | \epsilon_{l+k}] \cup [B | \epsilon_r] \Rightarrow \Delta$}
			\RightLabel{$L\land_1$}
			\UIC{$\mathcal{T} \cup [A^n | \epsilon_{l+k}] \cup [B \land C | \epsilon_r] \Rightarrow \Delta$}
			
			\dashedLine\RightLabel{$\nabla Cut$}
			\BIC{$[\mathcal{S} | \epsilon_l] \cup \mathcal{T} \cup [B \land C | \epsilon_r] \Rightarrow \Delta$}
		\end{prooftree}
		By induction hypothesis for $\mathbf{D_0}$ and $\mathbf{D_1}'$ we have
		\begin{prooftree}
			\noLine
			\AXC{$\mathbf{D_0}$}
			\UIC{$\mathcal{S} \Rightarrow \nabla^k A$}
			
			\noLine
			\AXC{$\mathbf{D_1}'$}
			\UIC{$\mathcal{T} [A^n | \epsilon_{l+k}] \cup [B | \epsilon_r] \Rightarrow \Delta$}
			\RightLabel{IH}
			\BIC{$[\mathcal{S} | \epsilon_l] \cup \mathcal{T} \cup [B | \epsilon_r] \Rightarrow \Delta$}
			
			\RightLabel{$L\land_1$}
			\UIC{$[\mathcal{S} | \epsilon_l] \cup \mathcal{T} \cup [B \land C | \epsilon_r] \Rightarrow \Delta$}
		\end{prooftree}
		
		\item \label{c:*-la2} ($R*, L\land_2$)
\begin{prooftree}
	\noLine
	\AXC{$\mathbf{D_0}$}
	\UIC{$\Gamma \Rightarrow A$}

	\noLine
	\AXC{$\mathbf{D_1}'$}
	\UIC{$\Sigma , (\nabla^l A)^n , \nabla^r C \Rightarrow \Delta$}
	\RightLabel{$L\land_2$}
	\UIC{$\Sigma , (\nabla^l A)^n , \nabla^r (B \land C) \Rightarrow \Delta$}

	\dashedLine\RightLabel{$MC$}
	\BIC{$\nabla^l \Gamma , \Sigma , \nabla^r (B \land C) \Rightarrow \Delta$}
\end{prooftree}
By induction hypothesis for $\mathbf{D_0}$ and $\mathbf{D_1}'$ we have
\begin{prooftree}
	\noLine
	\AXC{$\mathbf{D_0}$}
	\UIC{$\Gamma \Rightarrow A$}

	\noLine
	\AXC{$\mathbf{D_1}'$}
	\UIC{$\Sigma , (\nabla^l A)^n , \nabla^r C \Rightarrow \Delta$}
	\RightLabel{IH}
	\BIC{$\nabla^l \Gamma , \Sigma , \nabla^r C \Rightarrow \Delta$}

	\RightLabel{$L\land_2$}
	\UIC{$\nabla^l \Gamma , \Sigma , \nabla^r (B \land C) \Rightarrow \Delta$}
\end{prooftree}
		
		\item \label{c:*-ra} ($R*, R\land$)
		\begin{prooftree}
			\noLine
			\AXC{$\mathbf{D_0}$}
			\UIC{$\mathcal{S} \Rightarrow \nabla^k A$}
			
			\noLine
			\AXC{$\mathbf{D_1}'$}
			\UIC{$\mathcal{T} \cup [A^n | \epsilon_{l+k}] \Rightarrow B$}
			\noLine
			\AXC{$\mathbf{D_1}''$}
			\UIC{$\mathcal{T} \cup [A^n | \epsilon_{l+k}] \Rightarrow C$}
			\RightLabel{$R\land$}
			\BIC{$\mathcal{T} \cup [A^n | \epsilon_{l+k}] \Rightarrow B \land C$}
			
			\dashedLine\RightLabel{$\nabla Cut$}
			\BIC{$[\mathcal{S} | \epsilon_l] \cup \mathcal{T} \Rightarrow B \land C$}
		\end{prooftree}
		By induction hypothesis, once for $\mathbf{D_0}$ and $\mathbf{D_1}'$ and again for $\mathbf{D_0}$ and $\mathbf{D_1}''$
		\begin{prooftree}
			\noLine
			\AXC{$\mathbf{D_0}$}
			\UIC{$\mathcal{S} \Rightarrow \nabla^k A$}
			
			\noLine
			\AXC{$\mathbf{D_1}'$}
			\UIC{$\mathcal{T} \cup [A^n | \epsilon_{l+k}] \Rightarrow B$}
			
			\RightLabel{IH}
			\BIC{$[\mathcal{S} | \epsilon_l] \cup \mathcal{T} \Rightarrow B$}
			
			\noLine
			\AXC{$\mathbf{D_0}$}
			\UIC{$\mathcal{S} \Rightarrow A$}
			
			\noLine
			\AXC{$\mathbf{D_1}''$}
			\UIC{$\mathcal{T} \cup [A^n | \epsilon_{l+k}] \Rightarrow C$}
			
			\RightLabel{IH}
			\BIC{$[\mathcal{S} | \epsilon_l] \cup \mathcal{T} \Rightarrow C$}
			
			\RightLabel{$R\land$}
			\BIC{$[\mathcal{S} | \epsilon_l] \cup \mathcal{T} \Rightarrow B \land C$}
		\end{prooftree}
		
		\item \label{c:*-lo} ($R*, L\lor$)
		\begin{prooftree}
			\noLine
			\AXC{$\mathbf{D_0}$}
			\UIC{$\mathcal{S} \Rightarrow \nabla^k A$}
			
			\noLine
			\AXC{$\mathbf{D_1}'$}
			\UIC{$\mathcal{T} , \p^{l+k} A^n , \p^r B \Rightarrow \Delta$}
			\noLine
			\AXC{$\mathbf{D_1}''$}
			\UIC{$\mathcal{T} , \p^{l+k} A^n , \p^r C \Rightarrow \Delta$}
			\RightLabel{$L\lor$}
			\BIC{$\mathcal{T} , \p^{l+k} A^n , \p^r B \lor C \Rightarrow \Delta$}
			
			\dashedLine\RightLabel{$\nabla Cut$}
			\BIC{$\p^l \mathcal{S} ,  \mathcal{T} , \p^r B \lor C \Rightarrow \Delta$}
		\end{prooftree}
		By induction hypothesis, once for $\mathbf{D_0}$ and $\mathbf{D_1}'$ and again for $\mathbf{D_0}$ and $\mathbf{D_1}''$
		\begin{prooftree}
			\noLine
			\AXC{$\mathbf{D_0}$}
			\UIC{$\mathcal{S} \Rightarrow \nabla^k A$}
			
			\noLine
			\AXC{$\mathbf{D_1}'$}
			\UIC{$\mathcal{T} , \p^{l+k} A^n , \p^r B \Rightarrow \Delta$}
			
			\RightLabel{IH}
			\BIC{$\p^l \mathcal{S} , \mathcal{T} , \p^r B \Rightarrow \Delta$}
			
			\noLine
			\AXC{$\mathbf{D_0}$}
			\UIC{$\mathcal{S} \Rightarrow \nabla^k A$}
			
			\noLine
			\AXC{$\mathbf{D_1}''$}
			\UIC{$\mathcal{T} , \p^{l+k} A^n , \p^r C \Rightarrow \Delta$}
			
			\RightLabel{IH}
			\BIC{$\p^l \mathcal{S} , \mathcal{T} , \p^r C \Rightarrow \Delta$}
			
			\RightLabel{$L\lor$}
			\BIC{$\p^l \mathcal{S} , \mathcal{T} , \p^r B \lor C \Rightarrow \Delta$}
		\end{prooftree}
		
		\item \label{c:*-ro1} ($R*, R\lor_1$)
		\begin{prooftree}
			\noLine
			\AXC{$\mathbf{D_0}$}
			\UIC{$\Gamma \Rightarrow A$}

			\noLine
			\AXC{$\mathbf{D_1}'$}
			\UIC{$\Sigma , (\nabla^l A)^n \Rightarrow B$}
			\RightLabel{$R\lor_1$}
			\UIC{$\Sigma , (\nabla^l A)^n \Rightarrow B \lor C$}

			\dashedLine\RightLabel{$MC$}
			\BIC{$\nabla^l \Gamma , \Sigma \Rightarrow B \lor C$}
		\end{prooftree}
		By induction hypothesis for $\mathbf{D_0}$ and $\mathbf{D_1}'$
		\begin{prooftree}
			\noLine
			\AXC{$\mathbf{D_0}$}
			\UIC{$\Gamma \Rightarrow A$}

			\noLine
			\AXC{$\mathbf{D_1}'$}
			\UIC{$\Sigma , (\nabla^l A)^n \Rightarrow B$}
			\RightLabel{IH}
			\BIC{$\nabla^l \Gamma , \Sigma \Rightarrow B$}

			\RightLabel{$R\lor_1$}
			\UIC{$\nabla^l \Gamma , \Sigma \Rightarrow B \lor C$}
		\end{prooftree}
		
		\item \label{c:*-ro2} ($R*, R\lor_2$)
\begin{prooftree}
	\noLine
	\AXC{$\mathbf{D_0}$}
	\UIC{$\Gamma \Rightarrow A$}

	\noLine
	\AXC{$\mathbf{D_1}'$}
	\UIC{$\Sigma , (\nabla^l A)^n \Rightarrow C$}
	\RightLabel{$R\lor_2$}
	\UIC{$\Sigma , (\nabla^l A)^n \Rightarrow B \lor C$}

	\dashedLine\RightLabel{$MC$}
	\BIC{$\nabla^l \Gamma , \Sigma \Rightarrow B \lor C$}
\end{prooftree}
By induction hypothesis for $\mathbf{D_0}$ and $\mathbf{D_1}'$
\begin{prooftree}
	\noLine
	\AXC{$\mathbf{D_0}$}
	\UIC{$\Gamma \Rightarrow A$}

	\noLine
	\AXC{$\mathbf{D_1}'$}
	\UIC{$\Sigma , (\nabla^l A)^n \Rightarrow C$}
	\RightLabel{IH}
	\BIC{$\nabla^l \Gamma , \Sigma \Rightarrow C$}

	\RightLabel{$R\lor_2$}
	\UIC{$\nabla^l \Gamma , \Sigma \Rightarrow B \lor C$}
\end{prooftree}
		
		\item \label{c:*-li} ($R*, L\rightarrow$)
		\begin{prooftree}
			\noLine
			\AXC{$\mathbf{D_0}$}
			\UIC{$\mathcal{S} \Rightarrow \nabla^k A$}
			
			
			\noLine
			\AXC{$\mathbf{D_1}'$}
			\UIC{$\mathcal{T} , \p^{l+k} A^n \Rightarrow \nabla^r B$}
			
			\noLine
			\AXC{$\mathbf{D_1}''$}
			\UIC{$\mathcal{T} , \p^{l+k} A^n , \p^r C \Rightarrow \Delta$}
			
			\RightLabel{$L\rightarrow$}
			\BIC{$\mathcal{T} , \p^{l+k} A^n , \p^{r+1} B \rightarrow C \Rightarrow \Delta$}
			
			
			\dashedLine\RightLabel{$\nabla Cut$}
			\BIC{$\p^l \mathcal{S} , \mathcal{T} , \p^{r+1} B \rightarrow C \Rightarrow \Delta$}
		\end{prooftree}
		By induction hypothesis, once for $\mathbf{D_0}$ and $\mathbf{D_1}'$ and again for $\mathbf{D_0}$ and $\mathbf{D_1}''$
		\begin{prooftree}
			\noLine
			\AXC{$\mathbf{D_0}$}
			\UIC{$\mathcal{S} \Rightarrow \nabla^k A$}
			
			\noLine
			\AXC{$\mathbf{D_1}'$}
			\UIC{$\mathcal{T} , \p^{l+k} A^n \Rightarrow \nabla^r B$}
			
			\RightLabel{IH}
			\BIC{$\p^l \mathcal{S} , \mathcal{T} \Rightarrow \nabla^r B$}
			
			\noLine
			\AXC{$\mathbf{D_0}$}
			\UIC{$\mathcal{S} \Rightarrow \nabla^k A$}
			
			\noLine
			\AXC{$\mathbf{D_1}''$}
			\UIC{$\mathcal{T} , \p^{l+k} A^n, \p^r C \Rightarrow \Delta$}
			
			\RightLabel{IH}
			\BIC{$\p^l \mathcal{S} , \mathcal{T} , \p^r C \Rightarrow \Delta$}
			
			\RightLabel{$L\rightarrow$}
			\BIC{$\p^l \mathcal{S} , \mathcal{T} \p^{r+1} B \rightarrow C \Rightarrow \Delta$}
		\end{prooftree}
		
		\item \label{c:*-ri} ($R*, R\rightarrow$)
		\begin{prooftree}
			\noLine
			\AXC{$\mathbf{D_0}$}
			\UIC{$\Gamma \Rightarrow A$}
			
			\noLine
			\AXC{$\mathbf{D_1}'$}
			\UIC{$\nabla \Sigma , (\nabla^{l+1} A)^n , B \Rightarrow C$}
			\RightLabel{$R\rightarrow$}
			\UIC{$\Sigma , (\nabla^l A)^n \Rightarrow B \rightarrow C$}
			
			\dashedLine\RightLabel{$MC$}
			\BIC{$\nabla^l \Gamma , \Sigma \Rightarrow B \rightarrow C$}
		\end{prooftree}
		By induction hypothesis for $\mathbf{D_0}$ and $\mathbf{D_1}'$
		\begin{prooftree}
			\noLine
			\AXC{$\mathbf{D_0}$}
			\UIC{$\Gamma \Rightarrow A$}
			
			\noLine
			\AXC{$\mathbf{D_1}'$}
			\UIC{$\nabla \Sigma , (\nabla^{l+1} A)^n , B \Rightarrow C$}
			\RightLabel{IH}
			\BIC{$\nabla^{l+1} \Gamma , \nabla \Sigma , B \Rightarrow C$}
			
			\RightLabel{$R\rightarrow$}
			\UIC{$\nabla^l \Gamma , \Sigma \Rightarrow B \rightarrow C$}
		\end{prooftree}
		
		\item \label{c:*-ln} ($R*, L\nabla$)
		\begin{prooftree}
			\noLine
			\AXC{$\mathbf{D_0}$}
			\UIC{$\mathcal{S} \Rightarrow \nabla^k A$}
			
			\noLine
			\AXC{$\mathbf{D_1}'$}
			\UIC{$\mathcal{T} \cup [A^n | \epsilon_{l+k}] \cup [B | \epsilon_{r+1}] \Rightarrow \Delta$}
			\RightLabel{$L\nabla$}
			\UIC{$\mathcal{T} \cup [A^n | \epsilon_{l+k}] \cup [\nabla B | \epsilon_r] \Rightarrow \Delta$}
			
			\dashedLine\RightLabel{$\nabla Cut$}
			\BIC{$[\mathcal{S} | \epsilon_l] \cup \mathcal{T} \cup [\nabla B | \epsilon_r] \Rightarrow \Delta$}
		\end{prooftree}
		By induction hypothesis for $\mathbf{D_0}$ and $\mathbf{D_1}'$
		\begin{prooftree}
			\noLine
			\AXC{$\mathbf{D_0}$}
			\UIC{$\mathcal{S} \Rightarrow \nabla^k A$}
			
			\noLine
			\AXC{$\mathbf{D_1}'$}
			\UIC{$\mathcal{T} \cup [A^n | \epsilon_{l+k}] \cup [B | \epsilon_{r+1}] \Rightarrow \Delta$}
			\RightLabel{IH}
			\BIC{$[\mathcal{S} | \epsilon_l] \cup \mathcal{T} \cup [B | \epsilon_{r+1}] \Rightarrow \Delta$}
			
			\RightLabel{$L\nabla$}
			\UIC{$[\mathcal{S} | \epsilon_l] \cup \mathcal{T} \cup [\nabla B | \epsilon_r] \Rightarrow \Delta$}
		\end{prooftree}
		
		\item \label{c:*-rn} ($R*, R\nabla$) By the structure of $R \nabla$ we must have $l + k > 0$. Consider different cases for $l$.
		
		($\star$) If $l = 0$, then $k = k' + 1$ for some $k'$. Hence, the only possible right-rule as the last rule of $\mathbf{D_0}$ would be $R \nabla$. ($Rw$ for $\mathbf{D_0}$ is already discussed, other right-rules force a different structure for $A$)
		\begin{prooftree}
			\noLine
			\AXC{$\mathbf{D_0}'$}
			\UIC{$\mathcal{S} \Rightarrow \nabla^{k'} A$}
			\RightLabel{$R \nabla$}
			\UIC{$\p \mathcal{S} \Rightarrow \nabla^{k'+1} A$}
			
			\noLine
			\AXC{$\mathbf{D_1}'$}
			\UIC{$\mathcal{T} , \p^{k'} A^n \Rightarrow \Delta$}
			\RightLabel{$R\nabla$}
			\UIC{$\p \mathcal{T} , \p^{k'+1} A^n \Rightarrow \nabla \Delta$}
			
			\dashedLine\RightLabel{$\nabla Cut$}
			\BIC{$\p \mathcal{S} , \p \mathcal{T} \Rightarrow \nabla \Delta$}
		\end{prooftree}
		By induction hypothesis for $\mathbf{D_0}'$ and $\mathbf{D_1}'$
		\begin{prooftree}
			\noLine
			\AXC{$\mathbf{D_0}'$}
			\UIC{$\mathcal{S} \Rightarrow \nabla^{k'} A$}

			\noLine
			\AXC{$\mathbf{D_1}'$}
			\UIC{$\mathcal{T} , \p^{k'} A^n \Rightarrow \Delta$}

			\RightLabel{IH}
			\BIC{$\mathcal{S} , \mathcal{T} \Rightarrow \Delta$}
			
			\RightLabel{$R \nabla$}
			\UIC{$\p \mathcal{S} , \p \mathcal{T} \Rightarrow \nabla \Delta$}
		\end{prooftree}
		($\star \star$) Otherwise, if $l = l' + 1$ for some $l'$
		\begin{prooftree}
			\noLine
			\AXC{$\mathbf{D_0}$}
			\UIC{$\mathcal{S} \Rightarrow \nabla^k A$}
			
			\noLine
			\AXC{$\mathbf{D_1}'$}
			\UIC{$\mathcal{T} , \p^{l'+k} A^n \Rightarrow \Delta$}
			\RightLabel{$R\nabla$}
			\UIC{$\p \mathcal{T} , \p^{l'+k+1} A^n \Rightarrow \nabla \Delta$}
			
			\dashedLine\RightLabel{$\nabla Cut$}
			\BIC{$\p^{l'+1} \mathcal{S} , \p \mathcal{T} \Rightarrow \nabla \Delta$}
		\end{prooftree}
		then by the induction hypothesis for $\mathbf{D_0}$ and $\mathbf{D_1}'$
		\begin{prooftree}
			\noLine
			\AXC{$\mathbf{D_0}$}
			\UIC{$\mathcal{S} \Rightarrow \nabla^k A$}
			
			\noLine
			\AXC{$\mathbf{D_1}'$}
			\UIC{$\mathcal{T} , \p^{l'+k} A^n \Rightarrow \Delta$}
			
			\RightLabel{IH}
			\BIC{$\p^{l'} \mathcal{S} , \mathcal{T} \Rightarrow \Delta$}
			
			
			\RightLabel{$R\nabla$}
			\UIC{$\p^{l'+1} \mathcal{S} , \p \mathcal{T} \Rightarrow \nabla \Delta$}
		\end{prooftree}
	\end{enumerate}
	
		\paragraph{} II. These are the cases where the last rule in $\mathbf{D_1}$ is a logical left-rule with $A$ as its principal formula, which forces a specific structure for $A$. So in any case for $\mathbf{D_0}$, except $R \nabla$, we can assume $\mathbf{D_1}$ also ends with the corresponding logical rule and $k = 0$. The $R \nabla$ case with $k = 0$ is the same, but $k > 0$ tells us nothing more about $\mathbf{D_1}$.

	\begin{enumerate}[label={\Alph*.}]
		\item \label{c:ra-la1} ($R\land, L\land_1$)
		\begin{prooftree}
			\noLine
			\AXC{$\mathbf{D_0}'$}
			\UIC{$\Gamma \Rightarrow B$}
			\AXC{$\Gamma \Rightarrow C$}
			\RightLabel{$R\land$}
			\BIC{$\Gamma \Rightarrow B \land C$}
			
			\noLine
			\AXC{$\mathbf{D_1}'$}
			\UIC{$\Sigma , (\nabla^l (B \land C))^{n-1}, \nabla^l B \Rightarrow \Delta$}
			\RightLabel{$L\land_1$}
			\UIC{$\Sigma , (\nabla^l (B \land C))^n \Rightarrow \Delta$}
			
			\RightLabel{$MC$} \dashedLine
			\BIC{$\nabla^l \Gamma , \Sigma \Rightarrow \Delta$}
		\end{prooftree}
		Induction hypothesis for $\mathbf{D_0}$ and $\mathbf{D_1}'$ gives us $\nabla^l \Gamma , \Sigma , \nabla^l B \Rightarrow \Delta$. Then we remove $B$ by a low rank $MC$ on this sequent and $\mathbf{D_0}'$ to get $\nabla^l \Gamma^2 , \Sigma \Rightarrow \Delta$, which yields the desired sequent after enough applications of $Lc$.
		\begin{prooftree}
			\noLine
			\AXC{$\mathbf{D_0}'$}
			\UIC{$\Gamma \Rightarrow B$}

			\noLine
			\AXC{$\mathbf{D_0}$}
			\UIC{$\Gamma \Rightarrow B \land C$}
			\noLine
			\AXC{$\mathbf{D_1}'$}
			\UIC{$\Sigma , (\nabla^l (B \land C))^{n-1}, \nabla^l B \Rightarrow \Delta$}
			
			\RightLabel{IH}
			\BIC{$\nabla^l \Gamma , \Sigma , \nabla^l B \Rightarrow \Delta$}

			\RightLabel{$MC$}
			\BIC{$\nabla^l \Gamma^2 , \Sigma \Rightarrow \Delta$}
			
			\doubleLine \RightLabel{$Lc$}
			\UIC{$\nabla^l \Gamma , \Sigma \Rightarrow \Delta$}
		\end{prooftree}

		\item \label{c:ra-la2} ($R\land, L\land_1$)
\begin{prooftree}
	\noLine
	\AXC{$\Gamma \Rightarrow B$}
	\AXC{$\mathbf{D_0}'$}
	\UIC{$\Gamma \Rightarrow C$}
	\RightLabel{$R\land$}
	\BIC{$\Gamma \Rightarrow B \land C$}
	
	\noLine
	\AXC{$\mathbf{D_1}'$}
	\UIC{$\Sigma , (\nabla^l (B \land C))^{n-1}, \nabla^l C \Rightarrow \Delta$}
	\RightLabel{$L\land_2$}
	\UIC{$\Sigma , (\nabla^l (B \land C))^n \Rightarrow \Delta$}
	
	\RightLabel{$MC$} \dashedLine
	\BIC{$\nabla^l \Gamma , \Sigma \Rightarrow \Delta$}
\end{prooftree}
Induction hypothesis for $\mathbf{D_0}$ and $\mathbf{D_1}'$ gives us $\nabla^l \Gamma , \Sigma , \nabla^l C \Rightarrow \Delta$. Then we remove $C$ by a low rank $MC$ on this sequent and $\mathbf{D_0}'$ to get $\nabla^l \Gamma^2 , \Sigma \Rightarrow \Delta$, which yields the desired sequent after enough applications of $Lc$.
\begin{prooftree}
	\noLine
	\AXC{$\mathbf{D_0}'$}
	\UIC{$\Gamma \Rightarrow C$}
	
	\noLine
	\AXC{$\mathbf{D_0}$}
	\UIC{$\Gamma \Rightarrow B \land C$}
	\noLine
	\AXC{$\mathbf{D_1}'$}
	\UIC{$\Sigma , (\nabla^l (B \land C))^{n-1}, \nabla^l C \Rightarrow \Delta$}
	
	\RightLabel{IH}
	\BIC{$\nabla^l \Gamma , \Sigma , \nabla^l C \Rightarrow \Delta$}
	
	\RightLabel{$MC$}
	\BIC{$\nabla^l \Gamma^2 , \Sigma \Rightarrow \Delta$}
	
	\doubleLine \RightLabel{$Lc$}
	\UIC{$\nabla^l \Gamma , \Sigma \Rightarrow \Delta$}
\end{prooftree}

		\item \label{c:ro1-lo} ($R\lor_1, L\lor$)
		\begin{prooftree}
			\noLine
			\AXC{$\mathbf{D_0}'$}
			\UIC{$\mathcal{S} \Rightarrow B$}
			\RightLabel{$R\lor_1$}
			\UIC{$\mathcal{S} \Rightarrow B \lor C$}

			\noLine
			\AXC{$\mathbf{D_1}'$}
			\UIC{$\mathcal{T} | \Gamma , (B \lor C)^{n-1} , B | \mathcal{R} \Rightarrow \Delta$}
			\AXC{$\mathcal{T} | \Gamma , (B \lor C)^{n-1} , C | \mathcal{R} \Rightarrow \Delta$}
			\RightLabel{$L\lor$}
			\BIC{$\mathcal{T} | \Gamma , (B \lor C)^n | \mathcal{R} \Rightarrow \Delta$}

			\RightLabel{$\nabla Cut$} \dashedLine
			\BIC{$[ \mathcal{S} | \epsilon_{len(\mathcal{R})} ] \cup [ \mathcal{T} | \Gamma | \mathcal{R} ] \Rightarrow \Delta$}
		\end{prooftree}
		Induction hypothesis for $\mathbf{D_0}$ and $\mathbf{D_1}'$ gives us $[ \mathcal{S} | \epsilon_{len(\mathcal{R})} ] \cup [ \mathcal{T} | \Gamma , B | \mathcal{R} ] \Rightarrow \Delta$. Now removing $B$ with a low rank $\nabla Cut$ on this sequent and $\mathbf{D_0}'$ gives $[ \mathcal{S}^2 | \epsilon_{len(\mathcal{R})} ] \cup [ \mathcal{T} | \Gamma | \mathcal{R} ] \Rightarrow \Delta$, which yields the desired sequent with enough applications of $Lc$.
		\begin{prooftree}
			\noLine
			\AXC{$\mathbf{D_0}'$}
			\UIC{$\mathcal{S} \Rightarrow B$}

			\noLine
			\AXC{$\mathbf{D_0}$}
			\UIC{$\mathcal{S} \Rightarrow B \lor C$}

			\noLine
			\AXC{$\mathbf{D_1}'$}
			\UIC{$\mathcal{T} | \Gamma , (B \lor C)^{n-1} , B | \mathcal{R} \Rightarrow \Delta$}

			\RightLabel{IH}
			\BIC{$[ \mathcal{S} | \epsilon_{len(\mathcal{R})} ] \cup [ \mathcal{T} | \Gamma , B | \mathcal{R} ] \Rightarrow \Delta$}
			
			\RightLabel{$\nabla Cut$}
			\BIC{$[ \mathcal{S}^2 | \epsilon_{len(\mathcal{R})} ] \cup [ \mathcal{T} | \Gamma | \mathcal{R} ] \Rightarrow \Delta$}
			
			\doubleLine \RightLabel{$Lc$}
			\UIC{$[ \mathcal{S} | \epsilon_{len(\mathcal{R})} ] \cup [ \mathcal{T} | \Gamma | \mathcal{R} ] \Rightarrow \Delta$}
		\end{prooftree}

		\item \label{c:ro2-lo} ($R\lor_2, L\lor$)
\begin{prooftree}
	\noLine
	\AXC{$\mathbf{D_0}'$}
	\UIC{$\Gamma \Rightarrow C$}
	\RightLabel{$R\lor_2$}
	\UIC{$\Gamma \Rightarrow B \lor C$}
	
	\noLine
	\AXC{$\Sigma , (\nabla^l (B \lor C))^{n-1} , \nabla^l B \Rightarrow \Delta$}
	\AXC{$\mathbf{D_1}'$}
	\UIC{$\Sigma , (\nabla^l (B \lor C))^{n-1} , \nabla^l C \Rightarrow \Delta$}
	\RightLabel{$L\lor$}
	\BIC{$\Sigma , (\nabla^l (B \lor C))^n \Rightarrow \Delta$}
	
	\RightLabel{$MC$} \dashedLine
	\BIC{$\nabla^l \Gamma , \Sigma \Rightarrow \Delta$}
\end{prooftree}
Induction hypothesis for $\mathbf{D_0}$ and $\mathbf{D_1}'$ gives us $\nabla^l \Gamma , \Sigma , \nabla^l C \Rightarrow \Delta$. Then we remove $C$ by a low rank $MC$ on this sequent and $\mathbf{D_0}'$ to get $\nabla^l \Gamma^2 , \Sigma \Rightarrow \Delta$, which yields the desired sequent after enough applications of $Lc$.
\begin{prooftree}
	\noLine
	\AXC{$\mathbf{D_0}'$}
	\UIC{$\Gamma \Rightarrow C$}
	
	\noLine
	\AXC{$\mathbf{D_0}$}
	\UIC{$\Gamma \Rightarrow B \lor C$}
	
	\noLine
	\AXC{$\mathbf{D_1}'$}
	\UIC{$\Sigma , (\nabla^l (B \lor C))^{n-1} , \nabla^l C \Rightarrow \Delta$}
	
	\RightLabel{IH}
	\BIC{$\nabla^l \Gamma , \Sigma , \nabla^l C \Rightarrow \Delta$}
	
	\RightLabel{$MC$}
	\BIC{$\nabla^l \Gamma^2 , \Sigma \Rightarrow \Delta$}
	
	\doubleLine \RightLabel{$Lc$}
	\UIC{$\nabla^l \Gamma , \Sigma \Rightarrow \Delta$}
\end{prooftree}

		\item \label{c:rn-*} ($R\nabla, *$) This is the only case where $k$ can be more than $0$. First, in the case that $k = 0$, like previous cases, the last rule of $\mathbf{D_1}$ (and the construction of $A$) is determined by the last rule of $\mathbf{D_0}$; We have $A = \nabla B$ for some $B$.
		\begin{prooftree}
			\noLine
			\AXC{$\mathbf{D_0}$}
			\UIC{$\mathcal{S} | \epsilon_1 \Rightarrow \nabla B$}
			
			\noLine
			\AXC{$\mathbf{D_1}'$}
			\UIC{$\mathcal{T} \cup [B | (\nabla B)^{n-1} | \epsilon_l] \Rightarrow \Delta$}
			\RightLabel{$L \nabla$}
			\UIC{$\mathcal{T} \cup [(\nabla B)^n | \epsilon_l] \Rightarrow \Delta$}
			
			\RightLabel{$\nabla Cut$} \dashedLine
			\BIC{$[ \mathcal{S} | \epsilon_{l+1} ] \cup \mathcal{T} \Rightarrow \Delta$}
		\end{prooftree}
		First apply induction hypothesis for $\mathbf{D_0}$ and $\mathbf{D_1}'$ with $A' = A = \nabla B$ and $k' = 0$. Then remove $B$ by a $\nabla Cut$ on $\mathbf{D_0}$ and the resulting sequent.

		\begin{prooftree}
			\noLine
			\AXC{$\mathbf{D_0}$}
			\UIC{$\mathcal{S} | \epsilon_1 \Rightarrow \nabla B$}
			
			
			\noLine
			\AXC{$\mathbf{D_0}$}
			\UIC{$\mathcal{S} | \epsilon_1 \Rightarrow \nabla B$}
			
			\noLine
			\AXC{$\mathbf{D_1}'$}
			\UIC{$\mathcal{T} \cup [B | (\nabla B)^{n-1} | \epsilon_l] \Rightarrow \Delta$}
			
			\RightLabel{IH}
			\BIC{$[\mathcal{S} | \epsilon_{l+1}] \cup \mathcal{T} \cup [B | \epsilon_{l+1}] \Rightarrow \Delta$}
			
			
			\RightLabel{$\nabla Cut$}
			\BIC{$[\mathcal{S}^2 | \epsilon_{l+1}] \cup \mathcal{T} \Rightarrow \Delta$}
			
			\RightLabel{$Lc$}
			\UIC{$[\mathcal{S} | \epsilon_{l+1}] \cup \mathcal{T} \Rightarrow \Delta$}
		\end{prooftree}

		Now let $k = k' + 1$ for some $k'$. This does not necessarily determine the structure of $A$. Our construction, however, does not depend on then last rule of $\mathbf{D_1}$ or whether $A$ is principal there.
		\begin{prooftree}
			\noLine
			\AXC{$\mathbf{D_0}'$}
			\UIC{$\mathcal{S} \Rightarrow \nabla^{k'} A$}
			\RightLabel{$R \nabla$}
			\UIC{$\mathcal{S} | \epsilon_1 \Rightarrow \nabla^{k'+1} A$}
			
			\noLine
			\AXC{$\mathbf{D_1}$}
			\UIC{$\mathcal{T} \cup [A^n | \epsilon_{l+k'+1}] \Rightarrow \Delta$}
			
			\RightLabel{$\nabla Cut$} \dashedLine
			\BIC{$[ \mathcal{S} | \epsilon_{l+1} ] \cup \mathcal{T} \Rightarrow \Delta$}
		\end{prooftree}
		Induction hypothesis on $\mathbf{D_0}'$ and $\mathbf{D_1}$ does exactly the same.
		\begin{prooftree}
			\noLine
			\AXC{$\mathbf{D_0}'$}
			\UIC{$\mathcal{S} \Rightarrow \nabla^{k'} A$}

			\noLine
			\AXC{$\mathbf{D_1}$}
			\UIC{$\mathcal{T} \cup [A^n | \epsilon_{l+k'+1}] \Rightarrow \Delta$}
			
			\RightLabel{IH}
			\BIC{$[ \mathcal{S} | \epsilon_{l+1} ] \cup \mathcal{T} \Rightarrow \Delta$}
		\end{prooftree}

		\item \label{c:ri-li} ($R\rightarrow, L\rightarrow$)
		\begin{prooftree}
			\noLine
			\AXC{$\mathbf{D_0}'$}
			\UIC{$\nabla \Gamma , B \Rightarrow C$}
			\RightLabel{$R\rightarrow$}
			\UIC{$\Gamma \Rightarrow B \rightarrow C$}

			\noLine
			\AXC{$\mathbf{D_1}'$}
			\UIC{$\Sigma , (\nabla^{l+1} (B \rightarrow C))^{n-1} \Rightarrow \nabla^l B$}
			\noLine
			\AXC{$\mathbf{D_1}''$}
			\UIC{$\Sigma , (\nabla^{l+1} (B \rightarrow C))^{n-1} , \nabla^l C \Rightarrow \Delta$}
			\RightLabel{$L\rightarrow$}
			\BIC{$\Sigma , (\nabla^{l+1} (B \rightarrow C))^n \Rightarrow \Delta$}
			
			\RightLabel{$MC$} \dashedLine
			\BIC{$\nabla^{l+1} \Gamma , \Sigma \Rightarrow \Delta$}
		\end{prooftree}
		From induction hypothesis for $\mathbf{D_0}$ and $\mathbf{D_1}'$ we have $\nabla^{l+1} \Gamma , \Sigma \Rightarrow \nabla^l B$. Call it $\mathbf{D}'$.
		\begin{prooftree}
			\noLine
			\AXC{$\mathbf{D_0}$}
			\UIC{$\Gamma \Rightarrow B \rightarrow C$}
			
			\noLine
			\AXC{$\mathbf{D_1}'$}
			\UIC{$\Sigma , (\nabla^{l+1} (B \rightarrow C))^{n-1} \Rightarrow \nabla^l B$}
			
			\LeftLabel{$\mathbf{D}':~$}
			\RightLabel{IH}
			\BIC{$\nabla^{l+1} \Gamma , \Sigma \Rightarrow \nabla^l B$}
		\end{prooftree}
		Again from IH, this time for $\mathbf{D_0}$ and $\mathbf{D_1}''$ we have $\nabla^{l+1} \Gamma , \Sigma , \nabla^l C \Rightarrow \Delta$. Call it $\mathbf{D}''$.
		\begin{prooftree}
			\noLine
			\AXC{$\mathbf{D_0}$}
			\UIC{$\Gamma \Rightarrow B \rightarrow C$}
			
			\noLine
			\AXC{$\mathbf{D_1}''$}
			\UIC{$\Sigma , (\nabla^{l+1} (B \rightarrow C))^{n-1} , \nabla^l C \Rightarrow \Delta$}
			
			\LeftLabel{$\mathbf{D}'':~$}
			\RightLabel{IH}
			\BIC{$\nabla^{l+1} \Gamma , \Sigma , \nabla^l C \Rightarrow \Delta$}
		\end{prooftree}
		Cut $C$ from $\mathbf{D_0}'$ and $\mathbf{D}''$, then cut $\nabla^l B$ from the resulting sequent and $\mathbf{D}'$. Notice that $\nabla$ does not increase the rank of a formula, so $\rho(\nabla^l B) < \rho(B \rightarrow C)$. Applying enough contractions, we can derive the desired sequent.
		\begin{prooftree}
			\noLine
			\AXC{$\mathbf{D}'$}
			\UIC{$\nabla^{l+1} \Gamma , \Sigma \Rightarrow \nabla^l B$}


			\noLine
			\AXC{$\mathbf{D_0}'$}
			\UIC{$\nabla \Gamma , B \Rightarrow C$}

			\noLine
			\AXC{$\mathbf{D}''$}
			\UIC{$\nabla^{l+1} \Gamma , \Sigma , \nabla^l C \Rightarrow \Delta$}

			\RightLabel{$MC$}
			\BIC{$(\nabla^{l+1} \Gamma)^2 , \nabla^l B , \Sigma \Rightarrow \Delta$}



			\RightLabel{$MC$}
			\BIC{$(\nabla^{l+1} \Gamma)^3 , \Sigma^2 \Rightarrow \Delta$}



			\doubleLine \RightLabel{$Lc$}
			\UIC{$\nabla^{l+1} \Gamma , \Sigma \Rightarrow \Delta$}
		\end{prooftree}

	\end{enumerate}

	III. If an instance of $A$ is principal in the last rule of $\mathbf{D_1}$, which is not logical.
	\begin{enumerate}[label={\roman*.}]
		\item \label{c:*-id} ($R*, Id$) $k$ and $l$ have to be $0$, so $\mathbf{D_0}$ proves the desired sequent.

		\item \label{c:*-ex} ($R \nabla, Ex$) The last rule in $\mathbf{D_1}$ can be $Ex$ only in the case that $A = \bot$. So $\mathbf{D_0}$ must end with $R \nabla$ and $k > 0$.
		\begin{prooftree}
			\noLine
			\AXC{$\mathbf{D_0}'$}
			\UIC{$\mathcal{S}' \Rightarrow \nabla^{k'} \bot$}
			\RightLabel{$R\nabla$}
			\UIC{$\mathcal{S}' | \epsilon_1 \Rightarrow \nabla^{k' + 1} \bot$}
			

			\AXC{$\mathbf{D_1}$}
			\RightLabel{$Ex$}
			\UIC{$\epsilon | \bot^n | \epsilon_{l+k'+1} \Rightarrow$}
			
			\dashedLine\RightLabel{$\nabla Cut$}
			\BIC{$\mathcal{S}' | \epsilon_{l+1} \Rightarrow$}
		\end{prooftree}
		The desired sequent is exactly what we get from induction hypothesis for $\mathbf{D_0}'$ and $\mathbf{D_1}$.
		\begin{prooftree}
			\noLine
			\AXC{$\mathbf{D_0}'$}
			\UIC{$\mathcal{S}' \Rightarrow \nabla^{k'} \bot$}
			
			\AXC{$\mathbf{D_1}$}
			\RightLabel{$Ex$}
			\UIC{$\epsilon | \bot^n | \epsilon_{l+k'+1} \Rightarrow$}
			
			\RightLabel{IH}
			\BIC{$\mathcal{S}' | \epsilon_{l+1} \Rightarrow$}
		\end{prooftree}

		\item \label{c:*-lw-p} ($R*, Lw$)
		\begin{prooftree}
			\noLine
			\AXC{$\mathbf{D_0}$}
			\UIC{$\mathcal{S} \Rightarrow \nabla^k A$}
			
			\noLine
			\AXC{$\mathbf{D_1}'$}
			\UIC{$\mathcal{T} , \p^{l+k} A^{n-1} \Rightarrow \Delta$}
			\RightLabel{$Lw$}
			\UIC{$\mathcal{T} , \p^{l+k} A^n \Rightarrow \Delta$}
			
			\dashedLine\RightLabel{$\nabla Cut$}
			\BIC{$\p^l \mathcal{S} , \mathcal{T} \Rightarrow \Delta$}
		\end{prooftree}
		($\star$) If $n = 1$, then by $Lw$ on $\mathbf{D_1}'$ we have
		\begin{prooftree}
			\noLine
			\AXC{$\mathbf{D_1}'$}
			\UIC{$\mathcal{T} \Rightarrow \Delta$}
			
			\RightLabel{$Lw$}
			\UIC{$\p^l \mathcal{S} , \mathcal{T} \Rightarrow \Delta$}
		\end{prooftree}
		($\star \star$) If $n > 1$, then by induction hypothesis for $\mathbf{D_0}$ and $\mathbf{D_1}'$ we have
		\begin{prooftree}
			\noLine
			\AXC{$\mathbf{D_0}$}
			\UIC{$\mathcal{S} \Rightarrow \nabla^k A$}
			
			\noLine
			\AXC{$\mathbf{D_1}'$}
			\UIC{$\mathcal{T} , \p^{l+k} A^{n-1} \Rightarrow \Delta$}
			
			\RightLabel{IH}
			\BIC{$\p^l \mathcal{S} , \mathcal{T} \Rightarrow \Delta$}
		\end{prooftree}
	
		\item \label{c:*-lc-p} ($R*, Lc$)
		\begin{prooftree}
			\noLine
			\AXC{$\mathbf{D_0}$}
			\UIC{$\mathcal{S} \Rightarrow \nabla^k A$}
			
			\noLine
			\AXC{$\mathbf{D_1}'$}
			\UIC{$\mathcal{T} \cup [A^{n+1} | \epsilon_{l+k}] \Rightarrow \Delta$}
			\RightLabel{$Lc$}
			\UIC{$\mathcal{T} \cup [A^n | \epsilon_{l+k}] \Rightarrow \Delta$}
			
			\dashedLine\RightLabel{$\nabla Cut$}
			\BIC{$[\mathcal{S} | \epsilon_l] \cup \mathcal{T} \Rightarrow \Delta$}
		\end{prooftree}
		By induction hypothesis for $\mathbf{D_0}$ and $\mathbf{D_1}'$ we have
		\begin{prooftree}
			\noLine
			\AXC{$\mathbf{D_0}$}
			\UIC{$\mathcal{S} \Rightarrow \nabla^k A$}
			
			\noLine
			\AXC{$\mathbf{D_1}'$}
			\UIC{$\mathcal{T} \cup [A^{n+1} | \epsilon_{l+k}] \Rightarrow \Delta$}
			
			\RightLabel{IH}
			\BIC{$[\mathcal{S} | \epsilon_l] \cup \mathcal{T} \Rightarrow \Delta$}
		\end{prooftree}
	\end{enumerate}
\end{enumerate}

\section{Corollary} \emph{$\nabla$Cut Elimination for GSTL: }
If $\small\text{GSTL}^- + \nabla Cut \vdash \Gamma \Rightarrow \Delta$, then $\small\text{GSTL}^- \vdash \Gamma \Rightarrow \Delta$.

\emph{Proof:} It suffices to show that for any proof tree of $\Gamma \Rightarrow \Delta$ like $\mathbf{D}$, there is another proof of it with a lower rank. Using induction on $h(\mathbf{D})$, the induction hypothesis states that for any proof of $\Gamma' \Rightarrow \Delta'$ like $\mathbf{D'}$ such that $h(\mathbf{D'}) < h(\mathbf{D})$, there is another proof of $\Gamma' \Rightarrow \Delta'$ like $\mathbf{D''}$ such that $\rho(\mathbf{D''}) < \rho(\mathbf{D'})$. Particularly for the immediate sub-trees of $\mathbf{D}$, this gives us sub-trees with lower cut-rank, which we call $\mathbf{D}_i$. We now consider two cases for the last rule of $\mathbf{D}$

\begin{enumerate}[label=\Roman*]
	\item If the last rule of $\mathbf{D}$ is of a lower rank than $\rho(\mathbf{D})$, i.e. the $\nabla Cut$ rule with the maximum rank is not the last rule in $\mathbf{D}$, then we can apply the same last rule on $\mathbf{D}_i$s and get a proof of $\Gamma \Rightarrow \Delta$ with a lower rank.
	
	\item If the last rule of $\mathbf{D}$ is an instance of $\nabla Cut$ rule with a cut-formula of rank $\rho(\mathbf{D})$, then we can apply theorem \ref{cut-admis} to $\mathbf{D_0}$ and $\mathbf{D_1}$ to get the same $\nabla Cut$ with a lower rank.
\end{enumerate}

\section{Rules} In the following, $\dotdiv$ is the truncated subtraction defined as $a \dotdiv b = max(a-b, 0)$.

\subsection{Theorem} \textit{Generalized Visser rules:} Let $\{ l_i \}_{i=1}^n$ be a sequence of natural numbers of length $n$. If $\vdash_{iSTL} \Rightarrow \bigwedge_{i=1}^n (\nabla^{l_i} (A_i \rightarrow B_i)) \rightarrow C \lor D$ then either $\vdash_{iSTL} \Rightarrow \bigwedge_{i=1}^n (\nabla^{l_i} (A_i \rightarrow B_i)) \rightarrow \nabla^{l_j \dotdiv 1} A_j$ for some $j \in \{ 1 , \dots , n \}$, or $\vdash_{iSTL} \Rightarrow \bigwedge_{i=1}^n (\nabla^{l_i} (A_i \rightarrow B_i)) \rightarrow C$, or $\vdash_{iSTL} \Rightarrow \bigwedge_{i=1}^n (\nabla^{l_i} (A_i \rightarrow B_i)) \rightarrow D$.

\textit{Proof}:
We have $\bigwedge_{i=1}^n (\nabla^{l_i} (A_i \rightarrow B_i)) \Rightarrow C \lor D$ by \ref{lem:impl-elim}. By $Cut$ and \ref{lem:conj-context} we have $\{ \nabla^{l_i} (A_i \rightarrow B_i) \}_{i=1}^n \Rightarrow$ $C \lor D$, which also has a proof like $\mathbf{D}$ in $GSTL^-$ by \ref{translation} and \ref{cut-elim}. The last rule in $\mathbf{D}$ can be:
\begin{itemize}[label=-]
	\item $R\lor_1$, applied on $\{ \nabla^{l_i} (A_i \rightarrow B_i) \}_{i=1}^n \Rightarrow C$. With enough applications of $L\land_1$, $L\land_2$\footnote{{\color{red} \textit{<convincing the reader about $\land$'s associativity/commutativity>} }}, $Lc$ and a $R\rightarrow$ we will have $\Rightarrow \bigwedge_{i=1}^n (\nabla^{l_i} (A_i \rightarrow B_i)) \rightarrow C$.
	
	\item $R\lor_2$, applied on $\{ \nabla^{l_i} (A_i \rightarrow B_i) \}_{i=1}^n \Rightarrow D$. Same as the previous case, we can derive $\Rightarrow \bigwedge_{i=1}^n (\nabla^{l_i} (A_i \rightarrow B_i)) $ $\rightarrow D$.
	
	\item $Rw$, applied on $\{ \nabla^{l_i} (A_i \rightarrow B_i) \}_{i=1}^n \Rightarrow$. A different $Rw$ gives $\{ \nabla^{l_i} (A_i \rightarrow B_i) \}_{i=1}^n \Rightarrow C$. Again, we can get $\Rightarrow \bigwedge_{i=1}^n (\nabla^{l_i} (A_i \rightarrow B_i)) \rightarrow C$.
	
	\item $Lw$, applied on $\{ \nabla^{l_i} (A_i \rightarrow B_i) \}_{i=1,i \neq k}^n \Rightarrow C \lor D$ for some $k \in \{ 1 , \dots , n \}$. Let $\mathbf{D}'$ be the immediate sup-tree of $\mathbf{D}$. By induction on $h(\mathbf{D})$, we can apply induction hypothesis on $\mathbf{D}'$ to get either $\{ \nabla^{l_i} (A_i \rightarrow B_i) \}_{i=1,i \neq k}^n \Rightarrow C$, $\{ \nabla^{l_i} (A_i \rightarrow B_i) \}_{i=1,i \neq k}^n \Rightarrow D$ or $\{ \nabla^{l_i} (A_i \rightarrow B_i) \}_{i=1,i \neq k}^n \Rightarrow \nabla^{l_j \dotdiv 1} A_j$ for some $j \in \{ 1 , \dots , n \} - \{k\}$. Then, after introducing $\nabla^{l_k} A_k \rightarrow B_k$ on the left with $Lw$ again, we can follow the same manner as previous cases to reach any of the desired sequents.
	
	\item $Lc$, applied on $\{ \nabla^{l_i} (A_i \rightarrow B_i) \}_{i=1}^n , \nabla^{l_k} (A_k \rightarrow B_k) \Rightarrow C \lor D$ for some $k \in \{ 1 , \dots , n \}$. This is just the same as the $Lw$ case, except this time we must remove the extra $\nabla^{l_k} (A_k \rightarrow B_k)$ with another $Lc$.
	
	\item $L\rightarrow$, applied on $\{ \nabla^{l_i} (A_i \rightarrow B_i) \}_{i=1, i \neq j}^n \Rightarrow \nabla^{l_j - 1} A_j$ and $\{ \nabla^{l_i} (A_i \rightarrow B_i) \}_{i=1, i \neq j}^n , \nabla^{l_j - 1} B_j \Rightarrow C \lor D$ for some $j \in \{ 1 , \dots , n \}$. So this implies $n>0$ and $l_j>0$ for at least one such $j$. Again, we can derive $\Rightarrow  \bigwedge_{i=1}^n (\nabla^{l_i} (A_i \rightarrow B_i)) \rightarrow \nabla^{l_j - 1} A_j$ using proper $L\land_{1/2}$, $Lc$ and $R\rightarrow$.
\end{itemize}
Notice that no other case is valid, since they all imply different structure for $\mathbf{D}$. Now from theorem \ref{translation}, all of the desired sequents also have a proof in iSTL.


\section{Interpolation}

\subsection{Theorem} \textit{Craig's Interpolation for GSTL$^-$: } For any $\Gamma_1$, $\Gamma_2$ and $\Delta$, if GSTL$^-\vdash \Gamma_1 , \Gamma_2 \Rightarrow \Delta$, then there is a formula $C$ such that $P(C) \subseteq P(\Gamma_1) \cap P(\Gamma_2 , \Delta)$, GSTL$^-\vdash \Gamma_1 \Rightarrow C$ and GSTL$^-\vdash \Gamma_2 , C \Rightarrow \Delta$.

\textit{Proof}: Let $\mathbf{D}$ be the GSTL$^-$ proof of $\Gamma_1 , \Gamma_2 \Rightarrow \Delta$. We will use induction on $h(\mathbf{D})$. So for any $\Gamma_1'$, $\Gamma_2'$ and $\Delta'$ such that GSTL$^-\vdash \Gamma_1' , \Gamma_2' \Rightarrow \Delta'$ with a proof smaller than $\mathbf{D}$, the induction hypothesis (IH) gives an interpolant $C_{\langle\Gamma_1'; \Gamma_2'; \Delta'\rangle}$ for which the statement of the theorem is true. We now build the desired interpolant $C$, in different cases for the last rule of $\mathbf{D}$. In cases for left-rules, we also need to consider whether the principal formula is in $\Gamma_1$ or $\Gamma_2$ in separate cases.
\begin{enumerate}
	\item ($Id$) We have $\Gamma_1,\Gamma_2 = \Delta = A$.
	\begin{enumerate}
		\item If $\Gamma_1 = \{\}$ and $\Gamma_2 = A$, then define $C = \top$. So we have $\Rightarrow \top$ by $Ta$ and $A , \top \Rightarrow A$ by $Id$ and $Lw$.
		
		\item If $\Gamma_1 = A$ and $\Gamma_2 = \{\}$ then define $C = A$. So we have $A \Rightarrow A$ by $Id$.
	\end{enumerate}
	\item ($Ta$) Take $C = \top$.
	
	\item ($Ex$) Take $C = \nabla^n \bot$.
	
	\item ($Lw$) $\mathbf{D}$ proves $\Gamma_1' , \Gamma_2' , A \Rightarrow \Delta$ and has a sub-proof for $\Gamma_1' , \Gamma_2' \Rightarrow \Delta$, for which IH gives an interpolant $C_{\langle\Gamma_1';\Gamma_2';\Delta\rangle}$ and proofs for $\Gamma_1' \Rightarrow C_{\langle\Gamma_1';\Gamma_2';\Delta\rangle}$ and $\Gamma_2 , C_{\langle\Gamma_1';\Gamma_2';\Delta\rangle} \Rightarrow \Delta$, such that $P(C_{\langle\Gamma_1';\Gamma_2';\Delta\rangle}) \subseteq$ $ P(\Gamma_1') \cap P(\Gamma_2' , \Delta)$.
	\begin{enumerate}
		\item If $\Gamma_1 = \Gamma_1'$ and $\Gamma_2 = \Gamma_2' , A$, take $C = C_{\langle\Gamma_1';\Gamma_2';\Delta\rangle}$. Then we have  $\Gamma_1' \Rightarrow C$ by IH and $\Gamma_2 , A , C \Rightarrow \Delta$ by $Lw$ and IH. From IH, we also have $P(C) \subseteq P(\Gamma_1') \cap P(\Gamma_2' , A , \Delta)$, since $P$ takes ``$,$'' to ``$\cup$'', which distributes over ``$\cap$'' and is increasing with respect to ``$\subseteq$''.
		
		\item If $\Gamma_1 = \Gamma_1' , A$ and $\Gamma_2 = \Gamma_2'$, again take $C = C_{\langle\Gamma_1';\Gamma_2';\Delta\rangle}$. Then we have  $\Gamma_1' , A \Rightarrow C$ by $Lw$ and IH, and $\Gamma_2 , C \Rightarrow \Delta$ by IH. We also have $P(C) \subseteq P(\Gamma_1' , A) \cap P(\Gamma_2' , \Delta)$ by IH and argument similar to the previous case.
	\end{enumerate}

	\item ($Lc$) $\mathbf{D}$ proves $\Gamma_1' , \Gamma_2' , A \Rightarrow \Delta$ and has a sub-proof for $\Gamma_1' , \Gamma_2' , A , A \Rightarrow \Delta$.
	\begin{enumerate}
		\item If $\Gamma_1 = \Gamma_1'$ and $\Gamma_2 = \Gamma_2' , A$, take $C = C_{\langle\Gamma_1';\Gamma_2',A,A;\Delta\rangle}$. Then we have $\Gamma_1' \Rightarrow C$ by IH and $\Gamma_2' , A \Rightarrow \Delta$ by IH and $Lc$. From IH, we also have $P(C) \subseteq P(\Gamma_1') \cap P(\Gamma_2',A,\Delta)$, since $P(\Gamma,X) = P(\Gamma,X,X)$.
		
		\item If $\Gamma_1 = \Gamma_1' , A$ and $\Gamma_2 = \Gamma_2'$, take $C = C_{\langle\Gamma_1',A,A;\Gamma_2';\Delta\rangle}$. Then we have $\Gamma_1' , A \Rightarrow C$ by IH and $Lc$, and $\Gamma_2' \Rightarrow \Delta$ by IH. We also have $P(C) \subseteq P(\Gamma_1',A) \cap P(\Gamma_2',\Delta)$ as justified before.
	\end{enumerate}

	\item[6,7.] ($L\land_i$, {\small$i \in \{1,2\}$}) $\mathbf{D}$ proves $\Gamma_1' , \Gamma_2' , \nabla^n (A_1 \land A_2) \Rightarrow \Delta$ and has a sub-proof for $\Gamma_1' , \Gamma_2' , \nabla^n A_i \Rightarrow \Delta$.
	\begin{enumerate}
		\item If $\Gamma_1 = \Gamma_1'$ and $\Gamma_2 = \Gamma_2' , \nabla^n (A_1 \land A_2)$, take $C = C_{\langle\Gamma_1';\Gamma_2',\nabla^n A_i;\Delta\rangle}$. Then we have $\Gamma_1' \Rightarrow C$ by IH and $\Gamma_2' , \nabla^n (A_1 \land A_2) \Rightarrow \Delta$ by IH and $L\land_i$. From IH, we also have $P(C) \subseteq$ $P(\Gamma_1') \cap P(\Gamma_2',\nabla^n(A_1 \land A_2),\Delta)$, since $P(\nabla^n X) = P(X)$ and $P$ takes sub-formula ordering to ``$\subseteq$''.
		
		\item If $\Gamma_1 = \Gamma_1' , \nabla^n (A_1 \land A_2)$ and $\Gamma_2 = \Gamma_2'$, take $C = C_{\langle\Gamma_1',\nabla^n A_i;\Gamma_2';\Delta\rangle}$. Then we have $\Gamma_1' , \nabla^n (A_1 \land A_2)$ $\Rightarrow C$ by IH and $L\land_i$. Also from IH we have $\Gamma_2' \Rightarrow \Delta$. We also have $P(C) \subseteq P(\Gamma_1',\nabla^n (A_1 \land A_2))$ $\cap P(\Gamma_2',\Delta)$ as justified in the previous case.
	\end{enumerate}
	\setcounter{enumi}{7}

	\item ($R\land$) $\mathbf{D}$ proves $\Gamma_1 , \Gamma_2 \Rightarrow A \land B$ and has sub-proofs for $\Gamma_1 , \Gamma_2 \Rightarrow A$ and $\Gamma_1 , \Gamma_2 \Rightarrow B$.\\
	Let $C_1 = C_{\langle\Gamma_1;\Gamma_2;A\rangle}$ and $C_2 = C_{\langle\Gamma_1;\Gamma_2;B\rangle}$, and then take $C = C_1 \land C_2$.
	We have $\Gamma_1 \Rightarrow C_1$ and $\Gamma_1 \Rightarrow C_2$, both from IH. Then by $R\land$ we have $\Gamma_1 \Rightarrow C_1 \land C_2$.
	We also have $\Gamma_2 , C_1 \Rightarrow A$ and $\Gamma_2 , C_2 \Rightarrow B$, again from IH.
	We can then derive $\Gamma_2 , C_1 \land C_2 \Rightarrow A$ and $\Gamma_2 , C_1 \land C_2 \Rightarrow B$, respectively by $L\land_1$ and $L\land_2$, and finally  $\Gamma_2 , C_1 \land C_2 \Rightarrow A \land B$ by $R\land$.
	We also have $P(C_1) \subseteq P(\Gamma_1) \cap P(\Gamma_2 , A)$ and $P(C_2) \subseteq P(\Gamma_1) \cap P(\Gamma_2 , B)$. So $P(C_1 , C_2) \subseteq P(\Gamma_1) \cap P(\Gamma_2 , A , B)$ as it was justified before, and then $P(C_1 \land C_2) \subseteq P(\Gamma_1) \cap P(\Gamma_2 , A \land B)$.
	
	\item ($L\lor$) $\mathbf{D}$ proves $\Gamma_1' , \Gamma_2' , \nabla^n (A \lor B) \Rightarrow \Delta$ and has sub-proofs for $\Gamma_1' , \Gamma_2' , \nabla^n A \Rightarrow \Delta$ and $\Gamma_1' , \Gamma_2' , \nabla^n B \Rightarrow \Delta$.
	\begin{enumerate}
		\item If $\Gamma_1 = \Gamma_1'$ and $\Gamma_2 = \Gamma_2' , \nabla^n (A \lor B)$, let $C_1 = C_{\langle\Gamma_1';\Gamma_2',\nabla^n A;\Delta\rangle}$ and $C_2 = C_{\langle\Gamma_1';\Gamma_2',\nabla^n B;\Delta\rangle}$, and then take $C = C_1 \land C_2$.
		We have $\Gamma_1' \Rightarrow C_1 \land C_2$ from IH and $R\land$.
		From IH, by $L\land_1$ and $L\land_2$ we can derive $\Gamma_2' , \nabla^n A , C_1 \land C_2 \Rightarrow \Delta$ and $\Gamma_2' , \nabla^n B , C_1 \land C_2 \Rightarrow \Delta$ respectively, to which we apply $L\lor$ to get to $\Gamma_2' , \nabla^n (A \lor B) , C_1 \land C_2 \Rightarrow \Delta$.
		From IH, we also have $P(C_1) \subseteq P(\Gamma_1') \cap P(\Gamma_2' , \nabla^n A , \Delta)$ and $P(C_2) \subseteq P(\Gamma_1') \cap P(\Gamma_2' , \nabla^n B , \Delta)$. Just like the previous case, we can deduce that $P(C_1 \land C_2) \subseteq P(\Gamma_1') \cap P(\Gamma_2' , \nabla^n (A \land B) , \Delta)$.

		\item If $\Gamma_1 = \Gamma_1' , \nabla^n (A \lor B)$ and $\Gamma_2 = \Gamma_2'$, let $C_1 = C_{\langle\Gamma_1',\nabla^n A;\Gamma_2';\Delta\rangle}$ and $C_2 = C_{\langle\Gamma_1',\nabla^n B;\Gamma_2';\Delta\rangle}$, and then take $C = C_1 \lor C_2$.
		From IH, by $R\lor_1$ and $R\lor_2$ we can derive $\Gamma_1' , \nabla^n A \Rightarrow C_1 \lor C_2$ and $\Gamma_1' , \nabla^n B \Rightarrow C_1 \lor C_2$ respectively, to which we apply $L\lor$ to get to $\Gamma_1' , \nabla^n (A \lor B) \Rightarrow C_1 \lor C_2$.
		We have $\Gamma_2' , C_1 \lor C_2 \Rightarrow \Delta$ from IH and $L\lor$.
		From IH, we also have $P(C_1) \subseteq P(\Gamma_1' , \nabla^n A) \cap$ $P(\Gamma_2' , \Delta)$ and $P(C_2) \subseteq P(\Gamma_1' , \nabla^n B) \cap P(\Gamma_2' , \Delta)$. Just like the previous case, we can deduce that $P(C_1 \lor C_2) \subseteq P(\Gamma_1' , \nabla^n (A \land B)) \cap P(\Gamma_2' , \Delta)$.
	\end{enumerate}

	\item[10,11.] ($R\lor_i$, {\small$i \in \{1,2\}$}) $\mathbf{D}$ proves $\Gamma_1 , \Gamma_2 \Rightarrow A_1 \lor A_2$ and has a sub-proof for $\Gamma_1 , \Gamma_2 \Rightarrow A_i$. Take $C = C_{\langle\Gamma_1;\Gamma_2;A_i\rangle}$. Then we have $\Gamma_1 \Rightarrow C$ from IH and $\Gamma_2 , C \Rightarrow A_1 \lor A_2$ from IH and $R\lor_i$.
	From IH, we also have $P(C) \subseteq P(\Gamma_1) \cap P(\Gamma_2 , A_1 \lor A_2)$, as was justified before.
	\setcounter{enumi}{11}
	
	\item ($L\rightarrow$) $\mathbf{D}$ proves $\Gamma_1' , \Gamma_2' , \nabla^{n+1} (A \rightarrow B) \Rightarrow \Delta$ and has sub-proofs for $\Gamma_1' , \Gamma_2' \Rightarrow \nabla^n A$ and $\Gamma_1' , \Gamma_2' , \nabla^n B \Rightarrow \Delta$.
	\begin{enumerate}
		\item If $\Gamma_1 = \Gamma_1'$ and $\Gamma_2 = \Gamma_2' , \nabla^{n+1} (A \rightarrow B)$, let $C_1 = C_{\langle\Gamma_1';\Gamma_2';\nabla^n A\rangle}$ and $C_2 = C_{\langle\Gamma_1';\Gamma_2',\nabla^n B;\Delta\rangle}$, and take $C = C_1 \land C_2$.
		We have $\Gamma_1' \Rightarrow C_1 \land C_2$ from IH and $R\land$.
		From IH, by $L\land_1$ and $L\land_2$ we can derive $\Gamma_2' , C_1 \land C_2 \Rightarrow \nabla^n A$ and $\Gamma_2' , \nabla^n B , C_1 \land C_2 \Rightarrow \Delta$ respectively, to which we apply $L\rightarrow$ to get to $\Gamma_2' , \nabla^{n+1} (A \rightarrow B) , C_1 \land C_2 \Rightarrow \Delta$.
		From IH, we also have $P(C_1) \subseteq P(\Gamma_1') \cap$ $P(\Gamma_2' , \nabla^n A)$ and $P(C_2) \subseteq P(\Gamma_1') \cap P(\Gamma_2' , \nabla^n B , \Delta)$. This implies $P(C_1 \land C_2) \subseteq P(\Gamma_1') \cap P(\Gamma_2' , \nabla^{n+1} (A \rightarrow B) , \Delta)$.

		\item If $\Gamma_1 = \Gamma_1' , \nabla^{n+1} (A \rightarrow B)$ and $\Gamma_2 = \Gamma_2'$, let $C_1 = C_{\langle\Gamma_2';\Gamma_1';\nabla^n A\rangle}$ and $C_2 = C_{\langle\Gamma_1',\nabla^n B;\Gamma_2';\Delta\rangle}$, and take $C = \nabla (C_1 \rightarrow C_2)$.
		From IH we have $\Gamma_1' , C_1 \Rightarrow \nabla^n A$. Also from IH, with a $Lw$ to add $C_1$ to the left, we have $\Gamma_1' , \nabla^n B , C_1 \Rightarrow C_2$. By applying $L\rightarrow$ we get $\Gamma_1 , \nabla^{n+1} (A \rightarrow B) , C_1 \Rightarrow C_2$.
		\todo{}
		{\color{red} If $\nabla C_1 \rightarrow \nabla C_2 \Rightarrow \nabla (C_1 \rightarrow C_2)$}
		\begin{prooftree}
			\AXC{$\Gamma_1 , \nabla^{n+1} (A \rightarrow B) , C_1 \Rightarrow C_2$}
			\RightLabel{$N'$}
			\UIC{$\nabla \Gamma_1 , \nabla^{n+2} (A \rightarrow B) , \nabla C_1 \Rightarrow \nabla C_2$}
			\RightLabel{$R\rightarrow$}
			\UIC{$\Gamma_1 , \nabla^{n+1} (A \rightarrow B) \Rightarrow \nabla C_1 \rightarrow \nabla C_2$}
			
			\AXC{}
			\RightLabel{{\color{red}Only if this was true}}
			\UIC{$\nabla C_1 \rightarrow \nabla C_2 \Rightarrow \nabla (C_1 \rightarrow C_2)$}

			\RightLabel{$Cut$\tiny of the other logic}
			\BIC{$\Gamma_1 , \nabla^{n+1} (A \rightarrow B) \Rightarrow \nabla (C_1 \rightarrow C_2)$}
		\end{prooftree}
		{\color{red} Or if there was an intuitionistic implication $\supset$, we could take $C = C_1 \supset C_2$}
		\begin{prooftree}
			\AXC{$\Gamma_1 , \nabla^{n+1} (A \rightarrow B) , C_1 \Rightarrow C_2$}
			\RightLabel{$\supset$}
			\UIC{$\Gamma_1 , \nabla^{n+1} (A \rightarrow B) \Rightarrow C_1 \supset C_2$}
		\end{prooftree}


		We have from IH $\Gamma_2' \Rightarrow C_1$ and $\Gamma_2' , C_2 \Rightarrow \Delta$, from which we can derive $\Gamma_2' , \nabla (C_1 \rightarrow C_2)$ by an application of $L\rightarrow$. We also have from IH $P(C_1) \subseteq P(\Gamma_2') \cap P(\Gamma_1' , \nabla^n A)$ and $P(C_2) \subseteq P(\Gamma_1' , \nabla^n B) \cap P(\Gamma_2' , \Delta)$. Then $P(\nabla (C_1 \rightarrow C_2)) \subseteq P(\Gamma_1' , \nabla^{n+1} (A \rightarrow B)) \cap P(\Gamma_2' , \Delta)$.
	\end{enumerate}

	\item ($R\rightarrow$) $\mathbf{D}$ proves $\Gamma_1 , \Gamma_2 \Rightarrow A \rightarrow B$ and has a sub-proof for $\nabla \Gamma_1 , \nabla \Gamma_2 , A \Rightarrow B$. Let $C = C_{\langle\Gamma_1';\Gamma_2',A;B\rangle}$.
	\todo{}
	{\color{red} Intuitionistic implication does not help.\\  If we strengthen the theorem so that interpolants are $\nabla$-irrelevant, i.e. $C \Leftrightarrow \nabla C$}
	\begin{prooftree}
		\AXC{} \RightLabel{IH}
		\UIC{$\nabla C \Rightarrow C$}

		\AXC{} \RightLabel{IH}
		\UIC{$\nabla \Gamma_2 , A , C \Rightarrow B$}

		\RightLabel{$Cut$}
		\BIC{$\nabla \Gamma_2 , A , \nabla C \Rightarrow B$}
		\RightLabel{$R\rightarrow$}
		\UIC{$\Gamma_2 , C , \Rightarrow A \rightarrow B$}
	\end{prooftree}
	IH also gives $\nabla \Gamma_1 \Rightarrow C$ and $C \Rightarrow \nabla C$. We cut them to $\nabla \Gamma_1 \Rightarrow \nabla C$. Since $N'$ is invertible {\color{red} (it is so)}, we have $\Gamma_1 \Rightarrow C$. {\color{red} Constructed interpolant in other cases are also $\nabla$-irrelevant, except $L\rightarrow$, that needs $\nabla C_1 \rightarrow \nabla C_2 \Rightarrow \nabla (C_1 \rightarrow C_2)$ again.}
	
	\item ($N'$) $\mathbf{D}$ proves $\nabla \Gamma_1 , \nabla \Gamma_2 \Rightarrow \nabla \Delta$ and has a sub-proof for $\Gamma_1 , \Gamma_2 \Rightarrow \Delta$. Just take $C = C(\Gamma_1;\Gamma_2;\Delta)$ and apply $N'$ on the sequents from IH. The variable condition is also trivial from IH.
\end{enumerate}
\end{document}
