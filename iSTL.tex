
\section{iSTL$^-$} An iSTL sequent $\Gamma \Rightarrow \Delta$ is a binary relation between $\Gamma$, a multi-set of formulas, and $\Delta$, a sub-singleton of some formula, defined inductively by the following rules.
\begin{multicols}{3}
	\begin{prooftree}
		\AXC{}
		\RightLabel{$Id$}
		\UIC{$A \Rightarrow A$}
	\end{prooftree}
\columnbreak
	\begin{prooftree}
		\AXC{}
		\RightLabel{$Ta$}
		\UIC{$\Rightarrow \top$}
	\end{prooftree}
\columnbreak
	\begin{prooftree}
		\AXC{}
		\RightLabel{$Ex$}
		\UIC{$\bot \Rightarrow$}
	\end{prooftree}
\end{multicols}
‌\\
\begin{multicols}{3}
	\begin{prooftree}
		\RightLabel{$Lw$}
		\AXC{$\Gamma \Rightarrow \Delta$}
		\UIC{$\Gamma , A \Rightarrow \Delta$}
	\end{prooftree}
	\columnbreak
	\begin{prooftree}
		\RightLabel{$Rw$}
		\AXC{$\Gamma \Rightarrow$}
		\UIC{$\Gamma \Rightarrow A$}
	\end{prooftree}
	\columnbreak
	\begin{prooftree}
		\RightLabel{$Lc$}
		\AXC{$\Gamma , A , A \Rightarrow \Delta$}
		\UIC{$\Gamma , A \Rightarrow \Delta$}
	\end{prooftree}
\end{multicols}
‌\\
\begin{multicols}{3}
	\begin{prooftree}
		\RightLabel{$L\land_1$}
		\AXC{$\Gamma , A \Rightarrow \Delta$}
		\UIC{$\Gamma , A \land B \Rightarrow \Delta$}
	\end{prooftree}
	\columnbreak
	\begin{prooftree}
		\RightLabel{$L\land_2$}
		\AXC{$\Gamma , B \Rightarrow \Delta$}
		\UIC{$\Gamma , A \land B \Rightarrow \Delta$}
	\end{prooftree}
	\columnbreak
	\begin{prooftree}
		\RightLabel{$R\land$}
		\AXC{$\Gamma \Rightarrow A$}
		\AXC{$\Gamma \Rightarrow B$}
		\BIC{$\Gamma \Rightarrow A \land B$}
	\end{prooftree}
\end{multicols}
‌\\
\begin{multicols}{3}
	\begin{prooftree}
		\RightLabel{$L\lor$}
		\AXC{$\Gamma , A \Rightarrow \Delta$}
		\AXC{$\Gamma , B \Rightarrow \Delta$}
		\BIC{$\Gamma , A \lor B \Rightarrow \Delta$}
	\end{prooftree}
	\columnbreak
	\begin{prooftree}
		\RightLabel{$R\lor_1$}
		\AXC{$\Gamma \Rightarrow A$}
		\UIC{$\Gamma \Rightarrow A \lor B$}
	\end{prooftree}
	\columnbreak
	\begin{prooftree}
		\RightLabel{$R\lor_2$}
		\AXC{$\Gamma \Rightarrow B$}
		\UIC{$\Gamma \Rightarrow A \lor B$}
	\end{prooftree}
\end{multicols}
‌\\
\begin{multicols}{2}
	\begin{prooftree}
		\RightLabel{$L\rightarrow$}
		\AXC{$\Gamma \Rightarrow A$}
		\AXC{$\Gamma , B \Rightarrow \Delta$}
		\BIC{$\Gamma, \nabla(A \rightarrow B) \Rightarrow \Delta$}
	\end{prooftree}
	\columnbreak
	\begin{prooftree}
		\RightLabel{$R\rightarrow$}
		\AXC{$\nabla\Gamma , A \Rightarrow B$}
		\UIC{$\Gamma \Rightarrow A \rightarrow B$}
	\end{prooftree}
\end{multicols}
‌\\
\begin{prooftree}
	\RightLabel{$N$}
	\AXC{$\Gamma \Rightarrow A$}
	\UIC{$\nabla\Gamma \Rightarrow \nabla A$}
\end{prooftree}
‌\\
\subsection{Cut}
\begin{prooftree}
	\RightLabel{$Cut$}
	\AXC{$\Gamma \Rightarrow A$}
	\AXC{$\Gamma' , A \Rightarrow \Delta$}
	\BIC{$\Gamma , \Gamma' \Rightarrow \Delta$}
\end{prooftree}
By $\text{iSTL}$ we mean $\text{iSTLL}^- + Cut$

\subsection{Lemma} iSTL proves

\subsubsection{}\label{lem:i-nabla-dist-and} $\nabla^n (A \land B) \Rightarrow \nabla^n A \land \nabla^n B$

\textit{Proof}:
\begin{prooftree}
	\AXC{}
	\RightLabel{$Id$}
	\UIC{$A \Rightarrow A$}
	\RightLabel{$L\land_1$}
	\UIC{$A \land B \Rightarrow A$}
	\RightLabel{$N$} \doubleLine
	\UIC{$\nabla^n (A \land B) \Rightarrow \nabla^n A$}

	\AXC{}
	\RightLabel{$Id$}
	\UIC{$B \Rightarrow B$}
	\RightLabel{$L\land_2$}
	\UIC{$A \land B \Rightarrow B$}
	\RightLabel{$N$} \doubleLine	
	\UIC{$\nabla^n (A \land B) \Rightarrow \nabla^n B$}
	
	\RightLabel{$R\land$}
	\BIC{$\nabla^n (A \land B) \Rightarrow \nabla^n A \land \nabla^n B$}
\end{prooftree}

\subsubsection{}\label{lem:i-nabla-dist-or} $\nabla^n (A \lor B) \Rightarrow \nabla^n A \lor \nabla^n B$ \todo{}

\subsubsection{}\label{lem:i-nabla-dist-imp} $\nabla^n (A \rightarrow B) \Rightarrow \nabla^n A \rightarrow \nabla^n B$

\textit{Proof}:
\begin{prooftree}
	\AXC{}
	\RightLabel{$Id$}
	\UIC{$A \Rightarrow A$}
	
	\AXC{}
	\RightLabel{$Id$}
	\UIC{$B \Rightarrow B$}
	\RightLabel{$Lw$}
	\UIC{$A , B \Rightarrow B$}
	
	\RightLabel{$L\rightarrow$}
	\BIC{$\nabla (A \rightarrow B) , A \Rightarrow B$}
	\RightLabel{$N$} \doubleLine
	\UIC{$\nabla^{n+1} (A \rightarrow B) , \nabla^n A \Rightarrow \nabla^n B$}
	\RightLabel{$R\rightarrow$}
	\UIC{$\nabla^n (A \rightarrow B) \Rightarrow \nabla^n A \rightarrow \nabla^n B$}
\end{prooftree}

\subsubsection{}\label{lem:i-nabla-fact-imp} $\nabla^n A \rightarrow \nabla^n B \Rightarrow \nabla^n (A \rightarrow B)$ \todo{is this even possible?}