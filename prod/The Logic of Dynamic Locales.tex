\documentclass[12pt,a4paper]{article}
\usepackage{amsmath}
\usepackage{amsthm}
\usepackage{amsfonts}
%\usepackage{fdsymbol}

\usepackage{authblk}

\usepackage[mathscr]{eucal}

\usepackage{bussproofs}
\EnableBpAbbreviations

\usepackage{amssymb}
\usepackage{tikz}
\usepackage{enumitem}
\tikzset{node distance=2cm, auto}

\theoremstyle{plain}
\newtheorem{thm}{Theorem}[section]

\renewcommand{\thethm}{\arabic{section}.\arabic{thm}}
\newtheorem{lem}[thm]{Lemma}

\newtheorem{cor}[thm]{Corollary}
\theoremstyle{definition}
\newtheorem{dfn}[thm]{Definition}
\newtheorem{exam}[thm]{Example}
\newtheorem{rem}[thm]{Remark}
\newtheorem{nota}[thm]{Notation}
\newtheorem{exer}[thm]{Exercise}

\def\d{\displaystyle}
\def\PA{\mathrm{PA}}
\def\Pr{\mathrm{Pr}}
\def\Prf{\mathrm{Prf}}
\def\PR{\mathrm{PR}}
\def\IPC{\mathrm{IPC}}
\def\Proofs{\mathrm{Proofs}}
\def\int{\mathrm{int}}
\def\WT{\mathrm{WT}}
\def\exp{\mathrm{exp}}
\def\CHaus{\mathrm{CHaus}}
\def\Fin{\mathrm{Fin}}
\def\E{\mathrm{E}}
\def\PR{\mathrm{PR}}
\def\Top{\mathrm{Top}}
\def\S4{\mathrm{S4}}
\def\Hom{\mathrm{Hom}}
\def\Set{\mathrm{Set}}

\begin{document}

\title{The Logic of Dynamic Locales}


\author[]{Amirhossein Akbar Tabatabai, Alireza Mahmoudian}

\affil[]{ }


\date{ }

\maketitle


\begin{abstract}
In this paper we will first introduce dynamic locales as the point-free intuitionistic version of the dynamic topological systems. Then we will introduce the logic of the dynamic locales, $\mathbf{LDL}$, as a base for the intuitionistic version for classical dynamic topological logic, introduced in \cite{Artemov}. We will then present cut-free sequent-style proof systems for the logic $\mathbf{LDL}$ and one of its important fragments. Using the systems, we will finally provide a syntactical proof for the admissibly of a natural extension of Visser rules and the appropriate interpolation property that these logic enjoy.
\end{abstract}

\section{Introduction}


\section{Dynamic Locales and Their Logic} \label{DynamicLocales}

\begin{dfn}
A locale $\mathscr{X}$ is a complete poset whose meet distributes over any arbitrary family of joins, i.e., $a \wedge \bigvee_{i \in I} b_i=\bigvee_{i \in I} (a \vee b_i)$, for any $a, b_i \in \mathscr{X}$. A localic map $f: \mathscr{X} \to \mathscr{Y}$ between two locales $\mathscr{X}$ and $\mathscr{Y}$ is a map preserving all finite meets and arbitrary joins. It is called an isomorphism if it has a converse as a localic map.
\end{dfn}

The canonical example of locales and the localic maps are the poset of the open subsets of a topological space and the inverse image of the continuous functions between spaces, respectively. In this sense, locales and the localic maps provide a point-free formalization for the topological discourse, focusing only on open subsets as the main ingredient of topology.

\begin{dfn}
Let $\mathscr{X}$ be a locale and $\nabla: \mathscr{X} \to \mathscr{X}$ be a localic map. Then $\mathcal{D}=(\mathscr{X}, \nabla)$ is called a dynamic locale. It is called invertible if $\nabla$ is an isomorphism.
\end{dfn}

Dynamic locales are the intuitionistic and symplified version of the dynamic topological spaces defined in \cite{}. Here, we force the formalization to restrict itself to the open subsets. However, to represent the non-elementary structure of dynamic locales in a syntactical elementary language, we need to capture the distributivity of meets over arbitrary joins and the fact that $\nabla$ preserves any arbitrary joins. The following easy observation helps:

\begin{thm}(Adjoint Functor Theorem for Posets) \label{AFT}
Let $\mathcal{A}$ and $\mathcal{B}$ be two complete posets and $f:\mathcal{A} \to \mathcal{B}$ be an order-preserving function. Then $f$ preserves arbitrary joins iff there exists an order-preserving function $g: \mathcal{B} \to \mathcal{A}$ such that
\[
f(a) \leq_{\mathcal{B}} b  \;\;\; \text{iff}  \;\;\; a \leq_{\mathcal{A}} g(b)
\]
This $g$ if exits is clearly unique and called the right adjoint of $f$.
\end{thm}
\begin{proof}
If there exists such $g$ then $f$ preserves all joins... For the other half, define $g(b)=\bigvee \{a \in \mathcal{A} | f(a) \leq_{\mathcal{B}} b \}$. One direction is obvious from the definition. The other is a consequence of the fact that $f$ preserves all joins.
\end{proof}

Using Theorem \ref{AFT} on a dynamic locale $\mathcal{D}=(\mathscr{X}, \nabla_{\mathcal{D}})$, we can observe that the facts that the maps $x \mapsto x \wedge a$ and $x \mapsto \nabla x$ preserve arbitrary joins, are representable by introducing their right adjoints into the language as the primitive connectives. For the former, the adjoint is nothing but the usual Heyting implication. For the latter, we need to add a new unary connective $\Box$ representing the function denoted by $\Box_{\mathcal{D}}$, satisfying
\[
\nabla_{\mathcal{D}} a \leq_{\mathscr{X}} b  \;\;\; \text{iff}  \;\;\; a \leq_{\mathscr{X}} \Box_{\mathcal{D}} b
\]

In the following, we will define the logic of dynamic locales, denoted by $\mathbf{LDL}$. Let $\mathcal{L}$ be the usual language of propositional logic plus the modalities $\nabla$ and $\Box$, i.e., $\mathcal{L}=\{\wedge, \vee, \to, \top, \bot, \nabla, \Box\}$. Define $\mathbf{LDL}$ as the logic of the sequent-style system defined by the following rules:
\begin{flushleft}
 \textbf{Axioms:}
\end{flushleft}
\begin{center}
 \begin{tabular}{c c c}
 \AxiomC{}
 \RightLabel{$Id$}
 \UnaryInfC{$ A \Rightarrow A$}
 \DisplayProof \;\;\;
 &
 \AxiomC{}
 \RightLabel{$Ta$}
 \UnaryInfC{$ \Rightarrow \top$}
 \DisplayProof\;\;\;
 &
 \AxiomC{}
 \RightLabel{$Ex$}
 \UnaryInfC{$ \bot \Rightarrow $}
 \DisplayProof
 \\[3ex]
\end{tabular}
\end{center}

\begin{flushleft}
 		\textbf{Structural Rules:}
\end{flushleft}

\begin{center}
 \begin{tabular}{c c c}
 \AxiomC{$ \Gamma \Rightarrow \Delta$}
 \RightLabel{$L w$}
 \UnaryInfC{$ \Gamma, A \Rightarrow \Delta$}
 \DisplayProof
 &
 \AxiomC{$ \Gamma \Rightarrow $}
\RightLabel{$R w$}
 \UnaryInfC{$\Gamma \Rightarrow A$}
 \DisplayProof
 &
 \AxiomC{$ \Gamma, A, A \Rightarrow \Delta$}
\RightLabel{$Lc$}
 \UnaryInfC{$\Gamma, A \Rightarrow \Delta$}
 \DisplayProof
  \\[3ex]
\end{tabular}
\end{center}

\begin{flushleft}
 		\textbf{Cut:}
\end{flushleft}
\begin{center}
  	\begin{tabular}{c}

		\AxiomC{$ \Gamma \Rightarrow A$}
		\AxiomC{$\Pi, A \Rightarrow \Delta$}
		\RightLabel{$cut$}
		\BinaryInfC{$ \Pi, \Gamma \Rightarrow \Delta$}
		\DisplayProof
		 \\[3ex]
		\end{tabular}
\end{center}

\begin{flushleft}
 \textbf{Conjunction Rules:}
\end{flushleft}
\begin{center}
 \begin{tabular}{c c c}
\AxiomC{$ \Gamma, A \Rightarrow \Delta$}
 \RightLabel{$L \wedge_1$}
 \UnaryInfC{$ \Gamma, A \wedge B \Rightarrow \Delta$}
 \DisplayProof
 &
 \AxiomC{$ \Gamma, B \Rightarrow \Delta$}
 \RightLabel{$L \wedge_2$}
 \UnaryInfC{$\Gamma, A \wedge B \Rightarrow \Delta$}
 \DisplayProof
	   		&
   		\AxiomC{$\Gamma \Rightarrow A$}
   		\AxiomC{$\Gamma \Rightarrow B$}
   		\RightLabel{$R \wedge$}
   		\BinaryInfC{$ \Gamma \Rightarrow A \wedge B $}
   		\DisplayProof
   			\\[3 ex]
\end{tabular}
\end{center}

\begin{flushleft}
 \textbf{Disjunction Rules:}
\end{flushleft}
\vspace{.001pt}
\begin{center}
 \begin{tabular}{c c c}
 \AxiomC{$ \Gamma, A \Rightarrow \Delta$}
 \AxiomC{$\Gamma, B \Rightarrow \Delta$}
 \RightLabel{$L \vee_1$}
 \BinaryInfC{$ \Gamma, A \vee B \Rightarrow \Delta$}
 \DisplayProof
 &
 \AxiomC{$\Gamma \Rightarrow A$}
 \RightLabel{$R \vee_2$}
 \UnaryInfC{$\Gamma \Rightarrow A \vee B$}
 \DisplayProof
 &
 \AxiomC{$\Gamma \Rightarrow B$}
 \RightLabel{$R \vee$}
 \UnaryInfC{$\Gamma \Rightarrow A \vee B$}
 \DisplayProof
 \\[3ex]
\end{tabular}
\end{center}

\begin{flushleft}
	\textbf{Implication Rules:}
 \end{flushleft}
 \vspace{.001pt}
 \begin{center}
	\begin{tabular}{c c c}
	\AxiomC{$ \Gamma \Rightarrow A$}
	\AxiomC{$\Gamma, B \Rightarrow \Delta$}
	\RightLabel{$L \rightarrow$}
	\BinaryInfC{$ \Gamma, A \rightarrow B \Rightarrow \Delta$}
	\DisplayProof
	&
	\AxiomC{$\Gamma , A \Rightarrow B$}
	\RightLabel{$R \rightarrow$}
	\UnaryInfC{$\Gamma \Rightarrow A \rightarrow B$}
	\DisplayProof
	\\[3ex]
 \end{tabular}
 \end{center}

\begin{flushleft}
  \textbf{$\nabla$ Rules:}
\end{flushleft}
\vspace{.001pt}
\begin{center}
 \begin{tabular}{c}
 \AxiomC{$\Gamma \Rightarrow A$}
 \RightLabel{$N$}
 \UnaryInfC{$\nabla \Gamma \Rightarrow \nabla A$}
 \DisplayProof
 \\[3ex]
\end{tabular}
\end{center}

\begin{flushleft}
 \textbf{$\Box$ Rules:}
\end{flushleft}
\vspace{.001pt}
\begin{center}
 \begin{tabular}{c c}
 \AxiomC{$\Gamma, A \Rightarrow \Delta$}
 \RightLabel{$L \Box$}
 \UnaryInfC{$\Gamma, \nabla \Box A \Rightarrow \Delta$}
 \DisplayProof
 &
 \AxiomC{$\nabla \Gamma \Rightarrow A$}
 \RightLabel{$R \Box$}
 \UnaryInfC{$\Gamma \Rightarrow \Box A$}
 \DisplayProof
 \\[3ex]
\end{tabular}
\end{center}
If we also add the following rule to the system we have the logic $\mathbf{LDL}(H)$:
\begin{flushleft}
 \textbf{The Rule $H$:}
\end{flushleft}
\vspace{.001pt}
\begin{center}
 \begin{tabular}{c}
 \AxiomC{$\nabla \Gamma \Rightarrow \nabla A$}
 \RightLabel{$H$}
 \UnaryInfC{$\Gamma \Rightarrow \nabla \Box A$}
 \DisplayProof
 \\[3ex]
\end{tabular}
\end{center}

\begin{dfn}\label{t4-1}(Topological Semantics)
Let $\mathcal{D}=(\mathscr{X}, \nabla_{\mathcal{D}})$ be a dynamic locale and $V:\mathcal{L} \to\mathscr{X}$ be an assignment. A tuple $(\mathcal{D}, V)$ is called a topological model if:
\begin{itemize}
\item[$\bullet$]
$V(\top)=1$ and $V(\bot)=0$,
\item[$\bullet$]
$V(A \wedge B)=V(A) \wedge V(B)$,
\item[$\bullet$]
$V(A \vee B)=V(A) \vee V(B)$,
\item[$\bullet$]
$V(\nabla A)=\nabla_{\mathcal{D}} V(A)$,
\item[$\bullet$]
$V(\Box A)= \Box_{\mathcal{D}} V(A)$.
\end{itemize}
We say $(\mathcal{D}, V) \vDash \Gamma \Rightarrow \Delta$ when $\bigwedge_{\gamma \in \Gamma} V(\gamma) \leq \bigvee_{\delta \in \Delta} V(\delta)$ and $\mathcal{D} \vDash \Gamma \Rightarrow \Delta$ when for all $V$, $(\mathcal{D}, V) \vDash \Gamma \Rightarrow \Delta$.
\end{dfn}

\begin{thm}\label{t4-2}(Soundness-Completeness)
\begin{itemize}
\item[$(i)$]
$ \mathbf{LDL} \vdash \Gamma \Rightarrow \Delta$ iff $\mathcal{D} \vDash \Gamma \Rightarrow \Delta$, for any dynamic locale $\mathcal{D}$.
\item[$(ii)$]
$ \mathbf{LDL}(H) \vdash \Gamma \Rightarrow \Delta$ iff $\mathcal{D} \vDash \Gamma \Rightarrow \Delta$,  for any invertible dynamic locale $\mathcal{D}$.
\end{itemize}

\end{thm}
\begin{proof}
Basically refer to \cite{Amir}.
\end{proof}

\subsection{Kripke Models} \label{KripkeModels}

\begin{dfn}
By a Kripke model for the language $\mathcal{L}$, we mean a tuple $\mathcal{K}=(W, \leq, R, V)$ where $(W, \leq)$ is a poset, $R \subseteq W \times W$ is a relation over $W$ (not necessarily transitive or reflexive) compatible with $\leq$, i.e., for all $u, u', v, v' \in W$ if $(u, v) \in R$ and $u' \leq u$ and $v \leq v'$ then $(u', v') \in R$ and $V: At(\mathcal{L}) \to U((W, \leq))$, where $At(\mathcal{L})$ is the set of atomic formulas of $\mathcal{L}$ and $U((W, \leq))$ is the set of all upsets of $(W, \leq)$. Define the forcing relation as usual using the relation $\leq$ for the intuitionistic implication and $R$ for $\Box$, and for the $\nabla$ let $u \Vdash \nabla A$ if there exists $v \in W$ such that $(v, u) \in R$ and $v \Vdash A$. A Kripke model is called normal if there exists an order preserving function $\pi : W \to W$ such that $(u, v) \in R$ iff $u \leq \pi(v)$. It is clear that if this $\pi$ exists, it would be unique. Finally, a sequent $\Gamma \Rightarrow \Delta$ is valid in a Kripke model if for all $w \in W$, $w \Vdash \bigwedge \Gamma$ implies $w \Vdash \bigvee \Delta$.
\end{dfn}

\begin{thm}(Soundness-Completeness)
\begin{itemize}
\item[$(i)$]
The logic $\mathbf{LDL}$ is sound and complete with respect to all normal Kripke models.
\item[$(ii)$]
The logic $\mathbf{LDL}(H)$ is sound and complete with respect to all normal Kripke models whose functions $\pi$ are order-isomorphism.
\end{itemize}
\end{thm}

\subsection{The Fragment $\mathbf{LDL}_{\rightsquigarrow}$}
In this subsection we will introduce a fragment of the logic $\mathbf{LDL}$. To explain why we find this fragment interesting, let us begin with a formalization for a general notion of implication, introduced in \cite{Amir}:
\begin{dfn}
Let $\mathcal{A}=(A, \leq, \wedge, 1)$ be a meet semi-lattice. A map $\rightsquigarrow : A^{op} \times A \to A$ is called a meet-internalizing implication if:
\begin{itemize}
\item[$\bullet$]
$a \rightsquigarrow a=1$, for any $a \in A$.
\item[$\bullet$]
$(a \rightsquigarrow b) \wedge (b \rightsquigarrow c) \leq (a \rightsquigarrow c)$, for any $a, b, c \in A$.
\item[$\bullet$]
$(a \rightsquigarrow b) \wedge (a \rightsquigarrow c)= a \rightsquigarrow (b \wedge c)$, for any $a, b, c \in A$.
\end{itemize}
The tuple $(A, \leq, \wedge, 1, \rightsquigarrow)$ is called a meet-internalizing strong algebra. An embedding from a meet-internalizing strong algebra $(A, \leq_A, \wedge_A, 1_A, \rightsquigarrow_A)$ to another meet-internalizing strong algebra $(B, \leq_B, \wedge_B, 1_B, \rightsquigarrow_B)$ is a map $i : A \to B$, preserving all structures such that if $i(a) \leq_B i(b)$ then $a \leq_A b$.
\end{dfn}
The main source to produce these implications is dynamic locales. Let $\mathcal{D}=(\mathscr{X}, \nabla)$ be a dynamic locale. Then define $a \rightsquigarrow_{\nabla} b=\Box_{\mathcal{D}}(a \to b)$. It is not hard to show that this binary operation is a meet-internalizing implication. Denote this meet-internalizing strong algebra by $\mathcal{A}(\mathcal{D})$. The following theorem, proved in \cite{Amir}, shows that this example is essentially the only example we may have:
\begin{thm}(Representation theorem)
Let $\mathcal{A}=(A, \wedge, \rightsquigarrow)$ be a meet-internalizing strong algebra. Then there exists a dynamic locale $\mathcal{D}=(\mathscr{X}, \nabla)$ and an embedding $i: \mathcal{A} \to \mathcal{A}(\mathcal{D})$.
\end{thm}

This representation theorem justifies focusing on the fragment of $\mathbf{LDL}$, where the implication $\rightsquigarrow$ is present and the usual intuitionistic implication and $\Box$ are both omitted. Note that the representation theorem implies that the study of such a fragment is actually the study of all possible meet-internalizing implications. It is also possible to omit the modality $\nabla$ to have a more faithful syntax for implications. This is an established approach in sub-intuitionistic logic community. However, the draw back here is that in the absence of $\nabla$, the implication becomes proof-theoretically ill-behaved.\\

Define the fragment $\mathbf{LDL}_{\rightsquigarrow}$ of $\mathbf{LDL}$ as the system over the language $\mathcal{L}_{\rightsquigarrow}=\mathcal{L}-\{\to, \Box\} \cup \{\rightsquigarrow\}$, consisting of the axioms, structural rules and propositional rules and the rule $N$ of $\mathbf{LDL}$,  together with the following rules for $\rightsquigarrow$:

\begin{flushleft}
 \textbf{Implication Rules:}
\end{flushleft}
\vspace{.001pt}
\begin{center}
 \begin{tabular}{c c}
 \AxiomC{$\Gamma \Rightarrow A$}
 \AxiomC{$\Gamma, B \Rightarrow \Delta$}
 \RightLabel{$L \rightsquigarrow$}
 \BinaryInfC{$\Gamma, \nabla (A \rightsquigarrow B) \Rightarrow \Delta$}
 \DisplayProof
 &
 \AxiomC{$\nabla \Gamma, A \Rightarrow B$}
 \RightLabel{$R \rightsquigarrow$}
 \UnaryInfC{$\Gamma \Rightarrow A \rightsquigarrow B$}
 \DisplayProof
 \\[3ex]
\end{tabular}
\end{center}

If we also add the following rule to the system, we denote the logic by $\mathbf{LDL}_{\rightsquigarrow}(H_\rightsquigarrow)$:

\begin{flushleft}
 \textbf{The Rule $H_{\rightsquigarrow}$:}
\end{flushleft}
\vspace{.001pt}
\begin{center}
 \begin{tabular}{c}
 \AxiomC{$\nabla \Gamma, \nabla A \Rightarrow \nabla B$}
 \RightLabel{$H$}
 \UnaryInfC{$\Gamma \Rightarrow \nabla (A \rightsquigarrow B)$}
 \DisplayProof
 \\[3ex]
\end{tabular}
\end{center}

\begin{thm}
The systems $\mathbf{LDL}$ and $\mathbf{LDL}(H)$ are conservative extensions of the systems $\mathbf{LDL}_{\rightsquigarrow}$ and $\mathbf{LDL}_{\rightsquigarrow}(H_\rightsquigarrow)$, respectively.
\end{thm}
\begin{proof}
Write the easier part. The harder part must be essentially to completeness theorems of \cite{Amir}.
\end{proof}


\section{Proof Theory of $\mathbf{LDL}$ and $\mathbf{LDL}_{\rightsquigarrow}$}

\subsection{Sequent-style Systems}
In this section we will introduce two slightly modified systems $\mathbf{LDL}^*$ and $\mathbf{LDL}_{\rightsquigarrow}^*$, equivalent to $\mathbf{LDL}$ and $\mathbf{LDL}_{\rightsquigarrow}$ respectively. Their advantage is that they are, as it will be shown, cut-free.

Define $\mathbf{LDL}^*$ on the language $\mathcal{L}$, as the logic of the sequent-style system define by the same rules of $\mathbf{LDL}$, excpet the rules $Ex$, $L\wedge_1$, $L\wedge_2$, $L\vee$, $L\rightarrow$, $N$ and $L\Box$, which are replaced by the following generalized rules respectively:

	 \begin{center}
		\begin{tabular}{c}
		\AxiomC{}
		\RightLabel{$Ex$}
		\UnaryInfC{$\nabla^n \bot \Rightarrow$}
		\DisplayProof
		\\[3ex]
	 \end{tabular}
	 \end{center}

	 \begin{center}
		\begin{tabular}{c c}
	 \AxiomC{$\Gamma, \nabla^n A \Rightarrow \Delta$}
		\RightLabel{$L \wedge_1$}
		\UnaryInfC{$\Gamma, \nabla^n (A \wedge B) \Rightarrow \Delta$}
		\DisplayProof
		&
		\AxiomC{$ \Gamma, \nabla^n B \Rightarrow \Delta$}
		\RightLabel{$L \wedge_2$}
		\UnaryInfC{$\Gamma, \nabla^n (A \wedge B) \Rightarrow \Delta$}
		\DisplayProof
		\\[3 ex]
	 \end{tabular}
	 \end{center}

	 \vspace{.001pt}
	 \begin{center}
		\begin{tabular}{c}
		\AxiomC{$\Gamma, \nabla^n A \Rightarrow \Delta$}
		\AxiomC{$\Gamma, \nabla^n B \Rightarrow \Delta$}
		\RightLabel{$L \vee$}
		\BinaryInfC{$\Gamma, \nabla^n (A \vee B) \Rightarrow \Delta$}
		\DisplayProof
		\\[3ex]
	 \end{tabular}
	 \end{center}

	 \vspace{.001pt}
	 \begin{center}
		\begin{tabular}{c c}
		\AxiomC{$ \Gamma \Rightarrow \nabla^n A$}
		\AxiomC{$\Gamma, \nabla^n B \Rightarrow \Delta$}
		\RightLabel{$L \rightarrow$}
		\BinaryInfC{$ \Gamma, \nabla^n (A \rightarrow B) \Rightarrow \Delta$}
		\DisplayProof
		\\[3ex]
	 \end{tabular}
	 \end{center}

	 \vspace{.001pt}
	 \begin{center}
		\begin{tabular}{c}
		\AxiomC{$\Gamma \Rightarrow \Delta$}
		\RightLabel{$N$}
		\UnaryInfC{$\nabla \Gamma \Rightarrow \nabla \Delta$}
		\DisplayProof
		\\[3ex]
	 \end{tabular}
	 \end{center}

	 \vspace{.001pt}
	 \begin{center}
		\begin{tabular}{c}
		\AxiomC{$\Gamma, \nabla^n A \Rightarrow \Delta$}
		\RightLabel{$L \Box$}
		\UnaryInfC{$\Gamma, \nabla^{n+1} \Box A \Rightarrow \Delta$}
		\DisplayProof
		\\[3ex]
	 \end{tabular}
	 \end{center}

Similarly for $\mathbf{LDL}_{\rightsquigarrow}$, the equivalent but cut-free logic $\mathbf{LDL}^*_{\rightsquigarrow}$ is defined on the language $\mathcal{L}_{\rightsquigarrow}$, by replacing $Ex$, $L\wedge_1$, $L\wedge_2$, $L\vee$ and $N$ by exactly the same generalized rules as above, and also $L\rightsquigarrow$ by the following rule:

\begin{prooftree}
	\AXC{$\Gamma \Rightarrow \nabla^n A$}
	\AXC{$\Gamma, \nabla^n B \Rightarrow \Delta$}
	\RightLabel{$L \rightsquigarrow$}
	\BIC{$\Gamma, \nabla^{n+1} (A \rightsquigarrow B) \Rightarrow \Delta$}
\end{prooftree}

The following lemmas are used in the subsequent theorem, to show that $\mathbf{LDL}^*$ has exactly the same power as $\mathbf{LDL}$.

\begin{lem}\label{lem:l-nabla-dist-and} For all $n > 0$, $\mathbf{LDL} \vdash \nabla^n (A \land B) \Rightarrow \nabla^n A \land \nabla^n B$.
\end{lem}
\begin{proof}\quad
	\begin{prooftree}
		\AXC{}
		\RightLabel{$Id$}
		\UIC{$A \Rightarrow A$}
		\RightLabel{$L\land_1$}
		\UIC{$A \land B \Rightarrow A$}
		\RightLabel{$N$} \doubleLine
		\UIC{$\nabla^n (A \land B) \Rightarrow \nabla^n A$}

		\AXC{}
		\RightLabel{$Id$}
		\UIC{$B \Rightarrow B$}
		\RightLabel{$L\land_2$}
		\UIC{$A \land B \Rightarrow B$}
		\RightLabel{$N$} \doubleLine	
		\UIC{$\nabla^n (A \land B) \Rightarrow \nabla^n B$}
		
		\RightLabel{$R\land$}
		\BIC{$\nabla^n (A \land B) \Rightarrow \nabla^n A \land \nabla^n B$}
	\end{prooftree}
\end{proof}

\begin{lem}\label{lem:l-nabla-dist-or} $\mathbf{LDL} \vdash \nabla (A \lor B) \Rightarrow \nabla A \lor \nabla B$.
\end{lem}
\begin{proof}\quad
	\begin{prooftree}
		\AXC{}
		\RightLabel{$Id$}
		\UIC{$\nabla A \Rightarrow \nabla A$}
		\RightLabel{$R\lor_1$}
		\UIC{$\nabla A \Rightarrow \nabla A \lor \nabla B$}
		\RightLabel{$R\Box$}
		\UIC{$A \Rightarrow \Box (\nabla A \lor \nabla B)$}

		\AXC{}
		\RightLabel{$Id$}
		\UIC{$\nabla B \Rightarrow \nabla B$}
		\RightLabel{$R\lor_2$}
		\UIC{$\nabla B \Rightarrow \nabla A \lor \nabla B$}
		\RightLabel{$R\Box$}
		\UIC{$B \Rightarrow \Box (\nabla A \lor \nabla B)$}

		\RightLabel{$L\lor$}
		\BIC{$A \lor B \Rightarrow \Box (\nabla A \lor \nabla B)$}
		\RightLabel{$N$}
		\UIC{$\nabla (A \lor B) \Rightarrow \nabla \Box (\nabla A \lor \nabla B)$}

		
		\AXC{} \RightLabel{$Id$}
		\UIC{$\nabla A \lor \nabla B \Rightarrow \nabla A \lor \nabla B$}
		\RightLabel{$L\Box$}
		\UIC{$\nabla \Box (\nabla A \lor \nabla B) \Rightarrow \nabla A \lor \nabla B$}



		\RightLabel{$Cut$}
		\BIC{$\nabla (A \lor B) \Rightarrow \nabla A \lor \nabla B$}
	\end{prooftree}
\end{proof}

\begin{lem}\label{lem:l-nabla-n-dist-or} For all $n > 0$, $\mathbf{LDL} \vdash \nabla^n (A \lor B) \Rightarrow \nabla^n A \lor \nabla^n B$.
\end{lem}
\begin{proof} Let $\mathcal{D}_1$ be the proof-tree of lemma \ref{lem:l-nabla-dist-or}, which proves $n = 1$ case. For $n > 1$

	\begin{prooftree}
		\AXC{$\mathcal{D}_{n-1}$}
		\noLine
		\UIC{$\nabla^{n-1} (A \lor B) \Rightarrow \nabla^{n-1} A \lor \nabla^{n-1} B$}
		\RightLabel{$N$}
		\UIC{$\nabla^n (A \lor B) \Rightarrow \nabla (\nabla^{n-1} A \lor \nabla^{n-1} B)$}

		\AXC{$\mathcal{D}_1$}
		\noLine
		\UIC{$\nabla (\nabla^{n-1} A \lor \nabla^{n-1} B) \Rightarrow \nabla^n A \lor \nabla^n B$}
		
		\RightLabel{$Cut$} \LeftLabel{$\mathcal{D}_n:$}
		\BIC{$\nabla^n (A \lor B) \Rightarrow \nabla^n A \lor \nabla^n B$}
	\end{prooftree}
\end{proof}

\begin{lem}\label{lem:l-nabla-bot} $\mathbf{LDL} \vdash \nabla \bot \Rightarrow \bot$.
\end{lem}
\begin{proof}\quad
	\begin{prooftree}
		\AXC{}
		\RightLabel{$Ex$}
		\UIC{$\bot \Rightarrow$}
		\RightLabel{$Rw$}
		\UIC{$\bot \Rightarrow \Box \bot$}
		\RightLabel{$N$}
		\UIC{$\nabla \bot \Rightarrow \nabla \Box \bot$}

		\AXC{} \RightLabel{$Id$}
		\UIC{$\bot \Rightarrow \bot$}
		\RightLabel{$L\Box$}
		\UIC{$\nabla \Box \bot \Rightarrow \bot$}
		
		\RightLabel{$Cut$}
		\BIC{$\nabla \bot \Rightarrow \bot$}
	\end{prooftree}
\end{proof}

\begin{lem}\label{lem:l-nabla-n-bot} For $n > 0$, $\mathbf{LDL} \vdash \nabla^n \bot \Rightarrow \bot$.
\end{lem}
\begin{proof} We will prove a stronger version: For $n \geq m > 0$, $\mathbf{LDL} \vdash \nabla^n \bot \Rightarrow \nabla^{n-m} \bot$. Let $\mathcal{D}_1$ be the proof-tree of lemma \ref{lem:l-nabla-bot} which handles $n = m = 1$. Using induction om $m$, and denoting by IH the proof-tree for $\nabla^n \bot \Rightarrow \nabla^{n-(m-1)} \bot$ from the induction hypothesis, we have for $n > 1$
	\begin{prooftree}
		\AXC{IH}
		\noLine
		\UIC{$\nabla^n \bot \Rightarrow \nabla^{n-(m-1)} \bot$}

		\AXC{$\mathcal{D}_1$}
		\noLine
		\UIC{$\nabla \bot \Rightarrow \bot$}
		\doubleLine \RightLabel{$N^{(n-m)}$}
		\UIC{$\nabla^{n-(m-1)} \bot \Rightarrow \nabla^{n-m} \bot$}

		\RightLabel{$Cut$}
		\BIC{$\nabla^n \bot \Rightarrow \nabla^{n-m} \bot$}
	\end{prooftree}
\end{proof}

\begin{cor}
	All proof-trees of the lemmas \ref{lem:l-nabla-dist-and}, \ref{lem:l-nabla-dist-or}, \ref{lem:l-nabla-n-dist-or}, \ref{lem:l-nabla-bot} and \ref{lem:l-nabla-n-bot} can be easily constructed in $\mathbf{LDL}_{\rightsquigarrow}$. Hence the these lemmas also hold for $\mathbf{LDL}_{\rightsquigarrow}$.
\end{cor}

\begin{lem}\label{lem:l-nabla-dist-imp} For any $n > 0$, $\mathbf{LDL} \vdash \nabla^n (A \rightarrow B) \Rightarrow \nabla^n A \rightarrow \nabla^n B$.
\end{lem}
\begin{proof}\quad
	\begin{prooftree}
		\AXC{}
		\RightLabel{$Id$}
		\UIC{$A \Rightarrow A$}
	
		\AXC{}
		\RightLabel{$Id$}
		\UIC{$B \Rightarrow B$}
		\RightLabel{$Lw$}
		\UIC{$A , B \Rightarrow B$}
	
		\RightLabel{$L\rightarrow$}
		\BIC{$A \rightarrow B , A \Rightarrow B$}
		\RightLabel{$N^{(n)}$} \doubleLine
		\UIC{$\nabla^n (A \rightarrow B) , \nabla^n A \Rightarrow \nabla^n B$}
		\RightLabel{$R\rightarrow$}
		\UIC{$\nabla^n (A \rightarrow B) \Rightarrow \nabla^n A \rightarrow \nabla^n B$}
	\end{prooftree}
\end{proof}

\begin{lem}\label{lem:l-nabla-box-comm} For all $n > 0$, $\mathbf{LDL} \vdash \nabla^{n+1} \Box A \Rightarrow \nabla \Box \nabla^n A$.
\end{lem}
\begin{proof}\quad
	\begin{prooftree}
		\AXC{} \RightLabel{$Id$}
		\UIC{$A \Rightarrow A$}
		\RightLabel{$L\Box$}
		\UIC{$\nabla \Box A \Rightarrow A$}
		\RightLabel{$N^{(n)}$} \doubleLine
		\UIC{$\nabla^{n+1} \Box A \Rightarrow \nabla^n A$}
		\RightLabel{$R\Box$}
		\UIC{$\nabla^n \Box A \Rightarrow \Box \nabla^n A$}
		\RightLabel{$N$}
		\UIC{$\nabla^{n+1} \Box A \Rightarrow \nabla \Box \nabla^n A$}
	\end{prooftree}
\end{proof}

\begin{lem}\label{lem:sl-nabla-dist-si} For any $n > 0$, $\mathbf{LDL}_{\rightsquigarrow} \vdash \nabla^n (A \rightsquigarrow B) \Rightarrow \nabla^n A \rightsquigarrow \nabla^n B$.
\end{lem}
\begin{proof}\quad
	\begin{prooftree}
		\AXC{}
		\RightLabel{$Id$}
		\UIC{$A \Rightarrow A$}
		
		\AXC{}
		\RightLabel{$Id$}
		\UIC{$B \Rightarrow B$}
		\RightLabel{$Lw$}
		\UIC{$A , B \Rightarrow B$}
		
		\RightLabel{$L\rightsquigarrow$}
		\BIC{$\nabla (A \rightsquigarrow B) , A \Rightarrow B$}
		\RightLabel{$N^{(n)}$} \doubleLine
		\UIC{$\nabla^{n+1} (A \rightsquigarrow B) , \nabla^n A \Rightarrow \nabla^n B$}
		\RightLabel{$R\rightsquigarrow$}
		\UIC{$\nabla^n (A \rightsquigarrow B) \Rightarrow \nabla^n A \rightsquigarrow \nabla^n B$}
	\end{prooftree}
\end{proof}

\begin{cor}
	Since $\mathbf{LDL}^*$ ($\mathbf{LDL}^*_{\rightsquigarrow}$) is just a generalization of $\mathbf{LDL}$ ($\mathbf{LDL}_{\rightsquigarrow}$), then all the above lemmas also hold for $\mathbf{LDL}^*$ ($\mathbf{LDL}^*_{\rightsquigarrow}$).
\end{cor}

In the rest of this section, by the \emph{length of a proof-tree} $\mathcal{D}$, denoted by $h(\mathcal{D})$, we mean the number of rule instances in its longest branch.

\begin{thm}\label{thm:ldl-eq-ldls}
	For any sequent $\Gamma \Rightarrow \Delta$ in the language $\mathcal{L}$, $\mathbf{LDL} \vdash \Gamma \Rightarrow \Delta$ iff $\mathbf{LDL}^* \vdash \Gamma \Rightarrow \Delta$.
\end{thm}
\begin{proof}
	One direction easily follows from the fact that all rules of $\mathbf{LDL}$ are just instances of $\mathbf{LDL}^*$'s rules.
	For the other direction, we will use case analysis for the last rule of the proof-tree of $\Gamma \Rightarrow \Delta$ in $\mathbf{LDL}^*$, which we call $\mathcal{D}$, and construct a proof-tree for it in $\mathbf{LDL}$ for each case.
	
	First notice that $Id$ and $Ta$ are present in $\mathbf{LDL}$ and lemma \ref{lem:l-nabla-n-bot} handles the $Ex$ case.
	For other rules, use induction on the length of $\mathcal{D}$.
	So for the sequents proved by sub-tree(s) of $\mathcal{D}$, the induction hypothesis provides proof of the same sequent(s) in $\mathbf{LDL}$.
	For common rules, just apply the same rule (in $\mathbf{LDL}$) on the proof-tree(s) from the induction hypothesis to reach the desired sequent.
	In cases of $\mathbf{LDL}^*$'s stronger rules, i.e. $L\land_1$, $L\land_2$, $L\lor$, $L\rightarrow$ and $L\Box$, do the same, and also $cut$ sequents from lemmas \ref{lem:l-nabla-dist-and}, \ref{lem:l-nabla-dist-or}, \ref{lem:l-nabla-dist-imp} or \ref{lem:l-nabla-box-comm} into the resulting sequent.
	The case for $N$ divides to two further cases. One case is when $\Delta = A$ for some formula $A$, which is again handled by applying $N$ on the sequent from the induction hypothesis.
	The other case is when $\Delta = \{\}$, so we have $\Gamma \Rightarrow$ from induction hypothesis, in which we first introduce $\bot$ on the right using $Rw$ and apply $N$.
	Then we can $cut$ it into the sequent from lemma \ref{lem:l-nabla-bot}, and then into $Ex$, just to derive $\nabla \Gamma \Rightarrow$ as was desired.
\end{proof}

\begin{thm}\label{thm:sldl-eq-sldls}
	For any sequent $\Gamma \Rightarrow \Delta$ in the language $\mathcal{L}_{\rightsquigarrow}$, $\mathbf{LDL}_{\rightsquigarrow} \vdash \Gamma \Rightarrow \Delta$ iff $\mathbf{LDL}^*_{\rightsquigarrow} \vdash \Gamma \Rightarrow \Delta$.
\end{thm}
\begin{proof}
	The proof is similar to the proof of previous theorem, except that we don't have a case for $L\Box$ and there is a case for $L\rightsquigarrow$ instead of $L\rightarrow$, which is handled similarly, with the help of lemma \ref{lem:sl-nabla-dist-si}.
\end{proof}

\subsection{Cut-elimination theorems}
Although the admissibility of cut in $\mathbf{LDL}^*$ and $\mathbf{LDL}^*_{\rightsquigarrow}$ does not imply it's admissibility in $\mathbf{LDL}$ or $\mathbf{LDL}_{\rightsquigarrow}$, but as we will see, their equivalence ---by theorems \ref{thm:ldl-eq-ldls} and \ref{thm:sldl-eq-sldls} respectively--- enables us to prove statements about the cut-free system, which are otherwise not provable in the presence of the cut rule, and translate the result back into statements about the original system.

For technical reasons, we will eliminate a stronger form of the cut rule, which nevertheless satisfies our goal here.

\begin{dfn}[Multi-cut rule] We denote by $\mathbf{LDL}^{*M}$ ($\mathbf{LDL}^{*M}_{\rightsquigarrow}$) the same system defined by $\mathbf{LDL}^*$ ($\mathbf{LDL}^*_{\rightsquigarrow}$), with the cut rule replaced by the following generalization:
\begin{prooftree}
	\AXC{$\Gamma \Rightarrow A$}
	\AXC{$\Sigma , (\nabla^l A)^n \Rightarrow \Delta$}
	\RightLabel{$MC$}
	\BIC{$\nabla^l \Gamma , \Sigma \Rightarrow \Delta$}
\end{prooftree}
where $\nabla^l$ means $\nabla$ applied $l$ times and $A^n$ means $n$ instances of $A$ in a multi-set.
\end{dfn}

\begin{cor}\label{cor:mc-riddance} Any sequent provable in $\mathbf{LDL}^*$ ($\mathbf{LDL}^*_{\rightsquigarrow}$) is also provable in $\mathbf{LDL}^{*M}$ ($\mathbf{LDL}^{*M}_{\rightsquigarrow}$). Indeed we can just replace any instance of $cut$ with similar instance of $MC$, with $l = 0$ and $n = 1$.
\end{cor}

Next, we should make precise what we mean by the \emph{rank} of formulas and proof-trees.

\begin{dfn}[Rank]
	Rank of a formula $A$ is defined as
	\[ \rho(A) = \begin{cases}
	1 & \quad ; A \in P \cup \{ \bot, \top \} \\
	\rho(B) & \quad ; A = \nabla B \\
	\rho(B) + 1 & \quad ; A = \Box B \\
	max(\rho(B), \rho(C)) + 1 & \quad ; A = B \circ C, \circ \in \{ \land , \lor, \rightarrow \}
	\end{cases} \]
	Notice that $\nabla$ does not increase rank.
	
	We also define rank for rule instances and proof-trees. Rank of a rule instance is the rank of its cut-formula if it is an instance of the $MC$ rule, and $0$ if it's not.
	For a proof tree $\mathcal{D}$, $\rho(\mathcal{D})$ is the maximum rank of its rule instances.
\end{dfn}

The following lemma will help us in the proof of the next theorem.

\begin{lem}\label{lem:ldls-top-redundant} If $\mathbf{LDL}^{*M}$ ($\mathbf{LDL}^{*M}_{\rightsquigarrow}$) proves $\Gamma , (\nabla^r \top)^n \Rightarrow \Delta$, then it also proves $\Gamma \Rightarrow \Delta$ with a proof-tree of at most the same rank.
\end{lem}
\begin{proof}
Suppose $\mathcal{D}$ is a proof-tree for $\Gamma , (\nabla^r \top)^n \Rightarrow \Delta$ in $\mathbf{LDL}^{*M}$. By induction on $h(\mathcal{D})$, induction hypothesis states that for an arbitrary proof tree $\mathcal{D}'$ with conclusion $\Gamma' , (\nabla^{r'} \top)^{n'} \Rightarrow \Delta'$ such that $h(\mathcal{D}') < h(\mathcal{D})$, there is a proof $\mathcal{D}''$ of $\Gamma' \Rightarrow \Delta'$ such that $\rho(\mathcal{D}'') \leq \rho(\mathcal{D}')$. Notice that induction hypothesis is applicable to any proof tree of lower length, with no restrictions on $\Gamma'$, $\Delta'$, $n'$ or $r'$.

Consider different cases for the last rule of $\mathcal{D}$, with possible sub-trees $\mathcal{D}_0$ and $\mathcal{D}_1$. $Ta$ and $Ex$ cases are trivially ruled out. In $Id$ case, which implies $n = 1$, we have $\Rightarrow \nabla^r \top$ by $r$ times applications of $N$ on $Ta$. In $Lw$ case, where an instance of $\top$ is principal and $n = 1$, $\mathcal{D}_0$ itself proves the desired sequent. If $n > 1$, then the induction hypothesis with $n' = n - 1$ gives the desired sequent. $Lc$ on an instance of $\top$ is similar, with $n' = n + 1$. In all other cases, just apply induction hypothesis on $\mathcal{D}_0$ (and possibly $\mathcal{D}_1$), then the same last rule. Notice that in all cases we must apply the induction hypothesis with $n' = n$ and $r' = r$, except for $N$, in which we apply it with $r' = r - 1$. Also notice that $MC$ is not used except in $MC$ case, where it is applied with the same cut-formula, so the resulting proof tree is not of a higher rank than $\mathcal{D}$.

The proof for $\mathbf{LDL}^{*M}_{\rightsquigarrow}$ is similar, except that there are no cases for $\Box$, and there are cases for $\rightsquigarrow$ instead of $\rightarrow$. The case for $L\rightsquigarrow$ is handled similar to $L\rightarrow$, but in $R\rightsquigarrow$ case, we must apply the induction hypothesis with $r = r + 1$.
\end{proof}

The following theorem shows that we can imitate any instance of the cut rule in a proof-tree of lower rank.

\begin{thm}\label{thm:ldls-cut-reduction}[Cut Reduction] If $\mathbf{LDL}^{*M}$ ($\mathbf{LDL}^{*M}_{\rightsquigarrow}$) proves $\Gamma \Rightarrow A$ and $\Sigma , (\nabla^l A)^n \Rightarrow \Delta$ with proof trees of ranks less than $\rho(A)$, then it also proves $\nabla^l \Gamma , \Sigma \Rightarrow \Delta$ also with a proof tree of a rank less than $\rho(A)$. 
\end{thm}
\begin{proof}
Let $\mathcal{D}_0$ and $\mathcal{D}_1$ be proof trees of ranks less than $\rho(A)$, with conclusions $\Gamma \Rightarrow A$ and $\Sigma , (\nabla^l A)^n \Rightarrow \Delta$ and heights $h(\mathcal{D}_0)$ and $h(\mathcal{D}_1)$ respectively.
\begin{prooftree}
 	\noLine
 	\AXC{$\mathcal{D}_0$}
 	\UIC{$\Gamma \Rightarrow A$}
 	
 	\noLine
 	\AXC{$\mathcal{D}_1$}
 	\UIC{$\Sigma , (\nabla^l A)^n \Rightarrow \Delta$}
 	
 	\dashedLine \RightLabel{$MC$}
 	\BIC{$\nabla^l \Gamma , \Sigma \Rightarrow \Delta$}
\end{prooftree}
We will destruct the proof for possible cases of the last rule instance in $\mathcal{D}_0$ and $\mathcal{D}_1$ in the following order: We will first handle the cases where $\mathcal{D}_0$ ends with a left-rule, in which the cut-formula is not altered, independent of $\mathcal{D}_1$. The cases for $N$ and $Rw$ are also independent of $\mathcal{D}_1$, although they change the cut-formula. Then for all of the other rules (which do alter the cut-formula in $\mathcal{D}_0$), we will also consider all the cases where the cut-formula is not altered in $\mathcal{D}_1$, but this time independent of $\mathcal{D}_1$. And at last, there are cases where the cut-formula is altered in both $\mathcal{D}_0$ and $\mathcal{D}_1$ in their last rule. In these cases we can also destruct the cut-formula to the specific form implied by the respective rules at the end of $\mathcal{D}_0$ and $\mathcal{D}_1$, so we can use a low rank cut rule on their sub-proofs.

In most the cases, we will use induction on the length of both $\mathcal{D}_0$ and $\mathcal{D}_1$ as follows; For any two proof-trees $\mathcal{D}_0'$ and $\mathcal{D}_1'$ such that $h(\mathcal{D}_0') + h(\mathcal{D}_1') < h(\mathcal{D}_0) + h(\mathcal{D}_1)$, where $\mathcal{D}_0'$ proves $\Gamma' \Rightarrow A'$ and $\mathcal{D}_1'$ proves $\Sigma', (\nabla^{l'} A')^{n'} \Rightarrow \Delta$ for arbitrary $\Gamma'$, $\Sigma'$, $\Delta'$, $A'$, $l'$ and $n'$, for which we have $\rho(\mathcal{D}_0'),\mathcal{D}_1' < \rho(A')$, the induction hypothesis give us a proof-tree $\text{IH}(\mathcal{D}_0', \mathcal{D}_1')$ which proves $\nabla^{l'}\Gamma', \Sigma' \Rightarrow \Delta'$, and we also have $\rho(\text{IH}(\mathcal{D}_0', \mathcal{D}_1')) < \rho(A')$.

First, let $\mathcal{D}_0$ be just an axiom. The case for $Id$ is trivial, $Ex$ won't happen and $Ta$ is handled by \ref{lem:ldls-top-redundant}. Now assume $\mathcal{D}_0$ ends with either of the left-rules $L\land_1$, $L\land_2$, $L\lor$, $L\rightarrow$ or $L\Box$. In these cases, all we have to do is to simply apply the same rule on the sequent(s) resulting from induction hypothesis on the sub-proof(s) of $\mathcal{D}_0$ and $\mathcal{D}_1$. For the sake of shortness we will only mention cases for $Lw$, $MC$ and $L\lor$ here ; the other cases are just the same.

Let $\mathcal{D}_0$ end with $Lw$.
$\mathcal{D}_0$ proves $\Gamma , B \Rightarrow A$ and has a subproof $\mathcal{D}_0'$ of $\Gamma \Rightarrow A$.
	\begin{prooftree}
		\noLine
		\AXC{$\mathcal{D}_0'$}
		\UIC{$\Gamma \Rightarrow A$}
		\RightLabel{$Lw$}
		\UIC{$\Gamma , B \Rightarrow A$}

		
		\noLine
		\AXC{$\mathcal{D}_1$}
		\UIC{$\Sigma , (\nabla^l A)^n \Rightarrow \Delta$}

		\dashedLine\RightLabel{$MC$}
		\BIC{$\nabla^l \Gamma , \nabla^l B , \Sigma \Rightarrow \Delta$}
	\end{prooftree}
	From the induction hypothesis for $\mathcal{D}_0'$ and $\mathcal{D}_1$, we have a proof of $\nabla^l \Gamma , \Sigma \Rightarrow \Delta$ with a lower rank than $\rho(A)$. By applying $Lw$ we have $\nabla^l \Gamma , \nabla^l B , \Sigma \Rightarrow \Delta$, without increasing the cut-rank.
	\begin{prooftree}
		\noLine
		\AXC{$\mathcal{D}_0'$}
		\UIC{$\Gamma \Rightarrow A$}
		
		\noLine
		\AXC{$\mathcal{D}_1$}
		\UIC{$\Sigma , (\nabla^l A)^n \Rightarrow \Delta$}
		
		\RightLabel{IH}
		\BIC{$\nabla^l \Gamma , \Sigma \Rightarrow \Delta$}
		
		\RightLabel{$Lw$}
		\UIC{ $\nabla^l \Gamma , \nabla^l B , \Sigma \Rightarrow \Delta$}
	\end{prooftree}

\end{proof}

\textbf{Problem.} Do we have cut elimination theorem for the sequent-style systems we developed for $\mathbf{LDL}_{\rightsquigarrow}$ but this time plus the rule $H_{\rightsquigarrow}$? What about the system of the previous problem for $\mathbf{LDL}$ plus the rule $H$?
\subsection{Admissible Rules}
\textbf{Problem.} What are the Harrop formulas here? Prove the disjunction property for the theories axiomatized by Harrop formulas, using the cut-free proofs over the logics $\mathbf{LDL}$ and $\mathbf{LDL}_{\rightsquigarrow}$. Prove the same thing via Kleene-type argument. Do we have similar thing for the logics with added rules $H$ and $H_{\rightsquigarrow}$, respectively?
\subsection{Interpolation}
\textbf{Problem.} As we discussed, the system $\mathbf{LDL}$ and possibly $\mathbf{LDL}(H)$ must enjoy Craig interpolation property. What are the proofs?\\
\\
\textbf{Problem.} As we discussed, the system $\mathbf{LDL}_{\rightsquigarrow}$ and possibly $\mathbf{LDL}_{\rightsquigarrow}(H_{\rightsquigarrow})$ must enjoy deductive interpolation property. What are the proofs?

\begin{thebibliography}{99}
\addcontentsline{toc}{section}{References}

\bibitem{Amir}
A. Akbar Tabatabai. ``Implication via Spacetime." arXiv preprint arXiv:2001.00997 (2019).
\bibitem{Artemov}
S. Artemov, J. Davoren, and A. Nerode. Modal logics and topological semantics for hybrid systems. Technical Report MSI 97-05, 1997.
\bibitem{Ewald}
W. Ewald. Intuitionistic tense and modal logic. The Journal of Symbolic Logic, 51(1):166–179, 1986.
\bibitem{Duque}
D. Fernández-Duque. The intuitionistic temporal logic of dynamical systems. Log. Methods Comput. Sci. 14(3) (2018)
\bibitem{Mints}
P. Kremer and G. Mints. Dynamic topological logic. Annals of Pure and Applied Logic, 131:133–158, 2005.

\end{thebibliography}



\end{document}
