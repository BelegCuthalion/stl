\section{Rules} In the following, $\dotdiv$ is the truncated subtraction defined as $a \dotdiv b = max(a-b, 0)$.

\subsection{Theorem} \textit{Generalized Visser rules:} Let $\{ l_i \}_{i=1}^n$ be a sequence of natural numbers of length $n$. If {\small iSTL} $\vdash \Rightarrow \bigwedge_{i=1}^n (\nabla^{l_i} (A_i \rightarrow B_i)) \rightarrow C \lor D$ then either {\small iSTL} $\vdash \Rightarrow \bigwedge_{i=1}^n (\nabla^{l_i} (A_i \rightarrow B_i)) \rightarrow \nabla^{l_j \dotdiv 1} A_j$ for some $j \in \{ 1 , \dots , n \}$, or {\small iSTL} $\vdash \Rightarrow \bigwedge_{i=1}^n (\nabla^{l_i} (A_i \rightarrow B_i)) \rightarrow C$, or {\small iSTL} $\vdash \Rightarrow \bigwedge_{i=1}^n (\nabla^{l_i} (A_i \rightarrow B_i)) \rightarrow D$.

\textit{Proof}:
We have $\bigwedge_{i=1}^n (\nabla^{l_i} (A_i \rightarrow B_i)) \Rightarrow C \lor D$ by \ref{lem:impl-elim}. By $Cut$ and \ref{lem:conj-context} we have $\{ \nabla^{l_i} (A_i \rightarrow B_i) \}_{i=1}^n \Rightarrow$ $C \lor D$, which also has a proof like $\mathbf{D}$ in $GSTL^-$ by \ref{translation} and \ref{cut-elim}. The last rule in $\mathbf{D}$ can be:
\begin{itemize}[label=-]
	\item $R\lor_1$, applied on $\{ \nabla^{l_i} (A_i \rightarrow B_i) \}_{i=1}^n \Rightarrow C$. With enough applications of $L\land_1$, $L\land_2$\footnote{{\color{red} \textit{<convincing the reader about $\land$'s associativity/commutativity>} }}, $Lc$ and a $R\rightarrow$ we will have $\Rightarrow \bigwedge_{i=1}^n (\nabla^{l_i} (A_i \rightarrow B_i)) \rightarrow C$.
	
	\item $R\lor_2$, applied on $\{ \nabla^{l_i} (A_i \rightarrow B_i) \}_{i=1}^n \Rightarrow D$. Same as the previous case, we can derive $\Rightarrow \bigwedge_{i=1}^n (\nabla^{l_i} (A_i \rightarrow B_i)) $ $\rightarrow D$.
	
	\item $Rw$, applied on $\{ \nabla^{l_i} (A_i \rightarrow B_i) \}_{i=1}^n \Rightarrow$. A different $Rw$ gives $\{ \nabla^{l_i} (A_i \rightarrow B_i) \}_{i=1}^n \Rightarrow C$. Again, we can get $\Rightarrow \bigwedge_{i=1}^n (\nabla^{l_i} (A_i \rightarrow B_i)) \rightarrow C$.
	
	\item $Lw$, applied on $\{ \nabla^{l_i} (A_i \rightarrow B_i) \}_{i=1,i \neq k}^n \Rightarrow C \lor D$ for some $k \in \{ 1 , \dots , n \}$. Let $\mathbf{D}'$ be the immediate sup-tree of $\mathbf{D}$. By induction on $h(\mathbf{D})$, we can apply induction hypothesis on $\mathbf{D}'$ to get either $\{ \nabla^{l_i} (A_i \rightarrow B_i) \}_{i=1,i \neq k}^n \Rightarrow C$, $\{ \nabla^{l_i} (A_i \rightarrow B_i) \}_{i=1,i \neq k}^n \Rightarrow D$ or $\{ \nabla^{l_i} (A_i \rightarrow B_i) \}_{i=1,i \neq k}^n \Rightarrow \nabla^{l_j \dotdiv 1} A_j$ for some $j \in \{ 1 , \dots , n \} - \{k\}$. Then, after introducing $\nabla^{l_k} A_k \rightarrow B_k$ on the left with $Lw$ again, we can follow the same manner as previous cases to reach any of the desired sequents.
	
	\item $Lc$, applied on $\{ \nabla^{l_i} (A_i \rightarrow B_i) \}_{i=1}^n , \nabla^{l_k} (A_k \rightarrow B_k) \Rightarrow C \lor D$ for some $k \in \{ 1 , \dots , n \}$. This is just the same as the $Lw$ case, except this time we must remove the extra $\nabla^{l_k} (A_k \rightarrow B_k)$ with another $Lc$.
	
	\item $L\rightarrow$, applied on $\{ \nabla^{l_i} (A_i \rightarrow B_i) \}_{i=1, i \neq j}^n \Rightarrow \nabla^{l_j - 1} A_j$ and $\{ \nabla^{l_i} (A_i \rightarrow B_i) \}_{i=1, i \neq j}^n , \nabla^{l_j - 1} B_j \Rightarrow C \lor D$ for some $j \in \{ 1 , \dots , n \}$. So this implies $n>0$ and $l_j>0$ for at least one such $j$. Again, we can derive $\Rightarrow  \bigwedge_{i=1}^n (\nabla^{l_i} (A_i \rightarrow B_i)) \rightarrow \nabla^{l_j - 1} A_j$ using proper $L\land_{1/2}$, $Lc$ and $R\rightarrow$.
\end{itemize}
Notice that no other case is valid, since they all imply different structure for $\mathbf{D}$. Now from theorem \ref{translation}, all of the desired sequents also have a proof in iSTL.
