\begin{dfn}\label{gstl}
	Define $\gstl$ over the language $\mathcal{L}$, to be the logic of sequent-style system defined by the same rules as $\stl$, except that the rules $Ex$, $L \wedge_1$, $L \wedge_2$, $L\vee$, $L \rightarrow$ and $N$ are replaced by the following generalized rules respectively.
\end{dfn}

\begin{prooftree}
	\AXC{}
	\RightLabel{$Ex$}
	\UIC{$\nabla^n \bot \Rightarrow$}
\end{prooftree}

\begin{multicols}{2}
	\begin{prooftree}
		\AXC{$\Gamma, \nabla^n A \Rightarrow \Delta$}
		\RightLabel{$L \wedge_1$}
		\UIC{$\Gamma, \nabla^n (A \wedge B) \Rightarrow \Delta$}
	\end{prooftree}
	\columnbreak
	\begin{prooftree}
		\AXC{$ \Gamma, \nabla^n B \Rightarrow \Delta$}
		\RightLabel{$L \wedge_2$}
		\UIC{$\Gamma, \nabla^n (A \wedge B) \Rightarrow \Delta$}		
	\end{prooftree}
\end{multicols}

\begin{prooftree}
	\AXC{$\Gamma, \nabla^n A \Rightarrow \Delta$}
	\AXC{$\Gamma, \nabla^n B \Rightarrow \Delta$}
	\RightLabel{$L \vee$}
	\BIC{$\Gamma, \nabla^n (A \vee B) \Rightarrow \Delta$}
\end{prooftree}

\begin{prooftree}
	\AXC{$\Gamma \Rightarrow \nabla^n A$}
	\AXC{$\Gamma, \nabla^n B \Rightarrow \Delta$}
	\RightLabel{$L \rightarrow$}
	\BIC{$\Gamma, \nabla^{n+1} (A \rightarrow B) \Rightarrow \Delta$}
\end{prooftree}	
Notice that all the rules above are just generalizations of their corresponding rules in $\stl$.\\