\begin{dfn}\label{stl}
	Consider the language $\mathcal{L}=\langle \wedge, \vee, \top, \bot, \nabla, \rightarrow, P \rangle$ where $P$ is the set of all atomic propositions. Define $\stl$ as the logic of the sequent-style system defined by the following rules, where in a sequent of the form $\Gamma \Rightarrow \Delta$, the left-hand side is a multiset of formulas $\Gamma$, and the right-hand side $\Delta$ is either a formula or empty.
\end{dfn}

\begin{flushleft}
 \textbf{Axioms:}
\end{flushleft}

\begin{multicols}{3}
	\begin{prooftree}
		\AXC{}
		\RightLabel{$Id$}
		\UIC{$ A \Rightarrow A$}
	\end{prooftree}
	\columnbreak
	\begin{prooftree}
		\AXC{}
		\RightLabel{$Ta$}
		\UIC{$ \Rightarrow \top$}
	\end{prooftree}
	\columnbreak
	\begin{prooftree}
		\AXC{}
		\RightLabel{$Ex$}
		\UIC{$ \bot \Rightarrow $}		
	\end{prooftree}
\end{multicols}

\begin{flushleft}
 		\textbf{Structural Rules:}
\end{flushleft}
\begin{multicols}{3}
	\begin{prooftree}
		\AXC{$ \Gamma \Rightarrow \Delta$}
		\RightLabel{$L w$}
		\UIC{$ \Gamma, A \Rightarrow \Delta$}
	\end{prooftree}
	\columnbreak
	\begin{prooftree}
		\AXC{$ \Gamma \Rightarrow $}
		\RightLabel{$R w$}
		 \UIC{$\Gamma \Rightarrow A$}		
	\end{prooftree}
	\columnbreak
	\begin{prooftree}
		\AXC{$ \Gamma, A, A \Rightarrow \Delta$}
		\RightLabel{$Lc$}
		\UIC{$\Gamma, A \Rightarrow \Delta$}		
	\end{prooftree}
\end{multicols}

\begin{flushleft}
 		\textbf{Cut:}
\end{flushleft}
\begin{center}
	\begin{prooftree}
		\AXC{$ \Gamma \Rightarrow A$}
		\AXC{$\Pi, A \Rightarrow \Delta$}
		\RightLabel{$cut$}
		\BIC{$ \Pi, \Gamma \Rightarrow \Delta$}
	\end{prooftree}
\end{center}

\begin{flushleft}
 \textbf{Conjunction Rules:}
\end{flushleft}
\begin{multicols}{3}
	\begin{prooftree}
		\AXC{$ \Gamma, A \Rightarrow \Delta$}
		\RightLabel{$L \wedge_1$}
		\UIC{$ \Gamma, A \wedge B \Rightarrow \Delta$}		
	\end{prooftree}
	\columnbreak
	\begin{prooftree}
		\AXC{$ \Gamma, B \Rightarrow \Delta$}
		\RightLabel{$L \wedge_2$}
		\UIC{$\Gamma, A \wedge B \Rightarrow \Delta$}		
	\end{prooftree}
	\columnbreak
	\begin{prooftree}
		\AXC{$\Gamma \Rightarrow A$}
		\AXC{$\Gamma \Rightarrow B$}
		\RightLabel{$R \wedge$}
		\BIC{$ \Gamma \Rightarrow A \wedge B $}		
	\end{prooftree}
\end{multicols}

\begin{flushleft}
 \textbf{Disjunction Rules:}
\end{flushleft}
\begin{multicols}{3}
	\begin{prooftree}
		\AXC{$ \Gamma, A \Rightarrow \Delta$}
		\AXC{$\Gamma, B \Rightarrow \Delta$}
		\RightLabel{$L \vee_1$}
		\BIC{$ \Gamma, A \vee B \Rightarrow \Delta$}		
	\end{prooftree}
	\columnbreak
	\begin{prooftree}
		\AXC{$\Gamma \Rightarrow A$}
		\RightLabel{$R \vee_2$}
		\UIC{$\Gamma \Rightarrow A \vee B$}		
	\end{prooftree}
	\columnbreak
	\begin{prooftree}
		\AXC{$\Gamma \Rightarrow B$}
		\RightLabel{$R \vee$}
		\UIC{$\Gamma \Rightarrow A \vee B$}		
	\end{prooftree}
\end{multicols}

\begin{flushleft}
	\textbf{Implication Rules:}
 \end{flushleft}
 \begin{multicols}{2}
	\begin{prooftree}
		\AXC{$\Gamma \Rightarrow A$}
		\AXC{$\Gamma, B \Rightarrow \Delta$}
		\RightLabel{$L \rightarrow$}
		\BIC{$\Gamma, \nabla (A \rightarrow B) \Rightarrow \Delta$}		
	\end{prooftree}
	\columnbreak
	\begin{prooftree}
		\AXC{$\nabla \Gamma, A \Rightarrow B$}
		\RightLabel{$R \rightarrow$}
		\UIC{$\Gamma \Rightarrow A \rightarrow B$}		
	\end{prooftree}
\end{multicols}

\begin{flushleft}
  \textbf{Modal Rule:}
\end{flushleft}
\begin{prooftree}
	\AXC{$\Gamma \Rightarrow \Delta$}
	\RightLabel{$N$}
	\UIC{$\nabla \Gamma \Rightarrow \nabla \Delta$}
\end{prooftree}


Notice that our naming scheme for the systems differs slightly with \cite{amir}. Also notice that the sequent-style systems that are introduced in the mentioned paper use sequences of formulas instead of multisets, and have an explicit exchange rule.

If we also add one or more of the following rules to the system, we will have an \emph{extension of} $\stl$, which are denoted by $\stl(S)$ where $S$ is any combination of these rules:

	\begin{prooftree}
		\RightLabel{$L$}
		\AXC{$\Gamma, A \Rightarrow \Delta$}
		\UIC{$\Gamma, \nabla A \Rightarrow \Delta$}
	\end{prooftree}

	\begin{prooftree}
		\RightLabel{$R$}
		\AXC{$\nabla \Gamma, \Sigma \Rightarrow \Delta$}
		\UIC{$\Gamma, \Sigma \Rightarrow \Delta$}
	\end{prooftree}



	\begin{prooftree}
		\RightLabel{$Fa$}
		\AXC{$\Gamma , A \Rightarrow B$}
		\UIC{$\Gamma \Rightarrow \nabla(A \rightarrow B)$}
	\end{prooftree}

	\begin{prooftree}
		\RightLabel{$Fu$}
		\AXC{$\nabla \Gamma \Rightarrow \nabla A$}
		\UIC{$\Gamma \Rightarrow A$}
	\end{prooftree}