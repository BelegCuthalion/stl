\documentclass[12pt,a4paper]{article}
\usepackage{amsmath}
\usepackage{amssymb}
\usepackage{mathabx}
\usepackage{titlesec}
\usepackage{fullpage}
\usepackage{tikz-cd}
\usepackage{rotating}
\usepackage{pdflscape}
\usepackage{multicol}
\usepackage{multirow}
\usepackage{diagbox}
\usepackage[left=.5in,right=.5in,top=.5in,bottom=.5in]{geometry}
\usepackage{enumitem}
\usepackage[colorlinks]{hyperref}
\usepackage{bussproofs}

\setitemize{topsep=3pt,parsep=5pt,partopsep=0pt,label=,leftmargin=1.3pc}
\titleformat{\section}[runin]{\normalfont\bfseries}{\thesection}{0.5em}{}
\titlespacing{\section}{0pc}{5ex plus .1ex minus .2ex}{1pc}
\titleformat{\subsection}[runin]{\normalfont\bfseries}{\thesubsection}{0.7em}{}
\titlespacing{\subsection}{0pc}{2ex plus .1ex minus .2ex}{1pc}
\titleformat{\subsubsection}[runin]{\normalfont\bfseries}{\thesubsubsection}{0.7em}{}
\titlespacing{\subsubsection}{0pc}{2ex plus .1ex minus .2ex}{1pc}
\newcommand\eqn{\refstepcounter{equation}\tag{\theequation}}
\binoppenalty=\maxdimen
\relpenalty=\maxdimen
\newcommand{\ul}{\ulcorner}
\newcommand{\ur}{\urcorner}
\newcommand{\val}[1]{\ulcorner len1 \urcorner}
\newcommand{\caseref}[1]{\hyperref[#1]{\ref{#1}}}
\newcommand{\rot}{\rotatebox{90}}
\newcommand{\p}{\partial}
\newcommand{\todo}[1]{{\color{red}\textbf{TODO} #1}}
\EnableBpAbbreviations

\begin{document}
    
\title{Interpolation for Logics of Spacetime}
\author[]{Amirhossein Akbar Tabatabai, Majid Alizadeh, Alireza Mahmoudian}
\affil[]{ }
\date{March 2, 2022}
\maketitle

\begin{abstract}
	\input{abstract}
\end{abstract}

\section{Introduction}
\input{introduction}

\section{Sequent calculi for the logics of spacetime}
In this section, will define the systems $\stl$ (defined as $\istl(N)$ in \cite{amir}) and $\gstl$, and show that these two systems are equivalent.

\[\begin{tikzcd}
	&& IPC \\
	Fu & L && R & Fa \\
	\\
	&& STL
	\arrow[from=4-3, to=2-1]
	\arrow[from=2-4, to=1-3]
	\arrow[from=2-2, to=1-3]
	\arrow[from=4-3, to=2-2]
	\arrow[from=4-3, to=2-4]
	\arrow[from=4-3, to=2-5]
\end{tikzcd}\]

Now, we will introduce $\gstl$, which we then show to be equivalent with $\stl$, and also cut-free.

\input{dfn/gstl}

To show the equivalnce of two systems, we will use the following lemmas. Notice that the proof-trees are constructible in $\stl(S)$ for $S \subseteq \{ L, R, Fa, Fu \}$.

\input{thm/l-nabla-dist-and}

\input{thm/l-nabla-box}

\begin{lem}\label{lem:l-box-nabla} $\stl(S) \vdash A \Rightarrow \top \rightarrow \nabla A$.
\end{lem}
\begin{proof}\quad
	\begin{prooftree}
		\AXC{}
		\RightLabel{$Id$}
		\UIC{$\nabla A \Rightarrow \nabla A$}
		
		\RightLabel{$Lw$}
		\UIC{$\nabla A , \top \Rightarrow \nabla A$}

		\RightLabel{$R \rightarrow$}
		\UIC{$A \Rightarrow \top \rightarrow \nabla A$}
	\end{prooftree}
\end{proof}

\begin{lem}\label{lem:l-nabla-dist-or} $\stl(S) \vdash \nabla (A \vee B) \Rightarrow \nabla A \vee \nabla B$.
\end{lem}
\begin{proof} Let $\mathcal{D}$ be a proof-tree we can construct for $\nabla (\top \rightarrow (\nabla A \vee \nabla B)) \Rightarrow \nabla A \vee \nabla B$ from Lemma \ref{lem:l-nabla-box}.

	\begin{prooftree}
		\AXC{}
		\RightLabel{$Id$}
		\UIC{$\nabla A \Rightarrow \nabla A$}
		\RightLabel{$R\vee_1$}
		\UIC{$\nabla A \Rightarrow \nabla A \vee \nabla B$}
		\RightLabel{$Lw$}
		\UIC{$\nabla A , \top \Rightarrow \nabla A \vee \nabla B$}
		\RightLabel{$R \rightarrow$}
		\UIC{$A \Rightarrow \top \rightarrow (\nabla A \vee \nabla B)$}

		\AXC{}
		\RightLabel{$Id$}
		\UIC{$\nabla B \Rightarrow \nabla B$}
		\RightLabel{$R\vee_2$}
		\UIC{$\nabla B \Rightarrow \nabla A \vee \nabla B$}
		\RightLabel{$Lw$}
		\UIC{$\nabla B , \top \Rightarrow \nabla A \vee \nabla B$}
		\RightLabel{$R \rightarrow$}
		\UIC{$B \Rightarrow \top \rightarrow (\nabla A \vee \nabla B)$}

		\RightLabel{$L\vee$}
		\BIC{$A \vee B \Rightarrow \top \rightarrow (\nabla A \vee \nabla B)$}
		\RightLabel{$N$}
		\UIC{$\nabla (A \vee B) \Rightarrow \nabla (\top \rightarrow (\nabla A \vee \nabla B))$}

		\AXC{$\mathcal{D}$}

		\RightLabel{$Cut$}
		\BIC{$\nabla (A \vee B) \Rightarrow \nabla A \vee \nabla B$}
	\end{prooftree}
\end{proof}

\begin{lem}\label{lem:l-nabla-n-dist-or} For all $n \geq 0$, $\stl(S) \vdash \nabla^n (A \vee B) \Rightarrow \nabla^n A \vee \nabla^n B$.
\end{lem}
\begin{proof} The proof is trivial when $n = 0$. Let $\mathcal{D}_1$ be the proof-tree of Lemma \ref{lem:l-nabla-dist-or}, which proves the case for $n = 1$. For any $n > 1$ we have

	\begin{prooftree}
		\AXC{$\mathcal{D}_{n-1}$}
		\noLine
		\UIC{$\nabla^{n-1} (A \vee B) \Rightarrow \nabla^{n-1} A \vee \nabla^{n-1} B$}
		\RightLabel{$N$}
		\UIC{$\nabla^n (A \vee B) \Rightarrow \nabla (\nabla^{n-1} A \vee \nabla^{n-1} B)$}

		\AXC{$\mathcal{D}_1$}
		\noLine
		\UIC{$\nabla (\nabla^{n-1} A \vee \nabla^{n-1} B) \Rightarrow \nabla^n A \vee \nabla^n B$}
		
		\RightLabel{$Cut$} \LeftLabel{$\mathcal{D}_n:$}
		\BIC{$\nabla^n (A \vee B) \Rightarrow \nabla^n A \vee \nabla^n B$}
	\end{prooftree}
\end{proof}

\begin{lem}\label{lem:l-nabla-bot} $\stl(S) \vdash \nabla \bot \Rightarrow \bot$.
\end{lem}
\begin{proof} Let $\mathcal{D}$ be a proof-tree for $\nabla (\top \rightarrow \bot) \Rightarrow \bot$ which we have by Lemma \ref{lem:l-nabla-box}.
	\begin{prooftree}
		\AXC{}
		\RightLabel{$Ex$}
		\UIC{$\bot \Rightarrow$}
		\RightLabel{$Rw$}
		\UIC{$\bot \Rightarrow \top \rightarrow \bot$}
		\RightLabel{$N$}
		\UIC{$\nabla \bot \Rightarrow \nabla (\top \rightarrow \bot)$}

		\AXC{$\mathcal{D}$}

		\RightLabel{$Cut$}
		\BIC{$\nabla \bot \Rightarrow \bot$}
	\end{prooftree}	
\end{proof}

\begin{lem}\label{lem:l-nabla-n-bot} For $n > 0$, $\stl(S) \vdash \nabla^n \bot \Rightarrow \bot$.
\end{lem}
\begin{proof} We will prove a stronger version: For $n \geq m > 0$, $\stl(S) \vdash \nabla^n \bot \Rightarrow \nabla^{n-m} \bot$. Let $\mathcal{D}_1$ be the proof-tree of Lemma \ref{lem:l-nabla-bot} which handles $n = m = 1$. Using induction on $m$, and denoting the proof-tree for $\nabla^n \bot \Rightarrow \nabla^{n-(m-1)} \bot$ from the induction hypothesis by IH, we have for $n > 1$
	\begin{prooftree}
		\AXC{IH}
		\noLine
		\UIC{$\nabla^n \bot \Rightarrow \nabla^{n-(m-1)} \bot$}

		\AXC{$\mathcal{D}_1$}
		\noLine
		\UIC{$\nabla \bot \Rightarrow \bot$}
		\doubleLine \RightLabel{$N^{(n-m)}$}
		\UIC{$\nabla^{n-(m-1)} \bot \Rightarrow \nabla^{n-m} \bot$}

		\RightLabel{$Cut$}
		\BIC{$\nabla^n \bot \Rightarrow \nabla^{n-m} \bot$}
	\end{prooftree}
\end{proof}

\begin{lem}\label{lem:l-nabla-dist-si} For any $n \geq 0$, $\stl(S) \vdash \nabla^n (A \rightarrow B) \Rightarrow \nabla^n A \rightarrow \nabla^n B$.
\end{lem}
\begin{proof}\quad
	\begin{prooftree}
		\AXC{}
		\RightLabel{$Id$}
		\UIC{$A \Rightarrow A$}
		
		\AXC{}
		\RightLabel{$Id$}
		\UIC{$B \Rightarrow B$}
		\RightLabel{$Lw$}
		\UIC{$A , B \Rightarrow B$}
		
		\RightLabel{$L \rightarrow$}
		\BIC{$\nabla (A \rightarrow B) , A \Rightarrow B$}
		\RightLabel{$N^{(n)}$} \doubleLine
		\UIC{$\nabla^{n+1} (A \rightarrow B) , \nabla^n A \Rightarrow \nabla^n B$}
		\RightLabel{$R \rightarrow$}
		\UIC{$\nabla^n (A \rightarrow B) \Rightarrow \nabla^n A \rightarrow \nabla^n B$}
	\end{prooftree}
\end{proof}


In the rest of this paper, by the \emph{height of a proof-tree} $\mathcal{D}$, denoted by $h(\mathcal{D})$, we mean the number of rule instances in its longest branch.

The theorem below shows that $\stl$ (and its extensions) deduce exactly the same sequents as $\gstl$ (and its extensions).

\begin{thm}\label{thm:stl-eq-gstl}
	Let $S \subseteq \{L, R, Fa, Fu\}$. For any sequent $\Gamma \Rightarrow \Delta$ in the language $\mathcal{L}$, $\stl(S) \vdash \Gamma \Rightarrow \Delta$ iff $\gstl(S) \vdash \Gamma \Rightarrow \Delta$.
\end{thm}
\begin{proof}
	One direction easily follows from the fact that all rules of $\stl(S)$ are just instances of $\gstl(S)$'s rules.
	For the other direction, we will use case analysis for the last rule in the proof-tree of $\Gamma \Rightarrow \Delta$ in $\gstl(S)$, which we call $\mathcal{D}$, and construct a proof-tree for it in $\stl(S)$ in each case.
	
	First, observe that $Id$ and $Ta$ are present in $\stl(S)$ and Lemma \ref{lem:l-nabla-n-bot} handles the $Ex$ case.
	For the other rules, use induction on the length of $\mathcal{D}$; the induction hypothesis will provide a proof-tree in $\stl(S)$ for the subtree(s) of $\mathcal{D}$.
	For the rules that are common between two systems, just apply the same rule (in $\stl(S)$) on the proof-tree(s) from the induction hypothesis to reach the desired sequent. For example, if $\mathcal{D}$ ends with $R \vee$, it suffices to apply $R \vee$ (of $\stl(S)$) on the proof-tree that we get from the induction hypothesis.
	In the cases for $\gstl(S)$'s stronger rules, which are $L \wedge_1$, $L \wedge_2$, $L \vee$ and $L \rightarrow$, do the same, and also cut the sequents proved in Lemmas \ref{lem:l-nabla-dist-and}, \ref{lem:l-nabla-dist-or} or \ref{lem:l-nabla-dist-si} into the resulting sequent.
	The case for $N$ divides into two further cases. One case is when $\Delta = A$ for some formula $A$, which is again handled by applying $N$ on the sequent from the induction hypothesis.
	The other case is when $\Delta = \{\}$, so we have $\Gamma \Rightarrow$ from the induction hypothesis, on the right of which we first introduce $\bot$ using $Rw$, and then apply $N$.
	Then we can cut it into the sequent proved in Lemma \ref{lem:l-nabla-bot}, and then into $Ex$, to derive $\nabla \Gamma \Rightarrow$ as was desired.
\end{proof}

\[\begin{tikzcd}
	&& IPC \\
	Fu & L && R & Fa \\
	\\
	&& STL
	\arrow[from=4-3, to=2-1]
	\arrow[from=2-4, to=1-3]
	\arrow[from=2-2, to=1-3]
	\arrow[from=4-3, to=2-2]
	\arrow[from=4-3, to=2-4]
	\arrow[from=4-3, to=2-5]
\end{tikzcd}\]

\subsection{Cut-elimination theorem}
Our proof of the admissibility of $cut$ in $\gstl(S)$ does not imply its admissibility in $\stl(S)$. But, as we will see, their equivalence (by theorem \ref{thm:stl-eq-gstl}) will allow us to use proof-theoretic methods for the cut-free system, and then translate its results back into a statement about the original system.

For technical reasons, we will work with a generalized form of the $cut$ rule, which nevertheless satisfies our goal here.

\begin{dfn}[$\nabla cut$ rule]\label{def:n-cut}
	Let $S \subseteq \{L, R, Fa, Fu\}$. We denote by $\gstl^+_{/\rightarrow}(S)$ the same systems defined as $\gstl(S)$, except that the $cut$ rule is replaced by the following generalization:
	\begin{prooftree}
		\AXC{$\Gamma \Rightarrow \nabla^m A$}
		\AXC{$\Sigma , \{\nabla^{n_i} A\}_{i \leq l} \Rightarrow \Delta$}
		\RightLabel{$\nabla cut$}
		\BIC{$\{\nabla^{n_i} \Gamma\}_{i \leq l} , \nabla^m \Sigma \Rightarrow \nabla^m \Delta$}
	\end{prooftree}
	where $\nabla^n$ means $\nabla$ applied $n$ times, $\{\nabla^{n_i} A\}_{i \leq l}$ means $\{\nabla^{n_0} A,\dotsb, \nabla^{n_l} A\}$ and $\{\nabla^{n_i} \Gamma\}_{i \leq l}$ means $\bigcup_{A \in \Gamma} \{\nabla^{n_i} A\}_{i \leq l}$ for some finite sequence of natural numbers $\{n_i\}_{i \leq l}$ of length $l+1$.
	We will refer to the set $\{\nabla^m A\} \cup \{\nabla^{n_i} A\}_{i \leq l}$ as the \emph{cut-burden} of this $cut$ instance. We will also call the tuple $(A, m, \{n_i\}_{i \leq l})$ the \emph{cut-data} of such instance.
\end{dfn}

\begin{thm}\label{cor:nc-riddance} Any sequent provable in $\gstl(S)$ is also provable in $\gstl^+(S)$ (for $S \subseteq \{L, R, Fa, Fu\}$).
\end{thm}
\begin{proof}
	We can just replace any instance of $cut$ with a cut-formula $A$ with similar instance of $\nabla cut$ with cut-data $(A, \{0\})$.
\end{proof}

Next, we need to define a measure for the complexity of the $\nabla cut$ rule. We will first define this measure for formulas, and then extend it to rule instances and proof-trees.

\begin{dfn}[Rank]\label{dfn:rank}
	Rank of a formula $A$ is defined as
	\[ \rho(A) = \begin{cases}
	1 & \quad ; A \in P \cup \{ \bot, \top \} \\
	\rho(B) & \quad ; A = \nabla B \\
	max(\rho(B), \rho(C)) + 1 & \quad ; A = B \circ C \quad (\circ \in \{ \land, \lor, \rightarrow \})
	\end{cases} \]
	Notice that $\nabla$ does not increase the rank of a formula.
	
	We also define rank for rule instances and proof-trees as follows. Rank of an instance of the $\nabla Cut$ rule with cut-data $(A, m, \{n_i\}_{i \leq l})$ is defined to be the rank of $A$. Rank of any other rule instance is $0$.
	For a proof tree $\mathcal{D}$, $\rho(\mathcal{D})$ is the maximum rank of all of its rule instances.
\end{dfn}

The next lemma will help in handling of a case in the main theorem.

\begin{lem}\label{lem:gstl-top-redundant} If $\gstl^+(S)$ proves $\Gamma , \{\nabla^{n_i} \top\}_{i \leq l} \Rightarrow \Delta$, then it also proves $\Gamma \Rightarrow \Delta$ with a proof-tree of at most the same rank (for $S \subseteq \{L, R, Fa, Fu\}$).
\end{lem}
\begin{proof}
Suppose $\mathcal{D}$ is a proof-tree for $\Gamma , \{\nabla^{n_i} \top\}_{i \leq l} \Rightarrow \Delta$ in $\gstl^+(S)$ and consider different cases for the last rule of $\mathcal{D}$ with possible subtrees $\mathcal{D}_0$ and $\mathcal{D}_1$.
By induction on $\mathcal{D}$ we can assume that the theorem holds for $\mathcal{D}_0$ and $\mathcal{D}_1$.
First, observe that $Ta$ and $Ex$ cases are trivially ruled out. In the cases for $Id$, which implies $l = 0$, we have $\Rightarrow \nabla^{n_0} \top$ by $n_0$ times applications of $N$ on $Ta$. In $Lw$ case, if $l = 0$ and $\nabla^{n_0} \top$ is principal, $\mathcal{D}_0$ itself proves the desired sequent, but if $l > 0$, then the induction hypothesis gives the desired sequent. The cases for $Lc$ or $L$ on some $\nabla^{n_j} \top$  are similar. In all other cases, just apply induction hypothesis on $\mathcal{D}_0$ (and possibly $\mathcal{D}_1$), and then apply the same last rule. Notice that $\nabla Cut$ is not used except in the case for $\nabla Cut$ itself, where it is applied with an instance of the same rank, so the resulting proof-tree will not be of a higher rank than that of $\mathcal{D}$.
\end{proof}

The following theorem shows that we can imitate any instance of the cut rule in a proof-tree of lower rank.

\input{thm/gstl-proof-reduction}

The next theorem shows that the $cut$ rule is redundant.

\begin{thm}[Cut Elimination]\label{thm:gstl-cut-elim}
	For any $\Gamma$ and $\Delta$, if $\Gamma \Rightarrow \Delta$ is provable by $\gstl(S)$, then it is also provable by $\gstl(S)-\{cut\}$ (for $S \subseteq \{ L, R, Fa, Fu \}$).
\end{thm}
\begin{proof}
		First, we will show that for any non-zero-rank proof of $\Gamma \Rightarrow \Delta$ like $\mathcal{D}$ in $\gstl(S)$, there is another proof of the same sequent with a lower rank. Suppose $\mathcal{D}$ has subtree(s) called $\mathcal{D}_0$ (and possibly $\mathcal{D}_1$, if the last rule has two assumptions). Using induction on $h(\mathcal{D})$, the induction hypothesis for $\mathcal{D}_i ~(i \in \{0,1\})$ gives us a proof-tree with the same conclusion, which we call $IH(\mathcal{D}_i)$, but with a lower rank, i.e., $\rho(IH(\mathcal{D}_i)) < \rho(\mathcal{D}_i)$. We now consider two cases for the last rule of $\mathcal{D}$.

	\begin{enumerate}[label=\Roman*]
		\item If the last rule of $\mathcal{D}$ is of a lower rank than $\mathcal{D}$ itself, i.e., the $cut$ instance in $\mathcal{D}$ with the maximum rank is not the last rule, then we can apply the same last rule on $IH(\mathcal{D}_0)$ (and possibly $\mathcal{D}_1$), to construct a proof of $\Gamma \Rightarrow \Delta$ with a lower rank.
		
		\item If the last rule of $\mathcal{D}$ is an instance of $cut$ rule with the same rank as $\mathcal{D}$ itself, then it is indeed the $cut$ instance with the maximum rank. So we can apply Theorem \ref{thm:gstl-cut-reduction} to $IH(\mathcal{D}_0)$ and $IH(\mathcal{D}_1)$ to prove the same sequent as it would be proved by $cut$, but with a lower rank. (Recall that $\nabla cut$ is just a generalization of $cut$, so the theorem applies.)
	\end{enumerate}
	So for any proof of $\Gamma \Rightarrow \Delta$ in $\gstl(S)$, we also have a proof of rank $0$, which is cut-free.
\end{proof}

\subsection{Logic of spacetime with intuitionistic implication}
A conservative extension of $\stl$ is introduced in the next definiotion. This extension, which is called $\istl$ here, enrisches $\stl$ with intuitionistic implication. $\istl$ is conservative in the sense that adding intuitionistic implication will give no more power to the system in proving propositions in the original language of $\stl$.

\begin{dfn}\label{istl}
	Let $\mathcal{L}_\supset=\langle \wedge, \vee, \top, \bot, \nabla, \rightarrow, \supset, P \rangle$ where $P$ is the set of all atomic propositions. Define $\istl$ as the logic of the sequent-style system defined by the same rules as Definition \ref{stl}, plus the following rules.
\end{dfn}
\begin{multicols}{2}
  \begin{prooftree}
    \AXC{$\Gamma, B \Rightarrow \Delta$}
    \AXC{$\Gamma \Rightarrow A$}
    \RightLabel{$L \supset$}
    \BIC{$\Gamma, A \supset B \Rightarrow \Delta$}
  \end{prooftree}
  \columnbreak
  \begin{prooftree}
    \AXC{$\Gamma, A \Rightarrow B$}
    \RightLabel{$R \supset$}
    \UIC{$\Gamma \Rightarrow A \supset B$}
  \end{prooftree}
\end{multicols}

[introducing iSTL, the conservative extension with intuitionistic implication]

[introducing the equivalent cut-free system, iGSTL]

\subsection{applications}
[visser rule]

[disjuction property]

[subformula property]

\section{Interpolation theorems}
[deductive interpolation for iSTL+L]

[craig's interpolation for intuitionistic extension]

\section{Conclusion and future works}

\section{Acknowledgement}

\bibliographystyle{abbrv}
\bibliography{refs}
\end{document}
