\documentclass[12pt,a4paper]{article}
\usepackage{amsmath}
\usepackage{amsthm}
\usepackage{amsfonts}
%\usepackage{fdsymbol}

\usepackage{paracol}

\usepackage{authblk}

\usepackage[mathscr]{eucal}

\usepackage{bussproofs}
\EnableBpAbbreviations

\usepackage{amssymb}
\usepackage{tikz}
\usepackage{enumitem}
\usepackage{multicol}
\tikzset{node distance=2cm, auto}

\theoremstyle{plain}
\newtheorem{thm}{Theorem}[section]

\renewcommand{\thethm}{\arabic{section}.\arabic{thm}}
\newtheorem{lem}[thm]{Lemma}

\newtheorem{cor}[thm]{Corollary}
\theoremstyle{definition}
\newtheorem{dfn}[thm]{Definition}
\newtheorem{exam}[thm]{Example}
\newtheorem{rem}[thm]{Remark}
\newtheorem{nota}[thm]{Notation}
\newtheorem{exer}[thm]{Exercise}

\def\d{\displaystyle}
\def\PA{\mathrm{PA}}
\def\Pr{\mathrm{Pr}}
\def\Prf{\mathrm{Prf}}
\def\PR{\mathrm{PR}}
\def\IPC{\mathrm{IPC}}
\def\Proofs{\mathrm{Proofs}}
\def\int{\mathrm{int}}
\def\WT{\mathrm{WT}}
\def\exp{\mathrm{exp}}
\def\CHaus{\mathrm{CHaus}}
\def\Fin{\mathrm{Fin}}
\def\E{\mathrm{E}}
\def\PR{\mathrm{PR}}
\def\Top{\mathrm{Top}}
\def\S4{\mathrm{S4}}
\def\Hom{\mathrm{Hom}}
\def\Set{\mathrm{Set}}

\newcommand{\stl}{\mathbf{STL}}
\newcommand{\gstl}{\mathbf{GSTL}}
\newcommand{\istl}{\mathbf{iSTL}}
\newcommand{\igstl}{\mathbf{iGSTL}}

\begin{document}
    
\title{Interpolation for Logics of Spacetime}
\author[]{Amirhossein Akbar Tabatabai, Majid Alizadeh, Alireza Mahmoudian}
\affil[]{ }
\date{\today}
\maketitle

\section{Introduction}
A generalization of the notion of implication is introduced in \cite{amir}, as a binary operator which internalizes the reflexive and transitive properties of the provability order on the set of all propositions. This generalization gives rise to a classification of some sub-intuitionistic and sub-structural logics with various behaviours of their respecting implication operators. One of its specifications, which is called the logic of \emph{spacetime} in \cite{amir} with the goal of subduing the impredicative nature of the implication. The usual interpretation of an intuitionistic proof for an implication is a structure that converts \emph{all} proofs of its antecedent to some proof for its consequent. This interpretation is impredicative in the sense that we are quantifying over the set of all proofs which we are just defining. The impredicativity of implication clearly shows up when one is using a conditional assumption in an intuitionistic proof, which amounts to an application of the rule of \emph{modus ponens}: What this rule essentially says is that if we have a proof for an implication and a proof of its antecedent at the same time, then we can conclude its consequent. The logic of spacetime remedies this by restricting the use of conditional assumption in the proof such that its modus ponens rule would not take a proof of conditional statement itself, but a proof of modal operator, called $\nabla$, applied to the conditional statement. We can read this application of $\nabla$ as a \emph{delay} in use of implication, and interpret the proof of conditional statement as a function that converts all \emph{later} proofs of its antecedent to some \emph{later} proof for its consequent.

The spacetime logic has a Kripke style semanitcs, via a triple $(W, \leq, R)$ consisting of a Kripke frame $(W, R)$ and a poset $(W, \leq)$, where $\nabla$ is interpreted as the modality that looks backward in the frame.\\

A summary of this paper is as follows:
In the next section, we will introduce a sequent style calculus for the spacetime logic, called $\stl$ and introduce an equivalent cut-free system $\gstl$, and then survey some extensions and their respecting semantics, including an extension of $\stl$ with a Heyting implication, called $\istl$, and simillarly, an equivalent cut-free system $\igstl$.


In the third section, the cut-elimination theorems for $\stl$ and its extensions will be proved. Then we will use this theorem to deduce impoertant results about $\stl$ and its extensinos, such as subformula property, disjuction property and admisibility of the Visser's rule.

And in the last section, we will show that some extensions of $\stl$ have interpolation property.

\section{Sequent calculi for the logics of spacetime}

Consider the language $\mathcal{L}=\langle \wedge, \vee, \top, \bot, \nabla, \rightarrow \rangle$. Define $\stl$ as the logic of the sequent-style system defined by the following rules, where in a sequent of the form $\Gamma \Rightarrow \Delta$, the left-hand side is a multiset of formulas $\Gamma$, and the right-hand side $\Delta$ is either a formula or empty.

\begin{flushleft}
 \textbf{Axioms:}
\end{flushleft}
\begin{center}
 \begin{tabular}{c c c}
 \AxiomC{}
 \RightLabel{$Id$}
 \UnaryInfC{$ A \Rightarrow A$}
 \DisplayProof \;\;\;
 &
 \AxiomC{}
 \RightLabel{$Ta$}
 \UnaryInfC{$ \Rightarrow \top$}
 \DisplayProof\;\;\;
 &
 \AxiomC{}
 \RightLabel{$Ex$}
 \UnaryInfC{$ \bot \Rightarrow $}
 \DisplayProof
 \\[3ex]
\end{tabular}
\end{center}

\begin{flushleft}
 		\textbf{Structural Rules:}
\end{flushleft}

\begin{center}
 \begin{tabular}{c c c}
 \AxiomC{$ \Gamma \Rightarrow \Delta$}
 \RightLabel{$L w$}
 \UnaryInfC{$ \Gamma, A \Rightarrow \Delta$}
 \DisplayProof
 &
 \AxiomC{$ \Gamma \Rightarrow $}
\RightLabel{$R w$}
 \UnaryInfC{$\Gamma \Rightarrow A$}
 \DisplayProof
 &
 \AxiomC{$ \Gamma, A, A \Rightarrow \Delta$}
\RightLabel{$Lc$}
 \UnaryInfC{$\Gamma, A \Rightarrow \Delta$}
 \DisplayProof
  \\[3ex]
\end{tabular}
\end{center}

\begin{flushleft}
 		\textbf{Cut:}
\end{flushleft}
\begin{center}
  	\begin{tabular}{c}

		\AxiomC{$ \Gamma \Rightarrow A$}
		\AxiomC{$\Pi, A \Rightarrow \Delta$}
		\RightLabel{$cut$}
		\BinaryInfC{$ \Pi, \Gamma \Rightarrow \Delta$}
		\DisplayProof
		 \\[3ex]
		\end{tabular}
\end{center}

\begin{flushleft}
 \textbf{Conjunction Rules:}
\end{flushleft}
\begin{center}
 \begin{tabular}{c c c}
\AxiomC{$ \Gamma, A \Rightarrow \Delta$}
 \RightLabel{$L \wedge_1$}
 \UnaryInfC{$ \Gamma, A \wedge B \Rightarrow \Delta$}
 \DisplayProof
 &
 \AxiomC{$ \Gamma, B \Rightarrow \Delta$}
 \RightLabel{$L \wedge_2$}
 \UnaryInfC{$\Gamma, A \wedge B \Rightarrow \Delta$}
 \DisplayProof
	   		&
   		\AxiomC{$\Gamma \Rightarrow A$}
   		\AxiomC{$\Gamma \Rightarrow B$}
   		\RightLabel{$R \wedge$}
   		\BinaryInfC{$ \Gamma \Rightarrow A \wedge B $}
   		\DisplayProof
   			\\[3 ex]
\end{tabular}
\end{center}

\begin{flushleft}
 \textbf{Disjunction Rules:}
\end{flushleft}
\vspace{.001pt}
\begin{center}
 \begin{tabular}{c c c}
 \AxiomC{$ \Gamma, A \Rightarrow \Delta$}
 \AxiomC{$\Gamma, B \Rightarrow \Delta$}
 \RightLabel{$L \vee_1$}
 \BinaryInfC{$ \Gamma, A \vee B \Rightarrow \Delta$}
 \DisplayProof
 &
 \AxiomC{$\Gamma \Rightarrow A$}
 \RightLabel{$R \vee_2$}
 \UnaryInfC{$\Gamma \Rightarrow A \vee B$}
 \DisplayProof
 &
 \AxiomC{$\Gamma \Rightarrow B$}
 \RightLabel{$R \vee$}
 \UnaryInfC{$\Gamma \Rightarrow A \vee B$}
 \DisplayProof
 \\[3ex]
\end{tabular}
\end{center}



\begin{flushleft}
	\textbf{Implication Rules:}
 \end{flushleft}
 \vspace{.001pt}
 \begin{center}
	\begin{tabular}{c c}
	\AxiomC{$\Gamma \Rightarrow A$}
	\AxiomC{$\Gamma, B \Rightarrow \Delta$}
	\RightLabel{$L \rightarrow$}
	\BinaryInfC{$\Gamma, \nabla (A \rightarrow B) \Rightarrow \Delta$}
	\DisplayProof
	&
	\AxiomC{$\nabla \Gamma, A \Rightarrow B$}
	\RightLabel{$R \rightarrow$}
	\UnaryInfC{$\Gamma \Rightarrow A \rightarrow B$}
	\DisplayProof
	\\[3ex]
 \end{tabular}
 \end{center}

\begin{flushleft}
  \textbf{Modal Rule:}
\end{flushleft}
\vspace{.001pt}
\begin{center}
 \begin{tabular}{c}
 \AxiomC{$\Gamma \Rightarrow A$}
 \RightLabel{$N$}
 \UnaryInfC{$\nabla \Gamma \Rightarrow \nabla A$}
 \DisplayProof
 \\[3ex]
\end{tabular}
\end{center}


It is worth mentioning that our naming scheme for the systems differs slightly with \cite{amir}. Also notice that the sequent-style systems that are introduced in the mentioned paper use sequences of formulas instead of multisets, and have an explicit exchange rule.

$\stl$ is shown to be sound and complete with respect to the set of all Kripke frames of the form $(W, \leq, R)$, where $(W, \leq)$ is a poset, and $R$ is binary relation which behaves functorial over $(W, \leq) \times (W, \leq)^{op}$, in the sense that for all $u, v, w, z \in W$ such that $v \leq u$, $w \leq z$ and $u R w$ implies $v R z$. For all connectives except $\nabla$, the forcing relation is defined inductively with respect to $R$, similar to the Kripke semantics for intuitionistic logic. For a formula $\nabla A$, define the forcing relation as follows:
\begin{center}
$w \Vdash \nabla A$ if and only if there exists $u \in W$ where $u R w$ and $w \Vdash A$.
\end{center}


If we also add one or more of the following rules to the system, we will have an \emph{extension of} $\stl$, which are denoted by $\stl(S)$ where $S$ is any combination of these rules:

	\begin{prooftree}
		\RightLabel{$L$}
		\AXC{$\Gamma, A \Rightarrow \Delta$}
		\UIC{$\Gamma, \nabla A \Rightarrow \Delta$}
	\end{prooftree}

	\begin{prooftree}
		\RightLabel{$R$}
		\AXC{$\nabla \Gamma, \Sigma \Rightarrow \Delta$}
		\UIC{$\Gamma, \Sigma \Rightarrow \Delta$}
	\end{prooftree}



	\begin{prooftree}
		\RightLabel{$Fa$}
		\AXC{$\Gamma , A \Rightarrow B$}
		\UIC{$\Gamma \Rightarrow \nabla(A \rightarrow B)$}
	\end{prooftree}

	\begin{prooftree}
		\RightLabel{$Fu$}
		\AXC{$\nabla \Gamma \Rightarrow \nabla A$}
		\UIC{$\Gamma \Rightarrow A$}
	\end{prooftree}




[introducing GSTL]

[two systems are equivalnt]

[extensions]

[introducing iSTL, the conservative extension with intuitionistic implication]

[introducing the equivalent cut-free system, iGSTL]

[bunch of lemmas]

[diagram]

\section{Cut-elimination theorem}
[cut reduction]

[cut elimination]

\subsection{applications}
[visser rule]

[disjuction property]

[subformula property]

\section{Interpolation theorems}
[deductive interpolation for iSTL+L]

[craig's interpolation for intuitionistic extension]

\section{Conclusion and future works}

\section{Acknowledgement}

\bibliographystyle{abbrv}
\bibliography{refs}
\end{document}