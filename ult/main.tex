\documentclass[10pt,a4paper]{amsart}
\usepackage{amsmath}
\usepackage{amssymb}
\usepackage{mathabx}
\usepackage{titlesec}
\usepackage{fullpage}
\usepackage{tikz-cd}
\usepackage{rotating}
\usepackage{pdflscape}
\usepackage{multicol}
\usepackage{multirow}
\usepackage{diagbox}
\usepackage[left=.5in,right=.5in,top=.5in,bottom=.5in]{geometry}
\usepackage{enumitem}
\usepackage[colorlinks]{hyperref}
\usepackage{bussproofs}

\setitemize{topsep=3pt,parsep=5pt,partopsep=0pt,label=,leftmargin=1.3pc}
\titleformat{\section}[runin]{\normalfont\bfseries}{\thesection}{0.5em}{}
\titlespacing{\section}{0pc}{5ex plus .1ex minus .2ex}{1pc}
\titleformat{\subsection}[runin]{\normalfont\bfseries}{\thesubsection}{0.7em}{}
\titlespacing{\subsection}{0pc}{2ex plus .1ex minus .2ex}{1pc}
\titleformat{\subsubsection}[runin]{\normalfont\bfseries}{\thesubsubsection}{0.7em}{}
\titlespacing{\subsubsection}{0pc}{2ex plus .1ex minus .2ex}{1pc}
\newcommand\eqn{\refstepcounter{equation}\tag{\theequation}}
\binoppenalty=\maxdimen
\relpenalty=\maxdimen
\newcommand{\ul}{\ulcorner}
\newcommand{\ur}{\urcorner}
\newcommand{\val}[1]{\ulcorner len1 \urcorner}
\newcommand{\caseref}[1]{\hyperref[#1]{\ref{#1}}}
\newcommand{\rot}{\rotatebox{90}}
\newcommand{\p}{\partial}
\newcommand{\todo}[1]{{\color{red}\textbf{TODO} #1}}
\EnableBpAbbreviations

\begin{document}
   
\title{Interpolation for Logics of Spacetime}
\author{Amirhossein Akbar Tabatabai}


\author{Majid Alizadeh}

\author{Alireza Mahmoudian}

\date{\today}
 
\begin{abstract}
	\input{abstract}
\end{abstract}

\maketitle

\keywords{Intuitionistic Logic, Cut-elimination, Interpolation}

\subjclass{03B20, 03F05, 03C40}

\section{Introduction}
\input{introduction}

\section{Sequent calculi for the logics of spacetime}
In this section, will define the systems $\stl$ (defined as $\istl(N)$ in \cite{amir}) and $\gstl$, and show that these two systems are equivalent.

\[\begin{tikzcd}
	&& IPC \\
	Fu & L && R & Fa \\
	\\
	&& STL
	\arrow[from=4-3, to=2-1]
	\arrow[from=2-4, to=1-3]
	\arrow[from=2-2, to=1-3]
	\arrow[from=4-3, to=2-2]
	\arrow[from=4-3, to=2-4]
	\arrow[from=4-3, to=2-5]
\end{tikzcd}\]

Now, we will introduce $\gstl$, which we then show to be equivalent with $\stl$, and also cut-free. In the rest of this paper, we will use the short hand $\nabla^n A$ for $\nabla (\nabla^{n-1} A)$, for $n > 0$, where $\nabla^0 A$ is $A$.

\input{dfn/gstl}

To show the equivalnce of two systems, we will use the following lemmas. Notice that the proof-trees are constructible in $\stl(S)$ for $S \subseteq \{ L, R, Fa, Fu \}$.

\input{thm/l-nabla-dist-and}

\input{thm/l-nabla-box}

\begin{lem}\label{lem:l-box-nabla} $\stl(S) \vdash A \Rightarrow \top \rightarrow \nabla A$.
\end{lem}
\begin{proof}\quad
	\begin{prooftree}
		\AXC{}
		\RightLabel{$Id$}
		\UIC{$\nabla A \Rightarrow \nabla A$}
		
		\RightLabel{$Lw$}
		\UIC{$\nabla A , \top \Rightarrow \nabla A$}

		\RightLabel{$R \rightarrow$}
		\UIC{$A \Rightarrow \top \rightarrow \nabla A$}
	\end{prooftree}
\end{proof}

\begin{lem}\label{lem:l-nabla-dist-or} $\stl(S) \vdash \nabla (A \vee B) \Rightarrow \nabla A \vee \nabla B$.
\end{lem}
\begin{proof} Let $\mathcal{D}$ be a proof-tree we can construct for $\nabla (\top \rightarrow (\nabla A \vee \nabla B)) \Rightarrow \nabla A \vee \nabla B$ from Lemma \ref{lem:l-nabla-box}.

	\begin{prooftree}
		\AXC{}
		\RightLabel{$Id$}
		\UIC{$\nabla A \Rightarrow \nabla A$}
		\RightLabel{$R\vee_1$}
		\UIC{$\nabla A \Rightarrow \nabla A \vee \nabla B$}
		\RightLabel{$Lw$}
		\UIC{$\nabla A , \top \Rightarrow \nabla A \vee \nabla B$}
		\RightLabel{$R \rightarrow$}
		\UIC{$A \Rightarrow \top \rightarrow (\nabla A \vee \nabla B)$}

		\AXC{}
		\RightLabel{$Id$}
		\UIC{$\nabla B \Rightarrow \nabla B$}
		\RightLabel{$R\vee_2$}
		\UIC{$\nabla B \Rightarrow \nabla A \vee \nabla B$}
		\RightLabel{$Lw$}
		\UIC{$\nabla B , \top \Rightarrow \nabla A \vee \nabla B$}
		\RightLabel{$R \rightarrow$}
		\UIC{$B \Rightarrow \top \rightarrow (\nabla A \vee \nabla B)$}

		\RightLabel{$L\vee$}
		\BIC{$A \vee B \Rightarrow \top \rightarrow (\nabla A \vee \nabla B)$}
		\RightLabel{$N$}
		\UIC{$\nabla (A \vee B) \Rightarrow \nabla (\top \rightarrow (\nabla A \vee \nabla B))$}

		\AXC{$\mathcal{D}$}

		\RightLabel{$Cut$}
		\BIC{$\nabla (A \vee B) \Rightarrow \nabla A \vee \nabla B$}
	\end{prooftree}
\end{proof}

\begin{lem}\label{lem:l-nabla-n-dist-or} For all $n \geq 0$, $\stl(S) \vdash \nabla^n (A \vee B) \Rightarrow \nabla^n A \vee \nabla^n B$.
\end{lem}
\begin{proof} The proof is trivial when $n = 0$. Let $\mathcal{D}_1$ be the proof-tree of Lemma \ref{lem:l-nabla-dist-or}, which proves the case for $n = 1$. For any $n > 1$ we have

	\begin{prooftree}
		\AXC{$\mathcal{D}_{n-1}$}
		\noLine
		\UIC{$\nabla^{n-1} (A \vee B) \Rightarrow \nabla^{n-1} A \vee \nabla^{n-1} B$}
		\RightLabel{$N$}
		\UIC{$\nabla^n (A \vee B) \Rightarrow \nabla (\nabla^{n-1} A \vee \nabla^{n-1} B)$}

		\AXC{$\mathcal{D}_1$}
		\noLine
		\UIC{$\nabla (\nabla^{n-1} A \vee \nabla^{n-1} B) \Rightarrow \nabla^n A \vee \nabla^n B$}
		
		\RightLabel{$Cut$} \LeftLabel{$\mathcal{D}_n:$}
		\BIC{$\nabla^n (A \vee B) \Rightarrow \nabla^n A \vee \nabla^n B$}
	\end{prooftree}
\end{proof}

\begin{lem}\label{lem:l-nabla-bot} $\stl(S) \vdash \nabla \bot \Rightarrow \bot$.
\end{lem}
\begin{proof} Let $\mathcal{D}$ be a proof-tree for $\nabla (\top \rightarrow \bot) \Rightarrow \bot$ which we have by Lemma \ref{lem:l-nabla-box}.
	\begin{prooftree}
		\AXC{}
		\RightLabel{$Ex$}
		\UIC{$\bot \Rightarrow$}
		\RightLabel{$Rw$}
		\UIC{$\bot \Rightarrow \top \rightarrow \bot$}
		\RightLabel{$N$}
		\UIC{$\nabla \bot \Rightarrow \nabla (\top \rightarrow \bot)$}

		\AXC{$\mathcal{D}$}

		\RightLabel{$Cut$}
		\BIC{$\nabla \bot \Rightarrow \bot$}
	\end{prooftree}	
\end{proof}

\begin{lem}\label{lem:l-nabla-n-bot} For $n > 0$, $\stl(S) \vdash \nabla^n \bot \Rightarrow \bot$.
\end{lem}
\begin{proof} We will prove a stronger version: For $n \geq m > 0$, $\stl(S) \vdash \nabla^n \bot \Rightarrow \nabla^{n-m} \bot$. Let $\mathcal{D}_1$ be the proof-tree of Lemma \ref{lem:l-nabla-bot} which handles $n = m = 1$. Using induction on $m$, and denoting the proof-tree for $\nabla^n \bot \Rightarrow \nabla^{n-(m-1)} \bot$ from the induction hypothesis by IH, we have for $n > 1$
	\begin{prooftree}
		\AXC{IH}
		\noLine
		\UIC{$\nabla^n \bot \Rightarrow \nabla^{n-(m-1)} \bot$}

		\AXC{$\mathcal{D}_1$}
		\noLine
		\UIC{$\nabla \bot \Rightarrow \bot$}
		\doubleLine \RightLabel{$N^{(n-m)}$}
		\UIC{$\nabla^{n-(m-1)} \bot \Rightarrow \nabla^{n-m} \bot$}

		\RightLabel{$Cut$}
		\BIC{$\nabla^n \bot \Rightarrow \nabla^{n-m} \bot$}
	\end{prooftree}
\end{proof}

\begin{lem}\label{lem:l-nabla-dist-si} For any $n \geq 0$, $\stl(S) \vdash \nabla^n (A \rightarrow B) \Rightarrow \nabla^n A \rightarrow \nabla^n B$.
\end{lem}
\begin{proof}\quad
	\begin{prooftree}
		\AXC{}
		\RightLabel{$Id$}
		\UIC{$A \Rightarrow A$}
		
		\AXC{}
		\RightLabel{$Id$}
		\UIC{$B \Rightarrow B$}
		\RightLabel{$Lw$}
		\UIC{$A , B \Rightarrow B$}
		
		\RightLabel{$L \rightarrow$}
		\BIC{$\nabla (A \rightarrow B) , A \Rightarrow B$}
		\RightLabel{$N^{(n)}$} \doubleLine
		\UIC{$\nabla^{n+1} (A \rightarrow B) , \nabla^n A \Rightarrow \nabla^n B$}
		\RightLabel{$R \rightarrow$}
		\UIC{$\nabla^n (A \rightarrow B) \Rightarrow \nabla^n A \rightarrow \nabla^n B$}
	\end{prooftree}
\end{proof}


In the rest of this paper, by the \emph{height of a proof-tree} $\mathcal{D}$, denoted by $h(\mathcal{D})$, we mean the number of rule instances in its longest branch.

The theorem below shows that $\stl$ (and its extensions) deduce exactly the same sequents as $\gstl$ (and its extensions).

\begin{thm}\label{thm:stl-eq-gstl}
	Let $S \subseteq \{L, R, Fa, Fu\}$. For any sequent $\Gamma \Rightarrow \Delta$ in the language $\mathcal{L}$, $\stl(S) \vdash \Gamma \Rightarrow \Delta$ iff $\gstl(S) \vdash \Gamma \Rightarrow \Delta$.
\end{thm}
\begin{proof}
	One direction easily follows from the fact that all rules of $\stl(S)$ are just instances of $\gstl(S)$'s rules.
	For the other direction, we will use case analysis for the last rule in the proof-tree of $\Gamma \Rightarrow \Delta$ in $\gstl(S)$, which we call $\mathcal{D}$, and construct a proof-tree for it in $\stl(S)$ in each case.
	
	First, observe that $Id$ and $Ta$ are present in $\stl(S)$ and Lemma \ref{lem:l-nabla-n-bot} handles the $Ex$ case.
	For the other rules, use induction on the length of $\mathcal{D}$; the induction hypothesis will provide a proof-tree in $\stl(S)$ for the subtree(s) of $\mathcal{D}$.
	For the rules that are common between two systems, just apply the same rule (in $\stl(S)$) on the proof-tree(s) from the induction hypothesis to reach the desired sequent. For example, if $\mathcal{D}$ ends with $R \vee$, it suffices to apply $R \vee$ (of $\stl(S)$) on the proof-tree that we get from the induction hypothesis.
	In the cases for $\gstl(S)$'s stronger rules, which are $L \wedge_1$, $L \wedge_2$, $L \vee$ and $L \rightarrow$, do the same, and also cut the sequents proved in Lemmas \ref{lem:l-nabla-dist-and}, \ref{lem:l-nabla-dist-or} or \ref{lem:l-nabla-dist-si} into the resulting sequent.
	The case for $N$ divides into two further cases. One case is when $\Delta = A$ for some formula $A$, which is again handled by applying $N$ on the sequent from the induction hypothesis.
	The other case is when $\Delta = \{\}$, so we have $\Gamma \Rightarrow$ from the induction hypothesis, on the right of which we first introduce $\bot$ using $Rw$, and then apply $N$.
	Then we can cut it into the sequent proved in Lemma \ref{lem:l-nabla-bot}, and then into $Ex$, to derive $\nabla \Gamma \Rightarrow$ as was desired.
\end{proof}


\subsection{Cut-elimination theorem}
Our proof of the admissibility of $cut$ in $\gstl(S)$ does not imply its admissibility in $\stl(S)$. But, as we will see, their equivalence (by theorem \ref{thm:stl-eq-gstl}) will allow us to use proof-theoretic methods for the cut-free system, and then translate its results back into a statement about the original system.

For technical reasons, we will work with a generalized form of the $cut$ rule, which nevertheless satisfies our goal here.

\begin{dfn}[$\gstl^+$]\label{def:gstlp}
	Let $S \subseteq \{L, R, Fa, Fu\}$. We denote by $\gstl^+(S)$ the same systems defined as $\gstl(S)$, except that the $cut$ rule is replaced by the following generalization:
	\begin{prooftree}
		\AXC{$\Gamma \Rightarrow A$}
		\AXC{$\Sigma , \{\nabla^{n_i} A\}_{i \leq l} \Rightarrow \Delta$}
		\RightLabel{$\nabla cut$}
		\BIC{$\{\nabla^{n_i} \Gamma\}_{i \leq l} , \Sigma \Rightarrow \Delta$}
	\end{prooftree}
	where $\nabla^n$ means $\nabla$ applied $n$ times, $\{\nabla^{n_i} A\}_{i \leq l}$ means $\{\nabla^{n_0} A,\dotsb, \nabla^{n_l} A\}$ and $\{\nabla^{n_i} \Gamma\}_{i \leq l}$ means $\bigcup_{A \in \Gamma} \{\nabla^{n_i} A\}_{i \leq l}$ for some finite sequence of natural numbers $\{n_i\}_{i \leq l}$ of length $l+1$.
	We will refer to the set $\{\nabla^{n_i} A\}_{i \leq l}$ as the \emph{cut-burden} of this $cut$ instance. We will also call the pair $(A, \{n_i\}_{i \leq l})$ the \emph{cut-data} of such instance.
\end{dfn}

\begin{thm}\label{cor:nc-riddance} Any sequent provable in $\gstl(S)$ is also provable in $\gstl^+(S)$ (for $S \subseteq \{L, R, Fa, Fu\}$).
\end{thm}
\begin{proof}
	We can just replace any instance of $cut$ with a cut-formula $A$ with similar instance of $\nabla cut$ with cut-data $(A, \{0\})$.
\end{proof}

Next, we need to define a measure for the complexity of the $\nabla cut$ rule. We will first define this measure for formulas, and then extend it to rule instances and proof-trees.

\begin{dfn}[Rank]\label{dfn:rank}
	Rank of a formula $A$ is defined as
	\[ \rho(A) = \begin{cases}
	1 & \quad ; A \in P \cup \{ \bot, \top \} \\
	\rho(B) & \quad ; A = \nabla B \\
	max(\rho(B), \rho(C)) + 1 & \quad ; A = B \circ C \quad (\circ \in \{ \land, \lor, \rightarrow \})
	\end{cases} \]
	Notice that $\nabla$ does not increase the rank of a formula.
	
	We also define rank for rule instances and proof-trees as follows. Rank of an instance of the $\nabla Cut$ rule with cut-data $(A, m, \{n_i\}_{i \leq l})$ is defined to be the rank of $A$. Rank of any other rule instance is $0$.
	For a proof tree $\mathcal{D}$, $\rho(\mathcal{D})$ is the maximum rank of all of its rule instances.
\end{dfn}

The next lemma will help in handling of a case in the main theorem.

\begin{lem}\label{lem:gstl-top-redundant} If $\gstl^+(S)$ proves $\Gamma , \{\nabla^{n_i} \top\}_{i \leq l} \Rightarrow \Delta$, then it also proves $\Gamma \Rightarrow \Delta$ with a proof-tree of at most the same rank (for $S \subseteq \{L, R, Fa, Fu\}$).
\end{lem}
\begin{proof}
Suppose $\mathcal{D}$ is a proof-tree for $\Gamma , \{\nabla^{n_i} \top\}_{i \leq l} \Rightarrow \Delta$ in $\gstl^+(S)$ and consider different cases for the last rule of $\mathcal{D}$ with possible subtrees $\mathcal{D}_0$ and $\mathcal{D}_1$.
By induction on $\mathcal{D}$ we can assume that the theorem holds for $\mathcal{D}_0$ and $\mathcal{D}_1$.
First, observe that $Ta$ and $Ex$ cases are trivially ruled out. In the cases for $Id$, which implies $l = 0$, we have $\Rightarrow \nabla^{n_0} \top$ by $n_0$ times applications of $N$ on $Ta$. In $Lw$ case, if $l = 0$ and $\nabla^{n_0} \top$ is principal, $\mathcal{D}_0$ itself proves the desired sequent, but if $l > 0$, then the induction hypothesis gives the desired sequent. The cases for $Lc$ or $L$ on some $\nabla^{n_j} \top$  are similar. In all other cases, just apply induction hypothesis on $\mathcal{D}_0$ (and possibly $\mathcal{D}_1$), and then apply the same last rule. Notice that $\nabla Cut$ is not used except in the case for $\nabla Cut$ itself, where it is applied with an instance of the same rank, so the resulting proof-tree will not be of a higher rank than that of $\mathcal{D}$.
\end{proof}

The following theorem shows that we can imitate any instance of the cut rule in a proof-tree of lower rank.

\begin{thm}\label{thm:gstl-cut-reduction}[cut Reduction]
  If $\gstl^+(S)$ proves $\Gamma \Rightarrow A$ and\\\ $\Sigma , \{\nabla^{n_i} A\}_{i \leq l} \Rightarrow \Delta$ with proof-trees of ranks less than $\rho(A)$, then it also proves $\{\nabla^{n_i} \Gamma\}_{i \leq l} , \Sigma \Rightarrow \Delta$ also with a proof tree of a rank less than $\rho(A)$ (for $S \subseteq \{L, R, Fa, Fu\}$).
  \end{thm}
  \begin{proof}
    We have two proof-trees
    \[
      \genfrac{}{}{0pt}{}{\mathcal{D}_0}{\Gamma \Rightarrow A}
      \hspace{3em}
      \genfrac{}{}{0pt}{}{\mathcal{D}_1}{\Sigma , \{\nabla^{n_i} A\}_{i \leq l} \Rightarrow \Delta}
    \]
    both of a lower rank than that of $A$, and we want to construct a proof-tree
    \[\genfrac{}{}{0pt}{}{\mathcal{D}}{\{\nabla^{n_i} \Gamma\}_{i \leq l} , \Sigma \Rightarrow \Delta} \]
    without increasing the cut rank.
  
    The construction takes place in different cases for the last rule that occurs in $\mathcal{D}_0$ and $\mathcal{D}_1$. Notice that the proof of the theorem is essentially the same for any choice of $S$, where $S$ is a subset of $\{L, R, Fa, Fu\}$, modulo the cases that are specific to the rules in $S$. Thus, for the sake of brevity, we will not repeat the cases which are common between all systems. Also notice that the resulting proof-tree in each case is constructed using the core system $\gstl^+$, plus the same rule of that case, so it will work for any extension containing that rule.
  
    In many cases, our construction would depend only on the last rule of one of the subtrees and it would work, no matter what the last rule in the other subtree is. Therefore, any of these cases will also cover all the cases for the other subtree. These cases constitute the first two parts of the proof. In the third part, we will address the cases where our construction depends on the last rule in \emph{both} subtrees, which are the cases that the cut-burden is altered on both sides. In these cases, the last rule in one of the subtrees determines a specific form for the formulas in the cut-burden, which will determine the last rule in the other subtree.
  
    In first two groups of the cases, we will need induction on the height of one of the subtrees. But in the third part, we will use induction simultaneously on both $\mathcal{D}_0$ and $\mathcal{D}_1$, which goes as follows. For any two proof-trees $\mathcal{D}_0'$ and $\mathcal{D}_1'$ such that $h(\mathcal{D}_0') + h(\mathcal{D}_1') < h(\mathcal{D}_0) + h(\mathcal{D}_1)$, where $\mathcal{D}_0'$ proves $\Gamma' \Rightarrow A'$ and $\mathcal{D}_1'$ proves $\Sigma', \{\nabla^{n'_i} A'\}_{i \leq l} \Rightarrow \Delta'$ for arbitrary $\Gamma'$, $\Sigma'$, $\Delta'$, $A'$ and $n_i'$ of length $l'+1$, for which we have $\rho(\mathcal{D}_0'),\rho(\mathcal{D}_1') < \rho(A')$, the induction hypothesis gives us a prooftree, denoted by $\text{IH}(\mathcal{D}_0', \mathcal{D}_1')$ where it matters, that proves $\{\nabla^{n_i'}\Gamma'\}_{i \leq l'}, \Sigma' \Rightarrow \Delta'$, and we will also have $\rho(\text{IH}(\mathcal{D}_0', \mathcal{D}_1')) < \rho(A')$.
  
    \textbf{Part I.} First, assume that $\mathcal{D}_0$ is an axiom. No matter what would be the last rule instance in $\mathcal{D}_1$, the case for $Id$ is trivial, $Ex$ won't happen and $Ta$ is handled by Lemma \ref{lem:gstl-top-redundant}.
    Now assume that $\mathcal{D}_0$ ends with an instance of the rules $Lw$, $Lc$, $Rw$, $L \wedge_1$, $L \wedge_2$, $L \vee$, $L \rightarrow$, $\nabla cut$, $N$, $Fu$, $L$ or $R$. In all these cases---again, independent of $\mathcal{D}_1$---it suffices to use induction on the assumption(s) of this rule and $\mathcal{D}_1$ to remove the cut-burden from both subtrees. Then, we can apply the same rule to get the desired sequent. Here we will only mention the cases for $L \wedge_1$, $L \vee$, $L \rightarrow$, $\nabla cut$, $Rw$, $N$ and $Fu$, the last three of which may be of special concern, since they also alter the cut-burden. The other cases are similar.
  
    $L \wedge_1$: If $\mathcal{D}_0$ ends with $L \wedge_1$, that is
    \begin{prooftree}
      \noLine
      \AXC{$\mathcal{D}_0'$}
      \UIC{$\Gamma, \nabla^r B \Rightarrow A$}
      
      \RightLabel{$L \wedge_1$}
      \UIC{$\Gamma, \nabla^r (B \wedge C) \Rightarrow A$}
   \end{prooftree}
   then by applying $L \wedge_1$ on what we get from induction
   \begin{prooftree}
    \noLine
    \AXC{$\mathcal{D}_0'$}
    \UIC{$\Gamma, \nabla^r B \Rightarrow A$}
    
    \noLine
    \AXC{$\mathcal{D}_1$}
    \UIC{$\Sigma , \{\nabla^{n_i} A\}_{i \leq l} \Rightarrow \Delta$}
    
    \RightLabel{IH}
    \BIC{$\{\nabla^{n_i} \Gamma, \nabla^{n_i+r} B\}_{i \leq l}, \Sigma \Rightarrow \Delta$}
  
    \RightLabel{$L \wedge_1$} \doubleLine
    \UIC{$\{\nabla^{n_i} \Gamma, \nabla^{n_i+r} (B \wedge C)\}_{i \leq l}, \Sigma \Rightarrow \Delta$}
   \end{prooftree}
  
   \noindent $L \vee$: If $\mathcal{D}_0$ ends with $L \vee$
     \begin{prooftree}
       \noLine
       \AXC{$\mathcal{D}_0'$}
       \UIC{$\Gamma, \nabla^r B \Rightarrow A$}
       
       \noLine
       \AXC{$\mathcal{D}_0''$}
       \UIC{$\Gamma, \nabla^r C \Rightarrow A$}
       
       \RightLabel{$L \vee$}
       \BIC{$\Gamma, \nabla^r (B \vee C) \Rightarrow A$}
    \end{prooftree}
    Applying $L \vee$ on the sequents that we get from induction
    \begin{prooftree}
      \noLine
      \AXC{$\mathcal{D}_0'$}
      \UIC{$\Gamma, \nabla^r B \Rightarrow A$}
      
      \noLine
      \AXC{$\mathcal{D}_1$}
      \UIC{$\Sigma , \{\nabla^{n_i} A\}_{i \leq l} \Rightarrow \Delta$}
      
      \RightLabel{IH}
      \BIC{$\{\nabla^{n_i} \Gamma, \nabla^{n_i+r} B\}_{i \leq l}, \Sigma \Rightarrow \Delta$}
      
  
      \noLine
      \AXC{$\mathcal{D}_0''$}
      \UIC{$\Gamma, \nabla^r C \Rightarrow A$}
      
      \noLine
      \AXC{$\mathcal{D}_1$}
      \UIC{$\Sigma , \{\nabla^{n_i} A\}_{i \leq l} \Rightarrow \Delta$}
      
      \RightLabel{IH}
      \BIC{$\{\nabla^{n_i} \Gamma, \nabla^{n_i+r} C\}_{i \leq l}, \Sigma \Rightarrow \Delta$}
  
      \RightLabel{$L \vee$}
      \BIC{$\{\nabla^{n_i} \Gamma, \nabla^{n_i+r} (B \vee C)\}_{i \leq l}, \Sigma \Rightarrow \Delta$}
     \end{prooftree}
  
   
  \noindent $L \rightarrow$: Suppose $\mathcal{D}_0$ ends with a $L \rightarrow$ as shown below.
   \begin{prooftree}
    \noLine
    \AXC{$\mathcal{D}_0'$}
    \UIC{$\Gamma \Rightarrow \nabla^r B$}
    \noLine
    \AXC{$\mathcal{D}_0''$}
    \UIC{$\Gamma , \nabla^r C \Rightarrow A$}
    \RightLabel{$L \rightarrow$}
    \BIC{$\Gamma , \nabla^{r+1} (B \rightarrow C) \Rightarrow A$}
   \end{prooftree}
   Let $IH(\mathcal{D}_0'', \mathcal{D}_1)$ be called $\mathcal{D}'$.
   \begin{prooftree}
    \noLine
    \AXC{$\mathcal{D}_0''$}
    \UIC{$\Gamma , \nabla^r C \Rightarrow A$}
    \noLine
    \AXC{$\mathcal{D}_1$}
    \UIC{$\Sigma , \{\nabla^{n_i} A\}_{i \leq l} \Rightarrow \Delta$}
    \RightLabel{IH} \LeftLabel{$\mathcal{D}':~~~~$}
    \BIC{$\{\nabla^{n_i} \Gamma , \nabla^{n_i+r} C\}_{i \leq l} , \Sigma \Rightarrow \Delta$}
   \end{prooftree}
   In order to apply $L \rightarrow$, we must prepare the context in $\mathcal{D}_0'$, for each of $\nabla^{n_i+r}C$'s. Beginning with $j = 0$, first apply $N$ on $\mathcal{D}_0'$ $n_0$ times to get $\nabla^{n_0}\Gamma \Rightarrow \nabla^{n_0+r} B$. Then we can just add the rest of the context by $Lw$.
   \begin{prooftree}
    \noLine
    \AXC{$\mathcal{D}_0'$}
    \UIC{$\Gamma \Rightarrow \nabla^r B$}
    \doubleLine \RightLabel{$N$}
    \UIC{$\nabla^{n_0} \Gamma \Rightarrow \nabla^{n_0+r} B$}
    \doubleLine \RightLabel{$Lw$}
    \UIC{$\{\nabla^{n_i} \Gamma\}_{i \leq l}, \{\nabla^{n_i+r}C\}_{i \leq l}^{i \neq 0} , \Sigma \Rightarrow \nabla^{n_0+r} B$}
   \end{prooftree}
   Let the outcome of applying $L \rightarrow$ on this sequent and $\mathcal{D}'$ be called $\mathcal{D}'_{n_0}$:
   \[\mathcal{D}'_{n_0}:~~~~\{\nabla^{n_i} \Gamma\}_{i \leq l}, \{\nabla^{n_i+r}C\}_{i \leq l}^{i \neq 0}, \nabla^{n_0+r+1} (B \rightarrow C) , \Sigma \Rightarrow \Delta\]
   Now for all $0 < j \leq l$, we construct $\mathcal{D}_{n_j}$ similarly.
   \begin{prooftree}
    \noLine
    \AXC{$\mathcal{D}_0'$}
    \UIC{$\Gamma \Rightarrow \nabla^r B$}
    \doubleLine \RightLabel{$N$}
    \UIC{$\nabla^{n_j} \Gamma \Rightarrow \nabla^{n_j+r} B$}
    \doubleLine \RightLabel{$Lw$}
    \UIC{$\{\nabla^{n_i} \Gamma\}_{i \leq l}, \{\nabla^{n_i+r}C\}_{j < i \leq l}, \{ \nabla^{n_i+r+1} (B \rightarrow C) \}_{i < j}, \Sigma \Rightarrow \nabla^{n_j+r} B$}
   \end{prooftree}
   Applying $L \rightarrow$ on this sequent and $\mathcal{D}_{n_{j-1}}$ we would get
   \[\mathcal{D}'_{n_j}:~~~~\{\nabla^{n_i} \Gamma\}_{i \leq l}, \{\nabla^{n_i+r}C\}_{j < i \leq l}, \{ \nabla^{n_i+r+1} (B \rightarrow C) \}_{i \leq j}, \Sigma \Rightarrow \Delta\]
   $\mathcal{D}_{n_l}$ is exactly what we want:
   \[\{\nabla^{n_i} \Gamma, \nabla^{n_i+r+1}(B \rightarrow C)\}_{i \leq l}, \Sigma \Rightarrow \Delta\]
  
  
   $\nabla cut$: Assume $\mathcal{D}_0$ ends with a $\nabla cut$ with cut-data $(A', \{n_i'\}_{i \leq l'})$. Recall that by assumption, $A'$ must have a lower rank than $A$.
   \begin{prooftree}
     \noLine
     \AXC{$\mathcal{D}_0'$}
     \UIC{$\Gamma \Rightarrow A'$}
     
     \noLine
     \AXC{$\mathcal{D}_0''$}
     \UIC{$\Pi, \{\nabla^{n_i'} A'\}_{i \leq l'} \Rightarrow A$}
     
     \RightLabel{$\nabla cut$}
     \BIC{$\{\nabla^{n_i'} \Gamma\}_{i \leq l'}, \Pi \Rightarrow A$}
   \end{prooftree}
   We must construct a proof-tree for $\{\nabla^{n_i + n_j'} \Gamma\}_{j \leq l'}^{i \leq l}, \{\nabla^{n_i} \Pi\}_{i \leq l} , \Sigma$ $\Rightarrow \Delta$. We can use the induction hypothesis first to remove $A$, and then use a low rank $\nabla cut$ to remove $A'$.
   \begin{prooftree}
     \noLine
     \AXC{$\mathcal{D}_0'$}
     \UIC{$\Gamma \Rightarrow A'$}
     
     \noLine
     \AXC{$\mathcal{D}_0''$}
     \UIC{$\Pi, \{\nabla^{n_i'} A'\}_{i \leq l'} \Rightarrow A$}
  
     \noLine
     \AXC{$\mathcal{D}_1$}
     \UIC{$\Sigma , \{\nabla^{n_i} A\}_{i \leq l} \Rightarrow \Delta$}
  
     \RightLabel{IH}
     \BIC{$\{\nabla^{n_i} \Pi\}_{i \leq l} , \{\nabla^{n_i + n_j'} A'\}_{j \leq l'}^{i \leq l}, \Sigma \Rightarrow \Delta$}
     
  
     \RightLabel{$\nabla cut$}
     \BIC{$\{\nabla^{n_i + n_j'} \Gamma\}_{j \leq l'}^{i \leq l}, \{\nabla^{n_i} \Pi\}_{i \leq l}, \Sigma$ $\Rightarrow \Delta$}
   \end{prooftree}
  
   $Rw$: In this case, $\mathcal{D}_0'$ proves $\Gamma \Rightarrow$, so we can simply construct the desired proof-tree using $N$, $Lw$ and $Rw$.
   \begin{prooftree}
     \noLine
     \AXC{$\mathcal{D}_0'$}
     \UIC{$\Gamma \Rightarrow$}
     \doubleLine \RightLabel{$N$}
     \UIC{$\nabla^{n_0} \Gamma \Rightarrow$}
     \doubleLine \RightLabel{$Lw$}
     \UIC{$\{\nabla^{n_i} \Gamma\}_{i \leq l} , \Sigma \Rightarrow$}
     \RightLabel{$Rw$}
     \UIC{$\{\nabla^{n_i} \Gamma\}_{i \leq l} , \Sigma \Rightarrow \Delta$}
   \end{prooftree}
  
   $N$: $\mathcal{D}_0$ proves $\nabla \Gamma \Rightarrow \nabla A$ and $\mathcal{D}_1$ proves $\Sigma, \{\nabla^{n_i} A\}_{i<l} \Rightarrow \Delta$. There are two cases: The cut-data could be $(A, \{n_i\}_{i \leq l})$, or if for all $i \leq l$ we have $0 < n_i$, then the cut-data could also be $(\nabla A, \{n_i-1\}_{i \leq l})$. Induction hypothesis for $\mathcal{D}_0$'s immediate sub-tree and $\mathcal{D}_1$ gives us $\{\nabla^{n_i}\Gamma\}_{i \leq l}, \Sigma \Rightarrow \Delta$, which handles the latter case, and an application of $N$ on this sequent would handle the former.
  
   {\color{red} $Fu$: It suffices to apply $Fu$ on the result of the induction.}\\
  
   \textbf{Part II.} The rest of the cases for $\mathcal{D}_0$ can't be solved independent of $\mathcal{D}_1$, so in the second part of the cases, we will consider the last rule of $\mathcal{D}_1$, again, where the solution could be constructed independent of $\mathcal{D}_0$. But this time we have less possibilities for the opposite subtree, since we've already solved most of them in the previous part of the proof. In fact the only possible rules as the last rule of $\mathcal{D}_0$ are now $R\star (\star \in \{\wedge, \vee_{1/2}, \rightarrow\})$ and $Fa$.
  
   Suppose $\mathcal{D}_1$ is an axiom. Again, the case for $Id$ is trivial, $Ta$ won't happen, and $Ex$ is also infeasible, since all possible cases for $\mathcal{D}_0$ alter the right side of the sequent, but none of them are able to introduce $\bot$ on the right side.
   In the remaining cases, if the cut-data is $(A, \{n_i\}_{i \leq l})$, in the cases where no member of the cut-burden is altered in the last rule of $\mathcal{D}_1$ modulo its number of $\nabla$'s, the construction is similar to the first part: Applying the same rule on the sequent that we get from the induction hypothesis. But if a member of the cut-burden is principal in the last rule of $\mathcal{D}_1$, which is to be handeld in the last part, we must also use the induction hypothesis for $\mathcal{D}_0$, both with a different cut-data. We now address the second part of the cases.
   
   For the sake of briefness, we will only explain the cases for $L \wedge_1$, $R \vee_1$, $R \rightarrow$ and $N$, the last two of which are of special concern, since we must use induction hypothesis with different cut-data in those cases. The rest would be handled similarly.
  
   $L \wedge$: Assume that $\mathcal{D}_1$ ends with $L \wedge_1$, but no member of the cut-burden is its principal formula.
   \begin{prooftree}
    \AXC{$\mathcal{D}_1'$} \noLine
    \UIC{$\Sigma, \{\nabla^{n_i} A\}_{i \leq l}, \nabla^r B \Rightarrow \Delta$}
    \RightLabel{$L \wedge_1$}
    \UIC{$\Sigma, \{\nabla^{n_i} A\}_{i \leq l}, \nabla^r (B \wedge C) \Rightarrow \Delta$}
   \end{prooftree}
   From induction hypothesis we have $\{\nabla^{n_i} \Gamma\}_{i \leq l}, \Sigma, \nabla^r B \Rightarrow \Delta$. By $L \wedge_1$ we have $\{\nabla^{n_i} \Gamma\}_{i \leq l}, \Sigma, \nabla^r (B \wedge C) \Rightarrow \Delta$.
  
   $R \vee_1$: Suppose that $\mathcal{D}_1$ ends with $R \vee_1$.
   \begin{prooftree}
    \AXC{$\mathcal{D}_1'$} \noLine
    \UIC{$\Sigma, \{\nabla^{n_i} A\}_{i \leq l} \Rightarrow B$}
    \RightLabel{$R \vee_1$}
    \UIC{$\Sigma, \{\nabla^{n_i} A\}_{i \leq l} \Rightarrow B \vee C$}
   \end{prooftree}
   Again, use the induction hypothesis to get $\{\nabla^{n_i} \Gamma\}_{i \leq l}, \Sigma \Rightarrow B$, then apply $R \vee_1$ to reach the desired sequent.
  
  $R \rightarrow$: In the case where $\mathcal{D}_1$ ends with an $R \rightarrow$, the cut-burden is altered in the premise.
  \begin{prooftree}
    \AXC{$\mathcal{D}_1'$} \noLine
    \UIC{$\nabla\Sigma, \{\nabla^{n_i+1} A\}_{i \leq l}, B \Rightarrow C$}
    \RightLabel{$R \rightarrow$}
    \UIC{$\Sigma, \{\nabla^{n_i} A\}_{i \leq l} \Rightarrow B \rightarrow C$}
   \end{prooftree}
   The induction hypothesis has a different cut-data, nevertheless, it still commutes with $R \rightarrow$.
  From induction hypothesis, we have $\{\nabla^{n_i+1} \Gamma\}_{i \leq l}, \nabla \Sigma, B \Rightarrow C$. We can simply apply $R \rightarrow$ to get $\{\nabla^{n_i} \Gamma\}_{i \leq l}, \Sigma \Rightarrow B \rightarrow C$.
  
  $N$: Suppose $\mathcal{D}_1$ ends with $N$.
  \begin{prooftree}
    \AXC{$\mathcal{D}_1'$} \noLine
    \UIC{$\Sigma, \{\nabla^{n_i} A\}_{i \leq l} \Rightarrow \Delta$}
    \RightLabel{$N$}
    \UIC{$\nabla \Sigma, \{\nabla^{n_i+1} A\}_{i \leq l} \Rightarrow \nabla \Delta$}
  \end{prooftree}
  If we assume that the cut-data is $(A, \{n_i+1\}_{i \leq l})$, from the induction hypothesis we have $\{\nabla^{n_i} \Gamma\}_{i \leq l}, \Sigma \Rightarrow \Delta$. By $N$ we have $\{\nabla^{n_i+1} \Gamma\}_{i \leq l},$ $\nabla \Sigma \Rightarrow \nabla \Delta$, which is the desired sequent.
  {\color{red} Notice the cut-data could not be $(\nabla A, \{n_i\}_{i \leq l})$, since none of possible rules for $\mathcal{D}_0$ would be able to introduce $\nabla$ on the right-hand side.}
  
   \textbf{Part III.} Now in the last part of the proof, we will show how the construction takes place in the cases where a member of the cut-burden is principal in the last rule of $\mathcal{D}_1$, which can be either of $L\star (\star \in \{\wedge, \vee_{1/2}, \rightarrow\})$.
   Any of these rules also determine the rule at the end of the other proof-tree, because $A$ would also be principal in the last rule of $\mathcal{D}_0$. Recall that the only possible rules as the last rule of $\mathcal{D}_0$ are now $R\star (\star \in \{\wedge, \vee_{1/2}, \rightarrow\})$ and $Fa$, which all have a principal formula on the right side of the sequent.

   $R \wedge$ and $L \wedge$: Suppose $\mathcal{D}_0$ ends with $R \wedge$ and $\mathcal{D}_1$ ends with either of $L \wedge_c ~ (c \in \{1,2\})$.
   \begin{prooftree}
     \noLine
     \AXC{$\mathcal{D}_0'$}
     \UIC{$\Gamma \Rightarrow A_1$}
     \noLine
     \AXC{$\mathcal{D}_0''$}
     \UIC{$\Gamma \Rightarrow A_2$}
     \RightLabel{$R \wedge$}
     \BIC{$\Gamma \Rightarrow A_1 \wedge A_2$}
     
     \noLine
     \AXC{$\mathcal{D}_1'$}
     \UIC{$\Sigma , \{\nabla^{n_i} (A_1 \wedge A_2)\}_{i \leq l}^{i \neq j}, \nabla^{n_j} A_c \Rightarrow \Delta$}
     \RightLabel{$L \wedge_1$}
     \UIC{$\Sigma , \{\nabla^{n_i} (A_1 \wedge A_2)\}_{i \leq l} \Rightarrow \Delta$}
     
     \noLine
     \BIC{}
   \end{prooftree}
   $IH(\mathcal{D}_0, \mathcal{D}_1')$ proves $\{\nabla^{n_i} \Gamma\}_{i \leq l}^{i \neq j}, \Sigma , \nabla^{n_j} A_c \Rightarrow \Delta$. Remove $\nabla^{n_j} A_c$ with a low rank $\nabla cut$ on this sequent and either of $\mathcal{D}_0'$ (if $c = 1$) or $\mathcal{D}_0''$ (if $c = 2$) to get $\{\nabla^{n_i} \Gamma\}_{i \leq l}, \Sigma \Rightarrow \Delta$.
  
   $R \vee$ and $L \vee$: Suppose that $\mathcal{D}_0$ ends with either of $R \vee_c ~ (c \in \{1,2\})$ and $\mathcal{D}_1$ ends with $L \vee$.
   \begin{prooftree}
     \noLine
     \AXC{$\mathcal{D}_0'$}
     \UIC{$\Gamma \Rightarrow A_c$}
     \RightLabel{$R \vee_c$}
     \UIC{$\Gamma \Rightarrow A_1 \vee A_2$}
   \end{prooftree}
   \begin{prooftree}
    \noLine
    \AXC{$\mathcal{D}_1'$}
    \UIC{$\Sigma , \{\nabla^{n_i} (A_1 \vee A_2)\}_{i \leq l}^{i \neq j} , \nabla^{n_j} A_1 \Rightarrow \Delta$}
    \noLine
    \AXC{$\mathcal{D}_1''$}
    \UIC{$\Sigma , \{\nabla^{n_i} (A_1 \vee A_2)\}_{i \leq l}^{i \neq j} , \nabla^{n_j} A_2 \Rightarrow \Delta$}
    \RightLabel{$L \vee$}
    \BIC{$\Sigma ,  \{\nabla^{n_i} (A_1 \vee A_2)\}_{i \leq l} \Rightarrow \Delta$}
   \end{prooftree}
   Using induction hypothesis, first, remove $\{\nabla^{n_i} (A_1 \vee A_2)\}_{i \leq l}^{i \neq j}$ from the subtree of $\mathcal{D}_1$ which has $\nabla^{n_j} A_c$ on its left side (by $IH(\mathcal{D}_0, \mathcal{D}_1')$ for $c = 1$, $IH(\mathcal{D}_0, \mathcal{D}_1'')$ for $c = 2$), to get $\{\nabla^{n_i} \Gamma\}_{i \leq l}^{i \neq j}, \Sigma , \nabla^{n_j} A_c \Rightarrow \Delta$. Then, remove $\nabla^{n_j} A_c$ by a low rank $\nabla cut$ on this sequent and $\mathcal{D}_0'$ to get $\{\nabla^{n_i} \Gamma\}_{i \leq l}, \Sigma \Rightarrow \Delta$.
  
   $R \rightarrow$ and $L \rightarrow$: Suppose that $\mathcal{D}_0$ and $\mathcal{D}_1$ end with $R \rightarrow$ and $L \rightarrow$ respectively. So there must be $j \leq l$ such that $n_j > 0$.
   \begin{prooftree}
     \noLine
     \AXC{$\mathcal{ D}_0'$}
     \UIC{$\nabla \Gamma, A_1 \Rightarrow A_2$}
     \RightLabel{$R \rightarrow$}
     \UIC{$\Gamma \Rightarrow A_1 \rightarrow A_2$}        
     \end{prooftree}
     \begin{prooftree}
     \noLine
     \AXC{$\mathcal{D}_1'$}
     \UIC{$\Sigma, \{\nabla^{n_i} (A_1 \rightarrow A_2)\}_{i \leq l}^{i \neq j} \Rightarrow \nabla^{n_j-1} A_1$}
     \noLine
     \AXC{$\mathcal{D}_1''$}
     \UIC{$\Sigma, \{\nabla^{n_i} (A_1 \rightarrow A_2)\}_{i \leq l}^{i \neq j}, \nabla^{n_j-1} A_2 \Rightarrow \Delta$}
     \RightLabel{$L \rightarrow$}
     \BIC{$\Sigma,  \{\nabla^{n_i} (A_1 \rightarrow A_2)\}_{i \leq l} \Rightarrow \Delta$}
   \end{prooftree}
   
   $IH(\mathcal{D}_0, \mathcal{D}_1'')$ proves $\{\nabla^{n_i-1} \Gamma\}_{i \leq l}^{i \neq j}, \Sigma, \nabla^{n_j} A_2 \Rightarrow \Delta$. Applying a low rank $\nabla cut$ on $\mathcal{D}_0'$ and $IH(\mathcal{D}_0, \mathcal{D}_1'')$ removes $\nabla^{n_j-1} A_2$ and introduces $\nabla^{n_j-1} \Gamma$ and $\nabla^{n_j-1} A_1$ in the left. On the other hand $IH(\mathcal{D}_0, \mathcal{D}_1')$ proves $\{\nabla^{n_i} \Gamma\}_{i \leq l}^{i \neq j}, \Sigma, \Rightarrow \nabla^{n_j-1} A_1$, which we can use to also remove $\nabla^{n_j-1} A_1$ with another low rank cut. Then it suffices to remove the extra $\{\nabla^{n_i} \Gamma\}_{i \leq l}^{i \neq j}$ and $\Sigma$ with $Lc$.
  
   $Fa$ and $L \rightarrow$: Suppose $\mathcal{D}_0$ ends with $Fa$ and $\mathcal{D}_1$ ends with $L \rightarrow$.
   \begin{prooftree}
     \AXC{$\mathcal{D}_0'$}
     \noLine
     \UIC{$\Gamma, A_1 \Rightarrow A_2$}
     \RightLabel{$Fa$}
     \UIC{$\Gamma \Rightarrow \nabla (A_1 \rightarrow A_2)$}
   \end{prooftree}
   The cut-data must be of the form $(\nabla (A_1 \rightarrow A_2), \{n_i\}_{i \leq l})$, so the only option for $\mathcal{D}_1$ is $L \rightarrow$, with a principal formula from the cut-burden, like $\nabla^{n_j+1} (A_1 \rightarrow A_2)$ for some $j \leq l$.
   \begin{prooftree}
     \AXC{$\mathcal{D}_1'$}
     \noLine
     \UIC{$\Sigma, \{\nabla^{n_i+1} (A_1 \rightarrow A_2) \}_{i \leq l}^{i \neq j}, \Rightarrow \nabla^{n_j} A_1$}
     \AXC{$\mathcal{D}_1''$}
     \noLine
     \UIC{$\Sigma, \{\nabla^{n_i+1} (A_1 \rightarrow A_2) \}_{i \leq l}^{i \neq j}, \nabla^{n_j} A_2 \Rightarrow \Delta$}
     \RightLabel{$L \rightarrow$}
     \BIC{$\Sigma, \{\nabla^{n_i+1} (A_1 \rightarrow A_2)\}_{i \leq l} \Rightarrow \Delta$}
   \end{prooftree}
   First, apply a low rank $\nabla cut$ (with $(A_2, \{n_j\})$ as the cut-data) on $\mathcal{D}_0'$ and $IH(\mathcal{D}_0, \mathcal{D}_1'')$. Let the resulting sequent be called $\mathcal{D}'$.
   \begin{prooftree}
     \AXC{$\mathcal{D}_0'$}
     \noLine
     \UIC{$\Gamma, A_1 \Rightarrow A_2$}
     \AXC{$\mathcal{D}_0$}
     \noLine
     \UIC{$\Gamma \Rightarrow \nabla (A_1 \rightarrow A_2)$}
     \AXC{$\mathcal{D}_1''$}
     \noLine
     \UIC{$\Sigma, \{\nabla^{n_i+1} (A_1 \rightarrow A_2) \}_{i \leq l}^{i \neq j}, \nabla^{n_j} A_2 \Rightarrow \Delta$}
     \RightLabel{IH}
     \BIC{$\{\nabla^{n_i} \Gamma\}_{i \leq l}^{i \neq j}, \Sigma, \nabla^{n_j} A_2 \Rightarrow \Delta$}
     \RightLabel{$\nabla cut$} \LeftLabel{$\mathcal{D}':~~~~~$}
     \BIC{$\nabla^{n_j} A_1, \{\nabla^{n_i} \Gamma\}_{i \leq l}, \Sigma \Rightarrow \Delta$}
   \end{prooftree}
   Then cut $IH(\mathcal{D}_0, \mathcal{D}_1')$ (this time with $(\nabla^{n_j} A_1, \{0\})$ as the cut-data) into the resulting sequent.
   \begin{prooftree}
    \AXC{$\mathcal{D}_0$}
    \noLine
    \UIC{$\Gamma \Rightarrow \nabla (A_1 \rightarrow A_2)$}
     \AXC{$\mathcal{D}_1'$}
     \noLine
     \UIC{$\Sigma, \{\nabla^{n_i+1} (A_1 \rightarrow A_2) \}_{i \leq l}^{i \neq j}, \Rightarrow \nabla^{n_j} A_1$}
     \RightLabel{IH}
     \BIC{$\{\nabla^{n_i} \Gamma\}_{i \leq l}^{i \neq j}, \Sigma, \Rightarrow \nabla^{n_j} A_1$}
  
     \AXC{$\mathcal{D}'$}
  
     \RightLabel{$\nabla cut$}
     \BIC{$\{\nabla^{n_i} \Gamma\}_{i \leq l}^{i \neq j}, \{\nabla^{n_i} \Gamma\}_{i \leq l}, \Sigma, \Sigma \Rightarrow \Delta$}
     \doubleLine \RightLabel{$Lc$}
     \UIC{$\{\nabla^{n_i} \Gamma\}_{i \leq l}, \Sigma \Rightarrow \Delta$}
   \end{prooftree}
   And that's the sequent that we wanted.

   \vspace{5mm}
  
   Now we have a construction for any two possible pair of rules, in $\gstl^+$ and all its extensions. This concludes the proof of the theorem in all cases.
  
  \end{proof}

The next theorem shows that the $cut$ rule is redundant.

\begin{thm}[Cut Elimination]\label{thm:gstl-cut-elim}
	For any $\Gamma$ and $\Delta$, if $\Gamma \Rightarrow \Delta$ is provable by $\gstl(S)$, then it is also provable by $\gstl(S)-\{cut\}$ (for $S \subseteq \{ L, R, Fa, Fu \}$).
\end{thm}
\begin{proof}
		First, we will show that for any non-zero-rank proof of $\Gamma \Rightarrow \Delta$ like $\mathcal{D}$ in $\gstl(S)$, there is another proof of the same sequent with a lower rank. Suppose $\mathcal{D}$ has subtree(s) called $\mathcal{D}_0$ (and possibly $\mathcal{D}_1$, if the last rule has two assumptions). Using induction on $h(\mathcal{D})$, the induction hypothesis for $\mathcal{D}_i ~(i \in \{0,1\})$ gives us a proof-tree with the same conclusion, which we call $IH(\mathcal{D}_i)$, but with a lower rank, i.e., $\rho(IH(\mathcal{D}_i)) < \rho(\mathcal{D}_i)$. We now consider two cases for the last rule of $\mathcal{D}$.

	\begin{enumerate}[label=\Roman*]
		\item If the last rule of $\mathcal{D}$ is of a lower rank than $\mathcal{D}$ itself, i.e., the $cut$ instance in $\mathcal{D}$ with the maximum rank is not the last rule, then we can apply the same last rule on $IH(\mathcal{D}_0)$ (and possibly $\mathcal{D}_1$), to construct a proof of $\Gamma \Rightarrow \Delta$ with a lower rank.
		
		\item If the last rule of $\mathcal{D}$ is an instance of $cut$ rule with the same rank as $\mathcal{D}$ itself, then it is indeed the $cut$ instance with the maximum rank. So we can apply Theorem \ref{thm:gstl-cut-reduction} to $IH(\mathcal{D}_0)$ and $IH(\mathcal{D}_1)$ to prove the same sequent as it would be proved by $cut$, but with a lower rank. (Recall that $\nabla cut$ is just a generalization of $cut$, so the theorem applies.)
	\end{enumerate}
	So for any proof of $\Gamma \Rightarrow \Delta$ in $\gstl(S)$, we also have a proof of rank $0$, which is cut-free.
\end{proof}

\subsection{Logic of spacetime with intuitionistic implication}
A conservative extension of $\stl$ is introduced in the next definition. This extension, which is called $\istl$ here, enrisches $\stl$ with intuitionistic implication. $\istl$ is conservative in the sense that adding intuitionistic implication will give no more power to the system in proving propositions in the original language of $\stl$.

\begin{dfn}\label{istl}
	Let $\mathcal{L}_\supset=\langle \wedge, \vee, \top, \bot, \nabla, \rightarrow, \supset, P \rangle$ where $P$ is the set of all atomic propositions. Define $\istl$ as the logic of the sequent-style system defined by the same rules as Definition \ref{stl}, plus the following rules.
\end{dfn}
\begin{multicols}{2}
  \begin{prooftree}
    \AXC{$\Gamma, B \Rightarrow \Delta$}
    \AXC{$\Gamma \Rightarrow A$}
    \RightLabel{$L \supset$}
    \BIC{$\Gamma, A \supset B \Rightarrow \Delta$}
  \end{prooftree}
  \columnbreak
  \begin{prooftree}
    \AXC{$\Gamma, A \Rightarrow B$}
    \RightLabel{$R \supset$}
    \UIC{$\Gamma \Rightarrow A \supset B$}
  \end{prooftree}
\end{multicols}

The following theorem shows that $\istl$ is just a conservative extension of $\stl$.

\begin{thm}
  For any $\Gamma$ and $\varphi$ in the language $\mathcal{L}$ we have $\stl \vdash \Gamma \Rightarrow \varphi$ if and only if $\istl \vdash \Gamma \Rightarrow \varphi$.
\end{thm}
\begin{proof}
  $(\Rightarrow)$ If $\stl \vdash \Gamma \Rightarrow \varphi$, then obviously $\istl \vdash \Gamma \Rightarrow \varphi$, since $\istl$ is just an extension of $\stl$.

  $(\Leftarrow)$ Assume $\istl \vdash \Gamma \Rightarrow \varphi$ by a proof-tree $\mathcal{D}$. Now, consider different cases for the last rule in $\mathcal{D}$. We will construct a proof-tree for $\varphi$ in $\stl$ in different cases for the last rule in $\mathcal{D}$, using induction on the height of $\mathcal{D}$. All cases are trivial except the cases for $L \supset$ and $R \supset$. But since $\Gamma$ and $\varphi$ are $\mathcal{L}$-formulas, none of them may contain $\supset$, so both these two cases are not feasible.
\end{proof}

Now, we will introduce a counterpart for $\gstl$ with intuitionistic implication, which we call $\igstl$. In the following, we will also show that theorems \ref{thm:stl-eq-gstl}, \ref{thm:gstl-cut-reduction}, and \ref{thm:gstl-cut-elim} can be easily extended to also contain the cases for rules for intuitionistic implication, yielding the same results for $\istl$ and $\igstl$.

\begin{dfn}\label{dfn:igstl}
	Define $\igstl$ over the language $\mathcal{L}_\supset$, to be the logic of sequent-style system as the logic of the sequent-style system defined by the same rules as $\istl$, as in Definition \ref{dfn:istl}, except the rule $L \supset$, which is replaced with the following rule.
\end{dfn}
\begin{prooftree}
	\AXC{$\Gamma \Rightarrow \nabla^n A$}
	\AXC{$\Gamma, \nabla^n B \Rightarrow \Delta$}
	\RightLabel{$L \supset$}
	\BIC{$\Gamma, \nabla^n (A \supset B) \Rightarrow \Delta$}
\end{prooftree}

To show the equivalence between $\istl$ and $\igstl$, we need the following lemma.

\begin{lem}\label{lem:l-nabla-dist-imp} For any $n \ge 0$, $\mathbf{iSTL}(S) \vdash \nabla^n (A \supset B) \Rightarrow \nabla^n A \supset \nabla^n B$.
\end{lem}
\begin{proof}\quad
	\begin{prooftree}
		\AXC{}
		\RightLabel{$Id$}
		\UIC{$A \Rightarrow A$}
	
		\AXC{}
		\RightLabel{$Id$}
		\UIC{$B \Rightarrow B$}
		\RightLabel{$Lw$}
		\UIC{$A , B \Rightarrow B$}
	
		\RightLabel{$L \supset$}
		\BIC{$A \supset B , A \Rightarrow B$}
		\RightLabel{$N^{(n)}$} \doubleLine
		\UIC{$\nabla^n (A \supset B) , \nabla^n A \Rightarrow \nabla^n B$}
		\RightLabel{$R \supset$}
		\UIC{$\nabla^n (A \supset B) \Rightarrow \nabla^n A \supset \nabla^n B$}
	\end{prooftree}
\end{proof}

\begin{thm}\label{thm:istl-eq-igstl}
	Let $S \subseteq \{L, R, Fa, Fu\}$. For any sequent $\Gamma \Rightarrow \Delta$ in the language $\mathcal{L}_\supset$, $\istl(S) \vdash \Gamma \Rightarrow \Delta$ iff $\igstl(S) \vdash \Gamma \Rightarrow \Delta$.
\end{thm}
\begin{proof}
	The proof is essentially the same as Theorem \ref{thm:stl-eq-gstl}, with additional cases for intuitionistic implication rules. Again, the left-to-right direction is obvious from the fact that all rules of $\istl(S)$ are just instances of $\igstl(S)$'s rules. For the other direction, we will apply induction on the height of the proof-tree for $\Gamma \Rightarrow \Delta$ in $\igstl(S)$, which we call $\mathcal{D}$. We can construct a proof-tree for $\Gamma \Rightarrow \Delta$ in $\istl(S)$ in different cases for the last rule in $\mathcal{D}$, just as we did for $\stl$ in Theorem \ref{thm:stl-eq-gstl}. Recall that we can use lemmas \ref{lem:l-nabla-dist-and} to \ref{lem:l-nabla-n-bot} for $\istl$ as well.
	It remains to handle the cases where the last rule in $\mathcal{D}$ is $L \supset$ and $R \supset$. The case for $R \supset$ is obvoius: We have $R \supset$ also in $\istl$, so we can apply it again on the proof-tree from induction hypothesis. In the case for $L \supset$, we have two proof-trees for $\Gamma \Rightarrow \nabla^n A$ and $\Gamma, \nabla^n B \Rightarrow \Delta$. Applying $L \supset$ in $\istl$ proves $\Gamma, \nabla^n A \supset \nabla^n B \Rightarrow \Delta$. Using Lemma \ref{lem:l-nabla-dist-imp} and $cut$, we would have the desired sequent.
\end{proof}

We can also show that $cut$ is admissible in $\igstl$. To this end, we will first introduce a slightly altered version of $\igstl$ with the generalized $cut$ rule, just as we did for $\gstl$.

\begin{dfn}[$\igstl^+$]\label{def:igstlp}
	Let $S \subseteq \{L, R, Fa, Fu\}$. Define $\igstl^+(S)$ to be the same systems defined as $\igstl(S)$, except that $cut$ is replaced by the generalization $\nabla cut$, as was defined in Definition \ref{def:gstlp}. \emph{Cut-data} and \emph{cut-burden} are also defined similarly. We will also assign a natural number, called \emph{Rank}, to any formula, rule instance and proof-tree in $\igstl^+(S)$, just as it was defined in Definition \ref{dfn:rank}.
\end{dfn}

\begin{thm}\label{cor:nc-riddance-i} Any sequent provable in $\igstl(S)$ is also provable in $\igstl^+(S)$ (for $S \subseteq \{L, R, Fa, Fu\}$).
\end{thm}
\begin{proof}
	Replace all instances of $cut$ with a cut-formula $A$, with a similar instance of $\nabla cut$ with cut-data $(A, 0, \{0\})$.
\end{proof}

\begin{lem}\label{lem:igstl-top-redundant} If $\igstl^+(S)$ proves $\Gamma , \{\nabla^{n_i} \top\}_{i \leq l} \Rightarrow \Delta$, then it also proves $\Gamma \Rightarrow \Delta$ with a proof-tree of at most the same rank (for $S \subseteq \{L, R, Fa, Fu\}$).
\end{lem}
\begin{proof}
  The proof is similar to the proof of Lemma \ref{lem:gstl-top-redundant}. Assuming a proof-tree $\mathcal{D}$ for $\Gamma , \{\nabla^{n_i} \top\}_{i \leq l} \Rightarrow \Delta$, we will construct a proof-tree for $\Gamma \Rightarrow \Delta$ in cases for the last rule in $\mathcal{D}$, without increasing the rank. In all cases except $L \supset$ and $R \supset$, the construction takes place using induction on the height of $\mathcal{D}$, just as it was in Lemma \ref{lem:gstl-top-redundant}. The cases for $L \supset$ and $R \supset$ are similar: Just apply the same rule again on the sequent from induction hypothesis. Obviously, this will not increase the rank of the resulting proof-tree.
\end{proof}

Next theorem is just a generalization of Theorem \ref{thm:gstl-cut-reduction} for $\igstl$.

\begin{thm}\label{thm:igstl-cut-reduction}[cut Reduction]
  If $\igstl^+(S)$ proves $\Gamma \Rightarrow \nabla^m A$ and\\\ $\Sigma , \{\nabla^{n_i} A\}_{i \leq l} \Rightarrow \Delta$ with proof-trees of ranks less than $\rho(A)$, then it also proves $\{\nabla^{n_i} \Gamma\}_{i \leq l} , \nabla^m\Sigma \Rightarrow \nabla^m\Delta$ with a proof tree of a rank less than $\rho(A)$ (for $S \subseteq \{L, R, Fa, Fu\}$).
\end{thm}


Now we are ready to eliminate the $cut$ rule from $\igstl$, just as we did for $\gstl$.

\begin{thm}[Cut Elimination]\label{thm:igstl-cut-elim}
  For any $\Gamma$ and $\Delta$, if $\igstl(S) \vdash \Gamma \Rightarrow \Delta$ then $\igstl-\{cut\} \vdash \Gamma \Rightarrow \Delta$ (for $S \subseteq \{ L, R, Fa, Fu \}$).
\end{thm}
\begin{proof}
  Our strategy is to reduce the rank of any sequent down to zero, just as we did in the proof of Theorem \ref{thm:gstl-cut-elim}.
  Suppose $\Gamma \Rightarrow \Delta$ has an $\igstl$ prooftree called $\mathcal{D}$. We claim that if $\rho(\mathcal{D}) \neq 0$, then there must exist some other $\igstl$ prooftree for $\Gamma \Rightarrow \Delta$ called $\mathcal{D}'$, such that $\rho(\mathcal{D}') < \rho(\mathcal{D})$. By induction on the height of $\mathcal{D}$, we can suppose for any prooftree shorter than $\mathcal{D}$, there exists a prooftree with the same conclusion, but a lower rank. Now, consider two cases: First, if the last rule in $\mathcal{D}$ is the $cut$ instance with the highest rank in $\mathcal{D}$, apply Theorem \ref{thm:igstl-cut-reduction} to the low rank prooftree that we get from induction hypothesis for two subtrees of $\mathcal{D}$. In the second case, if the last rule in $\mathcal{D}$ is not the maximum rank $cut$ instance, then just apply this rule on the low rank prooftree that we get from induction hypothese for possible subtrees of $\mathcal{D}$.
\end{proof}

\subsection{Applications}
In this section, we will see some direct applications of our cut-elimination results. First, we will show that a generalization of Visser's rule is admissible in our system. To this aim, we would need the following lemmas.

Throughout this section, by $\Gamma \vdash A$ we mean there is a proof in $\stl(S)$ of $A$ with possible assumptions from $\Gamma$.

\begin{lem}\label{lem:modus-ponens}
  $\vdash A , \nabla (A \rightarrow B) \Rightarrow B$.
\end{lem}
\begin{proof}\quad
  \begin{prooftree}
    \AXC{}
    \RightLabel{$Id$}
    \UIC{$A \Rightarrow A$}
  
    \AXC{}
    \RightLabel{$Id$}
    \UIC{$B \Rightarrow B$}
    \RightLabel{$Lw$}
    \UIC{$A , B \Rightarrow B$}
  
    \RightLabel{$L \rightarrow$}
    \BIC{$A , \nabla (A \rightarrow B) \Rightarrow B$}
  \end{prooftree}  
\end{proof}

\begin{lem}\label{lem:impl-elim}
  $\Rightarrow A \rightarrow B \vdash A \Rightarrow B$.
\end{lem}
\begin{proof}\quad
  \textit{Proof}:
  \begin{prooftree}
    \AXC{$\Rightarrow A \rightarrow B$}
    \RightLabel{$N$}
    \UIC{$\Rightarrow \nabla (A \rightarrow B)$}

    \AXC{Lemma \ref{lem:modus-ponens}} \noLine
    \UIC{$A, \nabla (A \rightarrow B) \Rightarrow B$}
    
    \RightLabel{$cut$}
    \BIC{$A \Rightarrow B$}
  \end{prooftree}
\end{proof}

\begin{lem}\label{lem:conj-context}
  $\vdash \{ A_i \}_{i=1}^n \Rightarrow \bigwedge_{i=1}^n A_i$, for $n \geq 0$.
\end{lem}

\begin{proof}
  For $n = 1$ we have $A_1 \Rightarrow A_1$ by $Id$. By induction on $n$, let IH be the proof tree for $\{ A_n \}_{i=1}^{n-1} \Rightarrow \bigwedge_{i=1}^{n-1}$, which we have from the induction hypothesis. For $n > 1$
  \begin{prooftree}
    \AXC{IH}
    \noLine
    \UIC{$\{ A_i \}_{i=1}^{n-1} \Rightarrow \bigwedge_{i=1}^{n-1} A_i$}
    \RightLabel{$Lw$}
    \UIC{$\{ A_i \}_{i=1}^n \Rightarrow \bigwedge_{i=1}^{n-1} A_i$}

    \AXC{}
    \RightLabel{$Id$}
    \UIC{$A_n \Rightarrow A_n$}
    \doubleLine \RightLabel{$Lw$}
    \UIC{$\{ A_i \}_{i=1}^n \Rightarrow A_n$}

    \RightLabel{$R\land$}
    \BIC{$\{ A_i \}_{i=1}^n \Rightarrow \bigwedge_{i=1}^n A_i$}
  \end{prooftree}
\end{proof}


\begin{rem}
	In the following, $\dotdiv$ is the truncated subtraction defined as $a \dotdiv b = max(a-b, 0)$.
\end{rem}

\begin{thm}[Generalized Visser rules]\label{thm:visser}
  Let $\{ l_i \}_{i=1}^n$ be a sequence of natural numbers of length $n$. If $\vdash \Rightarrow \bigwedge_{i=1}^n (\nabla^{l_i} (A_i \rightarrow B_i)) \rightarrow C \lor D$ then either $\vdash \Rightarrow \bigwedge_{i=1}^n (\nabla^{l_i} (A_i \rightarrow B_i)) \rightarrow \nabla^{l_j \dotdiv 1} A_j$ for some $j \in \{ 1 , \dots , n \}$, or $\vdash \Rightarrow \bigwedge_{i=1}^n (\nabla^{l_i} (A_i \rightarrow B_i)) \rightarrow C$, or $\vdash \Rightarrow \bigwedge_{i=1}^n (\nabla^{l_i} (A_i \rightarrow B_i)) \rightarrow D$.
\end{thm}

\textit{Proof}:
We have $\bigwedge_{i=1}^n (\nabla^{l_i} (A_i \rightarrow B_i)) \Rightarrow C \lor D$ by \ref{lem:impl-elim}. By $cut$ and \ref{lem:conj-context} we have $\{ \nabla^{l_i} (A_i \rightarrow B_i) \}_{i=1}^n \Rightarrow$ $C \lor D$, which also has a proof like $\mathcal{D}$ in $GSTL^-$ by \ref{thm:stl-eq-gstl} and \ref{thm:gstl-cut-elim}. The last rule in $\mathcal{D}$ can be:
\begin{itemize}[label=-]
	\item $R\lor_1$, applied on $\{ \nabla^{l_i} (A_i \rightarrow B_i) \}_{i=1}^n \Rightarrow C$. With enough applications of $L\land_1$, $L\land_2$, $Lc$ and a $R\rightarrow$ we will have $\Rightarrow \bigwedge_{i=1}^n (\nabla^{l_i} (A_i \rightarrow B_i)) \rightarrow C$.
	
	\item $R\lor_2$, applied on $\{ \nabla^{l_i} (A_i \rightarrow B_i) \}_{i=1}^n \Rightarrow D$. Same as the previous case, we can derive $\Rightarrow \bigwedge_{i=1}^n (\nabla^{l_i} (A_i \rightarrow B_i)) $ $\rightarrow D$.
	
	\item $Rw$, applied on $\{ \nabla^{l_i} (A_i \rightarrow B_i) \}_{i=1}^n \Rightarrow$. A different $Rw$ gives $\{ \nabla^{l_i} (A_i \rightarrow B_i) \}_{i=1}^n \Rightarrow C$. Again, we can get $\Rightarrow \bigwedge_{i=1}^n (\nabla^{l_i} (A_i \rightarrow B_i)) \rightarrow C$.
	
	\item $Lw$, applied on $\{ \nabla^{l_i} (A_i \rightarrow B_i) \}_{i=1,i \neq k}^n \Rightarrow C \lor D$ for some $k \in \{ 1 , \dots , n \}$. Let $\mathcal{D}'$ be the immediate subtree of $\mathcal{D}$. By induction on $h(\mathcal{D})$, we can apply induction hypothesis on $\mathcal{D}'$ to get either $\{ \nabla^{l_i} (A_i \rightarrow B_i) \}_{i=1,i \neq k}^n \Rightarrow C$, $\{ \nabla^{l_i} (A_i \rightarrow B_i) \}_{i=1,i \neq k}^n \Rightarrow D$ or $\{ \nabla^{l_i} (A_i \rightarrow B_i) \}_{i=1,i \neq k}^n \Rightarrow \nabla^{l_j \dotdiv 1} A_j$ for some $j \in \{ 1 , \dots , n \} - \{k\}$. Then, after introducing $\nabla^{l_k} A_k \rightarrow B_k$ on the left with $Lw$ again, we can follow the same manner as previous cases to reach any of the desired sequents.
	
	\item $Lc$, applied on $\{ \nabla^{l_i} (A_i \rightarrow B_i) \}_{i=1}^n , \nabla^{l_k} (A_k \rightarrow B_k) \Rightarrow C \lor D$ for some $k \in \{ 1 , \dots , n \}$. This is just the same as the $Lw$ case, except this time we must remove the extra $\nabla^{l_k} (A_k \rightarrow B_k)$ with another $Lc$.
	
	\item $L\rightarrow$, applied on $\{ \nabla^{l_i} (A_i \rightarrow B_i) \}_{i=1, i \neq j}^n \Rightarrow \nabla^{l_j - 1} A_j$ and $\{ \nabla^{l_i} (A_i \rightarrow B_i) \}_{i=1, i \neq j}^n , \nabla^{l_j - 1} B_j \Rightarrow C \lor D$ for some $j \in \{ 1 , \dots , n \}$. So this implies $n>0$ and $l_j>0$ for at least one such $j$. Again, we can derive $\Rightarrow  \bigwedge_{i=1}^n (\nabla^{l_i} (A_i \rightarrow B_i)) \rightarrow \nabla^{l_j - 1} A_j$ using proper $L\land_{1/2}$, $Lc$ and $R\rightarrow$.
\end{itemize}
Notice that no other case is valid, since they all imply different structure for $\mathcal{D}$. Now from theorem \ref{thm:stl-eq-gstl}, all of the desired sequents also have a proof in $\stl$.



\begin{cor}[Disjuntion property]
	For all formulas $B$ and $C$, if $\vdash B \vee C$ then either $\vdash B$ or $\vdash C$.
\end{cor}
\begin{proof}
	Just apply Theorem \ref{thm:visser} for $n = 0$.
\end{proof}

\section{Interpolation theorems}
{\color{red} Interpolation theorems do not hold if the extension contains $Fa$ or $Fu$, since they assume cut-ellimination results.}

Existence of \emph{interpolant} formulas was proved by W. Craig \cite{CraigA} for classical predicate logic. In this section, we prove \emph{Craig's interpolation} theorem for $\istl$ and all its extensions using a method similar to Maehara \cite{maehara1960interpolation}. We will also prove a weaker version called \emph{deductive interpolation} for any extension of $\stl$ containing $L$. For some formula $A$, by $V(A)$ we mean the set of all atomic variables of $A$ and for a multi-set of formulas $\Gamma$, by $V(\Gamma)$ we denote the set of all atomic variables occuring in some formula in $\Gamma$, i.e. $\bigcup_{A \in \Gamma} V(A)$.

\begin{thm}[Craig's Interpolation for $\istl$]\label{thm:istl-craig} For any $\Gamma_1$, $\Gamma_2$ and $\Delta$, and for $S \subseteq \{L, R, Fa, Fu\}$, if $\istl(S) \vdash \Gamma_1 , \Gamma_2 \Rightarrow \Delta$, then there is a formula $C$ such that $V(C) \subseteq V(\Gamma_1) \cap V(\Gamma_2 , \Delta)$, $\istl(S) \vdash \Gamma_1 \Rightarrow C$ and $\istl(S) \vdash \Gamma_2 , C \Rightarrow \Delta$.
\end{thm}

\begin{proof}
Let $\mathcal{D}$ be a cut-free proof for $\Gamma_1 , \Gamma_2 \Rightarrow \Delta$ in $\igstl(S)$, from Theorems \ref{thm:istl-eq-igstl} and \ref{thm:igstl-cut-elim}. We will find an interpolant which satisfies the statement of the theorem for $\igstl(S)$, nevertheless, it can be translated back to $\istl(S)$, again by Theorem \ref{thm:istl-eq-igstl}.

We will use induction on the length of $\mathcal{D}$: For any smaller proof-tree which proves $\Gamma_1' , \Gamma_2' \Rightarrow \Delta'$, the induction hypothesis (IH) provides an interpolant, which we call $C_{\langle\Gamma_1'; \Gamma_2'; \Delta'\rangle}$, for which the statement of the theorem is true. We now build the desired interpolant $C$, in different cases for the last rule of $\mathcal{D}$. In cases for left-rules, we also need to consider whether the principal formula is in $\Gamma_1$ or $\Gamma_2$ in separate cases.
\begin{enumerate}
	\item ($Id$) We have $\Gamma_1,\Gamma_2 = \Delta = A$.
	\begin{enumerate}
		\item If $\Gamma_1 = \{\}$ and $\Gamma_2 = A$, then define $C = \top$. So we have $\Rightarrow \top$ by $Ta$ and $A , \top \Rightarrow A$ by $Id$ and $Lw$.
		
		\item If $\Gamma_1 = A$ and $\Gamma_2 = \{\}$ then define $C = A$. So we have $A \Rightarrow A$ by $Id$.
	\end{enumerate}
	\item ($Ta$) Take $C = \top$.
	
	\item ($Ex$) Take $C = \nabla^n \bot$.
	
	\item ($Lw$) $\mathcal{D}$ proves $\Gamma_1' , \Gamma_2' , A \Rightarrow \Delta$ and has a sub-proof for $\Gamma_1' , \Gamma_2' \Rightarrow \Delta$, for which IH gives an interpolant $C_{\langle\Gamma_1';\Gamma_2';\Delta\rangle}$ and proofs for $\Gamma_1' \Rightarrow C_{\langle\Gamma_1';\Gamma_2';\Delta\rangle}$ and $\Gamma_2' , C_{\langle\Gamma_1';\Gamma_2';\Delta\rangle} \Rightarrow \Delta$, such that $V(C_{\langle\Gamma_1';\Gamma_2';\Delta\rangle}) \subseteq$ $ V(\Gamma_1') \cap V(\Gamma_2' , \Delta)$.
	\begin{enumerate}
		\item If $\Gamma_1 = \Gamma_1'$ and $\Gamma_2 = \Gamma_2' , A$, then take $C = C_{\langle\Gamma_1';\Gamma_2';\Delta\rangle}$. We have  $\Gamma_1' \Rightarrow C$ by IH and $\Gamma_2 , A , C \Rightarrow \Delta$ by $Lw$ and IH. From IH, we also have $V(C) \subseteq V(\Gamma_1') \cap V(\Gamma_2' , A , \Delta)$, since $P$ takes ``$,$'' to ``$\cup$'', which distributes over ``$\cap$'' and is increasing with respect to ``$\subseteq$''.
		
		\item If $\Gamma_1 = \Gamma_1' , A$ and $\Gamma_2 = \Gamma_2'$, again take $C = C_{\langle\Gamma_1';\Gamma_2';\Delta\rangle}$. Then we have  $\Gamma_1' , A \Rightarrow C$ by $Lw$ and IH, and $\Gamma_2 , C \Rightarrow \Delta$ by IH. We also have $V(C) \subseteq V(\Gamma_1' , A) \cap V(\Gamma_2' , \Delta)$ by IH and argument similar to the previous case.
	\end{enumerate}

	\item ($Lc$) $\mathcal{D}$ proves $\Gamma_1' , \Gamma_2' , A \Rightarrow \Delta$ and has a sub-proof for $\Gamma_1' , \Gamma_2' , A , A \Rightarrow \Delta$.
	\begin{enumerate}
		\item If $\Gamma_1 = \Gamma_1'$ and $\Gamma_2 = \Gamma_2' , A$, take $C = C_{\langle\Gamma_1';\Gamma_2',A,A;\Delta\rangle}$. Then we have $\Gamma_1' \Rightarrow C$ by IH and $\Gamma_2' , A \Rightarrow \Delta$ by IH and $Lc$. From IH, we also have $V(C) \subseteq V(\Gamma_1') \cap V(\Gamma_2',A,\Delta)$, since $V(\Gamma,X) = V(\Gamma,X,X)$.
		
		\item If $\Gamma_1 = \Gamma_1' , A$ and $\Gamma_2 = \Gamma_2'$, take $C = C_{\langle\Gamma_1',A,A;\Gamma_2';\Delta\rangle}$. Then we have $\Gamma_1' , A \Rightarrow C$ by IH and $Lc$, and $\Gamma_2' \Rightarrow \Delta$ by IH. We also have $V(C) \subseteq V(\Gamma_1',A) \cap V(\Gamma_2',\Delta)$ as justified before.
	\end{enumerate}

	\item[6,7.] ($L\land_i$, {\small$i \in \{1,2\}$}) $\mathcal{D}$ proves $\Gamma_1' , \Gamma_2' , \nabla^n (A_1 \land A_2) \Rightarrow \Delta$ and has a sub-proof for $\Gamma_1' , \Gamma_2' , \nabla^n A_i \Rightarrow \Delta$.
	\begin{enumerate}
		\item If $\Gamma_1 = \Gamma_1'$ and $\Gamma_2 = \Gamma_2' , \nabla^n (A_1 \land A_2)$, take $C = C_{\langle\Gamma_1';\Gamma_2',\nabla^n A_i;\Delta\rangle}$. Then we have $\Gamma_1' \Rightarrow C$ by IH and $\Gamma_2' , \nabla^n (A_1 \land A_2) \Rightarrow \Delta$ by IH and $L\land_i$. From IH, we also have $V(C) \subseteq$ $V(\Gamma_1') \cap V(\Gamma_2',\nabla^n(A_1 \land A_2),\Delta)$, since $V(\nabla^n X) = V(X)$ and $P$ takes sub-formula ordering to ``$\subseteq$''.
		
		\item If $\Gamma_1 = \Gamma_1' , \nabla^n (A_1 \land A_2)$ and $\Gamma_2 = \Gamma_2'$, take $C = C_{\langle\Gamma_1',\nabla^n A_i;\Gamma_2';\Delta\rangle}$. Then we have $\Gamma_1' , \nabla^n (A_1 \land A_2)$ $\Rightarrow C$ by IH and $L\land_i$. Also from IH we have $\Gamma_2' \Rightarrow \Delta$. We also have $V(C) \subseteq V(\Gamma_1',\nabla^n (A_1 \land A_2))$ $\cap V(\Gamma_2',\Delta)$ as justified in the previous case.
	\end{enumerate}
	\setcounter{enumi}{7}

	\item ($R\land$) $\mathcal{D}$ proves $\Gamma_1 , \Gamma_2 \Rightarrow A \land B$ and has sub-proofs for $\Gamma_1 , \Gamma_2 \Rightarrow A$ and $\Gamma_1 , \Gamma_2 \Rightarrow B$.\\
	Let $C_1 = C_{\langle\Gamma_1;\Gamma_2;A\rangle}$ and $C_2 = C_{\langle\Gamma_1;\Gamma_2;B\rangle}$, and then take $C = C_1 \land C_2$.
	We have $\Gamma_1 \Rightarrow C_1$ and $\Gamma_1 \Rightarrow C_2$, both from IH. Then by $R\land$ we have $\Gamma_1 \Rightarrow C_1 \land C_2$.
	We also have $\Gamma_2 , C_1 \Rightarrow A$ and $\Gamma_2 , C_2 \Rightarrow B$, again from IH.
	We can then derive $\Gamma_2 , C_1 \land C_2 \Rightarrow A$ and $\Gamma_2 , C_1 \land C_2 \Rightarrow B$, respectively by $L\land_1$ and $L\land_2$, and finally  $\Gamma_2 , C_1 \land C_2 \Rightarrow A \land B$ by $R\land$.
	We also have $V(C_1) \subseteq V(\Gamma_1) \cap V(\Gamma_2 , A)$ and $V(C_2) \subseteq V(\Gamma_1) \cap V(\Gamma_2 , B)$. So $V(C_1 , C_2) \subseteq V(\Gamma_1) \cap V(\Gamma_2 , A , B)$ as it was justified before, and then $V(C_1 \land C_2) \subseteq V(\Gamma_1) \cap V(\Gamma_2 , A \land B)$.
	
	\item ($L\lor$) $\mathcal{D}$ proves $\Gamma_1' , \Gamma_2' , \nabla^n (A \lor B) \Rightarrow \Delta$ and has sub-proofs for $\Gamma_1' , \Gamma_2' , \nabla^n A \Rightarrow \Delta$ and $\Gamma_1' , \Gamma_2' , \nabla^n B \Rightarrow \Delta$.
	\begin{enumerate}
		\item If $\Gamma_1 = \Gamma_1'$ and $\Gamma_2 = \Gamma_2' , \nabla^n (A \lor B)$, let $C_1 = C_{\langle\Gamma_1';\Gamma_2',\nabla^n A;\Delta\rangle}$ and $C_2 = C_{\langle\Gamma_1';\Gamma_2',\nabla^n B;\Delta\rangle}$, and then take $C = C_1 \land C_2$.
		We have $\Gamma_1' \Rightarrow C_1 \land C_2$ from IH and $R\land$.
		From IH, by $L\land_1$ and $L\land_2$ we can derive $\Gamma_2' , \nabla^n A , C_1 \land C_2 \Rightarrow \Delta$ and $\Gamma_2' , \nabla^n B , C_1 \land C_2 \Rightarrow \Delta$ respectively, to which we apply $L\lor$ to get to $\Gamma_2' , \nabla^n (A \lor B) , C_1 \land C_2 \Rightarrow \Delta$.
		From IH, we also have $V(C_1) \subseteq V(\Gamma_1') \cap V(\Gamma_2' , \nabla^n A , \Delta)$ and $V(C_2) \subseteq V(\Gamma_1') \cap V(\Gamma_2' , \nabla^n B , \Delta)$. Just like the previous case, we can deduce that $V(C_1 \land C_2) \subseteq V(\Gamma_1') \cap V(\Gamma_2' , \nabla^n (A \land B) , \Delta)$.

		\item If $\Gamma_1 = \Gamma_1' , \nabla^n (A \lor B)$ and $\Gamma_2 = \Gamma_2'$, let $C_1 = C_{\langle\Gamma_1',\nabla^n A;\Gamma_2';\Delta\rangle}$ and $C_2 = C_{\langle\Gamma_1',\nabla^n B;\Gamma_2';\Delta\rangle}$, and then take $C = C_1 \lor C_2$.
		From IH, by $R\lor_1$ and $R\lor_2$ we can derive $\Gamma_1' , \nabla^n A \Rightarrow C_1 \lor C_2$ and $\Gamma_1' , \nabla^n B \Rightarrow C_1 \lor C_2$ respectively, to which we apply $L\lor$ to get to $\Gamma_1' , \nabla^n (A \lor B) \Rightarrow C_1 \lor C_2$.
		We have $\Gamma_2' , C_1 \lor C_2 \Rightarrow \Delta$ from IH and $L\lor$.
		From IH, we also have $V(C_1) \subseteq V(\Gamma_1' , \nabla^n A) \cap$ $V(\Gamma_2' , \Delta)$ and $V(C_2) \subseteq V(\Gamma_1' , \nabla^n B) \cap V(\Gamma_2' , \Delta)$. Just like the previous case, we can deduce that $V(C_1 \lor C_2) \subseteq V(\Gamma_1' , \nabla^n (A \land B)) \cap V(\Gamma_2' , \Delta)$.
	\end{enumerate}

	\item[10,11.] ($R\lor_i$, {\small$i \in \{1,2\}$}) $\mathcal{D}$ proves $\Gamma_1 , \Gamma_2 \Rightarrow A_1 \lor A_2$ and has a sub-proof for $\Gamma_1 , \Gamma_2 \Rightarrow A_i$. Take $C = C_{\langle\Gamma_1;\Gamma_2;A_i\rangle}$. Then we have $\Gamma_1 \Rightarrow C$ from IH and $\Gamma_2 , C \Rightarrow A_1 \lor A_2$ from IH and $R\lor_i$.
	From IH, we also have $V(C) \subseteq V(\Gamma_1) \cap V(\Gamma_2 , A_1 \lor A_2)$, as was justified before.
	\setcounter{enumi}{11}
	
	\item ($L\supset$) $\mathcal{D}$ proves $\Gamma_1' , \Gamma_2' , \nabla^n (A \supset B) \Rightarrow \Delta$ and has sub-proofs for $\Gamma_1' , \Gamma_2' \Rightarrow \nabla^n A$ and $\Gamma_1' , \Gamma_2' , \nabla^n B \Rightarrow \Delta$.
	\begin{enumerate}
		\item If $\Gamma_1 = \Gamma_1'$ and $\Gamma_2 = \Gamma_2' , \nabla^n (A \supset B)$, let $C_1 = C_{\langle\Gamma_1';\Gamma_2';\nabla^n A\rangle}$ and $C_2 = C_{\langle\Gamma_1';\Gamma_2',\nabla^n B;\Delta\rangle}$, and take $C = C_1 \land C_2$.
		We have $\Gamma_1' \Rightarrow C_1 \land C_2$ from IH and $R\land$.
		From IH, by $L\land_1$ and $L\land_2$ we can derive $\Gamma_2' , C_1 \land C_2 \Rightarrow \nabla^n A$ and $\Gamma_2' , \nabla^n B , C_1 \land C_2 \Rightarrow \Delta$ respectively, to which we apply $L\supset$ to get $\Gamma_2' , \nabla^n (A \supset B) , C_1 \land C_2 \Rightarrow \Delta$.
		From IH, we also have $V(C_1) \subseteq V(\Gamma_1') \cap$ $V(\Gamma_2' , \nabla^n A)$ and $V(C_2) \subseteq V(\Gamma_1') \cap V(\Gamma_2' , \nabla^n B , \Delta)$. This implies $V(C_1 \land C_2) \subseteq V(\Gamma_1') \cap V(\Gamma_2' , \nabla^n (A \supset B) , \Delta)$.

		\item If $\Gamma_1 = \Gamma_1' , \nabla^n (A \supset B)$ and $\Gamma_2 = \Gamma_2'$, let $C_1 = C_{\langle\Gamma_2';\Gamma_1';\nabla^n A\rangle}$ and $C_2 = C_{\langle\Gamma_1',\nabla^n B;\Gamma_2';\Delta\rangle}$, and take $C = C_1 \supset C_2$.
		From IH we have $\Gamma_1' , C_1 \Rightarrow \nabla^n A$ and $\Gamma_1' , \nabla^n B , C_1 \Rightarrow C_2$, with a $Lw$ to introduce $C_1$ to the left. From $L\supset$ we get $\Gamma_1 , \nabla^n (A \supset B) , C_1 \Rightarrow C_2$, to which we can apply $R\supset$ to get $\Gamma_1' , \nabla^n (A \supset B) \Rightarrow C_1 \supset C_2$.

		From IH, we have also $\Gamma_2' \Rightarrow C_1$ and $\Gamma_2' , C_2 \Rightarrow \Delta$, from which we can derive $\Gamma_2' , C_1 \supset C_2 \Rightarrow \Delta$ by an application of $L\supset$. IH also states that $V(C_1) \subseteq V(\Gamma_2') \cap V(\Gamma_1' , \nabla^n A)$ and $V(C_2) \subseteq V(\Gamma_1' , \nabla^n B) \cap V(\Gamma_2' , \Delta)$. Then $V(C_1 \supset C_2) \subseteq V(\Gamma_1' , \nabla^n (A \supset B)) \cap V(\Gamma_2' , \Delta)$.
	\end{enumerate}

	\item ($R\supset$) $\mathcal{D}$ proves $\Gamma_1 , \Gamma_2 \Rightarrow A \supset B$ and has a sub-proof for $\Gamma_1 , \Gamma_2 , A \Rightarrow B$. Let $C = C_{\langle\Gamma_1;\Gamma_2,A;B\rangle}$. So we have $\Gamma_1 \Rightarrow C$ and $\Gamma_2 , C \Rightarrow A \supset B$ from IH and an application of $R\supset$.
	We also have $V(C) \subseteq V(\Gamma_1) \cap V(\Gamma_2 , A \supset B)$ from IH and the fact that $P$ preserves sub-formula ordering in $\subseteq$.

	\item ($L\rightarrow$) This case is similar to $L\supset$. $\mathcal{D}$ proves $\Gamma_1' , \Gamma_2' , \nabla^{n+1} (A \rightarrow B) \Rightarrow \Delta$ and has sub-proofs for $\Gamma_1' , \Gamma_2' \Rightarrow \nabla^n A$ and $\Gamma_1' , \Gamma_2' , \nabla^n B \Rightarrow \Delta$.
	\begin{enumerate}
		\item If $\Gamma_1 = \Gamma_1'$ and $\Gamma_2 = \Gamma_2' , \nabla^{n+1} (A \rightarrow B)$, let $C_1 = C_{\langle\Gamma_1';\Gamma_2';\nabla^n A\rangle}$ and $C_2 = C_{\langle\Gamma_1';\Gamma_2',\nabla^n B;\Delta\rangle}$, and take $C = C_1 \land C_2$.
		We have $\Gamma_1' \Rightarrow C_1 \land C_2$ from IH and $R\land$.
		From IH, by $L\land_1$ and $L\land_2$ we can derive $\Gamma_2' , C_1 \land C_2 \Rightarrow \nabla^n A$ and $\Gamma_2' , \nabla^n B , C_1 \land C_2 \Rightarrow \Delta$ respectively, to which we apply $L\rightarrow$ to get to $\Gamma_2' , \nabla^{n+1} (A \rightarrow B) , C_1 \land C_2 \Rightarrow \Delta$.
		From IH, we also have $V(C_1) \subseteq V(\Gamma_1') \cap$ $V(\Gamma_2' , \nabla^n A)$ and $V(C_2) \subseteq V(\Gamma_1') \cap V(\Gamma_2' , \nabla^n B , \Delta)$, which implies $V(C_1 \land C_2) \subseteq V(\Gamma_1') \cap V(\Gamma_2' , \nabla^{n+1} (A \rightarrow B) , \Delta)$.

		\item If $\Gamma_1 = \Gamma_1' , \nabla^{n+1} (A \rightarrow B)$ and $\Gamma_2 = \Gamma_2'$, let $C_1 = C_{\langle\Gamma_2';\Gamma_1';\nabla^n A\rangle}$ and $C_2 = C_{\langle\Gamma_1',\nabla^n B;\Gamma_2';\Delta\rangle}$, and take $C = C_1 \supset C_2$.
		From IH we have $\Gamma_1' , C_1 \Rightarrow \nabla^n A$. Also from IH, with a $Lw$ to add $C_1$ to the left, we have $\Gamma_1' , \nabla^n B , C_1 \Rightarrow C_2$. By $L\rightarrow$ and $R\supset$ we get $\Gamma_1' , \nabla^{n+1} (A \rightarrow B) \Rightarrow C_1 \supset C_2$.
		We also have $\Gamma'_2, C_1 \supset C_2 \Rightarrow \Delta$ from IH and $L\supset$. Again from IH, we have $V(C_1) \subseteq V(\Gamma_2') \cap V(\Gamma_1' , \nabla^n A)$ and $V(C_2) \subseteq V(\Gamma_1' , \nabla^n B) \cap V(\Gamma_2' , \Delta)$, thus $V(C_1 \supset C_2) \subseteq V(\Gamma_1' , \nabla^{n+1} (A \supset B)) \cap V(\Gamma_2' , \Delta)$.
	\end{enumerate}

	\item ($R\rightarrow$) This is also similar to the $R\supset$ case, except that here $\mathcal{D}$'s sub-proof proves $\nabla \Gamma_1 , \nabla \Gamma_2 , A \Rightarrow B$. Let $C' = C_{\langle\nabla\Gamma_1;\nabla\Gamma_2,A;B\rangle}$ amd take $C = \top \rightarrow C'$. So we have $\nabla \Gamma_1 \Rightarrow C'$ from IH, at the right of which we can introduce $\top$ by $Lw$, and then apply $R\rightarrow$ to get $\Gamma_1 \Rightarrow \top \rightarrow C'$. From IH, we also have $\nabla \Gamma_2, A, C' \Rightarrow B$. On the other hand, we have $\nabla (\top \rightarrow C') \Rightarrow C'$ by applying $L\rightarrow$ on $\Rightarrow \top$ and $C' \Rightarrow C'$. Using this sequent and $Cut$, we can replace $C'$ with $\nabla (\top \rightarrow C')$ in the former sequent to get $\nabla \Gamma_2 , A , \nabla (\top \rightarrow C') \Rightarrow B$. By $R\rightarrow$ we would get $\Gamma_2 , \top \supset C' \Rightarrow A \rightarrow B$.
	We also have $V(\top \rightarrow C) \subseteq V(\Gamma_1) \cap V(\Gamma_2 , A \rightarrow B)$ from IH and the fact that $P$ preserves sub-formula ordering in $\subseteq$ and $\top$ does not introduce new atomic formulas.

	\item ($N$) $\mathcal{D}$ proves $\nabla \Gamma_1 , \nabla \Gamma_2 \Rightarrow \nabla \Delta$ and has a sub-proof for $\Gamma_1 , \Gamma_2 \Rightarrow \Delta$. Just take $C = C(\Gamma_1;\Gamma_2;\Delta)$ and apply $N$ on the sequents from IH. The variable condition is also trivial.
	
	\item ($L$) $\mathcal{D}$ proves $\Gamma_1' , \Gamma_2' , \nabla A \Rightarrow \Delta$ and has a sub-proof for $\Gamma_1' , \Gamma_2' , A \Rightarrow \Delta$.
	\begin{enumerate}
		\item If $\Gamma_1 = \Gamma_1'$ and $\Gamma_2 = \Gamma_2' , \nabla A$, take $C = C_{\langle\Gamma_1';\Gamma_2',A;\Delta\rangle}$. Then we have $\Gamma_1' \Rightarrow C$ by IH and $\Gamma_2' , \nabla A \Rightarrow \Delta$ by IH and $L$. From IH, it's also trivial that $V(C) \subseteq V(\Gamma_1') \cap V(\Gamma_2',\nabla A,\Delta)$.
		
		\item If $\Gamma_1 = \Gamma_1' , \nabla A$ and $\Gamma_2 = \Gamma_2'$, take $C = C_{\langle\Gamma_1',A;\Gamma_2';\Delta\rangle}$. Then we have $\Gamma_1' , \nabla A \Rightarrow C$ by IH and $L$, and $\Gamma_2' \Rightarrow \Delta$ by IH. We also have $V(C) \subseteq V(\Gamma_1', \nabla A) \cap V(\Gamma_2',\Delta)$
	\end{enumerate}

	\item ($R$) Assume $\Gamma_1 = \Pi_1, \Sigma_1$ and $\Gamma_2 = \Pi_2, \Sigma_2$. $\mathcal{D}$ proves $\Pi_1, \Sigma_1, \Pi_2, \Sigma_2 \Rightarrow \Delta$ and has a sub-proof for $\Pi_1, \nabla\Sigma_1, \Pi_2, \nabla\Sigma_2 \Rightarrow \Delta$.
	Take $C =$\\ $C_{\langle\Pi_1\nabla\Sigma_1;\Pi_2\nabla\Sigma_2;\Delta\rangle}$. Then from IH and $R$ we have $\Pi_1, \Sigma_1 \Rightarrow C$ and $\Pi_2, \Sigma_2, C \Rightarrow \Delta$. We also have $V(C) \subseteq V(\Pi_1,\Sigma_1) \cap V(\Pi_2,\Sigma_2,\Delta)$, since $\nabla$ does not introduce new atomic formulas and we can drop it.

	\item ($Fa$) This is similar to the $R\supset$ case. $\mathcal{D}$ proves $\Gamma_1 , \Gamma_2 \Rightarrow \nabla(A \rightarrow B)$ and has a sub-proof for $\Gamma_1 , \Gamma_2 , A \Rightarrow B$. Let $C = C_{\langle\Gamma_1;\Gamma_2,A;B\rangle}$. So we have $\Gamma_1 \Rightarrow C$ and $\Gamma_2 , C \Rightarrow \nabla (A \rightarrow B)$ from IH and an application of $Fa$.
	It is easy to deduce $V(C) \subseteq V(\Gamma_1) \cap V(\Gamma_2 , \nabla (A \rightarrow B))$ from IH.

	\item ($Fu$) $\mathcal{D}$ proves $\Gamma_1, \Gamma_2 \Rightarrow \Delta$ and has a sub-proof for $\nabla \Gamma_1, \nabla \Gamma_2 \Rightarrow \nabla \Delta$. Let $C' = C_{\langle\nabla\Gamma_1;\nabla\Gamma_2;\nabla\Delta\rangle}$ and take $C = \top \rightarrow C'$. We can derive $\Gamma_1 \Rightarrow \top \rightarrow C'$ by $Lw$ and $R\rightarrow$ on the sequent that we get from IH. On the other hand, Lemma \ref{lem:l-nabla-box} gives us $\nabla (\top \rightarrow C') \Rightarrow C'$, which can be cut into $\nabla \Gamma_2, C' \Rightarrow \nabla \Delta$, to get $\nabla \Gamma_2, \nabla (\top \rightarrow C') \Rightarrow \nabla \Delta$. By $Fu$ we have $\Gamma_2, \top \rightarrow C' \Rightarrow \Delta$. The variable condition also holds, since $V(C') = V(\top \rightarrow C')$.
\end{enumerate}
\end{proof}


\subsection{Interpolation for $\stl$}
We can not prove the same result for the fragment $\stl$, since we ought to use $\supset$ to construct the interpolant in the  $L\rightarrow$ case of the last theorem. But if we add the rule $L$ to this fragment, we can prove a weaker form of interpolation using $\rightarrow$, as we will do in this section.
Throughout this section, when we use plain $\vdash$ symbol without specifying the deductive system, we mean deduction in $\stl(S, L)$ where $S \subseteq \{L\supset, R\supset, R, Fu, Fa\}$. So by $\Gamma' \Rightarrow \Delta' \vdash \Gamma \Rightarrow \Delta$ we mean that $\Gamma \Rightarrow \Delta$ is provable in $\stl(S)+\{\Gamma' \Rightarrow \Delta'\}$.
\begin{lem}
	\label{lem:vdash} $\Rightarrow A \vdash \Gamma \Rightarrow \Delta$ if and only if $\Gamma, \nabla^n A \Rightarrow \Delta$, for some $n$.
\end{lem}
\begin{proof}
	I. Suppose we have $\Gamma, \nabla^n A \Rightarrow \Delta$ for some $n$. Assuning $\Rightarrow A$, we can first prove $\Rightarrow \nabla^n A$ by $n$ applications of $N$, and then $Cut$ it into $\Gamma, \nabla^n A \Rightarrow \Delta$ to get $\Gamma \Rightarrow \Delta$.

	II. For the other direction, suppose $\Gamma \Rightarrow \Delta$ has a proof-tree like $\mathcal{D}$ in $\stl(S,L)+\{\Rightarrow A\}$. It suffices to prove $\nabla^n A, \Gamma \Rightarrow \Delta$ for some $n$ in $\stl(S,L)$. By induction on the length of $\mathcal{D}$, we can assume that the statement of the theorem holds for the possible subtree(s) of $\mathcal{D}$.

	Now for different cases for the last rule of $\mathcal{D}$, we will construct the desired proof. In all cases, the possible immediate subtrees of $\mathcal{D}$ are called $\mathcal{D}_0$ and $\mathcal{D}_1$.
	
	If $\mathcal{D}$ is just $\Rightarrow A$, then $\Gamma$ and $\Delta$ must be $\{\}$ and $A$ respectively. So we would have $A \Rightarrow A$ by $Id$ with $n = 0$.

	If $\mathcal{D}$ is any of the axioms $Id$, $Ta$ or $Ex$, then we can add $A$ to the left side of the same axioms to construct the desired sequent with $n = 0$.

	If $\mathcal{D}$ ends with any of the rules with one assumption, which are $Lw$, $Lc$,$Rw$, $L\land_1$, $L\land_2$, $R\lor_1$, $R\lor_2$, $N$, $L$, $R$ or $Fa$, just apply the same rule on IH($\mathcal{D}_0$).

	If $\mathcal{D}$ ends with $Cut$, we can also do the same, applying $Cut$ on IH($\mathcal{D}_0$) and IH($\mathcal{D}_1$), since $Cut$ is a context-free rule.

	For the context-sensitive rules with two assumptions, which are $R\land$, $L\lor$, and $L\rightarrow$, we must first equalize the number of $\nabla$'s on $A$ in the left side of the sequents that we get from IH, then we can apply the same rule.

	If $\mathcal{D}$ ends with $R\rightarrow$ or $Fu$, we can do the same only if there is at least one $\nabla$ on $A$ in the left side of the sequent that we get from IH. Otherwise we can just add it with $L$, and then apply $R\rightarrow$ or $Fu$.
\end{proof}

The deductive interpolation is stated in the next corollary. First we need to prove a slightly stronger version for technical reasons.

\begin{thm}\label{thm:ldls-dedint}
	For any $\Gamma_1$, $\Gamma_2$ and $\Delta$, if $\vdash \Gamma_1 , \Gamma_2 \Rightarrow \Delta$ then there is a formula $C$ and natural numbers $m_1$ and $m_2$ such that
	\begin{enumerate}[label=(\arabic**)]
		\item $\vdash \nabla^{m_1} \Gamma_1 \Rightarrow C$,
		\item $\vdash \nabla^{m_2} \Gamma_2 , C \Rightarrow \Delta$,
		\item $V(C) \subseteq V(\Gamma_1) \cap V(\Gamma_2,\Delta)$, and
	\end{enumerate}
\end{thm}
\begin{proof}
	By theorem \ref{thm:gstl-cut-elim}, we can assume that $\Gamma_1, \Gamma_2 \Rightarrow \Delta$ has a cut-free proof-tree like $\mathcal{D}$. Now using induction on the length of $\mathcal{D}$, we can assume that the stronger statement holds for the assumption(s) of $\mathcal{D}$ like $\Gamma_1', \Gamma_2' \Rightarrow \Delta'$ by an interpolant that we call $C_{\langle\Gamma_1';\Gamma_2';\Delta'\rangle}$.
	Now construct proper $C$ for different cases for the last rule of $\mathcal{D}$. In each case, the immediate subtrees of $\mathcal{D}$ are denoted by $\mathcal{D}_i (i \in \{0,1\})$.
	\begin{enumerate}
		\item ($Id$) We have $\Gamma_1,\Gamma_2 = \Delta = A$ for some formula $A$.
		\begin{enumerate}
			\item If $\Gamma_1 = \{\}$ and $\Gamma_2 = A$, then take $C = \top$. So we have $\Rightarrow \top$ by $Ta$ and $A , \top \Rightarrow A$ by $Id$ and $Lw$.

			\item If $\Gamma_1 = A$ and $\Gamma_2 = \{\}$, take $C = A$, and we have $A \Rightarrow A$ by $Id$.
		\end{enumerate}
		\item ($Ta$) Take $C = \top$.

		\item ($Ex$) Take $C = \nabla^n \bot$.

		\item ($Lw$) $\mathcal{D}$ proves $\Gamma_1' , \Gamma_2' , A \Rightarrow \Delta$ and has a sub-proof for $\Gamma_1' , \Gamma_2' \Rightarrow \Delta$, for which IH gives an interpolant $C_{\langle\Gamma_1';\Gamma_2';\Delta\rangle}$, natural numbers $m_1$ and $m_2$ and proofs for $\nabla^{m_1} \Gamma_1' \Rightarrow C_{\langle\Gamma_1';\Gamma_2';\Delta\rangle}$ and $\nabla^{m_2} \Gamma_2 , C_{\langle\Gamma_1';\Gamma_2';\Delta\rangle} \Rightarrow \Delta$, such that $V(C_{\langle\Gamma_1';\Gamma_2';\Delta\rangle}) \subseteq$ $ V(\Gamma_1') \cap V(\Gamma_2' , \Delta)$.
		\begin{enumerate}
			\item If $\Gamma_1 = \Gamma_1'$ and $\Gamma_2 = \Gamma_2' , A$, take $C = C_{\langle\Gamma_1';\Gamma_2';\Delta\rangle}$. Then by IH we have  $\nabla^{m_1} \Gamma_1' \Rightarrow C$ and $\nabla^{m_2} \Gamma_2' , A , C \Rightarrow \Delta$ by $Lw$ and IH. From IH, we also have $V(C) \subseteq V(\Gamma_1') \cap V(\Gamma_2' , A , \Delta)$, since $P$ takes ``$,$'' to ``$\cup$'', which distributes over ``$\cap$'' and is increasing with respect to ``$\subseteq$''.

			\item If $\Gamma_1 = \Gamma_1' , A$ and $\Gamma_2 = \Gamma_2'$, again take $C = C_{\langle\Gamma_1';\Gamma_2';\Delta\rangle}$. Then we have  $\nabla^{m_1} \Gamma_1' , A \Rightarrow C$ by $Lw$ and IH, and $\nabla^{m_2} \Gamma_2' , C \Rightarrow \Delta$ by IH. We also have $V(C) \subseteq V(\Gamma_1' , A) \cap V(\Gamma_2' , \Delta)$ by IH and argument similar to the previous case.
		\end{enumerate}

		\item ($Lc$) $\mathcal{D}$ proves $\Gamma_1' , \Gamma_2' , A \Rightarrow \Delta$ and has a sub-proof for $\Gamma_1' , \Gamma_2' , A , A \Rightarrow \Delta$.
		\begin{enumerate}
			\item If $\Gamma_1 = \Gamma_1'$ and $\Gamma_2 = \Gamma_2' , A$, take $C = C_{\langle\Gamma_1';\Gamma_2',A,A;\Delta\rangle}$. Then we have $\nabla^{m_1} \Gamma_1' \Rightarrow C$ by IH and $\nabla^{m_2}\Gamma_2' , \nabla^{m_2} A , C \Rightarrow \Delta$ by IH and $Lc$. From IH, we can also deduce $V(C) \subseteq V(\Gamma_1') \cap V(\Gamma_2',A,\Delta)$ similarly.
			
			\item If $\Gamma_1 = \Gamma_1' , A$ and $\Gamma_2 = \Gamma_2'$, take $C = C_{\langle\Gamma_1',A,A;\Gamma_2';\Delta\rangle}$. Then we have $\nabla^{m_1} \Gamma_1', \nabla^{m_1} A \Rightarrow C$ by IH and $Lc$, and $\nabla^{m_2} \Gamma_2' , C \Rightarrow \Delta$ by IH. We also have $V(C) \subseteq V(\Gamma_1',A) \cap$ $V(\Gamma_2',\Delta)$ as justified before.
		\end{enumerate}

		\item[6,7.] ($L\land_i$, {\small$i \in \{1,2\}$}) $\mathcal{D}$ proves $\Gamma_1' , \Gamma_2' , \nabla^n (A_1 \land A_2) \Rightarrow \Delta$ and has a sub-proof for $\Gamma_1' , \Gamma_2' , \nabla^n A_i \Rightarrow \Delta$.
		\begin{enumerate}
			\item If $\Gamma_1 = \Gamma_1'$ and $\Gamma_2 = \Gamma_2' , \nabla^n (A_1 \land A_2)$, take $C = C_{\langle\Gamma_1';\Gamma_2',\nabla^n A_i;\Delta\rangle}$. Then we have $\nabla^{m_1} \Gamma_1' \Rightarrow C$ by IH and $\nabla^{m_2} \Gamma_2' , \nabla^{n+m_2} (A_1 \land A_2), C \Rightarrow \Delta$ by IH and $L\land_i$. From IH, we also have $V(C) \subseteq$ $V(\Gamma_1') \cap V(\Gamma_2',\nabla^n(A_1 \land A_2),\Delta)$, since $V(\nabla^n X) = V(X)$ and $P$ takes sub-formula ordering to ``$\subseteq$''.
			
			\item If $\Gamma_1 = \Gamma_1' , \nabla^n (A_1 \land A_2)$ and $\Gamma_2 = \Gamma_2'$, take $C = C_{\langle\Gamma_1',\nabla^n A_i;\Gamma_2';\Delta\rangle}$. Then we have $\nabla^{m_1} \Gamma_1',\nabla^{n+m_1} (A_1 \wedge A_2) \Rightarrow C$ by IH and $L\wedge_i$. Also from IH we have $\nabla^{m_2} \Gamma_2', C \Rightarrow \Delta$. We also have $V(C) \subseteq V(\Gamma_1',\nabla^n (A_1 \land A_2))$ $\cap V(\Gamma_2',\Delta)$ as justified in the previous case.
		\end{enumerate}
		\setcounter{enumi}{7}

		\item ($R\land$) $\mathcal{D}$ proves $\Gamma_1 , \Gamma_2 \Rightarrow A \land B$ and has sub-proofs for $\Gamma_1 , \Gamma_2 \Rightarrow A$ and $\Gamma_1 , \Gamma_2 \Rightarrow B$.
		Let $C_1 = C_{\langle\Gamma_1;\Gamma_2;A\rangle}$ and $C_2 = C_{\langle\Gamma_1;\Gamma_2;B\rangle}$, and then take $C = C_1 \land C_2$.
		We have $\nabla^{m_1'} \Gamma_1 \Rightarrow C_1$ and $\nabla^{m_1''} \Gamma_1 \Rightarrow C_2$, both from IH, to which we apply proper number of $L$'s and equalize their contexts to be able to derive $\nabla^{m_1} \Gamma_1 \Rightarrow C_1 \wedge C_2$ by $R\wedge$ for some $m_1$.
		We also have $\nabla^{m_2'} \Gamma_2 , C_1 \Rightarrow A$ and $\nabla^{m_2''} \Gamma_2 , C_2 \Rightarrow B$, again from IH.
		We can then derive $\nabla^{m_2'} \Gamma_2 , C_1 \land C_2 \Rightarrow A$ and $\nabla^{m_2''} \Gamma_2 , C_1 \land C_2 \Rightarrow B$, respectively by $L\land_1$ and $L\land_2$, and finally  $\nabla^{m_2} \Gamma_2 , C_1 \land C_2 \Rightarrow A \land B$ by $L$ and $R\land$.
		We also have $V(C_1) \subseteq V(\Gamma_1) \cap V(\Gamma_2 , A)$ and $V(C_2) \subseteq V(\Gamma_1) \cap V(\Gamma_2 , B)$. So $V(C_1 \land C_2) \subseteq V(\Gamma_1) \cap V(\Gamma_2 , A \land B)$ as it was justified before.

		\item ($L\lor$) $\mathcal{D}$ proves $\Gamma_1' , \Gamma_2' , \nabla^n (A \lor B) \Rightarrow \Delta$ and has sub-proofs for $\Gamma_1' , \Gamma_2' , \nabla^n A \Rightarrow \Delta$ and $\Gamma_1' , \Gamma_2' ,$ $\nabla^n B \Rightarrow \Delta$.
		\begin{enumerate}
			\item If $\Gamma_1 = \Gamma_1'$ and $\Gamma_2 = \Gamma_2' , \nabla^n (A \lor B)$, let $C_1 = C_{\langle\Gamma_1';\Gamma_2',\nabla^n A;\Delta\rangle}$ and $C_2 = C_{\langle\Gamma_1';\Gamma_2',\nabla^n B;\Delta\rangle}$, and then take $C = C_1 \land C_2$.
			We have $\nabla^{m_1'} \Gamma_1' \Rightarrow C_1$ and $\nabla^{m_1''} \Gamma_1' \Rightarrow C_2$ from IH, to which we apply $R\land$ after equalizing their contexts with $L$, to get $\nabla^{m_1} \Gamma_1' \Rightarrow C_1 \land C_2$ for some $m_1$.
			From IH, by $L\land_1$ and $L\land_2$ we can derive $\nabla^{m_2'}\Gamma_2' , \nabla^n A , C_1 \land C_2 \Rightarrow \Delta$ and $\nabla^{m_2''} \Gamma_2' , \nabla^n B , C_1 \land C_2 \Rightarrow \Delta$ respectively, to which we apply $L$ and $L\lor$ to get to $\nabla^{m_2} \Gamma_2' , \nabla^n (A \lor B) , C_1 \land C_2 \Rightarrow \Delta$.
			From IH, we also have $V(C_1) \subseteq V(\Gamma_1') \cap V(\Gamma_2' , \nabla^n A , \Delta)$ and $V(C_2) \subseteq V(\Gamma_1') \cap V(\Gamma_2' , \nabla^n B , \Delta)$. Just like the previous case, we can deduce that $V(C_1 \land C_2) \subseteq V(\Gamma_1') \cap V(\Gamma_2' , \nabla^n (A \land B) , \Delta)$.

			\item If $\Gamma_1 = \Gamma_1' , \nabla^n (A \lor B)$ and $\Gamma_2 = \Gamma_2'$, let $C_1 = C_{\langle\Gamma_1',\nabla^n A;\Gamma_2';\Delta\rangle}$ and $C_2 = C_{\langle\Gamma_1',\nabla^n B;\Gamma_2';\Delta\rangle}$, and then take $C = C_1 \lor C_2$.
			From IH, by $R\lor_1$ and $R\lor_2$ we can derive $\nabla^{m_1'} \Gamma_1',\nabla^{n+m_1'} A \Rightarrow C_1 \lor C_2$ and $\nabla^{m_1''} \Gamma_1', \nabla^{n+m_1''} B \Rightarrow C_1 \lor C_2$ respectively, to which we apply $L$ and $L\vee$ to get to $\nabla^{m_1} \Gamma_1', \nabla^{n+m_1} (A \lor B) \Rightarrow C_1 \lor C_2$ for some $m_1$.
			From IH, we also have $\nabla^{m_2} \Gamma_2' , C_1 \lor C_2 \Rightarrow \Delta$ with $L$ and $L\lor$.
			Also $V(C_1) \subseteq V(\Gamma_1' , \nabla^n A) \cap$ $V(\Gamma_2' , \Delta)$ and $V(C_2) \subseteq V(\Gamma_1' , \nabla^n B) \cap V(\Gamma_2' , \Delta)$. Just like the previous case, we can deduce that $V(C_1 \lor C_2) \subseteq V(\Gamma_1' , \nabla^n (A \land B)) \cap V(\Gamma_2' , \Delta)$.
		\end{enumerate}

		\item[10,11.] ($R\lor_i$, {\small$i \in \{1,2\}$}) $\mathcal{D}$ proves $\Gamma_1 , \Gamma_2 \Rightarrow A_1 \lor A_2$ and has a sub-proof for $\Gamma_1 , \Gamma_2 \Rightarrow A_i$. Take $C = C_{\langle\Gamma_1;\Gamma_2;A_i\rangle}$. Then we have $\nabla^m \Gamma_1 \Rightarrow C$ from IH/ and $\nabla^{m_2} \Gamma_2 , C \Rightarrow A_1 \lor A_2$ from IH and $R\lor_i$.
		From IH, we also have $V(C) \subseteq V(\Gamma_1) \cap V(\Gamma_2 , A_1 \lor A_2)$, as was justified before.
		\setcounter{enumi}{11}

		\item ($L\rightarrow$) $\mathcal{D}$ proves $\Gamma_1' , \Gamma_2' , \nabla^{n+1} (A \rightarrow B) \Rightarrow \Delta$ and has sub-proofs for $\Gamma_1' , \Gamma_2' \Rightarrow \nabla^n A$ and $\Gamma_1' , \Gamma_2' , \nabla^n B \Rightarrow \Delta$.
		\begin{enumerate}
			\item If $\Gamma_1 = \Gamma_1'$ and $\Gamma_2 = \Gamma_2' , \nabla^{n+1} (A \rightarrow B)$, let $C_1 = C_{\langle\Gamma_1';\Gamma_2';\nabla^n A\rangle}$ and $C_2 = C_{\langle\Gamma_1';\Gamma_2',\nabla^n B;\Delta\rangle}$.
			We have $\nabla^{m_1} \Gamma_1' \Rightarrow C_1 \land C_2$ from IH and $R\land$.
			From IH we can derive $\nabla^{m_2'} \Gamma_2' , C_1 \land C_2 \Rightarrow \nabla^n A$ and $\nabla^{m_2''} \Gamma_2' , \nabla^{n+m_2''} B , C_1 \land C_2 \Rightarrow \Delta$, by $L\land_1$ and $L\land_2$ respectively. Applying $N$ for $m_2''$ times on the former sequent, and also applying $L$ for $m_2''$ times on $\nabla^{m_2'} \Gamma_2'$ and $m_2''$ times on $C_1 \wedge C_2$ in the latter sequent, we would get $\nabla^{m_2'+m_2''} \Gamma_2' , \nabla^{m_2''} (C_1 \land C_2) \Rightarrow \nabla^{n+m_2''} A$ and $\nabla^{m_2'+m_2''} \Gamma_2' , \nabla^{n+m_2''} B , \nabla^{m_2''} (C_1 \land C_2) \Rightarrow \Delta$, on which we can apply $L\rightarrow$ to derive $\nabla^{m_2'+m_2''} \Gamma_2' , \nabla^{n+m_2''+1} (A \rightarrow B) , \nabla^{m_2''} (C_1 \land C_2) \Rightarrow \Delta$. So we will take $C = \nabla^{m_2''}(C_1 \wedge C_2)$.
			From IH, we also have $V(C_1) \subseteq V(\Gamma_1') \cap$ $V(\Gamma_2' , \nabla^n A)$ and $V(C_2) \subseteq V(\Gamma_1') \cap V(\Gamma_2' , \nabla^n B , \Delta)$. This implies $V(\nabla^{m_2''}(C_1 \land C_2)) \subseteq V(\Gamma_1') \cap V(\Gamma_2' , \nabla^{n+1} (A \rightarrow B) , \Delta)$.

			\item If $\Gamma_1 = \Gamma_1' , \nabla^{n+1} (A \rightarrow B)$ and $\Gamma_2 = \Gamma_2'$, let $C_1 = C_{\langle\Gamma_2';\Gamma_1';\nabla^n A\rangle}$ and $C_2 = C_{\langle\Gamma_1',\nabla^n B;\Gamma_2';\Delta\rangle}$.
			From IH we have $\nabla^{m_1'} \Gamma_1', \nabla^{m_1'} C_1 \Rightarrow \nabla^{n+m_1'} A$ and $\nabla^{m_1''} \Gamma_1',\nabla^{n+m_1''} B \Rightarrow C_2$. Like the previous case, we can first equalize their contexts using $N$, $L$ and $Lw$, and then apply a $L\rightarrow$ to get $\nabla^{m_1} \Gamma_1',\nabla^{n+m_1+1}(A \rightarrow B) , \nabla^{m_1} C_1 \Rightarrow C_2$ where $m_1 = m_1'+m_1''$. We can make sure that the context has at least one $\nabla$ using $L$, and then move $\nabla^{m_1} C_1$ to the right, using $R\rightarrow$, and have $\nabla^{m_1} \Gamma_1',\nabla^{n+m_1+1}(A \rightarrow B) \Rightarrow \nabla^{m_1} C_1 \rightarrow C_2$. Finally, we can apply $N$ to also add a $\nabla$ to the right side.
			From IH we also have $\nabla^{m_2'} \Gamma_2' \Rightarrow C_1$ and $\nabla^{m_2''} \Gamma_2' , C_2 \Rightarrow \Delta$. By propers applications of $N$ and $L$ turn these sequents to $\nabla^{m_1+m_2} \Gamma_2' \Rightarrow \nabla^{m_1} C_1$ and $\nabla^{m_1+m_2} \Gamma_2' , C_2 \Rightarrow \Delta$, where $m_2 = m_2'+m_2''$. Then by $L\rightarrow$ we have $\nabla^{m_1+m_2} \Gamma_2' , \nabla (\nabla^{m_1} C_1 \rightarrow C_2) \Rightarrow \Delta$. So we can take $C = \nabla (\nabla^{m_1} C_1 \rightarrow C_2)$. We also have from IH $V(C_1) \subseteq V(\Gamma_2') \cap V(\Gamma_1' , \nabla^n A)$ and $V(C_2) \subseteq V(\Gamma_1' , \nabla^n B) \cap V(\Gamma_2' , \Delta)$. Then $V(\nabla (C_1 \rightarrow C_2)) \subseteq V(\Gamma_1' , \nabla^{n+1} (A \rightarrow B)) \cap V(\Gamma_2' , \Delta)$.
		\end{enumerate}

		\item ($R\rightarrow$) $\mathcal{D}$ proves $\Gamma_1, \Gamma_2 \Rightarrow A \rightarrow B$ and has a sub-proof for $\nabla \Gamma_1, \nabla \Gamma_2, A \Rightarrow B$. Take $C = C_{\langle\nabla\Gamma_1;\nabla\Gamma_2,A;B\rangle}$. We would have $\nabla^{m_1+1}\Gamma_1 \Rightarrow C$ from IH and $\nabla^{m_2}\Gamma_2, C \Rightarrow A \rightarrow B$ from IH, $L$ and $R\rightarrow$. From IH, we also have $V(C) \subseteq V(\nabla\Gamma_1) \cap V(\nabla\Gamma_2,A,B)$. It is easy to see that this implies $V(C) \subseteq V(\Gamma_1) \cap V(\Gamma_2,A \rightarrow B)$.

		\item ($N$) $\mathcal{D}$ proves $\nabla \Gamma_1 , \nabla \Gamma_2 \Rightarrow \nabla \Delta$ and has a sub-proof for $\Gamma_1 , \Gamma_2 \Rightarrow \Delta$. Just take $C = \nabla C_{\langle\Gamma_1;\Gamma_2;\Delta\rangle}$ and apply $N$ on the sequents from IH. The variable condition is also trivial from IH.
		
		\item ($L$) $\mathcal{D}$ proves $\Gamma_1' , \Gamma_2' , \nabla A \Rightarrow \Delta$ and has a sub-proof for $\Gamma_1' , \Gamma_2' , A \Rightarrow \Delta$.
		\begin{enumerate}
			\item If $\Gamma_1 = \Gamma_1'$ and $\Gamma_2 = \Gamma_2' , A$, take $C = C_{\langle\Gamma_1';\Gamma_2',A;\Delta\rangle}$. Then we have $\nabla^{m_1} \Gamma_1' \Rightarrow C$ by IH and $\nabla^{m_2}\Gamma_2' , \nabla A \Rightarrow \Delta$ by IH and $L$. From IH, we can also deduce $V(C) \subseteq V(\Gamma_1') \cap V(\Gamma_2',\nabla A,\Delta)$, since $\nabla$ does not introduce new propositional variables.
			
			\item If $\Gamma_1 = \Gamma_1' , A$ and $\Gamma_2 = \Gamma_2'$, take $C = C_{\langle\Gamma_1',A;\Gamma_2';\Delta\rangle}$. Then we have $\nabla^{m_1}\Gamma_1', \nabla^{m_1+1}A \Rightarrow C$ by IH and $L$, and $\nabla^{m_2}\Gamma_2' \Rightarrow \Delta$ by IH. We also have $V(C) \subseteq V(\Gamma_1',A) \cap$ $V(\Gamma_2',\Delta)$ as justified before.
		\end{enumerate}

		\item ($R$) Assume $\Gamma_1 = \Pi_1, \Sigma_1$ and $\Gamma_2 = \Pi_2, \Sigma_2$. $\mathcal{D}$ proves $\Pi_1, \Sigma_1, \Pi_2, \Sigma_2 \Rightarrow \Delta$ and has a sub-proof for $\Pi_1, \nabla\Sigma_1, \Pi_2, \nabla\Sigma_2 \Rightarrow \Delta$.
		We can take $C_{\langle\Pi_1\nabla\Sigma_1;\Pi_2\nabla\Sigma_2;\Delta\rangle}$ as the desired interpolant $C$, because we have $\nabla^{m_1}\Pi_1,$ $\nabla^{m_1}\Sigma_1 \Rightarrow C$ and $\nabla^{m_2}\Pi_2, \nabla^{m_2}\Sigma_2, C \Rightarrow \Delta$ from IH, so it is also an interpolant for $\mathcal{D}$. We also have $V(C) \subseteq V(\Pi_1,\Sigma_1) \cap V(\Pi_2,\Sigma_2,\Delta)$, since $\nabla$ does not introduce new atomic formulas and we can drop it.

		\item ($Fa$) $\mathcal{D}$ proves $\Gamma_1 , \Gamma_2 \Rightarrow \nabla(A \rightarrow B)$ and has a sub-proof for $\Gamma_1 , \Gamma_2 , A \Rightarrow B$. Let $C = C_{\langle\Gamma_1;\Gamma_2,A;B\rangle}$. So we have $\nabla^{m_1}\Gamma_1 \Rightarrow C$ and $\nabla^{m_2}\Gamma_2 , C \Rightarrow \nabla (A \rightarrow B)$ from IH and an application of $Fa$.
		It is easy to deduce $V(C) \subseteq V(\Gamma_1) \cap V(\Gamma_2 , \nabla (A \rightarrow B))$ from IH.

		\item ($Fu$) $\mathcal{D}$ proves $\Gamma_1, \Gamma_2 \Rightarrow \Delta$ and has a sub-proof for $\nabla \Gamma_1, \nabla \Gamma_2 \Rightarrow \nabla \Delta$. Take $C = C_{\langle\nabla\Gamma_1;\nabla\Gamma_2;\nabla\Delta\rangle}$. We have $\nabla^{m_1+1}\Gamma_1 \Rightarrow C$ from IH. We also have $\nabla^{m_2+1} \Gamma_2, C \Rightarrow \nabla \Delta$ from IH. We can derive $\nabla^{m_2} \Gamma_2, C \Rightarrow \Delta$ by $L$ and $Fu$. The variable condition is also resulted from IH.
	\end{enumerate}
\end{proof}

\begin{cor}[Deductive Interpolation for $\stl$] For any $\Delta'$ and $\Delta$, if $\Rightarrow \Delta' \vdash\ \Rightarrow \Delta$ then there exists a formula $C$ such that
	\begin{enumerate}[label=(\arabic*)]
		\item $\Rightarrow \Delta' \vdash\ \Rightarrow C$,
		\item $\Rightarrow C \vdash\ \Rightarrow \Delta$ and
		\item $V(C) \subseteq V(\Delta') \cap V(\Delta)$.
	\end{enumerate}
\end{cor}
\begin{proof}
	In case $\Delta' = \{\}$, just take $C = \bot$. Now let's assume $\Delta' = A$ for some formula $A$. First, convert the proof-tree for $\Rightarrow A \vdash \Rightarrow \Delta$ to $\nabla^n A \Rightarrow \Delta$ for some $n$ using Lemma \ref{lem:vdash}, and then use Theorem \ref{thm:ldls-dedint} with $\Gamma_1 = \{\nabla^n A \}$ and $\Gamma_2 = \{\}$. Notice that we can translate results $(\mathit{1}^*)$ and $(\mathit{2}^*)$ from Theorem \ref{thm:ldls-dedint} back to $(\mathit{1})$ and $(\mathit{2})$ respectively, using the other direction of Lemma \ref{lem:vdash}.
\end{proof}

\subsection{Lyndon Interpolation}

We can easily check that our previous construction of interpolants for both $\istl$ and $\stl$ also satisfy Lyndon's definition \cite{Lyndon1959AnIT}. First, we need the following definition. The notion of the \emph{polarity} for an occurence of a subformula in some formula, simply tells if the subformula is in the antecedent of some implication for an even number of times or not.

\begin{dfn}\label{dfn:polarity}
  Suppose $A$ is formula in the language $\mathcal{L}$ and $B$ is a subformula of $A$. Define the \emph{polarity} of an occurence of $B$ in some occurence of $A$ by mutual induction as follows.
  \begin{enumerate}
    \item If $A = B$ then $B$ has \emph{positive} polarity, of simply, is \emph{positive} in any positive occurence of $A$, and has \emph{negative} polarity, or simply, is \emph{negative} in any negative occurence of $A$.
    \item If $A$ is of the form $B \rightarrow C$, then $B$ is \emph{negative} in any positive occurence of $A$, and is \emph{positive} in any negative occurence of $A$.
    \item If $A$ is of the form $B \wedge C$, $C \wedge B$, $B \vee C$, $C \vee B$, $C \rightarrow B$ or $\nabla B$, then $B$ is \emph{positive} in any positive occurence of $A$, and is \emph{negative} in any negative occurence of $A$.
  \end{enumerate}

  We will denote the set of all atomic formulas with some positive occurrence in a formula $A$ by $V^+(A)$, and the set of all atomic formulas with some negative occurrence in $A$ by $V^-(A)$
\end{dfn}

So, for example, we have $p \in V^+((p \rightarrow q) \rightarrow r)$ and $p \in V^-((p \rightarrow q) \wedge r)$.

\begin{thm}[Lyndon's interpolation for $\stl$]\label{thm:stl-lyndon}
  {\color{red} Probably has a flaw!}
  The formula $C$ that was constructed as an interpolant for $\Delta'$ and $\Delta$ in Corollary \ref{cor:stl-dedint} has also this stronger property: $V^+(C) \subseteq V^+(\Delta') \cap V^+(\Delta)$ and $V^-(C) \subseteq V^-(\Delta') \cap V^-(\Delta)$.
\end{thm}
\begin{proof}
  It suffices to check that in the proof of Theorem \ref{thm:stl-dedint}, this stronger property is preserved by our constructions of $C$, in the cases that are used to deduce Corollary \ref{cor:stl-dedint}, which are the cases where $\Gamma_2$ could be empty.
\end{proof}

\section{Conclusion and future works}

\section{Acknowledgement}

\section{Appendix}
% \subsection{Lattice of extensions of $\mathbf{STL}$}
% \[\begin{tikzcd}
	&& IPC \\
	Fu & L && R & Fa \\
	\\
	&& STL
	\arrow[from=4-3, to=2-1]
	\arrow[from=2-4, to=1-3]
	\arrow[from=2-2, to=1-3]
	\arrow[from=4-3, to=2-2]
	\arrow[from=4-3, to=2-4]
	\arrow[from=4-3, to=2-5]
\end{tikzcd}\]

\subsection{Detailed proofs}

\bibliographystyle{abbrv}
\bibliography{refs}
\end{document}
