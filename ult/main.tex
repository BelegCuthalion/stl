\documentclass[12pt,a4paper]{article}
\usepackage{amsmath}
\usepackage{amsthm}
\usepackage{amsfonts}
%\usepackage{fdsymbol}

\usepackage{paracol}

\usepackage{authblk}

\usepackage[mathscr]{eucal}

\usepackage{bussproofs}
\EnableBpAbbreviations

\usepackage{amssymb}
\usepackage{tikz}
\usepackage{enumitem}
\usepackage{multicol}
\tikzset{node distance=2cm, auto}

\theoremstyle{plain}
\newtheorem{thm}{Theorem}[section]

\renewcommand{\thethm}{\arabic{section}.\arabic{thm}}
\newtheorem{lem}[thm]{Lemma}

\newtheorem{cor}[thm]{Corollary}
\theoremstyle{definition}
\newtheorem{dfn}[thm]{Definition}
\newtheorem{exam}[thm]{Example}
\newtheorem{rem}[thm]{Remark}
\newtheorem{nota}[thm]{Notation}
\newtheorem{exer}[thm]{Exercise}

\def\d{\displaystyle}
\def\PA{\mathrm{PA}}
\def\Pr{\mathrm{Pr}}
\def\Prf{\mathrm{Prf}}
\def\PR{\mathrm{PR}}
\def\IPC{\mathrm{IPC}}
\def\Proofs{\mathrm{Proofs}}
\def\int{\mathrm{int}}
\def\WT{\mathrm{WT}}
\def\exp{\mathrm{exp}}
\def\CHaus{\mathrm{CHaus}}
\def\Fin{\mathrm{Fin}}
\def\E{\mathrm{E}}
\def\PR{\mathrm{PR}}
\def\Top{\mathrm{Top}}
\def\S4{\mathrm{S4}}
\def\Hom{\mathrm{Hom}}
\def\Set{\mathrm{Set}}

\newcommand{\stl}{\mathbf{STL}}
\newcommand{\gstl}{\mathbf{GSTL}}
\newcommand{\istl}{\mathbf{iSTL}}
\newcommand{\igstl}{\mathbf{iGSTL}}

\begin{document}
    
\title{Interpolation for Logics of Spacetime}
\author[]{Amirhossein Akbar Tabatabai, Majid Alizadeh, Alireza Mahmoudian}
\affil[]{ }
\date{March 2, 2022}
\maketitle

\begin{abstract}
	A generalization of the notion of implication is introduced in \cite{amir}, as a binary operator which internalizes the reflexive and transitive properties of the provability order on the set of all propositions. This generalization gives rise to a classification of some sub-intuitionistic and sub-structural logics with various behaviors of their respecting implication operators. One of its specifications, which is called the logic of \emph{spacetime} in \cite{amir} makes a dynamic reading of the Proof Interpretation for conditional statements possible. This reading does not suffer from the well-known impredicativity in the BHK interpretation.

In this paper, we will introduc a sequent-style calculus for the logic of spacetime, called $\stl$ along with an equivalent cut-free system, called $\gstl$. Then we will study some properties of thses systems such as subformula property, disjuction property, admissibility of the Visser's rule and existence of imterpolants.
\end{abstract}

\section{Introduction}
The usual interpretation of an intuitionistic proof for an implication is a structure that converts \emph{all} proofs of its antecedent to some proof for its consequent. This interpretation is impredicative in the sense that we are quantifying over the set of all proofs, while we are just defining some members of it, which are, the proofs for implication. The impredicativity of implication shows up more clearly when one is using a conditional assumption in an intuitionistic proof, which amounts to an application of the rule of \emph{modus ponens}. What this rule essentially says is that if we have a proof for an implication and a proof of its antecedent at the same time, then we can conclude its consequent.

The logic of spacetime remedies this by restricting the use of conditional assumption; So that the modus ponens rule in the logic of spacetime would not take a proof of conditional statement itself, but a proof of a modal operator, called $\nabla$, applied to the conditional statement. We can read this application of $\nabla$ as a \emph{delay} in use of implication, and interpret the proof of conditional statement as a function that converts all \emph{later} proofs of its antecedent to some \emph{later} proof for its consequent. So we can prove the consequent of a conditional statement, only if we have proved the conditional statement itself \emph{sooner} than the antecedent.\\


\emph{The logic of spacetime} is defined in \cite{amir} as the logic of an algebraic structure called \emph{spacetime}, which consists of a local along with a unary operator $\nabla$, which preserving the localic structure and arbitrary joins. It is shown that any spacetime has a binary operator $\rightarrow$ which satisfies the following equivalence.
$$a \wedge \nabla b \leq c \iff b \leq a \rightarrow c$$

The logic of spacetime is represented by a sequent-style system called $\stl$ (known as $\istl(N)$\footnote{Note that in this paper $\istl$ is reserved for another system with intuitionistic implication.} in \cite{amir}), which has the same rules as the usual intuitionistic calculus $\mathbf{LJ}$, over the propositional language augumented with a unary connective $\nabla$, but with left and right implication rules replaced with the following rules, respectively.

\begin{multicols}{2}
	\begin{prooftree}
		\AXC{$\Gamma \Rightarrow A$}
		\AXC{$\Gamma, B \Rightarrow \Delta$}
		\RightLabel{$L \rightarrow$}
		\BIC{$\Gamma, \nabla (A \rightarrow B) \Rightarrow \Delta$}
	\end{prooftree}
	\columnbreak
	\begin{prooftree}
		\AXC{$\nabla \Gamma, A \Rightarrow B$}
		\RightLabel{$R \rightarrow$}
		\UIC{$\Gamma \Rightarrow A \rightarrow B$}
	\end{prooftree}
\end{multicols}
Furthermore, the system also has a rule to introduce $\nabla$ to both sides of a sequent.
\begin{prooftree}
	\AXC{$\Gamma \Rightarrow A$}
	\RightLabel{$N$}
	\UIC{$\nabla \Gamma \Rightarrow \nabla A$}
\end{prooftree}
So, generally, $\nabla (A \rightarrow B)$, and not $A \rightarrow B$, will be of use in proving $B$ from $A$.

In the next section of this paper, we will introduce a sequent style calculus for the spacetime logic, called $\stl$ and introduce an equivalent cut-free system $\gstl$, and then survey some extensions and their respecting semantics, including an extension of $\stl$ with a Heyting implication, called $\istl$, and simillarly, an equivalent cut-free system $\igstl$.
In the third section, the cut-elimination theorems for $\stl$ and its extensions will be proved. Then we will use this theorem to deduce important results about $\stl$ and its extensions, such as subformula property, disjuction property and admissibility of the Visser's rule.
And in the last section, we will show that some extensions of $\stl$ have interpolation property.

\section{Sequent calculi for the logics of spacetime}
In this section, will define the systems $\stl$ (defined as $\istl(N)$ in \cite{amir}) and $\gstl$, and show that these two systems are equivalent.

% https://tikzcd.yichuanshen.de/#N4Igdg9gJgpgziAXAbVABwnAlgFyxMJZAFgBoBWAXVJADcBDAGwFcYkQBlAFQBkQBfUuky58hFAE5SAZmp0mrdgDFmAoSAzY8BIgAYZchizaIQS+muFaxRAGwGaRxaYBKAfQA6HnBDSWNItriyAAcDvLG7O5eAEYQOP6aojooAOykAEyGCiYg7sBePmix8fyJgTZppACM2ZGu5dYpyNU1dc4gfIJWycHkme25LsSNvUTSA4457Dwj3QFNwWS1U-Wdw6NBRGS6g+wAkgAKAMICcjBQAObwRKAAZgBOEAC2SPogPkgTIHAAFlh3BKIXTzR4vN40T6IYigp6vYGQiBIaqw8EIj5IxAZVHw95QkI4r6IpDkQnQ4mIUnqMHw-oYpC2Ml4zESJkU1JkqT0xDVEHUuFIMLc3lk9LClH8tH2cX8Sj8IA
\begin{tikzcd}
	&                &  &               & IPC                                                                                                      &               &                   &                        &                   &    \\
	& L \arrow[rrru] &  &               & LR4                                                                                                      &               &                   & R \arrow[lllu]         &                   &    \\
	&                &  & L4 \arrow[ru] &                                                                                                          & R4 \arrow[lu] &                   & R_{\top\bot} \arrow[u] &                   &    \\
Fa &                &  &               &                                                                                                          &               & R_\top \arrow[ru] &                        & R_\bot \arrow[lu] & Fu \\
	&                &  &               &                                                                                                          &               &                   &                        &                   &    \\
	&                &  &               & STL \arrow[rruu] \arrow[rrrruu] \arrow[rrrrruu] \arrow[lllluu] \arrow[ruuu] \arrow[luuu] \arrow[llluuuu] &               &                   &                        &                   &   
\end{tikzcd}

Now, we will introduce $\gstl$, which we then show to be equivalent with $\stl$, and also cut-free.

\begin{dfn}\label{dfn:gstl}
	Define $\gstl$ over the language $\mathcal{L}$, to be the logic of sequent-style system defined by the same rules as $\stl$, except that the rules $Ex$, $L \wedge_1$, $L \wedge_2$, $L\vee$, $L \rightarrow$ and $N$ are replaced by the following generalized rules respectively.
\end{dfn}

\begin{prooftree}
	\AXC{}
	\RightLabel{$Ex$}
	\UIC{$\nabla^n \bot \Rightarrow$}
\end{prooftree}

\begin{multicols}{2}
	\begin{prooftree}
		\AXC{$\Gamma, \nabla^n A \Rightarrow \Delta$}
		\RightLabel{$L \wedge_1$}
		\UIC{$\Gamma, \nabla^n (A \wedge B) \Rightarrow \Delta$}
	\end{prooftree}
	\columnbreak
	\begin{prooftree}
		\AXC{$ \Gamma, \nabla^n B \Rightarrow \Delta$}
		\RightLabel{$L \wedge_2$}
		\UIC{$\Gamma, \nabla^n (A \wedge B) \Rightarrow \Delta$}		
	\end{prooftree}
\end{multicols}

\begin{prooftree}
	\AXC{$\Gamma, \nabla^n A \Rightarrow \Delta$}
	\AXC{$\Gamma, \nabla^n B \Rightarrow \Delta$}
	\RightLabel{$L \vee$}
	\BIC{$\Gamma, \nabla^n (A \vee B) \Rightarrow \Delta$}
\end{prooftree}

\begin{prooftree}
	\AXC{$\Gamma \Rightarrow \nabla^n A$}
	\AXC{$\Gamma, \nabla^n B \Rightarrow \Delta$}
	\RightLabel{$L \rightarrow$}
	\BIC{$\Gamma, \nabla^{n+1} (A \rightarrow B) \Rightarrow \Delta$}
\end{prooftree}	
Notice that all the rules above are just generalizations of their corresponding rules in $\stl$.\\

To show the equivalnce of two systems, we will use the following lemmas. Notice that the proof-trees are constructible in $\stl(S)$ for $S \subseteq \{ L, R, Fa, Fu \}$.

\begin{lem}\label{lem:l-nabla-dist-and} For all $n \geq 0$, $\stl(S) \vdash \nabla^n (A \wedge B) \Rightarrow \nabla^n A \wedge \nabla^n B$.
\end{lem}
\begin{proof}\quad
	\begin{prooftree}
		\AXC{}
		\RightLabel{$Id$}
		\UIC{$A \Rightarrow A$}
		\RightLabel{$L \wedge_1$}
		\UIC{$A \wedge B \Rightarrow A$}
		\RightLabel{$N$} \doubleLine
		\UIC{$\nabla^n (A \wedge B) \Rightarrow \nabla^n A$}

		\AXC{}
		\RightLabel{$Id$}
		\UIC{$B \Rightarrow B$}
		\RightLabel{$L \wedge_2$}
		\UIC{$A \wedge B \Rightarrow B$}
		\RightLabel{$N$} \doubleLine	
		\UIC{$\nabla^n (A \wedge B) \Rightarrow \nabla^n B$}
		
		\RightLabel{$R \wedge$}
		\BIC{$\nabla^n (A \wedge B) \Rightarrow \nabla^n A \wedge \nabla^n B$}
	\end{prooftree}
\end{proof}

\begin{lem} \label{lem:l-nabla-box} $\stl(S) \vdash \nabla (\top \rightarrow A) \Rightarrow A$.
\end{lem}
\begin{proof}\quad
	\begin{prooftree}
		\AXC{}
		\RightLabel{$Ta$}
		\UIC{$\Rightarrow \top$}
	
		\AXC{}
		\RightLabel{$Id$}
		\UIC{$A \Rightarrow A$}
	
		\RightLabel{$L \rightarrow$}
		\BIC{$\nabla (\top \rightarrow A) \Rightarrow A$}
	\end{prooftree}
\end{proof}

\begin{lem}\label{lem:l-box-nabla} $\stl(S) \vdash A \Rightarrow \top \rightarrow \nabla A$.
\end{lem}
\begin{proof}\quad
	\begin{prooftree}
		\AXC{}
		\RightLabel{$Id$}
		\UIC{$\nabla A \Rightarrow \nabla A$}
		
		\RightLabel{$Lw$}
		\UIC{$\nabla A , \top \Rightarrow \nabla A$}

		\RightLabel{$R \rightarrow$}
		\UIC{$A \Rightarrow \top \rightarrow \nabla A$}
	\end{prooftree}
\end{proof}

\begin{lem}\label{lem:l-nabla-dist-or} $\stl(S) \vdash \nabla (A \vee B) \Rightarrow \nabla A \vee \nabla B$.
\end{lem}
\begin{proof} Let $\mathcal{D}$ be a proof-tree we can construct for $\nabla (\top \rightarrow (\nabla A \vee \nabla B)) \Rightarrow \nabla A \vee \nabla B$ from Lemma \ref{lem:l-nabla-box}.

	\begin{prooftree}
		\AXC{}
		\RightLabel{$Id$}
		\UIC{$\nabla A \Rightarrow \nabla A$}
		\RightLabel{$R\vee_1$}
		\UIC{$\nabla A \Rightarrow \nabla A \vee \nabla B$}
		\RightLabel{$Lw$}
		\UIC{$\nabla A , \top \Rightarrow \nabla A \vee \nabla B$}
		\RightLabel{$R \rightarrow$}
		\UIC{$A \Rightarrow \top \rightarrow (\nabla A \vee \nabla B)$}

		\AXC{}
		\RightLabel{$Id$}
		\UIC{$\nabla B \Rightarrow \nabla B$}
		\RightLabel{$R\vee_2$}
		\UIC{$\nabla B \Rightarrow \nabla A \vee \nabla B$}
		\RightLabel{$Lw$}
		\UIC{$\nabla B , \top \Rightarrow \nabla A \vee \nabla B$}
		\RightLabel{$R \rightarrow$}
		\UIC{$B \Rightarrow \top \rightarrow (\nabla A \vee \nabla B)$}

		\RightLabel{$L\vee$}
		\BIC{$A \vee B \Rightarrow \top \rightarrow (\nabla A \vee \nabla B)$}
		\RightLabel{$N$}
		\UIC{$\nabla (A \vee B) \Rightarrow \nabla (\top \rightarrow (\nabla A \vee \nabla B))$}

		\AXC{$\mathcal{D}$}

		\RightLabel{$Cut$}
		\BIC{$\nabla (A \vee B) \Rightarrow \nabla A \vee \nabla B$}
	\end{prooftree}
\end{proof}

\begin{lem}\label{lem:l-nabla-n-dist-or} For all $n \geq 0$, $\stl(S) \vdash \nabla^n (A \vee B) \Rightarrow \nabla^n A \vee \nabla^n B$.
\end{lem}
\begin{proof} The proof is trivial when $n = 0$. Let $\mathcal{D}_1$ be the proof-tree of Lemma \ref{lem:l-nabla-dist-or}, which proves the case for $n = 1$. For any $n > 1$ we have

	\begin{prooftree}
		\AXC{$\mathcal{D}_{n-1}$}
		\noLine
		\UIC{$\nabla^{n-1} (A \vee B) \Rightarrow \nabla^{n-1} A \vee \nabla^{n-1} B$}
		\RightLabel{$N$}
		\UIC{$\nabla^n (A \vee B) \Rightarrow \nabla (\nabla^{n-1} A \vee \nabla^{n-1} B)$}

		\AXC{$\mathcal{D}_1$}
		\noLine
		\UIC{$\nabla (\nabla^{n-1} A \vee \nabla^{n-1} B) \Rightarrow \nabla^n A \vee \nabla^n B$}
		
		\RightLabel{$Cut$} \LeftLabel{$\mathcal{D}_n:$}
		\BIC{$\nabla^n (A \vee B) \Rightarrow \nabla^n A \vee \nabla^n B$}
	\end{prooftree}
\end{proof}

\begin{lem}\label{lem:l-nabla-bot} $\stl(S) \vdash \nabla \bot \Rightarrow \bot$.
\end{lem}
\begin{proof} Let $\mathcal{D}$ be a proof-tree for $\nabla (\top \rightarrow \bot) \Rightarrow \bot$ which we have by Lemma \ref{lem:l-nabla-box}.
	\begin{prooftree}
		\AXC{}
		\RightLabel{$Ex$}
		\UIC{$\bot \Rightarrow$}
		\RightLabel{$Rw$}
		\UIC{$\bot \Rightarrow \top \rightarrow \bot$}
		\RightLabel{$N$}
		\UIC{$\nabla \bot \Rightarrow \nabla (\top \rightarrow \bot)$}

		\AXC{$\mathcal{D}$}

		\RightLabel{$Cut$}
		\BIC{$\nabla \bot \Rightarrow \bot$}
	\end{prooftree}	
\end{proof}

\begin{lem}\label{lem:l-nabla-n-bot} For $n > 0$, $\stl(S) \vdash \nabla^n \bot \Rightarrow \bot$.
\end{lem}
\begin{proof} We will prove a stronger version: For $n \geq m > 0$, $\stl(S) \vdash \nabla^n \bot \Rightarrow \nabla^{n-m} \bot$. Let $\mathcal{D}_1$ be the proof-tree of Lemma \ref{lem:l-nabla-bot} which handles $n = m = 1$. Using induction on $m$, and denoting the proof-tree for $\nabla^n \bot \Rightarrow \nabla^{n-(m-1)} \bot$ from the induction hypothesis by IH, we have for $n > 1$
	\begin{prooftree}
		\AXC{IH}
		\noLine
		\UIC{$\nabla^n \bot \Rightarrow \nabla^{n-(m-1)} \bot$}

		\AXC{$\mathcal{D}_1$}
		\noLine
		\UIC{$\nabla \bot \Rightarrow \bot$}
		\doubleLine \RightLabel{$N^{(n-m)}$}
		\UIC{$\nabla^{n-(m-1)} \bot \Rightarrow \nabla^{n-m} \bot$}

		\RightLabel{$Cut$}
		\BIC{$\nabla^n \bot \Rightarrow \nabla^{n-m} \bot$}
	\end{prooftree}
\end{proof}

\begin{lem}\label{lem:l-nabla-dist-si} For any $n \geq 0$, $\stl(S) \vdash \nabla^n (A \rightarrow B) \Rightarrow \nabla^n A \rightarrow \nabla^n B$.
\end{lem}
\begin{proof}\quad
	\begin{prooftree}
		\AXC{}
		\RightLabel{$Id$}
		\UIC{$A \Rightarrow A$}
		
		\AXC{}
		\RightLabel{$Id$}
		\UIC{$B \Rightarrow B$}
		\RightLabel{$Lw$}
		\UIC{$A , B \Rightarrow B$}
		
		\RightLabel{$L \rightarrow$}
		\BIC{$\nabla (A \rightarrow B) , A \Rightarrow B$}
		\RightLabel{$N^{(n)}$} \doubleLine
		\UIC{$\nabla^{n+1} (A \rightarrow B) , \nabla^n A \Rightarrow \nabla^n B$}
		\RightLabel{$R \rightarrow$}
		\UIC{$\nabla^n (A \rightarrow B) \Rightarrow \nabla^n A \rightarrow \nabla^n B$}
	\end{prooftree}
\end{proof}


In the rest of this paper, by the \emph{height of a proof-tree} $\mathcal{D}$, denoted by $h(\mathcal{D})$, we mean the number of rule instances in its longest branch.

The theorem below shows that $\stl$ (and its extensions) deduce exactly the same sequents as $\gstl$ (and its extensions).

\begin{thm}\label{thm:stl-eq-gstl}
	Let $S \subseteq \{L, R, Fa, Fu\}$. For any sequent $\Gamma \Rightarrow \Delta$ in the language $\mathcal{L}$, $\stl(S) \vdash \Gamma \Rightarrow \Delta$ iff $\gstl(S) \vdash \Gamma \Rightarrow \Delta$.
\end{thm}
\begin{proof}
	One direction easily follows from the fact that all rules of $\stl(S)$ are just instances of $\gstl(S)$'s rules.
	For the other direction, we will use case analysis for the last rule in the proof-tree of $\Gamma \Rightarrow \Delta$ in $\gstl(S)$, which we call $\mathcal{D}$, and construct a proof-tree for it in $\stl(S)$ in each case.
	
	First, observe that $Id$ and $Ta$ are present in $\stl(S)$ and Lemma \ref{lem:l-nabla-n-bot} handles the $Ex$ case.
	For the other rules, use induction on the length of $\mathcal{D}$; the induction hypothesis will provide a proof-tree in $\stl(S)$ for the subtree(s) of $\mathcal{D}$.
	For the rules that are common between two systems, just apply the same rule (in $\stl(S)$) on the proof-tree(s) from the induction hypothesis to reach the desired sequent. For example, if $\mathcal{D}$ ends with $R \vee$, it suffices to apply $R \vee$ (of $\stl(S)$) on the proof-tree that we get from the induction hypothesis.
	In the cases for $\gstl(S)$'s stronger rules, which are $L \wedge_1$, $L \wedge_2$, $L \vee$ and $L \rightarrow$, do the same, and also cut the sequents proved in Lemmas \ref{lem:l-nabla-dist-and}, \ref{lem:l-nabla-dist-or} or \ref{lem:l-nabla-dist-si} into the resulting sequent.
	The case for $N$ divides into two further cases. One case is when $\Delta = A$ for some formula $A$, which is again handled by applying $N$ on the sequent from the induction hypothesis.
	The other case is when $\Delta = \{\}$, so we have $\Gamma \Rightarrow$ from the induction hypothesis, on the right of which we first introduce $\bot$ using $Rw$, and then apply $N$.
	Then we can cut it into the sequent proved in Lemma \ref{lem:l-nabla-bot}, and then into $Ex$, to derive $\nabla \Gamma \Rightarrow$ as was desired.
\end{proof}

% https://tikzcd.yichuanshen.de/#N4Igdg9gJgpgziAXAbVABwnAlgFyxMJZAFgBoBWAXVJADcBDAGwFcYkQBlAFQBkQBfUuky58hFAE5SAZmp0mrdgDFmAoSAzY8BIgAYZchizaIQS+muFaxRAGwGaRxaYBKAfQA6HnBDSWNItriyAAcDvLG7O5eAEYQOP6aojooAOykAEyGCiYg7sBePmix8fyJgTZppACM2ZGu5dYpyNU1dc4gfIJWycHkme25LsSNvUTSA4457Dwj3QFNwWS1U-Wdw6NBRGS6g+wAkgAKAMICcjBQAObwRKAAZgBOEAC2SPogPkgTIHAAFlh3BKIXTzR4vN40T6IYigp6vYGQiBIaqw8EIj5IxAZVHw95QkI4r6IpDkQnQ4mIUnqMHw-oYpC2Ml4zESJkU1JkqT0xDVEHUuFIMLc3lk9LClH8tH2cX8Sj8IA
\begin{tikzcd}
	&                &  &               & IPC                                                                                                      &               &                   &                        &                   &    \\
	& L \arrow[rrru] &  &               & LR4                                                                                                      &               &                   & R \arrow[lllu]         &                   &    \\
	&                &  & L4 \arrow[ru] &                                                                                                          & R4 \arrow[lu] &                   & R_{\top\bot} \arrow[u] &                   &    \\
Fa &                &  &               &                                                                                                          &               & R_\top \arrow[ru] &                        & R_\bot \arrow[lu] & Fu \\
	&                &  &               &                                                                                                          &               &                   &                        &                   &    \\
	&                &  &               & STL \arrow[rruu] \arrow[rrrruu] \arrow[rrrrruu] \arrow[lllluu] \arrow[ruuu] \arrow[luuu] \arrow[llluuuu] &               &                   &                        &                   &   
\end{tikzcd}

\section{Cut-elimination theorem}
Our proof of the admissibility of $cut$ in $\gstl(S)$ does not imply its admissibility in $\stl(S)$. But, as we will see, their equivalence (by theorem \ref{thm:stl-eq-gstl}) will allow us to use proof-theoretic methods for the cut-free system, and then translate its results back into a statement about the original system.

For technical reasons, we will work with a generalized form of the $cut$ rule, which nevertheless satisfies our goal here.

\begin{dfn}[$\nabla cut$ rule]\label{def:n-cut}
	Let $S \subseteq \{L, R, Fa, Fu\}$. We denote by $\gstl^+(S)$ the same systems defined as $\gstl(S)$, except that the $cut$ rule is replaced by the following generalization:
	\begin{prooftree}
		\AXC{$\Gamma \Rightarrow \nabla^m A$}
		\AXC{$\Sigma , \{\nabla^{n_i} A\}_{i \leq l} \Rightarrow \Delta$}
		\RightLabel{$\nabla cut$}
		\BIC{$\{\nabla^{n_i} \Gamma\}_{i \leq l} , \nabla^m \Sigma \Rightarrow \nabla^m \Delta$}
	\end{prooftree}
	where $\nabla^n$ means $\nabla$ applied $n$ times, $\{\nabla^{n_i} A\}_{i \leq l}$ means $\{\nabla^{n_0} A,\dotsb, \nabla^{n_l} A\}$ and $\{\nabla^{n_i} \Gamma\}_{i \leq l}$ means $\bigcup_{A \in \Gamma} \{\nabla^{n_i} A\}_{i \leq l}$ for some finite sequence of natural numbers $\{n_i\}_{i \leq l}$ of length $l+1$.
	We will refer to the set $\{\nabla^m A\} \cup \{\nabla^{n_i} A\}_{i \leq l}$ as the \emph{cut-burden} of this $cut$ instance. We will also call the tuple $(A, m, \{n_i\}_{i \leq l})$ the \emph{cut-data} of such instance.
\end{dfn}

\begin{thm}\label{cor:nc-riddance} Any sequent provable in $\gstl(S)$ is also provable in $\gstl^+(S)$ (for $S \subseteq \{L, R, Fa, Fu\}$).
\end{thm}
\begin{proof}
	We can just replace any instance of $cut$ with a cut-formula $A$ with similar instance of $\nabla cut$ with cut-data $(A, 0, \{0\})$.
\end{proof}

Next, we need to define a measure for the complexity of the $\nabla cut$ rule. We will first define this measure for formulas, and then extend it to rule instances and proof-trees.

\begin{dfn}[Rank]\label{dfn:rank}
	Rank of a formula $A$ is defined as
	\[ \rho(A) = \begin{cases}
	1 & \quad ; A \in P \cup \{ \bot, \top \} \\
	\rho(B) & \quad ; A = \nabla B \\
	max(\rho(B), \rho(C)) + 1 & \quad ; A = B \circ C \quad (\circ \in \{ \land, \lor, \rightarrow \})
	\end{cases} \]
	Notice that $\nabla$ does not increase the rank of a formula.
	
	We also define rank for rule instances and proof-trees as follows. Rank of an instance of the $\nabla cut$ rule with cut-data $(A, \{n_i\}_{i \leq l})$ is defined to be the rank of $A$. Rank of any other rule instance is $0$.
	For a proof tree $\mathcal{D}$, $\rho(\mathcal{D})$ is the maximum rank of all of its rule instances.
\end{dfn}

The next lemma will help in handling of a case in the main theorem.

\begin{lem}\label{lem:gstl-top-redundant} If $\gstl^+(S)$ proves $\Gamma , \{\nabla^{n_i} \top\}_{i \leq l} \Rightarrow \Delta$, then it also proves $\Gamma \Rightarrow \Delta$ with a proof-tree of at most the same rank (for $S \subseteq \{L, R, Fa, Fu\}$).
\end{lem}
\begin{proof}
Suppose $\mathcal{D}$ is a proof-tree for $\Gamma , \{\nabla^{n_i} \top\}_{i \leq l} \Rightarrow \Delta$ in $\gstl^+(S)$ and consider different cases for the last rule of $\mathcal{D}$ with possible subtrees $\mathcal{D}_0$ and $\mathcal{D}_1$.
By induction on $\mathcal{D}$ we can assume that the theorem holds for $\mathcal{D}_0$ and $\mathcal{D}_1$.
First, observe that $Ta$ and $Ex$ cases are trivially ruled out. In the cases for $Id$, which implies $l = 0$, we have $\Rightarrow \nabla^{n_0} \top$ by $n_0$ times applications of $N$ on $Ta$. In $Lw$ case, if $l = 0$ and $\nabla^{n_0} \top$ is principal, $\mathcal{D}_0$ itself proves the desired sequent, but if $l > 0$, then the induction hypothesis gives the desired sequent. The cases for $Lc$ or $L$ on some $\nabla^{n_j} \top$  are similar. In all other cases, just apply induction hypothesis on $\mathcal{D}_0$ (and possibly $\mathcal{D}_1$), and then apply the same last rule. Notice that $\nabla Cut$ is not used except in the case for $\nabla Cut$ itself, where it is applied with an instance of the same rank, so the resulting proof-tree will not be of a higher rank than that of $\mathcal{D}$.
\end{proof}

The following theorem shows that we can imitate any instance of the cut rule in a proof-tree of lower rank.

\begin{thm}\label{thm:gstl-cut-reduction}[cut Reduction]
  If $\gstl^+(S)$ proves $\Gamma \Rightarrow \nabla^m A$ and\\\ $\Sigma , \{\nabla^{n_i} A\}_{i \leq l} \Rightarrow \Delta$ with proof-trees of ranks less than $\rho(A)$, then it also proves $\{\nabla^{n_i} \Gamma\}_{i \leq l} , \nabla^m\Sigma \Rightarrow \nabla^m\Delta$ also with a proof tree of a rank less than $\rho(A)$ (for $S \subseteq \{L, R, Fa, Fu\}$).
  \end{thm}
  \begin{proof}
    We have two proof-trees
    \[
      {\mathcal{D}_0
      \atop
      \Gamma \Rightarrow \nabla^m A}
      \hspace{3em}
      {\mathcal{D}_1
      \atop
      \Sigma , \{\nabla^{n_i} A\}_{i \leq l} \Rightarrow \Delta}
    \]
    both of a lower rank than that of $A$, and we want to construct a proof-tree
    \[\mathcal{D} \atop \{\nabla^{n_i} \Gamma\}_{i \leq l} , \nabla^m \Sigma \Rightarrow \nabla^m \Delta \]
    without increasing the cut rank.
  
    The construction takes place in different cases for the last rule that occurs in $\mathcal{D}_0$ and $\mathcal{D}_1$. Notice that the proof of the theorem is essentially the same for any choice of $S$, where $S$ is a subset of $\{L, R, Fa, Fu\}$, modulo the cases that are specific to the rules in $S$. Thus, we will not repeat the cases which are common between all systems for the sake of brevity. Also notice that the resulting proof-tree in each case is constructed using the core system $\gstl^+$, plus the same rule of that case, so it will work for any extension containing that rule.
  
    In many cases, our construction would depend only on the last rule of one of the subtrees. So it would work no matter what the last rule in the other subtree is. Therefore, it will cover all the cases for the other subtree. These cases constitute the first two parts of the proof. In the third part, we will address the cases where our construction depends on the last rule in \emph{both} subtrees, which are the cases that the cut-burden is altered on both sides. In these cases, the last rule in one of the subtrees determines a specific form for the formulas in the cut-burden, which in turn determines the last rule in the other subtree.
  
    In first two groups of the cases, we will need induction on the height of one of the subtrees. But in the third part, we will use induction simultaneously on both $\mathcal{D}_0$ and $\mathcal{D}_1$, which goes as follows. For any two proof-trees $\mathcal{D}_0'$ and $\mathcal{D}_1'$ such that $h(\mathcal{D}_0') + h(\mathcal{D}_1') < h(\mathcal{D}_0) + h(\mathcal{D}_1)$, where $\mathcal{D}_0'$ proves $\Gamma' \Rightarrow \nabla^{m'} A'$ and $\mathcal{D}_1'$ proves $\Sigma', \{\nabla^{n'_i} A'\}_{i \leq l} \Rightarrow \Delta'$ for arbitrary $\Gamma'$, $\Sigma'$, $\Delta'$, $A'$, $m'$ and $n_i'$ of length $l'+1$, for which we have $\rho(\mathcal{D}_0'),\rho(\mathcal{D}_1') < \rho(A')$, the induction hypothesis gives us a prooftree, denoted by $\text{IH}(\mathcal{D}_0', \mathcal{D}_1')$ where it matters, that proves $\{\nabla^{n_i'}\Gamma'\}_{i \leq l'}, \nabla^{m'} \Sigma' \Rightarrow \nabla^{m'} \Delta'$, and we will also have $\rho(\text{IH}(\mathcal{D}_0', \mathcal{D}_1')) < \rho(A')$.
  
    \textbf{Part I.} First, assume that $\mathcal{D}_0$ is an axiom. No matter what would be the last rule instance in $\mathcal{D}_1$, the case for $Id$ is trivial, $Ex$ won't happen and $Ta$ is handled by Lemma \ref{lem:gstl-top-redundant}.
    Now assume that $\mathcal{D}_0$ ends with an instance of the rules $Lw$, $Lc$, $L \wedge_1$, $L \wedge_2$, $L \vee$, $L \rightarrow$, $\nabla cut$, $N$, $Fu$, $L$ or $R$. In all these cases---again, independent of $\mathcal{D}_1$---it suffices to use induction on the assumption(s) of this rule and $\mathcal{D}_1$ to remove the cut-burden from both subtrees. Then, we can apply the same rule to get the desired sequent. Here we will only mention the cases for $L \wedge_1$, $L \vee$, $L \rightarrow$, $\nabla cut$, $N$ and $Fu$, the last three of which may be of special concern, since they also alter the cut-burden. The other cases are similar.
  
    $L \wedge_1$: If $\mathcal{D}_0$ ends with $L \wedge_1$, that is
    \begin{prooftree}
      \noLine
      \AXC{$\mathcal{D}_0'$}
      \UIC{$\Gamma, \nabla^r B \Rightarrow \nabla^m A$}
      
      \RightLabel{$L \wedge_1$}
      \UIC{$\Gamma, \nabla^r (B \wedge C) \Rightarrow \nabla^m A$}
   \end{prooftree}
   then by applying $L \wedge_1$ on what we get from induction
   \begin{prooftree}
    \noLine
    \AXC{$\mathcal{D}_0'$}
    \UIC{$\Gamma, \nabla^r B \Rightarrow \nabla^m A$}
    
    \noLine
    \AXC{$\mathcal{D}_1$}
    \UIC{$\Sigma , \{\nabla^{n_i} A\}_{i \leq l} \Rightarrow \Delta$}
    
    \RightLabel{IH}
    \BIC{$\{\nabla^{n_i} \Gamma, \nabla^{n_i+r} B\}_{i \leq l}, \nabla^m \Sigma \Rightarrow \nabla^m \Delta$}
  
    \RightLabel{$L \wedge_1$} \doubleLine
    \UIC{$\{\nabla^{n_i} \Gamma, \nabla^{n_i+r} (B \wedge C)\}_{i \leq l}, \nabla^m \Sigma \Rightarrow \nabla^m \Delta$}
   \end{prooftree}
  
   \noindent $L \vee$: If $\mathcal{D}_0$ ends with $L \vee$
     \begin{prooftree}
       \noLine
       \AXC{$\mathcal{D}_0'$}
       \UIC{$\Gamma, \nabla^r B \Rightarrow \nabla^m A$}
       
       \noLine
       \AXC{$\mathcal{D}_0''$}
       \UIC{$\Gamma, \nabla^r C \Rightarrow \nabla^m A$}
       
       \RightLabel{$L \vee$}
       \BIC{$\Gamma, \nabla^r (B \vee C) \Rightarrow \nabla^m A$}
    \end{prooftree}
    Applying $L \vee$ on the sequents that we get from induction
    \begin{prooftree}
      \noLine
      \AXC{$\mathcal{D}_0'$}
      \UIC{$\Gamma, \nabla^r B \Rightarrow \nabla^m A$}
      
      \noLine
      \AXC{$\mathcal{D}_1$}
      \UIC{$\Sigma , \{\nabla^{n_i} A\}_{i \leq l} \Rightarrow \Delta$}
      
      \RightLabel{IH}
      \BIC{$\{\nabla^{n_i} \Gamma, \nabla^{n_i+r} B\}_{i \leq l}, \nabla^m \Sigma \Rightarrow \nabla^m \Delta$}
      
  
      \noLine
      \AXC{$\mathcal{D}_0''$}
      \UIC{$\Gamma, \nabla^r C \Rightarrow \nabla^m A$}
      
      \noLine
      \AXC{$\mathcal{D}_1$}
      \UIC{$\Sigma , \{\nabla^{n_i} A\}_{i \leq l} \Rightarrow \Delta$}
      
      \RightLabel{IH}
      \BIC{$\{\nabla^{n_i} \Gamma, \nabla^{n_i+r} C\}_{i \leq l}, \nabla^m \Sigma \Rightarrow \nabla^m \Delta$}
  
      \RightLabel{$L \vee$}
      \BIC{$\{\nabla^{n_i} \Gamma, \nabla^{n_i+r} (B \vee C)\}_{i \leq l}, \nabla^m \Sigma \Rightarrow \nabla^m \Delta$}
     \end{prooftree}
  
   
  \noindent $L \rightarrow$: Suppose $\mathcal{D}_0$ ends with a $L \rightarrow$ as shown below.
   \begin{prooftree}
    \noLine
    \AXC{$\mathcal{D}_0'$}
    \UIC{$\Gamma \Rightarrow \nabla^r B$}
    \noLine
    \AXC{$\mathcal{D}_0''$}
    \UIC{$\Gamma , \nabla^r C \Rightarrow \nabla^m A$}
    \RightLabel{$L \rightarrow$}
    \BIC{$\Gamma , \nabla^{r+1} (B \rightarrow C) \Rightarrow \nabla^m A$}
   \end{prooftree}
   Let $IH(\mathcal{D}_0'', \mathcal{D}_1)$ be called $\mathcal{D}'$.
   \begin{prooftree}
    \noLine
    \AXC{$\mathcal{D}_0''$}
    \UIC{$\Gamma , \nabla^r C \Rightarrow \nabla^m A$}
    \noLine
    \AXC{$\mathcal{D}_1$}
    \UIC{$\Sigma , \{\nabla^{n_i} A\}_{i \leq l} \Rightarrow \Delta$}
    \RightLabel{IH} \LeftLabel{$\mathcal{D}':~~~~$}
    \BIC{$\{\nabla^{n_i} \Gamma , \nabla^{n_i+r} C\}_{i \leq l} , \nabla^m \Sigma \Rightarrow \nabla^m \Delta$}
   \end{prooftree}
   In order to apply $L \rightarrow$, we must prepare the context in $\mathcal{D}_0'$, for each of $\nabla^{n_i+r}C$'s. Beginning with $j = 0$, first apply $N$ on $\mathcal{D}_0'$ $n_0$ times to get $\nabla^{n_0}\Gamma \Rightarrow \nabla^{n_0+r} B$. Then we can just add the rest of the context by $Lw$.
   \begin{prooftree}
    \noLine
    \AXC{$\mathcal{D}_0'$}
    \UIC{$\Gamma \Rightarrow \nabla^r B$}
    \doubleLine \RightLabel{$N$}
    \UIC{$\nabla^{n_0} \Gamma \Rightarrow \nabla^{n_0+r} B$}
    \doubleLine \RightLabel{$Lw$}
    \UIC{$\{\nabla^{n_i} \Gamma\}_{i \leq l}, \{\nabla^{n_i+r}C\}_{i \leq l}^{i \neq 0} , \nabla^m \Sigma \Rightarrow \nabla^{n_0+r} B$}
   \end{prooftree}
   Let the outcome of applying $L \rightarrow$ on this sequent and $\mathcal{D}'$ be called $\mathcal{D}'_{n_0}$:
   \[\mathcal{D}'_{n_0}:~~~~\{\nabla^{n_i} \Gamma\}_{i \leq l}, \{\nabla^{n_i+r}C\}_{i \leq l}^{i \neq 0}, \nabla^{n_0+r+1} (B \rightarrow C) , \nabla^m \Sigma \Rightarrow \nabla^m \Delta\]
   Now for all $0 < j \leq l$, we construct $\mathcal{D}_{n_j}$ similarly.
   \begin{prooftree}
    \noLine
    \AXC{$\mathcal{D}_0'$}
    \UIC{$\Gamma \Rightarrow \nabla^r B$}
    \doubleLine \RightLabel{$N$}
    \UIC{$\nabla^{n_j} \Gamma \Rightarrow \nabla^{n_j+r} B$}
    \doubleLine \RightLabel{$Lw$}
    \UIC{$\{\nabla^{n_i} \Gamma\}_{i \leq l}, \{\nabla^{n_i+r}C\}_{j < i \leq l}, \{ \nabla^{n_i+r+1} (B \rightarrow C) \}_{i < j}, \nabla^m \Sigma \Rightarrow \nabla^{n_j+r} B$}
   \end{prooftree}
   Applying $L \rightarrow$ on this sequent and $\mathcal{D}_{n_{j-1}}$ we would get
   \[\mathcal{D}'_{n_j}:~~~~\{\nabla^{n_i} \Gamma\}_{i \leq l}, \{\nabla^{n_i+r}C\}_{j < i \leq l}, \{ \nabla^{n_i+r+1} (B \rightarrow C) \}_{i \leq j}, \nabla^m \Sigma \Rightarrow \nabla^m \Delta\]
   $\mathcal{D}_{n_l}$ is exactly what we want:
   \[\{\nabla^{n_i} \Gamma, \nabla^{n_i+r+1}(B \rightarrow C)\}_{i \leq l}, \nabla^m \Sigma \Rightarrow \nabla^m \Delta\]
  
  
   $\nabla cut$: Assume $\mathcal{D}_0$ ends with a $\nabla cut$ with cut-data $(A', m', \{n_i'\}_{i \leq l'})$. Recall that by assumption, $A'$ must have a lower rank than $A$.
   \begin{prooftree}
     \noLine
     \AXC{$\mathcal{D}_0'$}
     \UIC{$\Gamma \Rightarrow \nabla^{m'} A'$}
     
     \noLine
     \AXC{$\mathcal{D}_0''$}
     \UIC{$\Pi , \{\nabla^{n_i'} A'\}_{i \leq l'} \Rightarrow \nabla^m A$}
     
     \RightLabel{$\nabla cut$}
     \BIC{$\{\nabla^{n_i'} \Gamma\}_{i \leq l'} , \nabla^{m'} \Pi \Rightarrow \nabla^{m+m'} A$}
   \end{prooftree}
   We must construct a proof-tree for $\{\nabla^{n_i + n_j'} \Gamma\}_{j \leq l'}^{i \leq l}, \{\nabla^{n_i+m'} \Pi\}_{i \leq l} , \nabla^{m+m'}\Sigma$ $\Rightarrow \nabla^{m+m'}\Delta$. We can use the induction hypothesis first to remove $A$, and then use a low rank $\nabla cut$ to remove $A'$.
   \begin{prooftree}
     \noLine
     \AXC{$\mathcal{D}_0'$}
     \UIC{$\Gamma \Rightarrow \nabla^{m'} A'$}
     
     \noLine
     \AXC{$\mathcal{D}_0''$}
     \UIC{$\Pi , \{\nabla^{n_i'} A'\}_{i \leq l'} \Rightarrow \nabla^m A$}
  
     \noLine
     \AXC{$\mathcal{D}_1$}
     \UIC{$\Sigma , \{\nabla^{n_i} A\}_{i \leq l} \Rightarrow \Delta$}
  
     \RightLabel{IH}
     \BIC{$\{\nabla^{n_i} \Pi\}_{i \leq l} , \{\nabla^{n_i + n_j'} A'\}_{j \leq l'}^{i \leq l} , \nabla^m \Sigma \Rightarrow \nabla^m \Delta$}
     
  
     \RightLabel{$\nabla cut$}
     \BIC{$\{\nabla^{n_i + n_j'} \Gamma\}_{j \leq l'}^{i \leq l}, \{\nabla^{n_i+m'} \Pi\}_{i \leq l} , \nabla^{m+m'}\Sigma$ $\Rightarrow \nabla^{m+m'}\Delta$}
   \end{prooftree}
  
   $Rw$: In this case, $\mathcal{D}_0'$ proves $\Gamma \Rightarrow$, so we can simply construct the desired proof-tree using $N$, $Lw$ and $Rw$.
   \begin{prooftree}
     \noLine
     \AXC{$\mathcal{D}_0'$}
     \UIC{$\Gamma \Rightarrow$}
     \doubleLine \RightLabel{$N$}
     \UIC{$\nabla^{n_0} \Gamma \Rightarrow$}
     \doubleLine \RightLabel{$Lw$}
     \UIC{$\{\nabla^{n_i} \Gamma\}_{i \leq l} , \nabla^m \Sigma \Rightarrow$}
     \RightLabel{$Rw$}
     \UIC{$\{\nabla^{n_i} \Gamma\}_{i \leq l} , \nabla^m \Sigma \Rightarrow \nabla^m \Delta$}
   \end{prooftree}
  
   $N$: $\mathcal{D}_0$ proves $\nabla \Gamma \Rightarrow \nabla^{m+1} A$ and $\mathcal{D}_1$ proves $\Sigma, \{\nabla^{n_i} A\}_{i<l} \Rightarrow \Delta$. There are two cases: The cut-data could be $(A, m+1, \{n_i\}_{i \leq l})$, or if for all $i \leq l$ we have $0 < n_i$, then the cut-data could also be $(\nabla A, m, \{n_i-1\}_{i \leq l})$. Induction hypothesis for $\mathcal{D}_0$'s immediate sub-tree and $\mathcal{D}_1$ gives us $\{\nabla^{n_i}\Gamma\}_{i \leq l}, \nabla^m \Sigma \Rightarrow \nabla^m \Delta$, which handles the latter case, and an application of $N$ on this sequent would handle the former.
  
   $Fu$: It suffices to apply $Fu$ on the result of the induction.\\
  
   \textbf{Part II.} The rest of the cases for $\mathcal{D}_0$ can't be solved independent of $\mathcal{D}_1$, so in the second part of the cases, we will consider the last rule of $\mathcal{D}_1$, again, where the solution could be constructed independent of $\mathcal{D}_0$. But this time we have less possibilities for the opposite subtree, since we've already solved most of them in the previous part of the proof. In fact the only possible rules as the last rule of $\mathcal{D}_0$ are now $R\star (\star \in \{\wedge, \vee_{1/2}, \rightarrow\})$ and $Fa$.
  
   Suppose $\mathcal{D}_1$ is an axiom. Again, the case for $Id$ is trivial, $Ta$ won't happen, and $Ex$ is also infeasible, since all possible cases for $\mathcal{D}_0$ alter the right side of the sequent, but none of them are able to introduce $\bot$ on the right side.
   In the remaining cases, if the cut-data is $(A, m, \{n_i\}_{i \leq l})$, in the cases where none of $\nabla^{n_i} A$'s are altered in the last rule of $\mathcal{D}_1$ modulo their number of $\nabla$'s, the construction is similar to the first part: Applying the same rule on the sequent that we get from the induction hypothesis. But if a member of the cut-burden is principal in the last rule of $\mathcal{D}_1$, which is to be handeld in the last part, we must also use the induction hypothesis for $\mathcal{D}_0$, both with a different cut-data. We now address the second part of the cases.
   
   For the sake of briefness, we will only explain the cases for $L \wedge_1$, $R \vee_1$, $R \rightarrow$ and $N$, the last two of which are of special concern, since we must use induction hypothesis with different $\{n_i\}_{i \leq l}$ in those cases. The rest would be handled similarly.
  
   $L \wedge$: Assume that $\mathcal{D}_1$ ends with $L \wedge_1$, but no member of the cut-burden is its principal formula.
   \begin{prooftree}
    \AXC{$\mathcal{D}_1'$} \noLine
    \UIC{$\Sigma, \{\nabla^{n_i} A\}_{i \leq l}, \nabla^r B \Rightarrow \Delta$}
    \RightLabel{$L \wedge_1$}
    \UIC{$\Sigma, \{\nabla^{n_i} A\}_{i \leq l}, \nabla^r (B \wedge C) \Rightarrow \Delta$}
   \end{prooftree}
   From induction hypothesis we have $\{\nabla^{n_i} \Gamma\}_{i \leq l}, \nabla^m \Sigma, \nabla^{r+m} B \Rightarrow \nabla^m \Delta$. By $L \wedge_1$ we have $\{\nabla^{n_i} \Gamma\}_{i \leq l}, \nabla^m \Sigma, \nabla^{r+m} (B \wedge C) \Rightarrow \nabla^m \Delta$.
  
   $R \vee_1$: Suppose that $\mathcal{D}_1$ ends with $R \vee_1$.
   \begin{prooftree}
    \AXC{$\mathcal{D}_1'$} \noLine
    \UIC{$\Sigma, \{\nabla^{n_i} A\}_{i \leq l} \Rightarrow \nabla^r B$}
    \RightLabel{$R \vee_1$}
    \UIC{$\Sigma, \{\nabla^{n_i} A\}_{i \leq l} \Rightarrow \nabla^r (B \vee C)$}
   \end{prooftree}
   Again, use the induction hypothesis to get $\{\nabla^{n_i} \Gamma\}_{i \leq l}, \nabla^m\Sigma \Rightarrow \nabla^{r+m} B$, then apply $R \vee_1$ to reach the desired sequent.
  
  $R \rightarrow$: In the case where $\mathcal{D}_1$ ends with an $R \rightarrow$, the cut-burden is altered in the premise.
  \begin{prooftree}
    \AXC{$\mathcal{D}_1'$} \noLine
    \UIC{$\nabla\Sigma, \{\nabla^{n_i+1} A\}_{i \leq l}, B \Rightarrow C$}
    \RightLabel{$R \rightarrow$}
    \UIC{$\Sigma, \{\nabla^{n_i} A\}_{i \leq l} \Rightarrow B \rightarrow C$}
   \end{prooftree}
   The induction hypothesis has a different cut-data, nevertheless, it still commutes with $R \rightarrow$.
  From induction hypothesis, we have $\{\nabla^{n_i+1} \Gamma\}_{i \leq l}, \nabla^{m+1} \Sigma,$ $\nabla^m B \Rightarrow \nabla^m C$. We can simply apply $R \rightarrow$ to get $\{\nabla^{n_i} \Gamma\}_{i \leq l}, \nabla^m \Sigma \Rightarrow \nabla^m (B \rightarrow C)$.
  
  $N$: Suppose $\mathcal{D}_1$ ends with $N$.
  \begin{prooftree}
    \AXC{$\mathcal{D}_1'$} \noLine
    \UIC{$\Sigma, \{\nabla^{n_i} A\}_{i \leq l} \Rightarrow \Delta$}
    \RightLabel{$N$}
    \UIC{$\nabla \Sigma, \{\nabla^{n_i+1} A\}_{i \leq l} \Rightarrow \nabla \Delta$}
  \end{prooftree}
  If we assume that the cut-data is $(A, m, \{n_i+1\}_{i \leq l})$, from the induction hypothesis we have $\{\nabla^{n_i} \Gamma\}_{i \leq l}, \nabla^m \Sigma \Rightarrow \nabla^m \Delta$. By $N$ we have $\{\nabla^{n_i+1} \Gamma\}_{i \leq l},$ $\nabla^{m+1} \Sigma \Rightarrow \nabla^{m+1} \Delta$, which is the desired sequent.
  But if $m>0$, the cut-data could also be $(\nabla A, m-1, \{n_i\}_{i \leq l})$, so using the induction hypothesis would suffice.
  
   \textbf{Part III.} Now in the last part of the proof, we will show how the construction takes place in the cases where a member of the cut-burden is principal in the last rule of $\mathcal{D}_1$, which can be either of $L\star (\star \in \{\wedge, \vee_{1/2}, \rightarrow\})$.
   Any of these rules also determine the rule at the end of the other proof-tree, because $\nabla^m A$ would also be principal in the last rule of $\mathcal{D}_0$. Recall that the only possible rules as the last rule of $\mathcal{D}_0$ are now $R\star (\star \in \{\wedge, \vee_{1/2}, \rightarrow\})$ and $Fa$, which all have a principal formula on the right side of the sequent.
   
   First, notice that $m > 1$ is impossible, since the only rule that introduces $\nabla$ in the right side of a sequent is $Fa$, which also introduces a $\rightarrow$ immediately after the $\nabla$ and implies $m = 1$. So suppose that $m = 1$ and $\mathcal{D}_0$ ends with $Fa$.
   \begin{prooftree}
     \AXC{$\mathcal{D}_0'$}
     \noLine
     \UIC{$\Gamma, A \Rightarrow B$}
     \RightLabel{$Fa$}
     \UIC{$\Gamma \Rightarrow \nabla (A \rightarrow B)$}
   \end{prooftree}
   The cut-data must be of the form $(A \rightarrow B, 1, \{n_i\}_{i \leq l})$, so the only option for $\mathcal{D}_1$ is $L \rightarrow$, with a principal formula from the cut-burden, like $\nabla^{n_j} (A \rightarrow B)$ for some $j \leq l$ such that $0 < n_j$.
   \begin{prooftree}
     \AXC{$\mathcal{D}_1'$}
     \noLine
     \UIC{$\Sigma, \{\nabla^{n_i} (A \rightarrow B) \}_{i \leq l}^{i \neq j}, \Rightarrow \nabla^{n_j-1} A$}
     \AXC{$\mathcal{D}_1''$}
     \noLine
     \UIC{$\Sigma, \{\nabla^{n_i} (A \rightarrow B) \}_{i \leq l}^{i \neq j}, \nabla^{n_j-1} B \Rightarrow \Delta$}
     \RightLabel{$L \rightarrow$}
     \BIC{$\Sigma, \{\nabla^{n_i} (A \rightarrow B)\}_{i \leq l}^{i \neq j}, \nabla^{n_j} (A \rightarrow B) \Rightarrow \Delta$}
   \end{prooftree}
   First, apply a low rank $\nabla cut$ (with $(B, 0, \{n_j\})$ as the cut-data) on $\mathcal{D}_0'$ and $IH(\mathcal{D}_0, \mathcal{D}_1'')$. Let the resulting sequent be called $\mathcal{D}'$.
   \begin{prooftree}
     \AXC{$\mathcal{D}_0'$}
     \noLine
     \UIC{$\Gamma, A \Rightarrow B$}
     \AXC{$\mathcal{D}_0$}
     \noLine
     \UIC{$\Gamma \Rightarrow \nabla (A \rightarrow B)$}
     \AXC{$\mathcal{D}_1''$}
     \noLine
     \UIC{$\Sigma, \{\nabla^{n_i} (A \rightarrow B) \}_{i \leq l}^{i \neq j}, \nabla^{n_j-1} B \Rightarrow \Delta$}
     \RightLabel{IH}
     \BIC{$\{\nabla^{n_i} \Gamma\}_{i \leq l}^{i \neq j}, \nabla \Sigma, \nabla^{n_j} B \Rightarrow \nabla \Delta$}
     \RightLabel{$\nabla cut$} \LeftLabel{$\mathcal{D}':~~~~~$}
     \BIC{$\nabla^{n_j} A, \{\nabla^{n_i} \Gamma\}_{i \leq l}, \nabla \Sigma \Rightarrow \nabla \Delta$}
   \end{prooftree}
   Then cut $IH(\mathcal{D}_0, \mathcal{D}_1')$ (this time with $(\nabla^{n_j} A, 0, \{0\})$ as the cut-data) into the resulting sequent.
   \begin{prooftree}
    \AXC{$\mathcal{D}_0$}
    \noLine
    \UIC{$\Gamma \Rightarrow \nabla (A \rightarrow B)$}
     \AXC{$\mathcal{D}_1'$}
     \noLine
     \UIC{$\Sigma, \{\nabla^{n_i} (A \rightarrow B) \}_{i \leq l}^{i \neq j}, \Rightarrow \nabla^{n_j-1} A$}
     \RightLabel{IH}
     \BIC{$\{\nabla^{n_i} \Gamma\}_{i \leq l}^{i \neq j}, \nabla \Sigma, \Rightarrow \nabla^{n_j} A$}
  
     \AXC{$\mathcal{D}'$}
  
     \RightLabel{$\nabla cut$}
     \BIC{$\{\nabla^{n_i} \Gamma\}_{i \leq l}^{i \neq j}, \{\nabla^{n_i} \Gamma\}_{i \leq l}, (\nabla \Sigma)^2 \Rightarrow \nabla \Delta$}
     \doubleLine \RightLabel{$Lc$}
     \UIC{$\{\nabla^{n_i} \Gamma\}_{i \leq l}, \nabla \Sigma \Rightarrow \nabla \Delta$}
   \end{prooftree}
   And that's the sequent that we wanted.
   
   Now we can assume $m = 0$. Each right-rule for $\mathcal{D}_1$ determines its corresponding left-rule for $\mathcal{D}_0$.
  
   $R \wedge$ and $L \wedge$: Suppose $\mathcal{D}_0$ ends with $R \wedge$ and $\mathcal{D}_1$ ends with either of $L \wedge_c ~ (c \in \{1,2\})$.
   \begin{prooftree}
     \noLine
     \AXC{$\mathcal{D}_0'$}
     \UIC{$\Gamma \Rightarrow A_1$}
     \noLine
     \AXC{$\mathcal{D}_0''$}
     \UIC{$\Gamma \Rightarrow A_2$}
     \RightLabel{$R \wedge$}
     \BIC{$\Gamma \Rightarrow A_1 \wedge A_2$}
     
     \noLine
     \AXC{$\mathcal{D}_1'$}
     \UIC{$\Sigma , \{\nabla^{n_i} (A_1 \wedge A_2)\}_{i \leq l}^{i \neq j}, \nabla^{n_j} A_c \Rightarrow \Delta$}
     \RightLabel{$L \wedge_1$}
     \UIC{$\Sigma , \{\nabla^{n_i} (A_1 \wedge A_2)\}_{i \leq l} \Rightarrow \Delta$}
     
     \noLine
     \BIC{}
   \end{prooftree}
   $IH(\mathcal{D}_0, \mathcal{D}_1')$ proves $\{\nabla^{n_i} \Gamma\}_{i \leq l}^{i \neq j}, \Sigma , \nabla^{n_j} A_c \Rightarrow \Delta$. Remove $\nabla^{n_j} A_c$ with a low rank $\nabla cut$ on this sequent and either of $\mathcal{D}_0'$ (if $c = 1$) or $\mathcal{D}_0''$ (if $c = 2$) to get $\{\nabla^{n_i} \Gamma\}_{i \leq l}, \Sigma \Rightarrow \Delta$.
  
   $R \vee$ and $L \vee$: Suppose that $\mathcal{D}_0$ ends with either of $R \vee_c ~ (c \in \{1,2\})$ and $\mathcal{D}_1$ ends with $L \vee$.
   \begin{prooftree}
     \noLine
     \AXC{$\mathcal{D}_0'$}
     \UIC{$\Gamma \Rightarrow A_c$}
     \RightLabel{$R \vee_c$}
     \UIC{$\Gamma \Rightarrow A_1 \vee A_2$}
   \end{prooftree}
   \begin{prooftree}
    \noLine
    \AXC{$\mathcal{D}_1'$}
    \UIC{$\Sigma , \{\nabla^{n_i} (A_1 \vee A_2)\}_{i \leq l}^{i \neq j} , \nabla^{n_j} A_1 \Rightarrow \Delta$}
    \noLine
    \AXC{$\mathcal{D}_1''$}
    \UIC{$\Sigma , \{\nabla^{n_i} (A_1 \vee A_2)\}_{i \leq l}^{i \neq j} , \nabla^{n_j} A_2 \Rightarrow \Delta$}
    \RightLabel{$L \vee$}
    \BIC{$\Sigma ,  \{\nabla^{n_i} (A_1 \vee A_2)\}_{i \leq l} \Rightarrow \Delta$}
   \end{prooftree}
   Using induction hypothesis, first, remove $\{\nabla^{n_i} (A_1 \vee A_2)\}_{i \leq l}^{i \neq j}$ from the subtree of $\mathcal{D}_1$ which has $\nabla^{n_j} A_c$ on its left side (by $IH(\mathcal{D}_0, \mathcal{D}_1')$ for $c = 1$, $IH(\mathcal{D}_0, \mathcal{D}_1'')$ for $c = 2$), to get $\{\nabla^{n_i} \Gamma\}_{i \leq l}^{i \neq j}, \Sigma , \nabla^{n_j} A_c \Rightarrow \Delta$. Then, remove $\nabla^{n_j} A_c$ by a low rank $\nabla cut$ on this sequent and $\mathcal{D}_0'$ to get $\{\nabla^{n_i} \Gamma\}_{i \leq l}, \Sigma \Rightarrow \Delta$.
  
   $R \rightarrow$ and $L \rightarrow$: Suppose that $\mathcal{D}_0$ and $\mathcal{D}_1$ end with $R \rightarrow$ and $L \rightarrow$ respectively.
   \begin{prooftree}
     \noLine
     \AXC{$\mathcal{ D}_0'$}
     \UIC{$\nabla \Gamma, A_1 \Rightarrow A_2$}
     \RightLabel{$R \rightarrow$}
     \UIC{$\Gamma \Rightarrow A_1 \rightarrow A_2$}        
     \end{prooftree}
     \begin{prooftree}
     \noLine
     \AXC{$\mathcal{D}_1'$}
     \UIC{$\Sigma, \{\nabla^{n_i} (A_1 \rightarrow A_2)\}_{i \leq l}^{i \neq j} \Rightarrow \nabla^{n_j-1} A_1$}
     \noLine
     \AXC{$\mathcal{D}_1''$}
     \UIC{$\Sigma, \{\nabla^{n_i} (A_1 \rightarrow A_2)\}_{i \leq l}^{i \neq j}, \nabla^{n_j-1} A_2 \Rightarrow \Delta$}
     \RightLabel{$L \rightarrow$}
     \BIC{$\Sigma,  \{\nabla^{n_i} (A_1 \rightarrow A_2)\}_{i \leq l} \Rightarrow \Delta$}
   \end{prooftree}
   
   $IH(\mathcal{D}_0, \mathcal{D}_1'')$ proves $\{\nabla^{n_i-1} \Gamma\}_{i \leq l}^{i \neq j}, \Sigma, \nabla^{n_j} A_2 \Rightarrow \Delta$. Applying a low rank $\nabla cut$ on $\mathcal{D}_0'$ and $IH(\mathcal{D}_0, \mathcal{D}_1'')$ removes $\nabla^{n_j-1} A_2$ and introduces $\nabla^{n_j-1} \Gamma$ and $\nabla^{n_j-1} A_1$ in the left. On the other hand $IH(\mathcal{D}_0, \mathcal{D}_1')$ proves $\{\nabla^{n_i} \Gamma\}_{i \leq l}^{i \neq j}, \Sigma, \Rightarrow \nabla^{n_j-1} A_1$, which we can use to also remove $\nabla^{n_j-1} A_1$ with another low rank cut. Then it suffices to remove the extra $\{\nabla^{n_i} \Gamma\}_{i \leq l}^{i \neq j}$ and $\Sigma$ with $Lc$.
  
   $Fa$ and $L \rightarrow$: This is also similar to the $m = 1$ case, except that here the cut-data is $(\nabla (A \rightarrow B), 0, \{n_i\}_{i \leq l})$.
   \vspace{5mm}
  
   Now we have a construction for any two possible pair of rules, in $\gstl^+$ and all its extensions. This concludes the proof of the theorem in all cases.
  
  \end{proof}

[introducing iSTL, the conservative extension with intuitionistic implication]

[introducing the equivalent cut-free system, iGSTL]

\subsection{applications}
[visser rule]

[disjuction property]

[subformula property]

\section{Interpolation theorems}
[deductive interpolation for iSTL+L]

[craig's interpolation for intuitionistic extension]

\section{Conclusion and future works}

\section{Acknowledgement}

\bibliographystyle{abbrv}
\bibliography{refs}
\end{document}
