\begin{thm}\label{thm:stl-eq-gstl}
	Let $S \subseteq \{L, R, Fa, Fu\}$. For any sequent $\Gamma \Rightarrow \Delta$ in the language $\mathcal{L}$, $\stl(S) \vdash \Gamma \Rightarrow \Delta$ iff $\gstl(S) \vdash \Gamma \Rightarrow \Delta$.
\end{thm}
\begin{proof}
	One direction easily follows from the fact that all rules of $\stl(S)$ are just instances of $\gstl(S)$'s rules.
	For the other direction, we will use case analysis for the last rule in the proof-tree of $\Gamma \Rightarrow \Delta$ in $\gstl(S)$, which we call $\mathcal{D}$, and construct a proof-tree for it in $\stl(S)$ in each case.
	
	First, observe that $Id$ and $Ta$ are present in $\stl(S)$ and Lemma \ref{lem:l-nabla-n-bot} handles the $Ex$ case.
	For the other rules, use induction on the length of $\mathcal{D}$; the induction hypothesis will provide a proof-tree in $\stl(S)$ for the subtree(s) of $\mathcal{D}$.
	For the rules that are common between two systems, just apply the same rule (in $\stl(S)$) on the proof-tree(s) from the induction hypothesis to reach the desired sequent. For example, if $\mathcal{D}$ ends with $R \vee$, it suffices to apply $R \vee$ (of $\stl(S)$) on the proof-tree that we get from the induction hypothesis.
	In the cases for $\gstl(S)$'s stronger rules, which are $L \wedge_1$, $L \wedge_2$, $L \vee$ and $L \rightarrow$, do the same, and also cut the sequents proved in Lemmas \ref{lem:l-nabla-dist-and}, \ref{lem:l-nabla-dist-or} or \ref{lem:l-nabla-dist-si} into the resulting sequent.
	The case for $N$ divides into two further cases. One case is when $\Delta = A$ for some formula $A$, which is again handled by applying $N$ on the sequent from the induction hypothesis.
	The other case is when $\Delta = \{\}$, so we have $\Gamma \Rightarrow$ from the induction hypothesis, on the right of which we first introduce $\bot$ using $Rw$, and then apply $N$.
	Then we can cut it into the sequent proved in Lemma \ref{lem:l-nabla-bot}, and then into $Ex$, to derive $\nabla \Gamma \Rightarrow$ as was desired.
\end{proof}