\begin{thm}[cut Reduction]\label{thm:gstl-cut-reduction}
  If $\gstl^+(S)$ proves $\Gamma \Rightarrow A$ and\\\ $\Sigma , \{\nabla^{n_i} A\}_{i \leq l} \Rightarrow \Delta$ with proof-trees of ranks less than $\rho(A)$, then it also proves $\{\nabla^{n_i} \Gamma\}_{i \leq l} , \Sigma \Rightarrow \Delta$ also with a proof tree of a rank less than $\rho(A)$ (for $S \subseteq \{L, R{\color{red}, Fa, Fu}\}$).

  {\color{red} This proof does not work if $S$ contains $Fa$ or $Fu$.}
  \end{thm}
  \begin{proof}
    We have two proof-trees
    \[
      \genfrac{}{}{0pt}{}{\mathcal{D}_0}{\Gamma \Rightarrow A}
      \hspace{3em}
      \genfrac{}{}{0pt}{}{\mathcal{D}_1}{\Sigma , \{\nabla^{n_i} A\}_{i \leq l} \Rightarrow \Delta}
    \]
    both of a lower rank than that of $A$, and we want to construct a proof-tree
    \[\genfrac{}{}{0pt}{}{\mathcal{D}}{\{\nabla^{n_i} \Gamma\}_{i \leq l} , \Sigma \Rightarrow \Delta} \]
    without increasing the cut rank.
  
    The construction takes place in different cases for the last rule that occurs in $\mathcal{D}_0$ and $\mathcal{D}_1$. Notice that the proof of the theorem is essentially the same for any choice of $S$, where $S$ is a subset of $\{L, R, Fa, Fu\}$, modulo the cases that are specific to the rules in $S$. Thus, for the sake of brevity, we will not repeat the cases which are common between all systems. Also notice that the resulting proof-tree in each case is constructed using the core system $\gstl^+$, plus the same rule of that case, so it will work for any extension containing that rule.
  
    In many cases, our construction would depend only on the last rule of one of the subtrees and it would work, no matter what the last rule in the other subtree is. Therefore, any of these cases will also cover all the cases for the other subtree. These cases constitute the first two parts of the proof. In the third part, we will address the cases where our construction depends on the last rule in \emph{both} subtrees, which are the cases that the cut-burden is altered on both sides. In these cases, the last rule in one of the subtrees determines a specific form for the formulas in the cut-burden, which will determine the last rule in the other subtree.
  
    In first two groups of the cases, we will need induction on the height of one of the subtrees. But in the third part, we will use induction simultaneously on both $\mathcal{D}_0$ and $\mathcal{D}_1$, which goes as follows. For any two proof-trees $\mathcal{D}_0'$ and $\mathcal{D}_1'$ such that $h(\mathcal{D}_0') + h(\mathcal{D}_1') < h(\mathcal{D}_0) + h(\mathcal{D}_1)$, where $\mathcal{D}_0'$ proves $\Gamma' \Rightarrow A'$ and $\mathcal{D}_1'$ proves $\Sigma', \{\nabla^{n'_i} A'\}_{i \leq l} \Rightarrow \Delta'$ for arbitrary $\Gamma'$, $\Sigma'$, $\Delta'$, $A'$ and $n_i'$ of length $l'+1$, for which we have $\rho(\mathcal{D}_0'),\rho(\mathcal{D}_1') < \rho(A')$, the induction hypothesis gives us a prooftree, denoted by $\text{IH}(\mathcal{D}_0', \mathcal{D}_1')$ where it matters, that proves $\{\nabla^{n_i'}\Gamma'\}_{i \leq l'}, \Sigma' \Rightarrow \Delta'$, and we will also have $\rho(\text{IH}(\mathcal{D}_0', \mathcal{D}_1')) < \rho(A')$.
  
    \textbf{Part I.} First, assume that $\mathcal{D}_0$ is an axiom. No matter what would be the last rule instance in $\mathcal{D}_1$, the case for $Id$ is trivial, $Ex$ won't happen and $Ta$ is handled by Lemma \ref{lem:gstl-top-redundant}.
    Now assume that $\mathcal{D}_0$ ends with an instance of the rules $Lw$, $Lc$, $Rw$, $L \wedge_1$, $L \wedge_2$, $L \vee$, $L \rightarrow$, $\nabla cut$, $N$, $Fu$, $L$ or $R$. In all these cases---again, independent of $\mathcal{D}_1$---it suffices to use induction on the assumption(s) of this rule and $\mathcal{D}_1$ to remove the cut-burden from both subtrees. Then, we can apply the same rule to get the desired sequent. Here we will only mention the cases for $L \wedge_1$, $L \vee$, $L \rightarrow$, $\nabla cut$, $Rw$, $N$ and $Fu$, the last three of which may be of special concern, since they also alter the cut-burden. The other cases are similar.
  
    $L \wedge_1$: If $\mathcal{D}_0$ ends with $L \wedge_1$, that is
    \begin{prooftree}
      \noLine
      \AXC{$\mathcal{D}_0'$}
      \UIC{$\Gamma, \nabla^r B \Rightarrow A$}
      
      \RightLabel{$L \wedge_1$}
      \UIC{$\Gamma, \nabla^r (B \wedge C) \Rightarrow A$}
   \end{prooftree}
   then by applying $L \wedge_1$ on what we get from induction
   \begin{prooftree}
    \noLine
    \AXC{$\mathcal{D}_0'$}
    \UIC{$\Gamma, \nabla^r B \Rightarrow A$}
    
    \noLine
    \AXC{$\mathcal{D}_1$}
    \UIC{$\Sigma , \{\nabla^{n_i} A\}_{i \leq l} \Rightarrow \Delta$}
    
    \RightLabel{IH}
    \BIC{$\{\nabla^{n_i} \Gamma, \nabla^{n_i+r} B\}_{i \leq l}, \Sigma \Rightarrow \Delta$}
  
    \RightLabel{$L \wedge_1$} \doubleLine
    \UIC{$\{\nabla^{n_i} \Gamma, \nabla^{n_i+r} (B \wedge C)\}_{i \leq l}, \Sigma \Rightarrow \Delta$}
   \end{prooftree}
  
   \noindent $L \vee$: If $\mathcal{D}_0$ ends with $L \vee$
     \begin{prooftree}
       \noLine
       \AXC{$\mathcal{D}_0'$}
       \UIC{$\Gamma, \nabla^r B \Rightarrow A$}
       
       \noLine
       \AXC{$\mathcal{D}_0''$}
       \UIC{$\Gamma, \nabla^r C \Rightarrow A$}
       
       \RightLabel{$L \vee$}
       \BIC{$\Gamma, \nabla^r (B \vee C) \Rightarrow A$}
    \end{prooftree}
    Applying $L \vee$ on the sequents that we get from induction
    \begin{prooftree}
      \noLine
      \AXC{$\mathcal{D}_0'$}
      \UIC{$\Gamma, \nabla^r B \Rightarrow A$}
      
      \noLine
      \AXC{$\mathcal{D}_1$}
      \UIC{$\Sigma , \{\nabla^{n_i} A\}_{i \leq l} \Rightarrow \Delta$}
      
      \RightLabel{IH}
      \BIC{$\{\nabla^{n_i} \Gamma, \nabla^{n_i+r} B\}_{i \leq l}, \Sigma \Rightarrow \Delta$}
      
  
      \noLine
      \AXC{$\mathcal{D}_0''$}
      \UIC{$\Gamma, \nabla^r C \Rightarrow A$}
      
      \noLine
      \AXC{$\mathcal{D}_1$}
      \UIC{$\Sigma , \{\nabla^{n_i} A\}_{i \leq l} \Rightarrow \Delta$}
      
      \RightLabel{IH}
      \BIC{$\{\nabla^{n_i} \Gamma, \nabla^{n_i+r} C\}_{i \leq l}, \Sigma \Rightarrow \Delta$}
  
      \RightLabel{$L \vee$}
      \BIC{$\{\nabla^{n_i} \Gamma, \nabla^{n_i+r} (B \vee C)\}_{i \leq l}, \Sigma \Rightarrow \Delta$}
     \end{prooftree}
  
   
  \noindent $L \rightarrow$: Suppose $\mathcal{D}_0$ ends with a $L \rightarrow$ as shown below.
   \begin{prooftree}
    \noLine
    \AXC{$\mathcal{D}_0'$}
    \UIC{$\Gamma \Rightarrow \nabla^r B$}
    \noLine
    \AXC{$\mathcal{D}_0''$}
    \UIC{$\Gamma , \nabla^r C \Rightarrow A$}
    \RightLabel{$L \rightarrow$}
    \BIC{$\Gamma , \nabla^{r+1} (B \rightarrow C) \Rightarrow A$}
   \end{prooftree}
   Let $IH(\mathcal{D}_0'', \mathcal{D}_1)$ be called $\mathcal{D}'$.
   \begin{prooftree}
    \noLine
    \AXC{$\mathcal{D}_0''$}
    \UIC{$\Gamma , \nabla^r C \Rightarrow A$}
    \noLine
    \AXC{$\mathcal{D}_1$}
    \UIC{$\Sigma , \{\nabla^{n_i} A\}_{i \leq l} \Rightarrow \Delta$}
    \RightLabel{IH} \LeftLabel{$\mathcal{D}':~~~~$}
    \BIC{$\{\nabla^{n_i} \Gamma , \nabla^{n_i+r} C\}_{i \leq l} , \Sigma \Rightarrow \Delta$}
   \end{prooftree}
   In order to apply $L \rightarrow$, we must prepare the context in $\mathcal{D}_0'$, for each of $\nabla^{n_i+r}C$'s. Beginning with $j = 0$, first apply $N$ on $\mathcal{D}_0'$ $n_0$ times to get $\nabla^{n_0}\Gamma \Rightarrow \nabla^{n_0+r} B$. Then we can just add the rest of the context by $Lw$.
   \begin{prooftree}
    \noLine
    \AXC{$\mathcal{D}_0'$}
    \UIC{$\Gamma \Rightarrow \nabla^r B$}
    \doubleLine \RightLabel{$N$}
    \UIC{$\nabla^{n_0} \Gamma \Rightarrow \nabla^{n_0+r} B$}
    \doubleLine \RightLabel{$Lw$}
    \UIC{$\{\nabla^{n_i} \Gamma\}_{i \leq l}, \{\nabla^{n_i+r}C\}_{i \leq l}^{i \neq 0} , \Sigma \Rightarrow \nabla^{n_0+r} B$}
   \end{prooftree}
   Let the outcome of applying $L \rightarrow$ on this sequent and $\mathcal{D}'$ be called $\mathcal{D}'_{n_0}$:
   \[\mathcal{D}'_{n_0}:~~~~\{\nabla^{n_i} \Gamma\}_{i \leq l}, \{\nabla^{n_i+r}C\}_{i \leq l}^{i \neq 0}, \nabla^{n_0+r+1} (B \rightarrow C) , \Sigma \Rightarrow \Delta\]
   Now for all $0 < j \leq l$, we construct $\mathcal{D}_{n_j}$ similarly.
   \begin{prooftree}
    \noLine
    \AXC{$\mathcal{D}_0'$}
    \UIC{$\Gamma \Rightarrow \nabla^r B$}
    \doubleLine \RightLabel{$N$}
    \UIC{$\nabla^{n_j} \Gamma \Rightarrow \nabla^{n_j+r} B$}
    \doubleLine \RightLabel{$Lw$}
    \UIC{$\{\nabla^{n_i} \Gamma\}_{i \leq l}, \{\nabla^{n_i+r}C\}_{j < i \leq l}, \{ \nabla^{n_i+r+1} (B \rightarrow C) \}_{i < j}, \Sigma \Rightarrow \nabla^{n_j+r} B$}
   \end{prooftree}
   Applying $L \rightarrow$ on this sequent and $\mathcal{D}_{n_{j-1}}$ we would get
   \[\mathcal{D}'_{n_j}:~~~~\{\nabla^{n_i} \Gamma\}_{i \leq l}, \{\nabla^{n_i+r}C\}_{j < i \leq l}, \{ \nabla^{n_i+r+1} (B \rightarrow C) \}_{i \leq j}, \Sigma \Rightarrow \Delta\]
   $\mathcal{D}_{n_l}$ is exactly what we want:
   \[\{\nabla^{n_i} \Gamma, \nabla^{n_i+r+1}(B \rightarrow C)\}_{i \leq l}, \Sigma \Rightarrow \Delta\]
  
  
   $\nabla cut$: Assume $\mathcal{D}_0$ ends with a $\nabla cut$ with cut-data $(A', \{n_i'\}_{i \leq l'})$. Recall that by assumption, $A'$ must have a lower rank than $A$.
   \begin{prooftree}
     \noLine
     \AXC{$\mathcal{D}_0'$}
     \UIC{$\Gamma \Rightarrow A'$}
     
     \noLine
     \AXC{$\mathcal{D}_0''$}
     \UIC{$\Pi, \{\nabla^{n_i'} A'\}_{i \leq l'} \Rightarrow A$}
     
     \RightLabel{$\nabla cut$}
     \BIC{$\{\nabla^{n_i'} \Gamma\}_{i \leq l'}, \Pi \Rightarrow A$}
   \end{prooftree}
   We must construct a proof-tree for $\{\nabla^{n_i + n_j'} \Gamma\}_{j \leq l'}^{i \leq l}, \{\nabla^{n_i} \Pi\}_{i \leq l} , \Sigma$ $\Rightarrow \Delta$. We can use the induction hypothesis first to remove $A$, and then use a low rank $\nabla cut$ to remove $A'$.
   \begin{prooftree}
     \noLine
     \AXC{$\mathcal{D}_0'$}
     \UIC{$\Gamma \Rightarrow A'$}
     
     \noLine
     \AXC{$\mathcal{D}_0''$}
     \UIC{$\Pi, \{\nabla^{n_i'} A'\}_{i \leq l'} \Rightarrow A$}
  
     \noLine
     \AXC{$\mathcal{D}_1$}
     \UIC{$\Sigma , \{\nabla^{n_i} A\}_{i \leq l} \Rightarrow \Delta$}
  
     \RightLabel{IH}
     \BIC{$\{\nabla^{n_i} \Pi\}_{i \leq l} , \{\nabla^{n_i + n_j'} A'\}_{j \leq l'}^{i \leq l}, \Sigma \Rightarrow \Delta$}
     
  
     \RightLabel{$\nabla cut$}
     \BIC{$\{\nabla^{n_i + n_j'} \Gamma\}_{j \leq l'}^{i \leq l}, \{\nabla^{n_i} \Pi\}_{i \leq l}, \Sigma$ $\Rightarrow \Delta$}
   \end{prooftree}
  
   $Rw$: In this case, $\mathcal{D}_0'$ proves $\Gamma \Rightarrow$, so we can simply construct the desired proof-tree using $N$, $Lw$ and $Rw$.
   \begin{prooftree}
     \noLine
     \AXC{$\mathcal{D}_0'$}
     \UIC{$\Gamma \Rightarrow$}
     \doubleLine \RightLabel{$N$}
     \UIC{$\nabla^{n_0} \Gamma \Rightarrow$}
     \doubleLine \RightLabel{$Lw$}
     \UIC{$\{\nabla^{n_i} \Gamma\}_{i \leq l} , \Sigma \Rightarrow$}
     \RightLabel{$Rw$}
     \UIC{$\{\nabla^{n_i} \Gamma\}_{i \leq l} , \Sigma \Rightarrow \Delta$}
   \end{prooftree}
  
   $N$: $\mathcal{D}_0$ proves $\nabla \Gamma \Rightarrow \nabla A$ and $\mathcal{D}_1$ proves $\Sigma, \{\nabla^{n_i} A\}_{i<l} \Rightarrow \Delta$. There are two cases: The cut-data could be $(A, \{n_i\}_{i \leq l})$, or if for all $i \leq l$ we have $0 < n_i$, then the cut-data could also be $(\nabla A, \{n_i-1\}_{i \leq l})$. Induction hypothesis for $\mathcal{D}_0$'s immediate sub-tree and $\mathcal{D}_1$ gives us $\{\nabla^{n_i}\Gamma\}_{i \leq l}, \Sigma \Rightarrow \Delta$, which handles the latter case, and an application of $N$ on this sequent would handle the former.
  
   {\color{red} $Fu$: This case fails actually. For example in case
   \begin{multicols}{2}
    \begin{prooftree}
      \AXC{$\nabla \Gamma \Rightarrow \nabla A$}
      \RightLabel{$Fu$}
      \UIC{$\Gamma \Rightarrow A$}
    \end{prooftree}
    \columnbreak
    $A \Rightarrow \Delta$
   \end{multicols}
   we can't use induction hypothesis to cut $A$.
   }\\
  
   \textbf{Part II.} The rest of the cases for $\mathcal{D}_0$ can't be solved independent of $\mathcal{D}_1$, so in the second part of the cases, we will consider the last rule of $\mathcal{D}_1$, again, where the solution could be constructed independent of $\mathcal{D}_0$. But this time we have less possibilities for the opposite subtree, since we've already solved most of them in the previous part of the proof. In fact the only possible rules as the last rule of $\mathcal{D}_0$ are now $R\star (\star \in \{\wedge, \vee_{1/2}, \rightarrow\})$ and $Fa$.
  
   Suppose $\mathcal{D}_1$ is an axiom. Again, the case for $Id$ is trivial, $Ta$ won't happen, and $Ex$ is also infeasible, since all possible cases for $\mathcal{D}_0$ alter the right side of the sequent, but none of them are able to introduce $\bot$ on the right side.
   In the remaining cases, if the cut-data is $(A, \{n_i\}_{i \leq l})$, in the cases where no member of the cut-burden is altered in the last rule of $\mathcal{D}_1$ modulo its number of $\nabla$'s, the construction is similar to the first part: Applying the same rule on the sequent that we get from the induction hypothesis. But if a member of the cut-burden is principal in the last rule of $\mathcal{D}_1$, which is to be handeld in the last part, we must also use the induction hypothesis for $\mathcal{D}_0$, both with a different cut-data. We now address the second part of the cases.
   
   For the sake of briefness, we will only explain the cases for $L \wedge_1$, $R \vee_1$, $R \rightarrow$ and $N$, the last two of which are of special concern, since we must use induction hypothesis with different cut-data in those cases. The rest would be handled similarly.
  
   $L \wedge$: Assume that $\mathcal{D}_1$ ends with $L \wedge_1$, but no member of the cut-burden is its principal formula.
   \begin{prooftree}
    \AXC{$\mathcal{D}_1'$} \noLine
    \UIC{$\Sigma, \{\nabla^{n_i} A\}_{i \leq l}, \nabla^r B \Rightarrow \Delta$}
    \RightLabel{$L \wedge_1$}
    \UIC{$\Sigma, \{\nabla^{n_i} A\}_{i \leq l}, \nabla^r (B \wedge C) \Rightarrow \Delta$}
   \end{prooftree}
   From induction hypothesis we have $\{\nabla^{n_i} \Gamma\}_{i \leq l}, \Sigma, \nabla^r B \Rightarrow \Delta$. By $L \wedge_1$ we have $\{\nabla^{n_i} \Gamma\}_{i \leq l}, \Sigma, \nabla^r (B \wedge C) \Rightarrow \Delta$.
  
   $R \vee_1$: Suppose that $\mathcal{D}_1$ ends with $R \vee_1$.
   \begin{prooftree}
    \AXC{$\mathcal{D}_1'$} \noLine
    \UIC{$\Sigma, \{\nabla^{n_i} A\}_{i \leq l} \Rightarrow B$}
    \RightLabel{$R \vee_1$}
    \UIC{$\Sigma, \{\nabla^{n_i} A\}_{i \leq l} \Rightarrow B \vee C$}
   \end{prooftree}
   Again, use the induction hypothesis to get $\{\nabla^{n_i} \Gamma\}_{i \leq l}, \Sigma \Rightarrow B$, then apply $R \vee_1$ to reach the desired sequent.
  
  $R \rightarrow$: In the case where $\mathcal{D}_1$ ends with an $R \rightarrow$, the cut-burden is altered in the premise.
  \begin{prooftree}
    \AXC{$\mathcal{D}_1'$} \noLine
    \UIC{$\nabla\Sigma, \{\nabla^{n_i+1} A\}_{i \leq l}, B \Rightarrow C$}
    \RightLabel{$R \rightarrow$}
    \UIC{$\Sigma, \{\nabla^{n_i} A\}_{i \leq l} \Rightarrow B \rightarrow C$}
   \end{prooftree}
   The induction hypothesis has a different cut-data, nevertheless, it still commutes with $R \rightarrow$.
  From induction hypothesis, we have $\{\nabla^{n_i+1} \Gamma\}_{i \leq l}, \nabla \Sigma, B \Rightarrow C$. We can simply apply $R \rightarrow$ to get $\{\nabla^{n_i} \Gamma\}_{i \leq l}, \Sigma \Rightarrow B \rightarrow C$.
  
  $N$: Suppose $\mathcal{D}_1$ ends with $N$.
  \begin{prooftree}
    \AXC{$\mathcal{D}_1'$} \noLine
    \UIC{$\Sigma, \{\nabla^{n_i} A\}_{i \leq l} \Rightarrow \Delta$}
    \RightLabel{$N$}
    \UIC{$\nabla \Sigma, \{\nabla^{n_i+1} A\}_{i \leq l} \Rightarrow \nabla \Delta$}
  \end{prooftree}
  If we assume that the cut-data is $(A, \{n_i+1\}_{i \leq l})$, from the induction hypothesis we have $\{\nabla^{n_i} \Gamma\}_{i \leq l}, \Sigma \Rightarrow \Delta$. By $N$ we have $\{\nabla^{n_i+1} \Gamma\}_{i \leq l},$ $\nabla \Sigma \Rightarrow \nabla \Delta$, which is the desired sequent.

  {\color{red} This case failes in presence of $Fa$. Notice the cut-data could then also be $(\nabla A, \{n_i\}_{i \leq l})$.}
  
   \textbf{Part III.} Now in the last part of the proof, we will show how the construction takes place in the cases where a member of the cut-burden is principal in the last rule of $\mathcal{D}_1$, which can be either of $L\star (\star \in \{\wedge, \vee_{1/2}, \rightarrow\})$.
   Any of these rules also determine the rule at the end of the other proof-tree, because $A$ would also be principal in the last rule of $\mathcal{D}_0$. Recall that the only possible rules as the last rule of $\mathcal{D}_0$ are now $R\star (\star \in \{\wedge, \vee_{1/2}, \rightarrow\})$ and $Fa$, which all have a principal formula on the right side of the sequent.

   $R \wedge$ and $L \wedge$: Suppose $\mathcal{D}_0$ ends with $R \wedge$ and $\mathcal{D}_1$ ends with either of $L \wedge_c ~ (c \in \{1,2\})$.
   \begin{prooftree}
     \noLine
     \AXC{$\mathcal{D}_0'$}
     \UIC{$\Gamma \Rightarrow A_1$}
     \noLine
     \AXC{$\mathcal{D}_0''$}
     \UIC{$\Gamma \Rightarrow A_2$}
     \RightLabel{$R \wedge$}
     \BIC{$\Gamma \Rightarrow A_1 \wedge A_2$}
     
     \noLine
     \AXC{$\mathcal{D}_1'$}
     \UIC{$\Sigma , \{\nabla^{n_i} (A_1 \wedge A_2)\}_{i \leq l}^{i \neq j}, \nabla^{n_j} A_c \Rightarrow \Delta$}
     \RightLabel{$L \wedge_1$}
     \UIC{$\Sigma , \{\nabla^{n_i} (A_1 \wedge A_2)\}_{i \leq l} \Rightarrow \Delta$}
     
     \noLine
     \BIC{}
   \end{prooftree}
   $IH(\mathcal{D}_0, \mathcal{D}_1')$ proves $\{\nabla^{n_i} \Gamma\}_{i \leq l}^{i \neq j}, \Sigma , \nabla^{n_j} A_c \Rightarrow \Delta$. Remove $\nabla^{n_j} A_c$ with a low rank $\nabla cut$ on this sequent and either of $\mathcal{D}_0'$ (if $c = 1$) or $\mathcal{D}_0''$ (if $c = 2$) to get $\{\nabla^{n_i} \Gamma\}_{i \leq l}, \Sigma \Rightarrow \Delta$.
  
   $R \vee$ and $L \vee$: Suppose that $\mathcal{D}_0$ ends with either of $R \vee_c ~ (c \in \{1,2\})$ and $\mathcal{D}_1$ ends with $L \vee$.
   \begin{prooftree}
     \noLine
     \AXC{$\mathcal{D}_0'$}
     \UIC{$\Gamma \Rightarrow A_c$}
     \RightLabel{$R \vee_c$}
     \UIC{$\Gamma \Rightarrow A_1 \vee A_2$}
   \end{prooftree}
   \begin{prooftree}
    \noLine
    \AXC{$\mathcal{D}_1'$}
    \UIC{$\Sigma , \{\nabla^{n_i} (A_1 \vee A_2)\}_{i \leq l}^{i \neq j} , \nabla^{n_j} A_1 \Rightarrow \Delta$}
    \noLine
    \AXC{$\mathcal{D}_1''$}
    \UIC{$\Sigma , \{\nabla^{n_i} (A_1 \vee A_2)\}_{i \leq l}^{i \neq j} , \nabla^{n_j} A_2 \Rightarrow \Delta$}
    \RightLabel{$L \vee$}
    \BIC{$\Sigma ,  \{\nabla^{n_i} (A_1 \vee A_2)\}_{i \leq l} \Rightarrow \Delta$}
   \end{prooftree}
   Using induction hypothesis, first, remove $\{\nabla^{n_i} (A_1 \vee A_2)\}_{i \leq l}^{i \neq j}$ from the subtree of $\mathcal{D}_1$ which has $\nabla^{n_j} A_c$ on its left side (by $IH(\mathcal{D}_0, \mathcal{D}_1')$ for $c = 1$, $IH(\mathcal{D}_0, \mathcal{D}_1'')$ for $c = 2$), to get $\{\nabla^{n_i} \Gamma\}_{i \leq l}^{i \neq j}, \Sigma , \nabla^{n_j} A_c \Rightarrow \Delta$. Then, remove $\nabla^{n_j} A_c$ by a low rank $\nabla cut$ on this sequent and $\mathcal{D}_0'$ to get $\{\nabla^{n_i} \Gamma\}_{i \leq l}, \Sigma \Rightarrow \Delta$.
  
   $R \rightarrow$ and $L \rightarrow$: Suppose that $\mathcal{D}_0$ and $\mathcal{D}_1$ end with $R \rightarrow$ and $L \rightarrow$ respectively. So there must be $j \leq l$ such that $n_j > 0$.
   \begin{prooftree}
     \noLine
     \AXC{$\mathcal{ D}_0'$}
     \UIC{$\nabla \Gamma, A_1 \Rightarrow A_2$}
     \RightLabel{$R \rightarrow$}
     \UIC{$\Gamma \Rightarrow A_1 \rightarrow A_2$}        
     \end{prooftree}
     \begin{prooftree}
     \noLine
     \AXC{$\mathcal{D}_1'$}
     \UIC{$\Sigma, \{\nabla^{n_i} (A_1 \rightarrow A_2)\}_{i \leq l}^{i \neq j} \Rightarrow \nabla^{n_j-1} A_1$}
     \noLine
     \AXC{$\mathcal{D}_1''$}
     \UIC{$\Sigma, \{\nabla^{n_i} (A_1 \rightarrow A_2)\}_{i \leq l}^{i \neq j}, \nabla^{n_j-1} A_2 \Rightarrow \Delta$}
     \RightLabel{$L \rightarrow$}
     \BIC{$\Sigma,  \{\nabla^{n_i} (A_1 \rightarrow A_2)\}_{i \leq l} \Rightarrow \Delta$}
   \end{prooftree}
   
   $IH(\mathcal{D}_0, \mathcal{D}_1'')$ proves $\{\nabla^{n_i} \Gamma\}_{i \leq l}^{i \neq j}, \Sigma, \nabla^{n_j-1} A_2 \Rightarrow \Delta$. Applying a low rank $\nabla cut$ on $\mathcal{D}_0'$ and $IH(\mathcal{D}_0, \mathcal{D}_1'')$ removes $\nabla^{n_j-1} A_2$ and introduces $\nabla^{n_j } \Gamma$ and $\nabla^{n_j-1} A_1$ in the left. On the other hand $IH(\mathcal{D}_0, \mathcal{D}_1')$ proves $\{\nabla^{n_i} \Gamma\}_{i \leq l}^{i \neq j}, \Sigma, \Rightarrow \nabla^{n_j-1} A_1$, which we can use to also remove $\nabla^{n_j-1} A_1$ with another low rank cut. Then it suffices to remove the extra $\{\nabla^{n_i} \Gamma\}_{i \leq l}^{i \neq j}$ and $\Sigma$ with $Lc$.
  
   $Fa$ and $L \rightarrow$: Suppose $\mathcal{D}_0$ ends with $Fa$ and $\mathcal{D}_1$ ends with $L \rightarrow$.
   \begin{prooftree}
     \AXC{$\mathcal{D}_0'$}
     \noLine
     \UIC{$\Gamma, A_1 \Rightarrow A_2$}
     \RightLabel{$Fa$}
     \UIC{$\Gamma \Rightarrow \nabla (A_1 \rightarrow A_2)$}
   \end{prooftree}
   The cut-data must be of the form $(\nabla (A_1 \rightarrow A_2), \{n_i\}_{i \leq l})$, so the only option for $\mathcal{D}_1$ is $L \rightarrow$, with a principal formula from the cut-burden, like $\nabla^{n_j+1} (A_1 \rightarrow A_2)$ for some $j \leq l$.
   \begin{prooftree}
     \AXC{$\mathcal{D}_1'$}
     \noLine
     \UIC{$\Sigma, \{\nabla^{n_i+1} (A_1 \rightarrow A_2) \}_{i \leq l}^{i \neq j}, \Rightarrow \nabla^{n_j} A_1$}
     \AXC{$\mathcal{D}_1''$}
     \noLine
     \UIC{$\Sigma, \{\nabla^{n_i+1} (A_1 \rightarrow A_2) \}_{i \leq l}^{i \neq j}, \nabla^{n_j} A_2 \Rightarrow \Delta$}
     \RightLabel{$L \rightarrow$}
     \BIC{$\Sigma, \{\nabla^{n_i+1} (A_1 \rightarrow A_2)\}_{i \leq l} \Rightarrow \Delta$}
   \end{prooftree}
   First, apply a low rank $\nabla cut$ (with $(A_2, \{n_j\})$ as the cut-data) on $\mathcal{D}_0'$ and $IH(\mathcal{D}_0, \mathcal{D}_1'')$. Let the resulting sequent be called $\mathcal{D}'$.
   \begin{prooftree}
     \AXC{$\mathcal{D}_0'$}
     \noLine
     \UIC{$\Gamma, A_1 \Rightarrow A_2$}
     \AXC{$\mathcal{D}_0$}
     \noLine
     \UIC{$\Gamma \Rightarrow \nabla (A_1 \rightarrow A_2)$}
     \AXC{$\mathcal{D}_1''$}
     \noLine
     \UIC{$\Sigma, \{\nabla^{n_i+1} (A_1 \rightarrow A_2) \}_{i \leq l}^{i \neq j}, \nabla^{n_j} A_2 \Rightarrow \Delta$}
     \RightLabel{IH}
     \BIC{$\{\nabla^{n_i} \Gamma\}_{i \leq l}^{i \neq j}, \Sigma, \nabla^{n_j} A_2 \Rightarrow \Delta$}
     \RightLabel{$\nabla cut$} \LeftLabel{$\mathcal{D}':~~~~~$}
     \BIC{$\nabla^{n_j} A_1, \{\nabla^{n_i} \Gamma\}_{i \leq l}, \Sigma \Rightarrow \Delta$}
   \end{prooftree}
   Then cut $IH(\mathcal{D}_0, \mathcal{D}_1')$ (this time with $(\nabla^{n_j} A_1, \{0\})$ as the cut-data) into the resulting sequent.
   \begin{prooftree}
    \AXC{$\mathcal{D}_0$}
    \noLine
    \UIC{$\Gamma \Rightarrow \nabla (A_1 \rightarrow A_2)$}
     \AXC{$\mathcal{D}_1'$}
     \noLine
     \UIC{$\Sigma, \{\nabla^{n_i+1} (A_1 \rightarrow A_2) \}_{i \leq l}^{i \neq j}, \Rightarrow \nabla^{n_j} A_1$}
     \RightLabel{IH}
     \BIC{$\{\nabla^{n_i} \Gamma\}_{i \leq l}^{i \neq j}, \Sigma, \Rightarrow \nabla^{n_j} A_1$}
  
     \AXC{$\mathcal{D}'$}
  
     \RightLabel{$\nabla cut$}
     \BIC{$\{\nabla^{n_i} \Gamma\}_{i \leq l}^{i \neq j}, \{\nabla^{n_i} \Gamma\}_{i \leq l}, \Sigma, \Sigma \Rightarrow \Delta$}
     \doubleLine \RightLabel{$Lc$}
     \UIC{$\{\nabla^{n_i} \Gamma\}_{i \leq l}, \Sigma \Rightarrow \Delta$}
   \end{prooftree}
   And that's the sequent that we wanted.

   \vspace{5mm}
  
   Now we have a construction for any two possible pair of rules, in $\gstl^+$ and all its extensions. This concludes the proof of the theorem in all cases.
  
  \end{proof}