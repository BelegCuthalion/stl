The usual interpretation of an intuitionistic proof for an implication is a structure that converts \emph{all} proofs of its antecedent to some proof for its consequent. This interpretation is impredicative in the sense that we are quantifying over the set of all proofs, while we are just defining some members of it, which are, the proofs for implication. The impredicativity of implication shows up more clearly when one is using a conditional assumption in an intuitionistic proof, which amounts to an application of the rule of \emph{modus ponens}. What this rule essentially says is that if we have a proof for an implication and a proof of its antecedent at the same time, then we can conclude its consequent.

The logic of spacetime remedies this by restricting the use of conditional assumption; So that the modus ponens rule in the logic of spacetime would not take a proof of conditional statement itself, but a proof of a modal operator, called $\nabla$, applied to the conditional statement. We can read this application of $\nabla$ as a \emph{delay} in use of implication, and interpret the proof of conditional statement as a function that converts all \emph{later} proofs of its antecedent to some \emph{later} proof for its consequent. So we can prove the consequent of a conditional statement, only if we have proved the conditional statement itself \emph{sooner} than the antecedent.\\

The logic of spacetime is represented by a sequent-style system called $\stl$ (known as $\istl(N)$\footnote{Note that in this paper $\istl$ is reserved for another system with intuitionistic implication.} in \cite{amir}), which has the same rules as the usual intuitionistic calculus $\mathbf{LJ}$, over the propositional language augumented with a unary connective $\nabla$, but with left and right implication rules replaced with the following rules, respectively.

\begin{multicols}{2}
	\begin{prooftree}
		\AXC{$\Gamma \Rightarrow A$}
		\AXC{$\Gamma, B \Rightarrow \Delta$}
		\RightLabel{$L \rightarrow$}
		\BIC{$\Gamma, \nabla (A \rightarrow B) \Rightarrow \Delta$}
	\end{prooftree}
	\columnbreak
	\begin{prooftree}
		\AXC{$\nabla \Gamma, A \Rightarrow B$}
		\RightLabel{$R \rightarrow$}
		\UIC{$\Gamma \Rightarrow A \rightarrow B$}
	\end{prooftree}
\end{multicols}
Furthermore, the system also has a rule to introduce $\nabla$ to both sides of a sequent.
\begin{prooftree}
	\AXC{$\Gamma \Rightarrow A$}
	\RightLabel{$N$}
	\UIC{$\nabla \Gamma \Rightarrow \nabla A$}
\end{prooftree}
So, generally, $\nabla (A \rightarrow B)$, and not $A \rightarrow B$, will be of use in proving $B$ from $A$.

In the next section of this paper, we will introduce a sequent style calculus for the spacetime logic, called $\stl$ and introduce an equivalent cut-free system $\gstl$, and then survey some extensions and their respecting semantics, including an extension of $\stl$ with a Heyting implication, called $\istl$, and simillarly, an equivalent cut-free system $\igstl$.
In the third section, the cut-elimination theorems for $\stl$ and its extensions will be proved. Then we will use this theorem to deduce important results about $\stl$ and its extensions, such as subformula property, disjuction property and admissibility of the Visser's rule.
And in the last section, we will show that some extensions of $\stl$ have interpolation property.