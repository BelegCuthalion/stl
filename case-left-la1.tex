($L\land_1$) $\mathbf{D_0}$ proves $\mathcal{S} , \p^r B \land C \Rightarrow \nabla^k A$ and has a subproof $\mathbf{D_0}'$ of $\mathcal{S} , \p^r B \Rightarrow \nabla^k A$.
	\begin{prooftree}
		\noLine
		\AXC{$\mathbf{D_0}'$}
		\UIC{$\mathcal{S} , \p^r B \Rightarrow \nabla^k A$}
		\UIC{$\mathcal{S} , \p^r B \land C \Rightarrow \nabla^k A$}
		
		
		\noLine
		\AXC{$\mathbf{D_1}$}
		\UIC{$\mathcal{T} , \p^{l+k} A^n \Rightarrow \Delta$}
		
		\dashedLine\RightLabel{$\nabla Cut$}
		\BIC{$\p^l \mathcal{S} , \p^{r+l} B \land C , \mathcal{T} \Rightarrow \Delta$}
	\end{prooftree}
	From the induction hypothesis for $\mathbf{D_0}'$ and $\mathbf{D_1}$, we have a proof of $\p^l \mathcal{S} , \p^{r+l} B , \mathcal{T} \Rightarrow \Delta$ with rank lower than that of $A$. By applying $L\land$ we have $\p^l \mathcal{S} , \p^{r+l} B \land C , \mathcal{T} \Rightarrow \Delta$, without increasing the cut-rank.
	\begin{prooftree}
		\noLine
		\AXC{$\mathbf{D_0}'$}
		\UIC{$\mathcal{S} , \p^r B \Rightarrow \nabla^k A$}
		
		\noLine
		\AXC{$\mathbf{D_1}$}
		\UIC{$\mathcal{T} , \p^{l+k} A^n \Rightarrow \Delta$}
		
		\RightLabel{IH}
		\BIC{$\p^l \mathcal{S} , \p^{r+l} B , \mathcal{T} \Rightarrow \Delta$}
		
		\RightLabel{$L\land$}
		\UIC{$\p^l \mathcal{S} , \p^{r+l} B \land C , \mathcal{T} \Rightarrow \Delta$}
	\end{prooftree}