($L\land_1$) $\mathbf{D_0}$ proves $\Gamma , \nabla^r (B \land C) \Rightarrow A$ and has a subproof $\mathbf{D_0}'$ of $\Gamma , \nabla^r B \Rightarrow A$.
	\begin{prooftree}
		\noLine
		\AXC{$\mathbf{D_0}'$}
		\UIC{$\Gamma , \nabla^r B \Rightarrow A$}
		\UIC{$\Gamma , \nabla^r (B \land C) \Rightarrow A$}
		
		
		\noLine
		\AXC{$\mathbf{D_1}$}
		\UIC{$\Sigma , (\nabla^l A)^n \Rightarrow \Delta$}
		
		\dashedLine\RightLabel{$MC$}
		\BIC{$\nabla^l \Gamma , \nabla^{r+l} B \land C , \Sigma \Rightarrow \Delta$}
	\end{prooftree}
	From the induction hypothesis for $\mathbf{D_0}'$ and $\mathbf{D_1}$, we have a proof of $\nabla^l \Gamma , \nabla^{r+l} B , \Sigma \Rightarrow \Delta$ with rank lower than that of $A$. By applying $L\land$ we have $\nabla^l \Gamma , \nabla^{r+l} (B \land C) , \Sigma \Rightarrow \Delta$, without increasing the cut-rank.
	\begin{prooftree}
		\noLine
		\AXC{$\mathbf{D_0}'$}
		\UIC{$\Gamma , \nabla^r B \Rightarrow A$}
		
		\noLine
		\AXC{$\mathbf{D_1}$}
		\UIC{$\Sigma , (\nabla^l A)^n \Rightarrow \Delta$}
		
		\RightLabel{IH}
		\BIC{$\nabla^l \Gamma , \nabla^{r+l} B , \Sigma \Rightarrow \Delta$}
		
		\RightLabel{$L\land$}
		\UIC{$\nabla^l \Gamma , \nabla^{r+l} (B \land C) , \Sigma \Rightarrow \Delta$}
	\end{prooftree}